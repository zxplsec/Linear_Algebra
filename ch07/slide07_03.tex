\section{获取逻辑性}

\begin{frame}[fragile]\ft{\secname}
\begin{li}
编写程序,首先输入一个句子,然后计算除单引号和双引号之外的字符出现的次数。
\end{li}
\end{frame}

\begin{frame}[fragile]\ft{\secname}
\lstinputlisting[numbers=left]{ch07/code/chcount.c}
\end{frame}

\begin{frame}[fragile]\ft{\secname}
\begin{lstlisting}
"I'm fine".
There are 7 non-quote characters.
\end{lstlisting}

\end{frame}

\begin{frame}[fragile]\ft{\secname}
\begin{itemize}
\item  首先,程序读入一个字符并检查它是不是一个句号。\\[0.1in]
\item 接下来的语句中,使用了\blue{逻辑与}运算符\&\&。
\item[] 此时,if语句的含义为“若字符不是双引号也不是单引号,则charcount增加1”。\\[0.1in]
\item 要使整个表达式为真,则两个条件都必须为真。逻辑运算符的优先级低于关系运算符,故不必使用圆括号。
\end{itemize}
\end{frame}

\begin{frame}[fragile]\ft{\secname}
\begin{table}
\centering
\caption{逻辑运算符}
\begin{tabular}{c|c}\hline\hline
运算符&含义 \\\hline
\lstinline|&&| & 与  \\[0.1in]
\lstinline|||| & 或  \\[0.1in]
\lstinline|!| & 非  \\\hline\hline
\end{tabular}
\end{table}
\end{frame}

\begin{frame}[fragile]\ft{\secname}
  设 \lstinline|exp1| 和 \lstinline|exp2| 为两个简单的关系表达式,则 \vspace{0.1in}

\begin{itemize}
\item 仅当 \lstinline|exp1| 和 \lstinline|exp2| 都为真时, \lstinline|exp1 && exp2| 才为真。\\[0.2in]
\item 若 \lstinline|exp1| 或 \lstinline|exp2| 为真或二者都为真, \lstinline|exp1| \lstinline||||  \lstinline|exp2| 为真。\\[0.2in]
\item 若 \lstinline|exp1| 为假,则 \lstinline|!exp1| 为真;若 \lstinline|exp1| 为真,则 \lstinline|!exp1| 为假。
\end{itemize}
\end{frame}


\begin{frame}[fragile]\ft{\secname}
请判断以下表达式的值。

\begin{itemize}
\item \lstinline|5 > 2 \&\& 4 > 7|\\[0.2in]
\item \lstinline|5 > 2 || 4 > 7|\\[0.2in]
\item \lstinline|!(4 > 7)|
\end{itemize}
\end{frame}


\begin{frame}[fragile]\ft{头文件iso646.h}
\begin{itemize}
\item 
  C99标准为逻辑运算符增加了可供选择的拼写法,它们在头文件iso646.h中定义。\\[0.1in]
\item 
  若包含了该头文件,可用 \lstinline|and| 代替 \lstinline|&&|,用 \lstinline|or| 代替 \lstinline||||,用 \lstinline|not| 代替 \lstinline|!|。
\end{itemize}
\end{frame}

\begin{frame}[fragile]\ft{头文件iso646.h}
若包含了头文件iso646.h,则
\begin{lstlisting}
if (ch != '"' && ch != '\'')
  charcount++;
\end{lstlisting}
可重写为
\begin{lstlisting}
if (ch != '"' and ch != '\'')
  charcount++;
\end{lstlisting}
\end{frame}


\begin{frame}[fragile]\ft{头文件iso646.h}
\begin{table}
\centering
\caption{逻辑运算符的可选表示法}
\begin{tabular}{c|c}\hline\hline
传统用法 & iso646.h  \\\hline
\lstinline|&&| & \lstinline|and|  \\[0.1in]
\lstinline|||| & \lstinline|or|  \\[0.1in]
\lstinline|!| &  \lstinline|not|  \\\hline\hline
\end{tabular}
\end{table}
\end{frame}


\begin{frame}[fragile]\ft{优先级}
\begin{itemize}
\item 
\lstinline|!| 为单目运算符,优先级同增量运算符相同,仅次于圆括号。\\[0.2in]
\item
\lstinline|&&| 的优先级高于 \lstinline||||,两者的优先级都低于关系运算符,高于赋值运算符。\\[0.2in]
\item[] 如
\begin{lstlisting}
a > b && b > c || b > d
\end{lstlisting}
会被视为
\begin{lstlisting}
((a > b) && (b > c)) || (b > d)
\end{lstlisting}
\end{itemize}
\end{frame}

\begin{frame}[fragile]\ft{求值顺序}
\begin{itemize}
\item 除了那些两个运算符共享一个操作数的情况外,C通常不保证复杂表达式的哪个部分首先被求值。\\[0.2in]
\item [] 
如以下语句中
\begin{lstlisting}
b = (5 + 3) * (9 + 6)
\end{lstlisting}
可能先计算 \lstinline|5 + 3| 的值,也可能先计算 \lstinline|9 + 6| 的值。\\[0.2in]
\item 
C允许这种不确定性,以便编译器设计者可以针对特定系统做出最有效率的选择。
\end{itemize}

\end{frame}


\begin{frame}[fragile]\ft{求值顺序}
\begin{itemize}
\item 但对逻辑运算符的处理是个例外,C保证逻辑表达式是从左到右求值的。\\[0.2in]
\item 
\lstinline|&&| 和 \lstinline|||| 是顺序点,故在程序从一个操作数前进到下一个操作数之前,所有副作用都会生效。\\[0.2in]
\item 
\blue{C保证一旦发现某个元素使表达式总体无效,求值会立即停止。}
\end{itemize}

\end{frame}


\begin{frame}[fragile]\ft{求值顺序}
\begin{lstlisting}
while ((c = getchar()) != ' ' && c != '\n') 
\end{lstlisting} 

\pause \vspace{0.5mm}

\begin{itemize}
\item
该结构用于循环读入字符,直到出现第一个空格符或换行符。
\item 
第一个子表达式给c赋值,然后该值用于第二个子表达式中。
\item
若没有顺序保障,计算机可能试图在c被赋值之前判断第二个表达式。
\end{itemize}
\end{frame}



\begin{frame}[fragile]\ft{求值顺序}
\begin{lstlisting}
while (x++ < 10 && x + y < 20) 
\end{lstlisting} 


\pause \vspace{0.5mm}

\lstinline|&&| 是顺序点,故保证了在对右边表达式求值之前,先把x的值增加1。
\end{frame}


\begin{frame}[fragile]\ft{求值顺序}
\begin{lstlisting}
if (number != 0 && 12/number == 2)
  printf("The number is 5 or 6.\n");
\end{lstlisting} 

\rule{\textwidth}{.5mm} \pause \vspace{0.5mm}

若 \lstinline|number| 的值为0,则第一个表达式为假,就不再对关系表达式求值。这就避免了计算机试图把0作为除数。
\end{frame}

\begin{frame}[fragile]\ft{范围}
可把 \lstinline|&&| 用于测试范围。如要检查90到100范围内的得分,可以这样做
\begin{lstlisting}
if (score >= 90 && score <= 100)
  printf("Excellent!\n");
\end{lstlisting} \pause \vspace{0.1in}
\end{frame}

\begin{frame}[fragile]\ft{范围}
请避免以下做法:
\begin{lstlisting}
if (90 <= score <= 100)
  printf("Excellent!\n");
\end{lstlisting}

\pause \vspace{0.5mm}

这段代码没有语法错误,但有语义错误。因对 \lstinline|<=| 的求值顺序是从左到右的,故测试表达式会被解释为
\begin{lstlisting}
(90 <= score) <= 100
\end{lstlisting}
而子表达式\lstinline| 90 <= score |的值为1或0,总小于100。故不管range取何值,整个表达式总为真。
\end{frame}

