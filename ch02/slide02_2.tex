\section{程序解释}
\begin{frame}\ft{\lstinline|\#include|与头文件}

\lstinputlisting[
language=c,
linerange={2-2},
firstnumber=2,
numbers=left,
]{ch02/code/first.c}

\begin{itemize}
\item
相当于在此处复制文件\lstinline|stdio.h|的完整内容,以方便在多个程序间共享公用信息。\\[0.1in]
\item
\lstinline|\#include|语句是C预处理器指令的一个例子。通常,C编译器在编译前要对源代码做一些准备工作,这称为预处理。
\\[0.1in]
\item
文件\lstinline|stdio.h|包含了有关输入和输出函数的信息,以供编译器使用。
\end{itemize}
\end{frame}

\begin{frame}\ft{\lstinline|main| 函数}
\lstinputlisting[
language=c,
linerange={3-3},
firstnumber=3,
numbers=left,
frame=none
]{ch02/code/first.c}
 
C程序至少包含一个函数,函数是C程序的基本模块。\\[0.1in]
\begin{table}
\centering
\begin{tabular}{ll}
表达式 & 含义 \\ \hline
\lstinline|( )| &  函数名为 \lstinline|main| \\[0.1in]
\lstinline|int| &  返回一整数\\[0.1in]
\lstinline|void| & 不接受任何参数 \\ \hline
\end{tabular}
\end{table}
 
注:\lstinline|main| 函数是任何C程序的唯一入口。
\end{frame}

\begin{frame}[fragile]\ft{\lstinline|main| 函数}
\lstinline|main| 函数有三种定义:
\begin{lstlisting}
// Definition 1: NOT RECOMMENDED
void main() { /* ... */ }

// Definition 2
int main() { /* ... */ }

// Definition 3
int main(int argc, char* argv[]) { /* ... */ }
\end{lstlisting}
\end{frame}

\begin{frame}[fragile]\ft{\lstinline|main| 函数}
考虑\lstinline|main| 函数的两种定义,它们的差别是什么?
\begin{lstlisting}
int main()
{
   /*  */
   return 0;
}

int main(void)
{
   /*  */
   return 0;
}
\end{lstlisting}
\pause
\begin{itemize}
\item 在C++中,两种没有差别,完全一致。
\item 在C中,两种定义都可以,但是第二种定义更好,因它清晰地表明main()在调用时不允许有参量。	
\end{itemize}
\end{frame}

\begin{frame}[fragile]\ft{\lstinline|main| 函数}
在C中,如果一个函数没有指定任何参量,就意味着该函数允许在调用时有任意多个参量或者无参量。
 
\begin{lstlisting}
// Program 1 (Compiles and runs fine in C, but not in C++)
void fun() {  } 
int main(void)
{
    fun(10, "GfG", "GQ");
    return 0;
}
\end{lstlisting}
上述程序能编译和运行,但以下程序编译会失败。
 
\begin{lstlisting}
// Program 2 (Fails in compilation in both C and C++)
void fun(void) {  }
int main(void)
{
    fun(10, "GfG", "GQ");
    return 0;
}
\end{lstlisting}
\end{frame}

\begin{frame}[fragile]\ft{\lstinline|main| 函数}

\begin{itemize}
\item 在C++中,以上两个程序在编译时都会失败,因为在C++中,\lstinline|fun()| 和 \lstinline|fun(void)| 无差别。\\[.1in]
\item 在C中, \lstinline|int main()| 和 \lstinline|int main(void)|的差别在于,前者允许在调用时有任意多个参量,而后者在调用时不能有任何参量。\red{尽管两者在大多数时候无任何差别,但在实践中更推荐使用 \lstinline|int main(void)|。}
\end{itemize}

\end{frame}

\begin{frame}[fragile]\ft{\lstinline|main| 函数}
以下C程序的输出是什么?
\lstinputlisting[]{ch02/code/question1.c}
 
\lstinputlisting[]{ch02/code/question2.c}

\end{frame}

\begin{frame}\ft{注释}
 
\lstinputlisting[
language=c,
linerange={4-4},
firstnumber=4,
numbers=left,
frame=none
]{ch02/code/first.c}
 


\begin{overprint}
\begin{itemize}
\item \lstinline|/* ... */| 之间的内容是程序注释。\\[0.1in]
\item 注释可以让阅读者更容易理解程序。\\[0.1in]
\item 注释可以放在任意位置,甚至和它要解释的语句在同一行。\\[0.1in]
\item 一个较长的注释可以单独放一行,也可以是多行。\\[0.1in]
\item \lstinline|/* ... */| 之间的所有内容都会被编译器忽略。
\end{itemize}
\end{overprint}

\end{frame}

\begin{frame}[fragile]\ft{注释}
\begin{lstlisting}[language=c]
/* This is a C comment.*/

/* This comment is spread over
   two lines. */

/*
   You can do this, too.
*/

/* invalid comment
\end{lstlisting}
\end{frame}

\begin{frame}[fragile]\ft{注释}
C99增加另一种风格的注释,使用 \lstinline|//| 符号,该风格在C++和Java中经常使用。 \vspace{0.2in}


\begin{lstlisting}[language=c]
// Here is a comment confined to one line

int n; // Such comments can go here, too.
\end{lstlisting}
\end{frame}

\begin{frame}[fragile]\ft{花括号,程序体和代码块}
 
\begin{lstlisting}[language=c,frame=none]
{
  ...
}
\end{lstlisting}
 
\begin{itemize}
\item
C函数使用花括号表示函数体的开始和结束。\\[0.2in]
\item
花括号还可以用来把函数中的语句聚集到一个单元或代码块中。
\end{itemize}
\end{frame}

\begin{frame}[fragile]\ft{声明}
\lstinputlisting[
language=c,
linerange={6-6},
firstnumber=6,
numbers=left,
frame=none
]{ch02/code/first.c}

 
声明语句(declaration statement),做两件事情:\vspace{0.1in}
\begin{itemize}
\item 
在内存中为变量 \lstinline|num| 分配了空间。\\[0.1in]
\item 
 \lstinline|int| 说明变量 \lstinline|num| 的类型(整型)。
\end{itemize} \vspace{0.1in}

\pause 

注意:分号指明该行是C的一个语句。分号是语句的一部分。
 
\end{frame}

\begin{frame}[fragile]\ft{声明}
Ansi C要求必须在一个代码块的开始处声明变量,在这之前不允许其他任何语句。

\begin{lstlisting}[
language=c,
numbers=left,
frame=none
]
int main(void)
{
  int n;
  int m;
  n = 5;
  m = 3;
  // other statements
}
\end{lstlisting}
\end{frame}

\begin{frame}[fragile]\ft{声明}
C99遵循C++的惯例,允许把声明放在代码块的任何位置。但是在首次使用变量之前仍必须先声明它。

\begin{lstlisting}[
language=c,
numbers=left,
frame=none
]
int main(void)
{
  int n;
  n = 5;
  // more statements
  int m;
  m = 3;
  // other statements
}
\end{lstlisting}
\end{frame}


\begin{frame}[fragile]\ft{声明}
\begin{wenti}
\begin{itemize}
\item 什么是数据类型?\\[.1in]
\item 怎么给变量取名? \\[.1in]
\item 为什么必须对变量进行声明?
\end{itemize}
\end{wenti}
\end{frame}

\begin{frame}[fragile]\ft{声明}
\begin{enumerate}[1]
\item \red{数据类型}\\[0.1in]
\begin{itemize}
\item   C可以处理多个数据种类,如整数、字符和浮点数。\\[0.1in]
\item   把一个变量声明为整数类型、字符类型或浮点数类型,是计算机能正确存储、获取和解释该数据的基本前提。
\end{itemize}

\item \red{取名}\\[0.1in]
\begin{itemize}
\item  “见其名只其意”。\\[0.1in]
\item  若名字不能表达清楚,可以用注释解释变量所代表的意思。\\[0.1in]
\end{itemize}

\item \red{命名规则:} \\[0.1in]
\begin{itemize}
\item 只能使用\red{字母、数字和下划线},且第一个字符不能为数字。 \\[0.1in]
\item 操作系统和C库通常使用以一个或两个下划线开始的名字,因此最好避免这种用法。 \\[0.1in]
\item \red{区分大小写},如stars不同于Stars或STARS。
\end{itemize}
\end{enumerate}

\end{frame}


\begin{frame}[fragile]\ft{声明}

\begin{table}
\centering
\caption{正确和错误的名字}
\begin{tabular}{c|c}\hline
${\checkmark}$&${\times}$\\[0.1in]\hline
\lstinline|wiggles |&   \lstinline|$zj**|\\[0.1in]
\lstinline|cat2| &  \lstinline|2cat|\\[0.1in]
\lstinline|Hot_Dog| &  \lstinline|Hot-Dog|\\[0.1in]
\lstinline|taxRate| &  \lstinline|tax Rate|\\[0.1in]
\lstinline|_kcab| &  \lstinline|don't|\\\hline
\end{tabular}
\end{table}


\end{frame}


\begin{frame}[fragile]\ft{声明变量的好处}
\begin{itemize} 
\item 把所有变量放在一起,可以让读者更容易掌握程序的内容。
\\[0.1in]
\item  在开始编写程序之前,考虑一下需要声明的变量会促使你做一些计划。\\[0.1in]
\item  声明变量可以帮助避免程序中出现一类很难发现的细微错误,即变量名的错误拼写。\\[0.1in]
\item  所有变量都需要声明,否则程序将不能编译。
\end{itemize}
\end{frame}


\begin{frame}[fragile]\ft{赋值}
\lstinputlisting[
language=c,
linerange={7-7},
firstnumber=7,
numbers=left,
frame=none
]{ch02/code/first.c}

 
这是一条赋值语句(Assignment statement),其含义:把值 \lstinline|1| 赋给变量 \lstinline|num|。
 
\end{frame}


\begin{frame}[fragile]\ft{\lstinline|printf| 函数}
\lstinputlisting[
language=c,
linerange={8-10},
firstnumber=8,
numbers=left,
frame=none
]{ch02/code/first.c}
 

\lstinline|printf| 是一个标准输出函数,其信息由头文件 \lstinline|stdio.h| 指定。
%\vspace{0.1in}
%
%圆括号表明printf为函数名,圆括号内为参数(argument)。这里的参数都是字符串,即双引号之间的内容。
% 
\end{frame}


\begin{frame}[fragile]\ft{转义字符}
 
\red{转义字符}通常用于代表难以表达或无法键入的字符,以 \lstinline|\| 开头。
 
\begin{table}
\centering
\begin{tabular}{c|c} \hline
转义字符 & 含义 \\[.1in] \hline  
\lstinline|\n| & 换行\\[.1in]
\lstinline|\t| & Tab键\\[.1in]
\lstinline|\b| & 退格\\[.1in]
\lstinline|\'| & 单引号\\[.1in]
\lstinline|\"| & 双引号\\[.1in]
$\backslash\backslash$ & 反斜杠\\[.1in]
\hline 
\end{tabular}
\end{table}
\end{frame}


\begin{frame}[fragile]\ft{格式化字符串}
\red{格式化字符串},也称\red{占位符},用以指定输出项的数据类型和输出格式,以 \lstinline|%| 开头。

\begin{table}
\centering
\begin{tabular}{c|c}\hline
占位符 & 含义 \\ \hline  
\lstinline|%d| & 用于输出十进制整数(实际长度)\\[0.1in]
\lstinline|%c| & 输出一个字符\\[0.1in]
\lstinline|%s| & 输出一个字符串\\[0.1in]
\lstinline|%f| & 以小数形式输出实数(整数部分全部输出,小数部分6位)\\
\hline 
\end{tabular}
\end{table}
 
\end{frame}

\begin{frame}[fragile]\ft{\lstinline|return|语句}
\lstinputlisting[
language=c,
linerange={11-11},
firstnumber=11,
numbers=left,
frame=none
]{ch02/code/first.c}

带有返回值的C函数要求使用一个\lstinline|return|语句,该语句包含关键字\lstinline|return|,后面紧跟要返回的值。

\end{frame}
