\section{行列式简介}



行列式出现于线性方程组的求解,它最早是一种速记的表达式,现在已经是数学中一种非常有用的工具。
\begin{itemize}
\item 行列式是由莱布尼茨和日本数学家关孝和分别发明的。 
  \begin{itemize}
  \item 1683年,日本数学家关孝和在其著作《解伏题之法》中也提出了行列式的概念与算法。《解伏题之法》的意思就是“解行列式问题的方法”,书里对行列式的概念和它的展开已经有了清楚的叙述。 
  \item 1693年4月,莱布尼茨在写给洛比达的一封信中使用并给出了行列式,并给出方程组的系数行列式为零的条件。

  \end{itemize}
  
\item
  1750年,瑞士数学家克莱姆在其著作《线性代数分析导引》中,对行列式的定义和展开法则给出了比较完整、明确的阐述,并给出了现在我们所称的解线性方程组的克莱姆法则。
\end{itemize}




\begin{itemize}

\item
  在行列式的发展史上,第一个对行列式理论做出连贯的逻辑的阐述,即把行列式理论与线性方程组求解相分离的人,是法国数学家范德蒙。
  范德蒙自幼在父亲的指导下学习音乐,但对数学有深厚的兴趣,后来终于成为法兰西科学院院士。他给出了用二阶子式和它们的余子式来展开行列式的法则,就对行列式本身这一点来说,他是这门理论的奠基人。
  \\[0.1in] 
\item
  1772年,拉普拉斯在一篇论文中证明了范德蒙提出的一些规则,推广了他的展开行列式的方法。 
  
\item
  继范德蒙之后,在行列式的理论方面,又一位做出贡献的就是另一位法国大数学家柯西。1815年,柯西在一篇论文中给出了行列式的第一个系统的、几乎是近代的处理。其中主要结果之一是行列式的乘法定理。另外,他第一个把行列式的元素排成方阵,采用双足标记法;引进了行列式特征方程的术语;给出了相似行列式的概念;改进了拉普拉斯的行列式展开定理并给出了一个证明等。
\end{itemize}




