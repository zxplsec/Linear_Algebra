\section{逆矩阵}
\begin{frame}
给定一个从$\vx$到$\vy$的线性变换
\begin{equation}\label{yax}
  \vy = \MA \vx  
\end{equation}       
% 即
% $$
% \left\{
%   \begin{array}{c}
%     y_1 = a_{11}x_1 + a_{12}x_2 + \cd + a_{1n}x_n\\[0.1in]
%     y_2 = a_{21}x_1 + a_{22}x_2 + \cd + a_{2n}x_n\\[0.1in]
%     \cd\cd \\[0.1in]
%     y_n = a_{n1}x_1 + a_{n2}x_2 + \cd + a_{nn}x_n
%   \end{array}
% \right.
% $$
\pause 
用$\MA$的伴随阵$\MA^*$左乘(\ref{yax}),得
$$
\MA^* \vy = \MA^* \MA \vx  = |\MA|\vx.
$$ \pause 
当$|\MA|\ne 0$时,有
$$
\vx = \frac{\MA^*}{|\MA|} \vy.
$$ \pause 

记
$$
\red{\MB = \frac{\MA^*}{|\MA|},}
$$
则上式可记为
\begin{equation}\label{xby}
  \vx = \MB \vy,
\end{equation}
它表示一个从$\vy$到$\vx$的线性变换,称为线性变换(\ref{yax})的逆变换。
\end{frame}

\begin{frame}
\begin{zhu}$\MA$与$\MB$的关系:
  \begin{enumerate}
  \item 将(\ref{xby})代入(\ref{yax})
    $$
    \vy = \MA(\MB\vy) = (\MA\MB)\vy
    $$
    可见$\MA\MB$为恒等变换对应的矩阵,故
    $$\MA\MB=\MI.$$    
  \item 将(\ref{yax})代入(\ref{xby})
    $$
    \vx = \MB(\MA\vx) = (\MB\MA)\vx
    $$
    可见$\MB\MA$为恒等变换对应的矩阵,故
    $$\MB\MA=\MI.$$
  \end{enumerate}
\end{zhu}
\pause 
$$
\red{
  \MA\MB = \MB\MA = \MI.
}
$$
\end{frame}

\begin{frame}
\begin{dingyi}[逆矩阵]
  对于$n$阶矩阵$\MA$,如果有一个$n$阶矩阵$\MB$,使
  $$
  \red{
    \MA\MB = \MB\MA = \MI.
  }
  $$
  则称$\MA$是\red{可逆}的,并把$\MB$称为$\MA$的\red{逆矩阵}。
\end{dingyi}
\pause 

\begin{zhu}
  \begin{enumerate}
  \item 可逆矩阵与其逆矩阵为同阶方阵。
  \item $\MA$与$\MB$地位相等,也可称$\MA$为$\MB$的逆矩阵。      
  \end{enumerate}
\end{zhu}
\end{frame}

\begin{frame}

\begin{dingli}
  若$\MA$可逆,则$\MA$的逆阵惟一。
\end{dingli}

\begin{proof}
\end{proof}
\end{frame}

\begin{frame}
\red{
  $\MA$的矩阵记作$\MA^{-1}$,即
  $$
  \MA\MB = \MB\MA = \MI ~ \Rightarrow ~ \MB = \MA^{-1}.
  $$
}


\end{frame}

\begin{frame}

\begin{dingli}
  若$\MA$可逆,则$|\MA|\ne 0$.
\end{dingli}
\begin{proof}

\end{proof}
\end{frame}

\begin{frame}
\begin{dingyi}{代数余子式矩阵,伴随矩阵}
  设$\MA=(a_{ij})_{n\times n}$,$A_{ij}$为行列式$|\MA|$中元素$a_{ij}$的代数余子式,称
  $$
  \mathrm{coef} \MA = (A_{ij})_{n\times n}
  $$
  为$\MA$的\red{代数余子式矩阵},并称$\mathrm{coef} \MA$的转置矩阵为$\MA$的\red{伴随矩阵},记为$\MA^*$,
  即
  $$\red{
    \MA^* = (\mathrm{coef}\MA)^T = \left(
      \begin{array}{cccc}
        A_{11} & A_{21} & \cd & A_{n1} \\
        A_{12} & A_{22} & \cd & A_{n2} \\
        \vd   & \vd   &     & \vd   \\
        A_{1n} & A_{2n} & \cd & A_{nn} \\
      \end{array}
    \right)
  }
  $$
\end{dingyi}
\end{frame}

\begin{frame}
之前已证
$$ \red{
  \MA\MA^* = |\MA|\MI
}
$$
同理可证
$$ \red{
  \MA^*\MA = |\MA|\MI
}
$$

\end{frame}

\begin{frame}
\begin{dingli}
  若$|\MA|\ne 0$,则$\MA$可逆,且
  $$
  \MA^{-1} = \frac1{|\MA|} \MA^*
  $$
\end{dingli}

\begin{proof}

\end{proof} \pause 

\red{
  该定理提供了求$\MA^{-1}$的一种方法。
}
\end{frame}

\begin{frame}
\begin{tuilun}
  若$\MA\MB = \MI$(或$\MB\MA=\MI$),则
  $$
  \MB = \MA^{-1}.
  $$
\end{tuilun}
\begin{proof}

\end{proof}

\red{
  该推论告诉我们,判断$\MB$是否为$\MA$的逆,只需验证$\MA\MB=\MI$或$\MB\MA=\MI$的一个等式成立即可。
}
\end{frame}

\begin{frame}
\begin{dingyi}[奇异阵与非奇异阵]
  当$|\MA|=0$时,$\MA$称为\red{奇异矩阵},否则称为\red{非奇异矩阵}。
\end{dingyi}

\pause 
\begin{zhu}
  \red{可逆矩阵就是非奇异矩阵。}
\end{zhu}
\end{frame}

\begin{frame}
\begin{dingli}可逆矩阵有如下运算规律:
  \begin{enumerate}
  \item[1] 若$\MA$可逆,则$\MA^{-1}$亦可逆,且
    $$(\MA^{-1})^{-1}=\MA.$$
  \item[2] 若$\MA$可逆,$k\ne 0$,则$k\MA$可逆,且
    $$(k\MA)^{-1}= k^{-1}A^{-1}.$$
  \item[3] 若$\MA, ~\MB$为同阶矩阵且均可逆,则$\MA\MB$可逆,且
    $$(\MA\MB)^{-1} = \MB^{-1}\MA^{-1}.$$
  \item[] 若$\MA_1,\MA_2,\cd,\MA_m$皆可逆,则
    $$
    (\MA_1\MA_2\cd\MA_m)^{-1}=\MA_m^{-1}\cd\MA_2^{-1}\MA_1^{-1}
    $$
  \item[4] 若$\MA$可逆,则$\MA^T$亦可逆,且
    $$(\MA^T)^{-1}=(\MA^{-1})^T.$$ 
  \item[5] 若$\MA$可逆,则
    $$|\MA^{-1}|=|\MA|^{-1}.$$
  \end{enumerate}
\end{dingli}
\end{frame}

\begin{frame}
\begin{li}
  已知$\MA = \left(
    \begin{array}{cc}
      a & b \\
      c & d
    \end{array}
  \right)$,求$\MA^{-1}$。
\end{li} \pause 

\begin{jie}
$$
|\MA| = ad-bc, \quad
|\MA^*| = \left(
  \begin{array}{rr}
    d & -b \\
    -c & a
  \end{array}
\right)
$$
\pause 
\begin{itemize}
\item[1] 当$|\MA|=ad-bc=0$时,逆阵不存在;  \pause 
\item[2] 当$|\MA|=ad-bc\ne0$时,
  $$
  \MA^{-1} = \frac1{|\MA|} \MA^* = \frac1{ad-bc}\left(
    \begin{array}{rr}
      d & -b \\
      -c & a
    \end{array}
  \right)
  $$
\end{itemize}
\end{jie}
\end{frame}

\begin{frame}
\begin{li}
  求方阵
  $
  \MA = \left(
    \begin{array}{ccc}
      1 & 2 & 3\\
      2 & 2 & 1\\
      3 & 4 & 3
    \end{array}
  \right)
  $
  的逆阵。
\end{li}
\begin{jie}
$|\MA| = 2$,故$\MA$可逆。 计算$\MA$的余子式
$$
\begin{array}{lll}
  M_{11}=2 & M_{12}=3 & M_{13}=2\\
  M_{21}=-6 & M_{22}=-6 & M_{23}=-2\\
  M_{31}=-4 & M_{32}=-5 & M_{33}=-2
\end{array}
$$ \pause 

$$
\mathrm{coef} \MA = \left(
  \begin{array}{rrr}
    M_{11} & -M_{12} &  M_{13}\\
    -M_{21} &  M_{22} & -M_{23}\\
    M_{31} & -M_{32} &  M_{33}
  \end{array}
\right)  = \left(
  \begin{array}{rrr}
    2 & -3 &  2\\
    6 & -6 &  2 \\
    -4 & 5 & -2
  \end{array}
\right)
$$ \pause 

$$
\MA^* =  \left(
  \begin{array}{rrr}
    2 & 6 &  -4\\
    -3 & -6 & 5 \\
    2 & 2 & -2
  \end{array}
\right)
$$\pause 

故
$$
\MA^{-1} = \frac1{|A|}\MA^* = \left(
  \begin{array}{rrr}
    1 & 3 &  -2\\
    -\frac32 & -3 & \frac52 \\
    1 & 1 & -1
  \end{array}
\right)
$$
\end{jie}
\end{frame}

\begin{frame}
\begin{li}
  设方阵$\MA$满足方程
  $$
  \MA^2 - 3\MA - 10 \MI = \M0,
  $$
  证明:$\MA, \MA-4\MI$都可逆,并求它们的逆矩阵。      
\end{li} \pause 
\begin{proof}
$$
\MA^2-3\MA-10\MI=\M0 \pause ~\Rightarrow~ \MA(\MA-3\MI) = 10\MI 
\pause ~\Rightarrow~ \MA\left[\frac1{10}(\MA-3\MI)\right] = \MI
$$  \pause 
故$\MA$可逆,且\red{$\ds \MA^{-1} = \frac1{10}(\MA-3\MI)$}. \pause 

$$
\MA^2-3\MA-10\MI=\M0 \pause ~\Rightarrow~ (\MA+\MI)(\MA-4\MI) = 6\MI 
\pause ~\Rightarrow~ \frac1{6}(\MA+\MI)(\MA-4\MI) = \MI
$$  \pause 
故$\MA-4\MI$可逆,且\red{$\ds (\MA-4\MI)^{-1} = \frac1{6}(\MA+\MI)$}.

\end{proof}
\end{frame}

\begin{frame}
\begin{li}
  证明:可逆对称矩阵的逆矩阵仍为对称矩阵;可逆反对称矩阵的逆矩阵仍为反对称矩阵。
\end{li}
\end{frame}

\begin{frame}
\begin{li}
  设$\MA=(a_{ij})_{n\times n}$为非零实矩阵,证明:若$\MA^*=\MA^T$,则$\MA$可逆。
\end{li} \pause 
\begin{proof}
欲证$\MA$可逆,只需证$|\MA|\ne 0$。 \pause 

由$\MA^* = \MA^T$及$\MA^*$的定义可知,$\MA$的元素$a_{ij}$等于自身的代数余子式$A_{ij}$。 \pause 

再根据行列式的按行展开定义,有
$$
|\MA| = \sum_{j=1}^n a_{ij} A_{ij} = \sum_{j=1}^n a_{ij}^2.
$$ \pause 

由于$\MA$为非零实矩阵,故$|\MA|\ne 0$,即$\MA$可逆。
\end{proof}
\end{frame}

\begin{frame}
\begin{li}
  设$\MA$可逆,且$\MA^*\MB = \MA^{-1}+\MB$,证明$\MB$可逆,当$\MA=\left(
    \begin{array}{ccc}
      2 & 6 & 0 \\
      0 & 2 & 6\\
      0 & 0 & 2
    \end{array}
  \right)$时,求$\MB$.
\end{li}
\begin{jie}
$$
\MA^*\MB = \MA^{-1}+\MB  \pause \Rightarrow (\MA^*-\MI)\MB = \MA^{-1} \pause 
\Rightarrow |\MA^*-\MI|\cdot |\MB| = |\MA^{-1}|\ne 0 
$$ \pause 
故$\MB$与$\MA^*-\MI$可逆。 \pause 

$$
\MB = (\MA^*-\MI)^{-1} \MA^{-1} = [\MA(\MA^*-\MI)]^{-1} = (\MA\MA^*-\MA)^{-1} = (|\MA|\MI-\MA)^{-1}.
$$
其中
$$
|\MA|\MI-\MA = \left(
  \begin{array}{ccc}
    8 & &\\
      & 8 &\\
      & & 8
  \end{array}
\right) - \left(
  \begin{array}{ccc}
    2 & 6 & 0 \\
    0 & 2 & 6\\
    0 & 0 & 2
  \end{array}
\right) = 6 \left(
  \begin{array}{rrr}
    1 & -1 & 0 \\
    0 &  1 & -1\\
    0 & 0 & 1
  \end{array}
\right)
$$ \pause 

易算得
$$
\MB = \frac16 \left(
  \begin{array}{rrr}
    1 &  1 & 1 \\
    0 &  1 & 1\\
    0 & 0 & 1
  \end{array}
\right)
$$
\end{jie}
\end{frame}

\begin{frame}
\begin{li}
  设$\MA,~\MB$均为$n$阶可逆矩阵,证明:
  \begin{itemize}
  \item[(1).] $(\MA\MB)^*=\MB^*\MA^*$
  \item[(2).] $(\MA^*)^*=|\MA|^{n-2}\MA$ 
  \end{itemize}
\end{li}
\begin{proof}
% \begin{block}{知识点}
%   $$
%   \red{\MA^{-1}=\frac1{|\MA|}\MA^* ~~\Rightarrow~~ \MA^*=|\MA|\MA^{-1}}
%   $$
% \end{block}

% \proofname
(1) 由$|\MA\MB| = |\MA||\MB| \ne 0$可知$\MA\MB$可逆,且有$(\MA\MB)(\MA\MB)^*=|\MA\MB|\MI$。\pause 
故
$$
\begin{array}{cl}
  (\MA\MB)^* &  =|\MA\MB| (\MA\MB)^{-1}
             = |\MA||\MB| \MB^{-1}\MA^{-1} \\[0.2cm]
           & \ds = |\MB| \MB^{-1} |\MA| \MA^{-1}   = \MB^*\MA^*.      
\end{array}
$$ \pause 


(2) 由$(\MA^*)^* \MA^* = |\MA^*|\MI$,得 
$$(\MA^*)^* \red{|\MA|\MA^{-1}} = |\MA|^{n-1}\MI.$$  \pause 
两边同时右乘$\MA$得
$$
(\MA^*)^*=|\MA|^{n-2}\MA.
$$
\end{proof}
\end{frame}

\begin{frame}
\begin{li}
  设$\MP = \left(
    \begin{array}{cc}
      1 & 2\\
      1 & 4
    \end{array}
  \right), ~~ \MLambda=\left(
    \begin{array}{cc}
      1 & \\
        & 2
    \end{array}
  \right), ~~ \MA\MP=\MP\MLambda$,求$\MA^n$.
\end{li} \pause 
\begin{jie}
$$
|\MP|=2, \quad \MP^{-1} = \frac12 \left(
  \begin{array}{rr}
    4 & -2\\
    -1 & 1
  \end{array}
\right).
$$ \pause 

$$
\MA = \MP\MLambda\MP^{-1}, \quad 
\MA^2 = \MP\MLambda\MP^{-1}\MP\MLambda\MP^{-1} = \MP\MLambda^2\MP^{-1}, \quad 
\cd, \quad 
\MA^n = \MP\MLambda^n\MP^{-1}.
$$ \pause 

$$
\MLambda^n = \left(
  \begin{array}{cc}
    1 & \\
      & 2^n
  \end{array}
\right).
$$

$$
\MA^n =  \left(
  \begin{array}{cc}
    1 & 2\\
    1 & 4
  \end{array}
\right) \cdot \left(
  \begin{array}{cc}
    1 & \\
      & 2^n
  \end{array}
\right) \cdot \frac12 \left(
  \begin{array}{rr}
    4 & -2\\
    -1 & 1
  \end{array}
\right) = \left(
  \begin{array}{cc}
    2-2^n & 2^n-1\\
    2-2^{n+1} & 2^{n+1}-1
  \end{array}
\right).
$$

\end{jie}
\end{frame}

\begin{frame}
\begin{jielun}
  令
  $$
  \varphi(\MA) = a_0 \MI + a_1 \MA + \cd + a_m \MA^m.
  $$ \pause 
  \begin{itemize}
  \item[(i)]
    若$\MA = \MP \MLambda \MP^{-1}$,则$\MA^k = \MP \MLambda^k \MP^{-1}$,从而
    $$
    \begin{array}{rcl}
      \varphi(\MA) &=& a_0 \MI + a_1 \MA + \cd + a_m \MA^m \\[0.2cm]
                  &=& \MP a_0 \MI \MP^{-1} + \MP a_1 \MLambda\MP^{-1} + \cd + \MP a_m \MLambda^m \MP^{-1} \\[0.2cm]
                  &=& \MP \varphi(\MLambda) \MP^{-1}.
    \end{array}
    $$ \pause 
  \item[(ii)] 若$\MLambda=\mathrm{diag}(\lambda_1,\lambda_2,\cd,\lambda_n)$为对角阵,则$\MLambda^k=\mathrm{diag}(\lambda_1^k,\lambda_2^k,\cd,\lambda_n^k)$,从而
    $$
    \begin{array}{l}
      \varphi(\MLambda) = a_0 \MI + a_1 \MLambda + \cd + a_m \MLambda^m \\[0.2cm]
      =  a_0 \left(
      \begin{array}{cccc}
        1 & & &\\
          & 1 & & \\
          & & \dd & \\
          & & & 1
      \end{array}
                \right)
                + a_1 \left(
                \begin{array}{cccc}
                  \lambda_1 & & &\\
                            & \lambda_2 & & \\
                            & & \dd & \\
                            & & & \lambda_n
                \end{array}
                                  \right) + \cd \\[0.2in]
      +  a_m \left(
      \begin{array}{cccc}
        \lambda_1^m & & &\\
                    & \lambda_2^m & & \\
                    & & \dd & \\
                    & & & \lambda_n^m
      \end{array}
                          \right)  \pause 
      =\left(
      \begin{array}{cccc}
        \varphi(\lambda_1) & & &\\
                           & \varphi(\lambda_2) & & \\
                           & & \dd & \\
                           & & & \varphi(\lambda_n)
      \end{array}
                                 \right)
    \end{array}
    $$
  \end{itemize}


\end{jielun}
\end{frame}