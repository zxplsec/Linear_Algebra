\section{矩阵的计算}
\subsection{矩阵的加法}
\begin{frame}
\begin{dingyi}[矩阵的加法]
  设有两个$m\times n$矩阵$\MA=(a_{ij})$和$\MB=(b_{ij})$,则矩阵$\MA$与$\MB$之和记为$\MA+\MB$,规定为
  $$
  \MA + \MB = 
  \left(
    \begin{array}{cccc}
      a_{11} + b_{11}  & a_{12} + b_{12}  & \cd & a_{1n} + b_{1n}  \\[0.2cm]
      a_{21} + b_{21}  & a_{22} + b_{22}  & \cd & a_{2n} + b_{2n}  \\[0.2cm]
      \cd            & \cd            &     & \cd            \\[0.2cm]
      a_{n1} + b_{n1}  & a_{n2} + b_{n2}  & \cd & a_{nn} + b_{nn}  
    \end{array}
  \right)
  $$
\end{dingyi} \pause 

\begin{zhu}
  \red{只有当两个矩阵同型时才能进行加法运算。}
\end{zhu}
\end{frame}

\begin{frame}
矩阵加法的运算律:
\begin{itemize}
\item[(i)] $\MA + \MB = \MB + \MA$;
\item[(ii)] $(\MA + \MB) + \MC = \MA + (\MB + \MC)$ 
\end{itemize}

设$\MA = (a_{ij})$,称
$$
-\MA = (-a_{ij}),
$$
为$\MA$的负矩阵,显然有
$$
\MA + (-\MA) = \mathbf{0}.
$$
由此规定矩阵的减法为
$$
\MA - \MB = \MA + (-\MB).
$$
\end{frame}
\subsection{矩阵的数乘}

\begin{frame} 
\begin{dingyi}[矩阵的数乘]
  数$k$与矩阵$\MA$的乘积记作$k \MA$或$\MA k$,规定为
  $$
  k \MA = 
  \left(
    \begin{array}{cccc}
      k a_{11}   & k a_{12}   & \cd & k a_{1n}  \\[0.2cm]
      k a_{21}   & k a_{22}   & \cd & k a_{2n}  \\[0.2cm]
      \cd     & \cd     &     & \cd    \\[0.2cm]
      k a_{m1}   & k a_{m2}   & \cd & k a_{mn}  
    \end{array}
  \right)
  $$
\end{dingyi}\pause 

\begin{zhu}
  \red{用数$k$乘一个矩阵,需要把数$k$乘矩阵的每一个元素,这与行列式的数乘性质不同。}
\end{zhu}
\end{frame}

\begin{frame}
矩阵数乘的运算律:
\begin{itemize}
\item[(i)] $(k l)\MA =  k(l \MA)$;
\item[(ii)] $(k + l) \MA  = k \MA + l \MA$;
\item[(iii)] $k (\MA + \MB)  = k \MA + k \MB$
\end{itemize}
\purple{矩阵加法与矩阵数乘统称为矩阵的线性运算}
\end{frame}

\subsection{矩阵的乘法}
\begin{frame}

设有两个线性变换
\begin{equation}\label{lt1}
  \left\{
    \begin{array}{l}
      y_1 = a_{11} x_1 + a_{12} x_2 + a_{13} x_3, \\[0.2cm]
      y_2 = a_{21} x_1 + a_{22} x_2 + a_{23} x_3,
    \end{array}
  \right.
\end{equation}
\begin{equation}\label{lt2}
  \left\{
    \begin{array}{l}
      x_1 = b_{11} t_1 + b_{12} t_2 , \\[0.2cm]
      x_2 = b_{21} t_1 + b_{22} t_2 , \\[0.2cm]
      x_3 = b_{31} t_1 + b_{32} t_2 , 
    \end{array}
  \right.
\end{equation}\pause 

若想求从$t_1, t_2$到$y_1, y_2$的线性变换,可将(\ref{lt2})代入(\ref{lt1}),便得
\begin{equation}\label{lt3}
  \left\{
    \begin{array}{l}
      y_1 = (\blue{a_{11}b_{11} + a_{12}b_{21} + a_{13}b_{31}}) t_1 + (\blue{a_{11}b_{12} + a_{12}b_{22} + a_{13}b_{32}})t_2, \\[0.2cm]
      y_2 = (\blue{a_{21}b_{11} + a_{22}b_{21} + a_{23}b_{31}}) t_1 + (\blue{a_{21}b_{12} + a_{22}b_{22} + a_{23}b_{32}})t_2.
    \end{array}
  \right.
\end{equation}
\red{线性变换(\ref{lt3})可看成是先作线性变换(\ref{lt2})再作线性变换(\ref{lt1})的结果}。
\end{frame}

\begin{frame}
把线性变换(\ref{lt3})叫做线性变换(\ref{lt1})和(\ref{lt2})的乘积,
相应地把线性变换(\ref{lt3})对应的矩阵定义为线性变换(\ref{lt1})与(\ref{lt2})所对应矩阵的乘积,即
$$
\begin{array}{ll}
  & \left(
    \begin{array}{lll}
      a_{11} & a_{12} & a_{13}\\[0.1cm]
      a_{21} & a_{22} & a_{23}
    \end{array}
                        \right)
                        \left(
                        \begin{array}{ll}
                          b_{11} & b_{12} \\[0.1cm]
                          b_{21} & b_{22} \\[0.1cm]
                          b_{31} & b_{32} 
                        \end{array}
                                   \right) \\[0.8cm]
  = & \left(
      \begin{array}{cc}
        a_{11}b_{11} + a_{12}b_{21} + a_{13}b_{31}  &  a_{11}b_{12} + a_{12}b_{22} + a_{13}b_{32} \\[0.1cm]
        a_{21}b_{11} + a_{22}b_{21} + a_{23}b_{31}  &  a_{21}b_{12} + a_{22}b_{22} + a_{23}b_{32}
      \end{array}
                                                      \right)
\end{array}
$$
\end{frame}

\begin{frame}
\begin{dingyi}[矩阵乘法]
  设$A$为$m\times n$矩阵,$B$为$n\times s$矩阵,即
  $$
  A = \left(
    \begin{array}{cccc}
      a_{11} & a_{12} & \cd & a_{1n}\\
      a_{21} & a_{22} & \cd & a_{2n}\\
      \vd   & \vd   &     & \vd \\
      a_{m1} & a_{m2} & \cd & a_{mn}
    \end{array}
  \right), ~~
  B = \left(
    \begin{array}{cccc}
      b_{11} & b_{12} & \cd & b_{1s}\\
      b_{21} & b_{22} & \cd & b_{2s}\\
      \vd   & \vd   &     & \vd \\
      b_{n1} & b_{n2} & \cd & b_{ns}
    \end{array}
  \right)
  $$
  则$A$与$B$之乘积$AB$(记为$C=(c_{ij})$)为$m\times s$矩阵,且
  $$
  c_{ij} = c_{i1}b_{1j} + c_{i2}b_{2j} + \cd + c_{in}b_{nj} = \sum_{k=1}^na_{ik}b_{kj}.
  $$
\end{dingyi} \pause 
\begin{zhu}
  两个矩阵$A$与$B$相乘有意义的前提是\red{$A$的列数等于$B$的行数}。
\end{zhu}

\end{frame}

\begin{frame}
% l' unite
\newcommand{\myunit}{0.5 cm}
\tikzset{
  node style sp/.style={draw,circle,minimum size=\myunit},
  node style ge/.style={circle,minimum size=\myunit},
  arrow style mul/.style={draw,sloped,midway,fill=white},
  arrow style plus/.style={midway,sloped,fill=white},
}
\begin{figure}[htbp]
  \centering
  \begin{tikzpicture}[scale=0.5,>=latex]
    % les matrices
    \matrix (A) [matrix of math nodes,%
    nodes = {node style ge},%
    ampersand replacement=\&,
    left delimiter  = (,%
    right delimiter = )] at (0,0)   {%
      a_{11} \& a_{12} \& \ldots \& a_{1n}  \\
      \blue{a_{21}}  \& \blue{a_{22}}; \& \ldots%
      \& \blue{a_{2n}} \\
      \vdots \& \vdots \& \ddots \& \vdots  \\
      a_{m1} \& a_{m2} \& \ldots \& a_{mn}  \\
    };
    
    % \node [draw,below=1pt] at (A.south) { 
    % $A$ : \textcolor{red}{$n$ rows} $p$ columns};

    \matrix (B) [matrix of math nodes,%
    % nodes = {node style ge},%
    ampersand replacement=\&,
    left delimiter  = (,%
    right delimiter =)] at (20*\myunit,18*\myunit){%
      b_{11} \& \blue{b_{12}}%
      \& \ldots \& b_{1s}  \\
      b_{21} \& \blue{b_{22}}%
      \& \ldots \& b_{2s}  \\
      \vdots \& \vdots \& \ddots \& \vdots  \\
      b_{n1} \& \blue{b_{n2}}%
      \& \ldots \& b_{ns}  \\
    };
    
    %% \node [draw,above=10pt] at (B.north) { 
    %% $B$ : $p$ rows \textcolor{red}{$q$ columns}
    %% };
    %   matrice 
    \matrix (C) [matrix of math nodes,%
    nodes = {node style ge},%
    ampersand replacement=\&,
    left delimiter  = (,%
    right delimiter = )] at (20*\myunit,0) {%
      c_{11} \& c_{12} \& \ldots \& c_{1s} \\
      c_{21} \& \red{c_{22}}%
      \& \ldots \& c_{2s} \\
      \vdots \& \vdots \& \ddots \& \vdots \\
      c_{m1} \& c_{m2} \& \ldots \& c_{ms} \\
    };
    
    % les fleches
    \draw[<->,red](A-2-1) to[in=180,out=90]
    node[arrow style mul] (x) {$a_{21}\times b_{12}$} (B-1-2);
    
    \draw[<->,red](A-2-2) to[in=180,out=90]
    node[arrow style mul] (y) {$a_{22}\times b_{22}$} (B-2-2);
    
    \draw[<->,red](A-2-4) to[in=180,out=90]
    node[arrow style mul] (z) {$a_{2n}\times b_{n2}$} (B-4-2);
    
    \draw[red,->] (x) to node[arrow style plus] {$+$} (y)%
    to node[arrow style plus] {$+\raisebox{.5ex}{\ldots}+$} (z)%
    to (C-2-2.north west);
    
    \draw[blue] (A-2-1.north) -- (C-2-2.north);
    \draw[blue] (A-2-1.south) -- (C-2-2.south);

    \draw[blue] (B-1-2.west)  -- (C-2-2.west);
    \draw[blue] (B-1-2.east)  -- (C-2-2.east);

    %% \node [draw,below=10pt] at (C.south) 
    %% {$ C=A\times B$ : \textcolor{red}{$n$ rows}  \textcolor{red}{$q$ columns}};
  \end{tikzpicture}
  \caption{矩阵乘法示意图}
\end{figure}
\end{frame}

\begin{frame}
\begin{li}
  求矩阵
  $
  \MA = \left(
    \begin{array}{rrrr}
      1&2&-1\\
      -1&3&4\\
      1&1&1
    \end{array}
  \right)
  $
  与
  $
  \MB = \left(
    \begin{array}{rrrr}
      5&6\\
      -5&-6\\
      6&0
    \end{array}
  \right)
  $
  的乘积$\MA\MB$
\end{li} \pause
\begin{jie}
  $$
  \begin{array}{cl}
    \MA\MB & = \left(
           \begin{array}{ccc}
             1\times5 + 2\times(-5) + (-1)\times6 &
                                                    1\times6 + 2\times(-6) + (-1)\times0 \\[0.1cm]
             (-1)\times5 + 3\times(-5) +    4\times6 &
                                                       (-1)\times6 + 3\times(-6) +    4\times0 \\[0.1cm]
             1\times5 + 1\times(-5) +    1\times6 &
                                                    1\times6 + 1\times(-6) +    1\times0 
           \end{array}
                                                    \right)  =\left(
                                                    \begin{array}{rr}
                                                      -11 & -6\\
                                                      4 & -24\\
                                                      6 & 0
                                                    \end{array}
                                                          \right)   
  \end{array}
  $$
\end{jie}
\end{frame}

\begin{frame}

\begin{li}
  设
  $$
  \MA = \left(
    \begin{array}{c}
      a_1\\
      a_2\\
      \vd\\
      a_n
    \end{array}
  \right), ~~
  \MB = \left(
    \begin{array}{cccc}
      b_1 & b_2 & \cd & b_n
    \end{array}
  \right)
  $$
  计算$\MA\MB$与$\MB\MA$.
\end{li} \pause 
\begin{jie}
  $$
  \MA\MB = \left(
    \begin{array}{c}
      a_1\\
      a_2\\
      \vd\\
      a_n
    \end{array}
  \right)\left(
    \begin{array}{cccc}
      b_1 & b_2 & \cd & b_n
    \end{array}
  \right) 
  = \left(
    \begin{array}{cccc}
      a_1b_1 & a_1b_2 & \cd & a_1b_n\\
      a_2b_1 & a_2b_2 & \cd & a_2b_n\\
      \vd & \vd & & \vd\\
      a_nb_1 & a_nb_2 & \cd & a_nb_n
    \end{array}
  \right)
  $$  \pause 
  $$
  \MB\MA = \left(
    \begin{array}{cccc}
      b_1 & b_2 & \cd & b_n
    \end{array}
  \right)\left(
    \begin{array}{c}
      a_1\\
      a_2\\
      \vd\\
      a_n
    \end{array}
  \right)  
  = a_1b_1+a_2b_2+\cd+a_nb_n.
  $$
\end{jie}
\end{frame}

\begin{frame}

\begin{li}
  设
  $$
  \MA = \left(
    \begin{array}{cc}
      a & a\\
      -a & -a
    \end{array}
  \right), ~
  \MB = \left(
    \begin{array}{cc}
      b & -b\\
      -b & b
    \end{array}
  \right),~
  \MC = \left(
    \begin{array}{cc}
      -1 & 1\\
      1 & -1
    \end{array}
  \right)
  $$
  计算$\MA\MB, ~\MA\MC$和$\MB\MA$.
\end{li}  \pause 

\begin{jie}
  $$
  \MA\MB = \MA\MC = \left(
    \begin{array}{cc}
      0 & 0\\
      0 & 0
    \end{array}
  \right)
  $$ \pause 
  $$
  \MB\MA = \left(
    \begin{array}{cc}
      2ab & 2ab\\
      -2ab & -2ab
    \end{array}
  \right)
  $$      
\end{jie}
\end{frame}

\begin{frame}

由以上例题可以看出一些结论: 
\begin{itemize}
\item[1] 矩阵乘法不满足交换律。
\item[] 若$\MA\MB\ne\MB\MA$,则称\red{$\MA$与$\MB$不可交换}。
\item[] 若$\MA\MB=\MB\MA$,则称\red{$\MA$与$\MB$可交换}。  \\[0.2in] \pause 
\item[2] $\MA\MB=\M0 ~\nRightarrow~ \MA = \M0 \mbox{或} \MB = \M0$
\item[]  $\MA\ne\M0\mbox{且} \MB\ne\M0 ~\xLongrightarrow[]{\mbox{有可能}}~ \MA \MB= \M0$ \\[0.2in] \pause 
\item[3] 矩阵乘法不满足消去律,即当$\MA\ne\M0$时,
  $$
  \MA\MB=\MA\MC ~\nRightarrow~ \MB=\MC
  $$ \pause 
\item[]当$\MA$为非奇异矩阵,即$|\MA|\ne 0$时,
  $$
  \MA\MB=\M0 ~\Rightarrow~ \MB = \M0; ~~~
  \MA\MB=\MA\MC ~\Rightarrow~ \MB = \MC.
  $$
\end{itemize}
\end{frame}

\begin{frame}
矩阵乘法的运算律:
\begin{itemize}
\item[(i)] 结合律
\item[] $$ (\MA\MB)\MC = \MA(\MB\MC)$$
\item[(ii)] 数乘结合律
\item[] $$ k(\MA\MB) = (k\MA)\MB = \MA(k\MB)$$
\item[(iii)] 左结合律
\item[] $$ \MA(\MB+\MC) = \MA\MB+\MA\MC$$
\item[] 右结合律
\item[] $$ (\MB+\MC)\MA = \MB\MA+\MC\MA$$
  
\end{itemize}
\end{frame}


\subsection{一些特殊矩阵及其运算}
\begin{frame}
% 
\begin{dingyi}[单位矩阵与数量矩阵]
  \begin{itemize}
  \item[1] 主对角元全为1,其余元素全为零的$n$阶方阵,称为$n$阶\red{单位矩阵},记为$\MI_n, \MI, \ME$
    $$
    \MI_n = \left(
      \begin{array}{cccc}
        1 & & &\\
          & 1 & & \\
          & & \dd & \\
          & & & 1
      \end{array}
    \right)
    $$ 
  \item[2] 主对角元全为非零数$k$,其余元素全为零的$n$阶方阵,称为$n$阶\red{数量矩阵},记为$k\MI_n, k\MI, k\ME$
    $$
    k\MI_n = \left(
      \begin{array}{cccc}
        k & & &\\
          & k & & \\
          & & \dd & \\
          & & & k
      \end{array}
    \right)~(k\ne 0)
    $$
  \end{itemize}
\end{dingyi}
\end{frame}

\begin{frame}
% 
\begin{zhu}
  \begin{itemize}
  \item[1] \red{单位阵在矩阵乘法中的作用相当于数$1$在数的乘法中的作用。}
  \item[2] 一些等式:
    $$
    (k\MI) \MA = k(\MI\MA) = k\MA, ~~
    \MA(k\MI) = k(\MA\MI) = k\MA.
    $$
  \end{itemize}
\end{zhu}
\end{frame}

\begin{frame}

\begin{dingyi}[对角矩阵]
  非对角元皆为零的$n$阶方阵称为$n$阶\red{对角矩阵},记作$\MLambda$,即
  $$
  \MLambda = \left(
    \begin{array}{cccc}
      \lambda_1 & & &\\
                & \lambda_2 & & \\
                & & \dd & \\
                & & & \lambda_n
    \end{array}
  \right)
  $$
  或记作$\mathrm{diag}(\lambda_1,\lambda_2,\cd,\lambda_n)$.
\end{dingyi}
\end{frame}

\begin{frame}
\begin{zhu}
  \begin{itemize}
  \item[1] 用对角阵$\MLambda$左乘$\MA$,就是用$\lambda_i(i=1,\cd,n)$乘$\MA$中第$i$行的每个元素;
    $$
    \left(
      \begin{array}{cccc}
        \lambda_1 & & &\\
                  & \lambda_2 & & \\
                  & & \dd & \\
                  & & & \lambda_n
      \end{array}
    \right)
    \left(
      \begin{array}{cccc}
        a_{11} & a_{12} & \cd & a_{1n}\\
        a_{21} & a_{22} & \cd & a_{2n}\\
        \vd & \vd &  & \vd\\
        a_{n1} & a_{n2} & \cd & a_{nn}
      \end{array}
    \right) = 
    \left(
      \begin{array}{cccc}
        \lambda_1 a_{11} & \lambda_1a_{12} & \cd & \lambda_1a_{1n}\\
        \lambda_2a_{21} & \lambda_2a_{22} & \cd & \lambda_2a_{2n}\\
        \vd & \vd &  & \vd\\
        \lambda_na_{n1} & \lambda_na_{n2} & \cd & \lambda_na_{nn}
      \end{array}
    \right)
    $$         \pause 
  \item[2] 用对角阵$\MLambda$右乘$\MA$,就是用$\lambda_i(i=1,\cd,n)$乘$\MA$中第$i$列的每个元素,即
    $$
    \left(
      \begin{array}{cccc}
        a_{11} & a_{12} & \cd & a_{1n}\\
        a_{21} & a_{22} & \cd & a_{2n}\\
        \vd & \vd &  & \vd\\
        a_{n1} & a_{n2} & \cd & a_{nn}
      \end{array}
    \right)  
    \left(
      \begin{array}{cccc}
        \lambda_1 & & &\\
                  & \lambda_2 & & \\
                  & & \dd & \\
                  & & & \lambda_n
      \end{array}
    \right)
    = 
    \left(
      \begin{array}{cccc}
        \lambda_1a_{11} & \lambda_2a_{12} & \cd & \lambda_na_{1n}\\
        \lambda_1a_{21} & \lambda_2a_{22} & \cd & \lambda_na_{2n}\\
        \vd & \vd &  & \vd\\
        \lambda_1a_{n1} & \lambda_2a_{n2} & \cd & \lambda_na_{nn}
      \end{array}
    \right)
    $$
  \end{itemize}
\end{zhu}

\end{frame}

\begin{frame}
% 
\begin{dingyi}[三角矩阵]
  \begin{itemize}
  \item[1] 主对角线以上的元素全为零的$n$阶方阵称为\red{上三角矩阵}($a_{ij}=0, ~i>j$)
    $$
    \left(
      \begin{array}{cccc}
        a_{11} & a_{12} & \cd & a_{1n} \\
               & a_{22} & \cd & a_{2n} \\
               &       & \dd & \vd   \\
               &       &     & a_{nn}
      \end{array}
    \right)
    $$
  \item[2] 主对角线以下的元素全为零的$n$阶方阵称为\red{下三角矩阵}($a_{ij}=0, ~i<j$)
    $$
    \left(
      \begin{array}{cccc}
        a_{11} &       &     &       \\
        a_{21} & a_{22} &     &  \\
        \vd   & \vd   & \dd &    \\
        a_{n1} & a_{n2} & \cd & a_{nn}
      \end{array}
    \right)
    $$
  \end{itemize}
\end{dingyi}
\end{frame}

\begin{frame}

\begin{li}
  证明:两个上三角矩阵的乘积仍为上三角矩阵。
\end{li} \pause 
\begin{proof}
  设
  $$
  \MA = \left(
    \begin{array}{cccc}
      a_{11} & a_{12} & \cd & a_{1n} \\
             & a_{22} & \cd & a_{2n} \\
             &       & \dd & \vd   \\
             &       &     & a_{nn}
    \end{array}
  \right), ~     \MB = \left(
    \begin{array}{cccc}
      b_{11} & b_{12} & \cd & b_{1n} \\
             & b_{22} & \cd & b_{2n} \\
             &       & \dd & \vd   \\
             &       &     & b_{nn}
    \end{array}
  \right)
  $$
  
  令$\MC = \MA\MB = (c_{ij})$,
  则当$i>j$时,
  $$
  c_{ij} = \sum_{k=1}^n a_{ik}b_{kj}   
  = \sum_{k=1}^{i-1} a_{ik}b_{kj}  + \sum_{k=i}^n a_{ik}b_{kj}  
  = \sum_{k=1}^{i-1} \red{\underbrace{a_{ik}}_{\Downarrow\atop0}}b_{kj}  
  + \sum_{k=i}^n a_{ik}\red{\underbrace{b_{kj}}_{\Downarrow\atop0}}  = 0.
  $$
\end{proof}
\end{frame}

\begin{frame}

\begin{zhu}
  两个下三角矩阵的乘积仍是下三角矩阵。
\end{zhu}
\end{frame}

\begin{frame}

设线性方程组
$$
\left\{
  \begin{array}{c}
    a_{11}x_1 + a_{12}x_2 + a_{1n}x_n = b_1, \\[0.2cm]
    a_{21}x_1 + a_{22}x_2 + a_{2n}x_n = b_2, \\[0.2cm]
    \vd \\[0.2cm]
    a_{m1}x_1 + a_{m2}x_2 + a_{mn}x_n = b_m.
  \end{array}
\right.
$$

第$i$个方程可表示为
$$
\left(
  \begin{array}{cccc}
    a_{i1} & a_{i2} & \cd &  a_{in}
  \end{array}
\right)
\left(
  \begin{array}{c}
    x_{1} \\
    x_{2} \\
    \cd   \\
    x_{n}
  \end{array}
\right) = b_i, \quad i=1,2,\cd,n.
$$ 
因此上述线性方程组可表示为
$$
\underbrace{
  \left(
    \begin{array}{cccc}
      a_{11} & a_{12} & \cd &  a_{1n} \\[0.1cm]
      a_{21} & a_{22} & \cd &  a_{2n} \\[0.1cm]
      \vd   & \vd   &     & \vd \\[0.1cm]
      a_{n1} & a_{n2} & \cd &  a_{nn} \\[0.1cm]
    \end{array}
  \right)
}_{\red{\MA}}
\underbrace{
  \left(
    \begin{array}{c}
      x_{1} \\[0.1cm]
      x_{2} \\[0.1cm]
      \vd  \\[0.1cm]
      x_{n}
    \end{array}
  \right)
}_{\red{\vx}}
=     
\underbrace{
  \left(
    \begin{array}{c}
      b_{1} \\[0.1cm]
      b_{2} \\[0.1cm]
      \vd  \\[0.1cm]
      b_{n}
    \end{array}
  \right)
}_{\red{\vb}},  
\quad \red{\MA\vx=\vb}.
$$
% 
% 
\end{frame}

\begin{frame}
\begin{dingli}
  设$\MA,\MB$是两个$n$阶方阵,则
  $$
  |\MA\MB| = |\MA||\MB|.
  $$
\end{dingli} \pause 
\begin{proof}

  设$\MA=(a_{ij})_{n\times n}, ~\MB=(b_{ij})_{n\times n}$,则
  $$
  \begin{aligned}
    |\MA||\MB| =& \left|
      \begin{array}{cccccccc}
        \blue{a_{11}} & \blue{a_{12}} & \blue{\cd} &  \blue{a_{1n}} & \purple{0} & \purple{0} & \purple{\cd} & \purple{0}\\
        \blue{a_{21}} & \blue{a_{22}} & \blue{\cd} &  \blue{a_{2n}} & \purple{0} & \purple{0} & \purple{\cd} & \purple{0}\\
        \blue{\vd}   & \blue{\vd}   &     & \blue{\vd}    & \purple{\vd} & \purple{\vd} & & \purple{\vd}\\
        \blue{a_{n1}} & \blue{a_{n2}} & \blue{\cd} &  \blue{a_{nn}} & \purple{0} & \purple{0} & \purple{\cd} & \purple{0}\\
        -1    & 0     & \cd &   0    & b_{11} & b_{12} & \cd & b_{1n} \\
        0 & -1 & \cd & 0& b_{21} & b_{22} & \cd & b_{2n} \\
        \vd&\vd & & \vd &  \vd  &\vd &  & \vd   \\
        0 & 0 & \cd & -1 & b_{n1} & b_{n2} &\cd     & b_{nn}
      \end{array}
    \right|\\ \pause 
    \xlongequal[i=1,\cd,n]{r_1+a_{1i}r_{n+i}}&
    \left|
      \begin{array}{cccccccc}
        \blue{0} & \blue{0} & \blue{\cd} &  \blue{0} & \purple{c_{11}} & \purple{c_{12}} & \purple{\cd} & \purple{c_{1n}}\\
        \blue{a_{21}} & \blue{a_{22}} & \blue{\cd} &  \blue{a_{2n}} & \purple{0} & \purple{0} & \purple{\cd} & \purple{0}\\
        \blue{\vd}   & \blue{\vd}   &     & \blue{\vd}    & \purple{\vd} & \purple{\vd} & & \purple{\vd}\\
        \blue{a_{n1}} & \blue{a_{n2}} & \blue{\cd} &  \blue{a_{nn}} & \purple{0} & \purple{0} & \purple{\cd} & \purple{0}\\
        -1    & 0     & \cd &   0    & b_{11} & b_{12} & \cd & b_{1n} \\
        0 & -1 & \cd & 0& b_{21} & b_{22} & \cd & b_{2n} \\
        \vd&\vd & & \vd &  \vd  &\vd &  & \vd   \\
        0 & 0 & \cd & -1 & b_{n1} & b_{n2} &\cd     & b_{nn}
      \end{array}
    \right|
  \end{aligned}
  $$
\end{proof} 
\end{frame}

\begin{frame}   
  仿照上述步骤,可将行列式的左上角元素全消为零,即得
  $$
  \begin{array}{rl}
    |\MA||\MB| &  = \left|
               \begin{array}{cc}
                 \MA & \M0\\
                 -\MI & \MB
               \end{array}
                        \right|  = \left|
                        \begin{array}{cc}
                          \M0 & \MA\MB \\
                          -\MI & \MB
                        \end{array}
                                 \right| = (-1)^n\left|
                                 \begin{array}{cc}
                                   \MA\MB & \M0\\
                                   \MB   & -\MI
                                 \end{array}
                                          \right| \\[0.2in]
             &  = (-1)^n |\MA\MB||-\MI_n|  = (-1)^n |\MA\MB|(-1)^n \\[0.1in]
             &  = |\MA\MB|.  
  \end{array}
  $$
\end{frame}

\begin{frame}
\begin{li}
  设
  $$
  \MA = \left(
    \begin{array}{cccc}
      a_{11} & a_{12} & \cd & a_{1n} \\
      a_{21} & a_{22} & \cd & a_{2n} \\
      \vd   & \vd   &     & \vd   \\
      a_{n1} & a_{n2} & \cd & a_{nn} \\
    \end{array}
  \right), ~~
  \MA^* = \left(
    \begin{array}{cccc}
      A_{11} & A_{21} & \cd & A_{n1} \\
      A_{12} & A_{22} & \cd & A_{n2} \\
      \vd   & \vd   &     & \vd   \\
      A_{1n} & A_{2n} & \cd & A_{nn} \\
    \end{array}
  \right)
  $$
  其中$A_{ij}$是行列式$|\MA|$中元素$a_{ij}$的代数余子式。
  证明:\red{当$|\MA|\ne 0$时,$|\MA^*|=|\MA|^{n-1}$}.      
\end{li}
\end{frame}

\begin{frame}
\begin{proof}      
  设$\MA\MA^*=\MC=(c_{ij})$,其中
  $$
  c_{ij} = a_{i1}A_{j1} + a_{i2}A_{j2} + \cd + a_{in}A_{jn} = \delta_{ij}|\MA|
  $$
  
  于是
  $$
  \MA\MA^* = \left(
    \begin{array}{cccc}
      |\MA|&&&\\
          &|\MA|&&\\
          &&\dd&\\
          &&&|\MA|
    \end{array}
  \right) = |\MA| \MI_n,
  $$ 
  因此,
  $$
  |\MA||\MA^*| = |\MA\MA^*| = |\MA|^n,
  $$ 
  由于$|\MA|\ne 0$,故$|\MA^*|=|\MA|^{n-1}$.
\end{proof}
% % 
\end{frame}

\begin{frame}

\begin{dingyi}[矩阵幂]
  设$\MA$是$n$阶矩阵,$k$个$\MA$的连乘积称为$\MA$的$k$次幂,记作$\MA^k$,即
  $$
  \MA^k = \underbrace{\MA~ \MA~ \cd ~\MA}_{k}
  $$
\end{dingyi}
% % 
\end{frame}

\begin{frame}
矩阵幂的运算律:
\begin{itemize}
\item[1] 当$m,k$为正整数时,
  $$
  \MA^m \MA^k = \MA^{m+k}, \quad
  (\MA^m)^k = \MA^{mk}.
  $$  
\item[2]
  当$\MA\MB$不可交换时,一般情况下,
  $$
  (\MA\MB)^k \ne \MA^k\MB^k 
  $$  
\item[3]
  当$\MA\MB$可交换时,
  $$
  (\MA\MB)^k = \MA^k\MB^k =  \MB^k\MA^k. 
  $$
\end{itemize}
\end{frame}

\begin{frame}
\begin{dingyi}[矩阵多项式]
  设$f(x)=a_kx^k+a_{k-1}x^{k-1}+\cd+a_1x+a_0$是$x$的$k$次多项式,$\MA$是$n$阶矩阵,则
  $$
  f(\MA)=a_k\MA^k+a_{k-1}\MA^{k-1}+\cd+a_1\MA+a_0\MI
  $$
  称为矩阵$\MA$的$k$次多项式。
\end{dingyi}
\pause 
% 
\begin{zhu}
  \begin{itemize}
  \item[1] 若$f(x), g(x)$为多项式,$\MA,\MB$皆是$n$阶矩阵,则
    $$
    f(\MA)g(\MA) = g(\MA)f(\MA).
    $$
  \item[2] 当$\MA\MB$不可交换时,一般
    $$f(\MA)g(\MB)\ne g(\MB)f(\MA)$$
  \end{itemize}
\end{zhu}
\end{frame}