\section{习题}

\begin{frame}
  \begin{footnotesize}
    \begin{li}{28}
      求与$\A=\left(
      \begin{array}{rrr}
        1&0&0\\
        0&1&2\\
        0&1&-2
      \end{array}
      \right)$可交换的全体三阶矩阵。
    \end{li}
  \end{footnotesize}
\end{frame}



\begin{frame}
  \begin{footnotesize}
    \begin{li}{29}
      已知$\A$是对角元互不相等的$n$阶对角矩阵,即
      $$
      \A=\left(
      \begin{array}{cccc}
        a_1&&&\\
        &a_2&&\\
        &&\dd&\\
        &&&a_n
      \end{array}
      \right)
      $$
      当$i\ne j$时,$a_i\ne a_j$。证明:与$\A$可交换的矩阵必是对角矩阵。
    \end{li}
    \pause\proofname
    设与$\A$可交换的矩阵为$\B=\left(
    \begin{array}{rrrr}
      b_{11}&b_{12}&\cd&b_{1n}\\
      b_{21}&b_{22}&\cd&b_{2n}\\
      \vd&\vd&\dd&\vd\\
      b_{n1}&b_{n2}&\cd&b_{nn}
    \end{array}
    \right)$
    由$\A\B=\B\A$,即
    $$
    \left(
    \begin{array}{rrrr}
      a_1b_{11}&a_1b_{12}&\cd&a_1b_{1n}\\
      a_2b_{21}&a_2b_{22}&\cd&a_2b_{2n}\\
      \vd&\vd&\dd&\vd\\
      a_nb_{n1}&a_nb_{n2}&\cd&a_nb_{nn}
    \end{array}
    \right)=\left(
    \begin{array}{rrrr}
      a_1b_{11}&a_2b_{12}&\cd&a_nb_{1n}\\
      a_1b_{21}&a_2b_{22}&\cd&a_nb_{2n}\\
      \vd&\vd&\dd&\vd\\
      a_1b_{n1}&a_2b_{n2}&\cd&a_nb_{nn}
    \end{array}
    \right)
    $$
  \end{footnotesize}
\end{frame}

\begin{frame}
  \begin{footnotesize}
    亦即
    $$
    \left(
    \begin{array}{rrrr}
      0&(a_1-a_2)b_{12}&\cd&(a_1-a_n)b_{1n}\\
      (a_2-a_1)b_{21}&0&\cd&(a_2-a_n)b_{2n}\\
      \vd&\vd&\dd&\vd\\
      (a_n-a_1)b_{n1}&(a_n-a_2)b_{n2}&\cd&0
    \end{array}
    \right)=\zero
    $$
    因为$a_i\ne a_j(i\ne j)$,故
    $$
    b_{ij}=0, \quad i\ne j
    $$
  \end{footnotesize}
\end{frame}

\begin{frame}
  \begin{footnotesize}
    \begin{li}{30}
      证明:两个$n$阶下三角矩阵的乘积仍是下三角矩阵。
    \end{li}
  \end{footnotesize}
\end{frame}



\begin{frame}
  \begin{footnotesize}
    \begin{li}{31}
      证明:若$\A$是对角元全为零的上三角矩阵,则$\A^2$也是主对角元全为零的上三角矩阵。
    \end{li}
  \end{footnotesize}
\end{frame}



\begin{frame}
  \begin{footnotesize}
    \begin{li}{32}
      证明:对角元全为1的上三角矩阵的乘积,仍是主对角元全为1的上三角矩阵。
    \end{li}
  \end{footnotesize}
\end{frame}


\begin{frame}
  \begin{footnotesize}
    \begin{li}{33}
      设$$\A=\left(
      \begin{array}{rrr}
        5&-2&1\\
        3&4&-1
      \end{array}
      \right),~~\B=\left(
      \begin{array}{rrr}
        -3&2&0\\
        -2&0&1
      \end{array}
      \right),$$
      计算$\A\B^T,~\B^T\A,~\A^T\A,~\B\B^T+\A\B^T$。
    \end{li}
  \end{footnotesize}
\end{frame}



\begin{frame}
  \begin{footnotesize}
    \begin{li}{34}
      证明:$(\A_1\A_2\cd\A_k)^T=\A_k^T\cd\A_2^T\A_1^T$。
    \end{li}
    \pause\proofname
    由$(\A\B)^T=\B^T\A^T$及数学归纳法容易证明。
  \end{footnotesize}
\end{frame}



\begin{frame}
  \begin{footnotesize}
    \begin{li}{35}
      证明:若$\A$和$\B$都是$n$阶对称矩阵,则$\A+\B,~~\A-2\B$也是对称矩阵。
    \end{li}
    \pause\proofname
    对称矩阵的线性组合仍是对称矩阵。
  \end{footnotesize}
\end{frame}



\begin{frame}
  \begin{footnotesize}
    \begin{li}{36}
      对于任意的$n$阶矩阵$\A$,证明:
      \begin{itemize}
      \item[(1)]$\A+\A^T$是对称矩阵,$\A-\A^T$是反对称矩阵。
      \item[(2)]$\A$可表示对称矩阵和反对称矩阵之和。
      \end{itemize}
    \end{li}
    \pause\proofname
    $$
    \A = \frac{\A+\A^T}2+\frac{\A-\A^T}2
    $$
  \end{footnotesize}
\end{frame}



\begin{frame}
  \begin{footnotesize}
    \begin{li}{37}
      证明:若$\A$和$\B$都是$n$阶对称矩阵,则$\A\B$是对称矩阵的充要条件是$\A$与$\B$可交换。
    \end{li}
    \pause\proofname
    \begin{itemize}
    \item[($\Rightarrow$)]
      $$\A\B\mbox{对称}\Rightarrow \A\B=(\A\B)^T=\B^T\A^T=\B\A$$
    \item[($\Leftarrow$)]
      $$\A\B\mbox{可交换}\Rightarrow\A\B=\B\A\Rightarrow (\A\B)^T=\B^T\A^T=\B\A=\A\B$$
    \end{itemize}
  \end{footnotesize}
\end{frame}


\begin{frame}
  \begin{footnotesize}
    \begin{li}{38}
      设$\A$是实对称矩阵,且$\A^2=\zero$,证明$\A=\zero$.
    \end{li}
    \pause\proofname
    因为$\A$实对称,故
    $$
    \A^2=\zero \Longleftrightarrow \A^T\A=\zero
    $$
    观察$\A^T\A$的主对角元,
    $$
    \sum_{i=1}^na_{ij}^2=0
    $$
    故
    $$
    a_{ij}=0 \quad \forall i, j.
    $$
  \end{footnotesize}
\end{frame}



\begin{frame}
  \begin{footnotesize}
    \begin{li}{39}
      已知$\A$是一个$n$阶对称矩阵,$\B$是一个$n$阶反对称矩阵。
      \begin{itemize}
      \item[(1)]问$\A^k,~~\B^k$是否为对称或反对称矩阵?
      \item[(2)]证明:$\A\B+\B\A$是一个反对称矩阵。
      \end{itemize}
    \end{li}
    \pause\proofname
    \begin{itemize}
    \item[(1)]
      $$
      (\A^k)^T=(\A\A\cd\A)^T=\A^T\cd\A^T\A^T=\A\cd\A\A=\A^k \red{~~\Longrightarrow~~ \A^k\mbox{对称}}
      $$

      $$
      (\B^k)^T=(-\B)\cd(-\B)(-\B)=(-1)^k\B^k \red{~~\Longrightarrow~~
      \left\{
      \begin{array}{ll}
        \B^k\mbox{对称}, & k\mbox{为偶数}\\
        \B^k\mbox{反对称}, & k\mbox{为奇数}
      \end{array}      
      \right.
    }
      $$
    \item[(2)]
      $$
      \begin{array}{rl}
        (\A\B+\B\A)^T&=(\A\B)^T+(\B\A)^T=\B^T\A^T+\A^T\B^T\\[0.05in]
        &=-\B\A-\A\B.
      \end{array}
      $$
    \end{itemize}
  \end{footnotesize}
\end{frame}



\begin{frame}
  \begin{footnotesize}
    \begin{li}{40(求逆矩阵)}
      \begin{itemize}
      \item[(1)]
        $$
        \left(
        \begin{array}{rr}
          8&-4\\
          -5&3
        \end{array}
        \right)
        $$
      \end{itemize}
    \end{li}
  \end{footnotesize}
\end{frame}

\begin{frame}
  \begin{footnotesize}
    \begin{li}{40(求逆矩阵)}
      \begin{itemize}
      \item[(2)]
        $$
        \left(
        \begin{array}{rr}
          \cos\theta&\sin\theta\\
          -\sin\theta&\cos\theta
        \end{array}
        \right)
        $$
      \end{itemize}
    \end{li}

    $$
    \left(
    \begin{array}{rr}
      \cos\theta&\sin\theta\\
      -\sin\theta&\cos\theta
    \end{array}
    \right)^{-1}=
    \left(
    \begin{array}{rr}
      \cos(-\theta)&\sin(-\theta)\\
      -\sin(-\theta)&\cos(-\theta)
    \end{array}
    \right)=     \left(
    \begin{array}{rr}
      \cos\theta&-\sin\theta\\
      \sin\theta&\cos\theta
    \end{array}
    \right)
    $$
  \end{footnotesize}
\end{frame}


\begin{frame}
  \begin{footnotesize}
    \begin{li}{40(求逆矩阵)}
      \begin{itemize}
      \item[(3)]
        $$
        \left(
        \begin{array}{rrr}
          1&2&-2\\
          2&1&-2\\
          2&-2&1
        \end{array}
        \right)
        $$
      \end{itemize}
    \end{li}
  \end{footnotesize}
\end{frame}

\begin{frame}
  \begin{footnotesize}
    \begin{li}{40(求逆矩阵)}
      \begin{itemize}
      \item[(4)]
        $$
        \left(
        \begin{array}{rrr}
          2&3&-1\\
          1&2&0\\
          -1&2&-2
        \end{array}
        \right)
        $$
      \end{itemize}
    \end{li}
  \end{footnotesize}
\end{frame}

\begin{frame}
  \begin{footnotesize}
    \begin{li}{40(求逆矩阵)}
      \begin{itemize}
      \item[(5)]
        $$
        \left(
        \begin{array}{rrr}
          1&0&0\\
          1&1&0\\
          1&1&1
        \end{array}
        \right)
        $$
      \end{itemize}
    \end{li}
  \end{footnotesize}
\end{frame}

\begin{frame}
  \begin{footnotesize}
    \begin{li}{40(求逆矩阵)}
      \begin{itemize}
      \item[(6)]
        $$
        \left(
        \begin{array}{rrrr}
          1&1&0&0\\
          0&1&1&0\\
          0&0&1&1\\
          0&0&0&1
        \end{array}
        \right)
        $$
      \end{itemize}
    \end{li}
  \end{footnotesize}
\end{frame}


\begin{frame}
  \begin{footnotesize}
    \begin{li}{41(解矩阵方程)}
      \begin{itemize}
      \item[(2)]
        $$
        \left(
        \begin{array}{rrr}
          2&3&-1\\
          1&2&0\\
          -1&2&-2
        \end{array}
        \right)\X=
        \left(
        \begin{array}{rr}
          2&1\\
          -1&0\\
          3&1
        \end{array}
        \right)        
        $$
      \end{itemize}
    \end{li}
  \end{footnotesize}
\end{frame}

\begin{frame}
  \begin{footnotesize}
    \begin{li}{41(解矩阵方程)}
      \begin{itemize}
      \item[(3)]
        $$
        \A\left(
        \begin{array}{rrr}
          1&1&1\\
          0&1&1\\
          0&0&1
        \end{array}
        \right)=
        \left(
        \begin{array}{rrr}
          1&-2&1\\
          0&1&-1
        \end{array}
        \right)        
        $$
      \end{itemize}
    \end{li}
  \end{footnotesize}
\end{frame}


\begin{frame}
  \begin{footnotesize}
    \begin{li}{42(解线性方程组)}     
      $$
      \left\{
      \begin{array}{rrrcr}
        x_1&+x_2&+x_3&=&1\\
           &2x_2&+2x_3&=&1\\
        x_1&-x_2&&=&2
      \end{array}
      \right.
      $$
    \end{li}
  \end{footnotesize}
\end{frame}



\begin{frame}
  \begin{footnotesize}
    \begin{li}{43}
      设$\A,~\B,~\C$为同阶方阵,
      \begin{itemize}
      \item[(1)]问$\A$满足什么条件时,命题\blue{$\A\B=\A\C~\Rightarrow~\B=\C$}成立;\\[0.05in]
      \item[(2)] 问:若$\B\ne\C$时,是否必有$\A\B\ne\A\C$?
      \end{itemize}
    \end{li}
    
    \begin{itemize}
    \item[(1)]
      当$\A$可逆时,该命题成立。
    \item[(2)]
      不一定。例如,当$\A=\zero$时,不论任何$\B,\C$,总有$\A\B=\A\C$。

    \end{itemize}

  \end{footnotesize}
\end{frame}


\begin{frame}
  \begin{footnotesize}
    \begin{li}{44}
      设$\A,~\B$都是$n$阶方阵,问:下列命题是否成立?若成立,给出证明;若不成立,举反例说明。
      \begin{itemize}
      \item[(1)]若$\A,\B$皆不可逆,则$\A+\B$也不可逆;
      \item[(2)]若$\A\B$可逆,则$\A,~\B$都可逆;
      \item[(3)]若$\A\B$不可逆,则$\A,~\B$都不可逆;
      \item[(4)]若$\A$可逆,则$k\A$可逆($k$是数)。
      \end{itemize}
    \end{li}
    \begin{itemize}
    \item[(1)]不成立。\purple{例如,令$\A=\left(
      \begin{array}{cc}
        1&0\\
        0&0
      \end{array}
      \right),~~\B=\left(
      \begin{array}{cc}
        0&0\\
        0&1
      \end{array}
      \right)$皆不可逆,但$\A+\B=\left(
      \begin{array}{cc}
        1&0\\
        0&1
      \end{array}
      \right)$可逆。}
    \item[(2)]成立。$\purple{\A\B\mbox{可逆}\Rightarrow |\A||\B|=|\A\B|\ne 0 \Rightarrow |\A|\ne0\mbox{且}|\B|\ne 0
      \Rightarrow \A,~\B\mbox{都可逆。}}$
    \item[(3)]不成立。\purple{例如,令$\A=\left(
      \begin{array}{cc}
        0&0\\
        0&0
      \end{array}
      \right),~~\B=\left(
      \begin{array}{cc}
        1&0\\
        0&1
      \end{array}
      \right)$,则$\A\B=\zero$不可逆,但$\A$不可逆,但$\B$可逆。}
    \item[(4)]不成立。\purple{$\A\mbox{可逆} \Rightarrow \left\{
      \begin{array}{rl}
        k\A\mbox{可逆},& k\ne 0, \\
        k\A\mbox{不可逆},& k= 0.
      \end{array}
      \right.
      $}
    \end{itemize}

  \end{footnotesize}
\end{frame}



\begin{frame}
  \begin{footnotesize}
    \begin{li}{45}
      设方阵$\A$满足$\A^2-\A-2\II=\zero$,证明:
      \begin{itemize}
      \item[(1)]$\A$和$\II-\A$都可逆,并求它们的逆;
      \item[(2)]$\A+\II$和$\A-2\II$不同时可逆。
      \end{itemize}
    \end{li}
  \end{footnotesize}
\end{frame}



\begin{frame}
  \begin{footnotesize}
    \begin{li}{46}
      设方阵$\A$满足$\A^2-2\A+4\II=\zero$,证明$\A+\II$和$\A-3\II$都可逆,并求它们的逆。
    \end{li}
  \end{footnotesize}
\end{frame}



\begin{frame}
  \begin{footnotesize}
    \begin{li}{47}
      试求上(或下)三角矩阵可逆的充要条件,并证明:可逆上(或下)三角矩阵的逆矩阵也是可逆上(或下)三角矩阵。
    \end{li}
  \end{footnotesize}
\end{frame}



\begin{frame}
  \begin{footnotesize}
    \begin{li}{28}
      
    \end{li}
  \end{footnotesize}
\end{frame}


\begin{frame}
  \begin{footnotesize}
    \begin{li}{28}
      
    \end{li}
  \end{footnotesize}
\end{frame}



\begin{frame}
  \begin{footnotesize}
    \begin{li}{28}
      
    \end{li}
  \end{footnotesize}
\end{frame}



\begin{frame}
  \begin{footnotesize}
    \begin{li}{28}
      
    \end{li}
  \end{footnotesize}
\end{frame}



\begin{frame}
  \begin{footnotesize}
    \begin{li}{28}
      
    \end{li}
  \end{footnotesize}
\end{frame}



\begin{frame}
  \begin{footnotesize}
    \begin{li}{28}
      
    \end{li}
  \end{footnotesize}
\end{frame}


\begin{frame}
  \begin{footnotesize}
    \begin{li}{28}
      
    \end{li}
  \end{footnotesize}
\end{frame}



\begin{frame}
  \begin{footnotesize}
    \begin{li}{28}
      
    \end{li}
  \end{footnotesize}
\end{frame}



\begin{frame}
  \begin{footnotesize}
    \begin{li}{28}
      
    \end{li}
  \end{footnotesize}
\end{frame}



\begin{frame}
  \begin{footnotesize}
    \begin{li}{28}
      
    \end{li}
  \end{footnotesize}
\end{frame}



\begin{frame}
  \begin{footnotesize}
    \begin{li}{28}
      
    \end{li}
  \end{footnotesize}
\end{frame}
