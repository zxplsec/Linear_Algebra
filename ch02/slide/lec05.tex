\section{矩阵的初等变换与初等矩阵}

\begin{frame}
  用高斯消去法求解线性方程组,其步骤是对增广矩阵做以下三种行变换:
  \begin{itemize}
  \item[(i)] 对调两行;
  \item[(ii)] 以非零常数$k$乘矩阵的某一行;
  \item[(iii)] 将矩阵的某一行乘以常数$k$并加到另一行。
  \end{itemize}
  这三类行变换统称为\red{矩阵的初等行变换},\pause 分别称为
  \begin{itemize}
  \item[(i)] \red{对换变换}  $\quad r_i \leftrightarrow r_j$;
  \item[(ii)] \red{倍乘变换}       $\quad r_i \times k$;
  \item[(iii)] \red{倍加变换} $\quad r_i + r_j \times k $。
  \end{itemize}
  \pause 对应的还有\red{初等列变换}。 \pause \blue{初等行变换与初等列变换统称为初等变换。}
\end{frame}


\begin{frame}
  三种初等变换都是可逆的,
  \begin{table}[htbp]
    \centering
    \begin{tabular}{|c|c|} \hline
      初等变换 &  逆变换 \\\hline
      $r_i \leftrightarrow r_j$ & $r_i \leftrightarrow r_j$ \\[0.2cm]\hline
      $r_i \times k$ & $\ds r_i \div k$ \\[0.2cm]\hline
      $r_i + r_j \times k$ & $r_i - r_j\times k$ \\[0.2cm]\hline
    \end{tabular}
    \caption{初等变换及其逆变换}
  \end{table}

\end{frame}


\begin{frame}

  \begin{dingyi}[矩阵的等价]
    
    \begin{itemize}
    \item[(i)] 如果$\MA$经过有限次初等行变换变成$\MB$,就称\blue{$\MA$与$\MB$行等价},记为$\red{\MA\overset{r}{\sim} \MB}$;
    \item[(ii)] 如果$\MA$经过有限次初等列变换变成$\MB$,就称\blue{$\MA$与$\MB$列等价},记为$\red{\MA\overset{c}{\sim} \MB}$;
    \item[(iii)] 如果$\MA$经过有限次初等变换变成$\MB$,就称\blue{$\MA$与$\MB$等价},记为$\red{\MA\sim \MB}$。
    \end{itemize}
  \end{dingyi}
\end{frame}


\begin{frame}
  \begin{xingzhi}
    矩阵的等价满足以下三条性质:
    \begin{itemize}
    \item[(i)] \red{反身性}:$\MA \sim \MA$;
    \item[(ii)] \red{对称性}:若$\MA \sim \MB$,则$\MB \sim \MA$;
    \item[(iii)] \red{传递性}:若$\MA \sim \MB, ~\MB \sim \MC$,则$\MA \sim \MC$。
    \end{itemize}
  \end{xingzhi}
\end{frame}


\begin{frame}

  \begin{dingyi}[初等矩阵]
    将单位矩阵$\MI$做一次初等变换所得的矩阵称为\red{初等矩阵}。
    对应于$3$类初等行、列变换,有$3$种类型的初等矩阵。
  \end{dingyi}
\end{frame}


\begin{frame}
  以下介绍三种初等矩阵:
  \begin{enumerate}
  \item 初等对调矩阵;
  \item 初等倍乘矩阵;
  \item 初等倍加矩阵。
  \end{enumerate}
\end{frame}


\begin{frame}
  1、 对调$\MI$的两行或两列(\red{初等对调矩阵})
  \begin{figure}[htbp]
    \centering
    \begin{tikzpicture}[scale=0.8]
      \matrix (M) [matrix of math nodes]  { 
        \ME_{ij} = \\
      };
      \matrix(MM) [right=2pt of M, matrix of math nodes,nodes in empty cells, ampersand replacement=\&,left delimiter=(,right delimiter=)] {
        1 \&     \&   \&   \&     \&   \&   \& \& \\
        \& \dd \&   \&   \&     \&   \&   \& \& \\
        \&     \& 0 \&   \& \cd \&   \& 1 \& \& \\
        \&     \&   \& 1 \&     \&   \&   \& \& \\
        \&     \&\vd\&   \& \dd \&   \&\vd\& \& \\
        \&     \&   \&   \&     \& 1 \&   \& \& \\
        \&     \& 1 \&   \& \cd \&   \& 0 \& \& \\
        \&     \&   \&   \&     \&   \&   \& \dd \& \\
        \&     \&   \&   \&     \&   \&   \& \& 1\\
      };
      \node[right=16pt  of MM-3-9, blue]  {第$i$行};
      \node[right=16pt  of MM-7-9, blue]  {第$j$行};
      \node[below=5pt  of MM-9-3, blue]  {第$i$列};
      \node[below=5pt  of MM-9-7, blue]  {第$j$列};
    \end{tikzpicture}
  \end{figure}
\end{frame}


\begin{frame}
  a、用$m$阶初等矩阵$\ME_{ij}$左乘$\MA=(a_{ij})_{m\times n}$,得
  \begin{figure}[htbp]
    \centering
    \begin{tikzpicture}
      \matrix (M) [matrix of math nodes]  { 
        \ME_{ij}\MA = \\
      };
      \matrix(MM) [right=2pt of M, matrix of math nodes,nodes in empty cells,
      ampersand replacement=\&,left delimiter=(,right delimiter=)] {
        a_{11} \& a_{12}    \& \cd   \&  a_{1n} \\
        \vd   \& \vd      \&   \&  \vd \\          
        a_{j1} \& a_{j2}    \& \cd  \&  a_{jn} \\
        \vd   \& \vd      \&   \&  \vd \\
        a_{i1} \& a_{i2}    \& \cd   \&  a_{in} \\
        \vd   \& \vd      \&   \&  \vd \\
        a_{m1} \& a_{m2}    \& \cd  \&  a_{mn} \\
      };
      \node[right=12pt  of MM-3-4, blue]  {第$i$行};
      \node[right=12pt  of MM-5-4, blue]  {第$j$行};
    \end{tikzpicture}
  \end{figure}
  相当于
  \red{把$\MA$的第$i$行与第$j$行对调($r_i \leftrightarrow r_j$).}
\end{frame}


\begin{frame}
  b、用$n$阶初等矩阵$\ME_{ij}$右乘$\MA$,得
  \begin{figure}[htbp]
    \centering
    \begin{tikzpicture}
      \matrix (M) [matrix of math nodes]  { 
        \MA \ME_{ij}= \\
      };
      \matrix(MM) [right=2pt of M, matrix of math nodes,nodes in empty cells,
      ampersand replacement=\&,left delimiter=(,right delimiter=)] {
        a_{11} \& \cd \&a_{1j}    \& \cd \&a_{1i}    \& \cd  \&  a_{1n} \\
        a_{21} \& \cd \&a_{2j}    \& \cd \&a_{2i}    \& \cd  \&  a_{jn} \\
        \vd    \&     \&\vd       \&     \&\vd       \&      \&  \vd \\
        a_{m1} \& \cd \&a_{mj}    \& \cd \&a_{mi}    \& \cd  \&  a_{mn} \\
      };
      \node[below=12pt  of MM-4-3, blue]  {第$i$列};
      \node[below=12pt  of MM-4-5, blue]  {第$j$列};
    \end{tikzpicture}
  \end{figure}

  相当于\red{把$\MA$的第$i$列与第$j$列对调($c_i \leftrightarrow c_j$).}
\end{frame}


\begin{frame}
  2、 以非零常数$k$乘$\MI$的某行或某列(\red{初等倍乘矩阵})
  \begin{figure}[htbp]
    \centering
    \begin{tikzpicture}
      \matrix (M) [matrix of math nodes]  { 
        \ME_{i}(k) = \\
      };
      \matrix(MM) [right=2pt of M, matrix of math nodes,nodes in empty cells,
      ampersand replacement=\&,left delimiter=(,right delimiter=)] {
        1 \&     \&   \&   \&     \&   \& \\
        \& \dd \&   \&   \&     \&   \& \\
        \&     \& 1 \&   \&     \&   \& \\
        \&     \&   \& k \&     \&   \& \\
        \&     \&   \&   \& 1   \&   \& \\
        \&     \&   \&   \&     \& \dd \& \\
        \&     \&   \&   \&     \&   \& 1\\
      };
      \node[right=12pt  of MM-4-7, blue]  {第$i$行};
      \node[below=12pt  of MM-7-4, blue]  {第$i$列};
    \end{tikzpicture}
  \end{figure}
\end{frame}


\begin{frame}
  a、以$m$阶初等矩阵$\ME_i(k)$左乘$\MA$,得
  \begin{figure}[htbp]
    \centering
    \begin{tikzpicture}
      \matrix (M) [matrix of math nodes]  { 
        \ME_{i}(k)\MA = \\
      };
      \matrix(MM) [right=2pt of M, matrix of math nodes,nodes in empty cells,
      ampersand replacement=\&,left delimiter=(,right delimiter=)] {
        a_{11} \& a_{12}    \& \cd   \&  a_{1n} \\
        \vd   \& \vd      \&   \&  \vd \\          
        ka_{i1} \& ka_{i2}    \& \cd  \&  ka_{in} \\
        \vd   \& \vd      \&   \&  \vd \\
        a_{m1} \& a_{m2}    \& \cd  \&  a_{mn} \\
      };
      \node[right=12pt  of MM-3-4, blue]  {第$i$行};
    \end{tikzpicture}
  \end{figure} 
  相当于\red{以数$k$乘$\MA$的第$i$行($r_i\times k$)};
\end{frame}


\begin{frame}
  b、 以$n$阶初等矩阵$\ME_i(k)$右乘$\MA$,得
  \begin{figure}[htbp]
    \centering
    \begin{tikzpicture}
      \matrix (M) [matrix of math nodes]  { 
        \MA \ME_{i}(k)= \\
      };
      \matrix(MM) [right=2pt of M, matrix of math nodes,nodes in empty cells,
      ampersand replacement=\&,left delimiter=(,right delimiter=)] {
        a_{11} \& \cd \&ka_{1i}      \& \cd  \&  a_{1n} \\
        a_{21} \& \cd \&ka_{2i}      \& \cd  \&  a_{jn} \\
        \vd    \&     \&\vd         \&      \&  \vd \\
        a_{m1} \& \cd \&ka_{mi}      \& \cd  \&  a_{mn} \\
      };
      \node[below=12pt  of MM-4-3, blue]  {第$i$列};
    \end{tikzpicture}
  \end{figure}
  
  相当于\red{以数$k$乘$\MA$的第$i$列($c_i\times k$)}。
\end{frame}


\begin{frame}
  3、将非零常数$k$乘$\MI$的某行再加到另一行上(\red{初等倍加矩阵})
  \begin{figure}[htbp]
    \centering
    \begin{tikzpicture}
      \matrix (M) [matrix of math nodes]  { 
        \ME_{ij}(k) = \\
      };
      \matrix(MM) [right=2pt of M, matrix of math nodes,nodes in empty cells,
      ampersand replacement=\&,left delimiter=(,right delimiter=)] {
        1 \&     \&   \&   \&     \&   \& \\
        \& \dd \&   \&   \&     \&   \& \\
        \&     \& 1 \&\cd\& k    \&   \& \\
        \&     \&   \&\dd\& \vd  \&   \& \\
        \&     \&   \&   \& 1   \&   \& \\
        \&     \&   \&   \&     \& \dd \& \\
        \&     \&   \&   \&     \&   \& 1\\
      };
      \node[right=12pt  of MM-3-7, blue]  {第$i$行};
      \node[right=12pt  of MM-5-7, blue]  {第$j$行};
    \end{tikzpicture}
  \end{figure} 
\end{frame}


\begin{frame}
  a、 以$m$阶初等矩阵$\ME_{ij}(k)$左乘$\MA$,得
  \begin{figure}[htbp]
    \centering
    \begin{tikzpicture}
      \matrix (M) [matrix of math nodes]  { 
        \ME_{ij}\MA = \\
      };
      \matrix(MM) [right=2pt of M, matrix of math nodes,nodes in empty cells,
      ampersand replacement=\&,left delimiter=(,right delimiter=)] {
        a_{11} \& a_{12}    \& \cd   \&  a_{1n} \\
        \vd   \& \vd      \&   \&  \vd \\          
        a_{i1}+ka_{j1} \& a_{i2}+ka_{j2}    \& \cd  \&  a_{in}+ka_{jn} \\
        \vd   \& \vd      \&   \&  \vd \\
        a_{j1} \& a_{j2}    \& \cd   \&  a_{jn} \\
        \vd   \& \vd      \&   \&  \vd \\
        a_{m1} \& a_{m2}    \& \cd  \&  a_{mn} \\
      };
      \node[right=12pt  of MM-3-4, blue]  {第$i$行};
      \node[right=28pt  of MM-5-4, blue]  {第$j$行};
    \end{tikzpicture}
  \end{figure}
  相当于\red{把$\MA$的第$j$行乘以数$k$加到第$i$行上($r_i+r_j\times k$)};
\end{frame}


\begin{frame}
  b、 以$n$阶初等矩阵$\ME_{ij}(k)$右乘$\MA$,得
  \begin{figure}[htbp]
    \centering
    \begin{tikzpicture}
      \matrix (M) [matrix of math nodes]  { 
        \MA \ME_{ij}= \\
      };
      \matrix(MM) [right=2pt of M, matrix of math nodes,nodes in empty cells,
      ampersand replacement=\&,left delimiter=(,right delimiter=)] {
        a_{11} \& \cd \&a_{1i}    \& \cd \&a_{1j}+ka_{1i}  \& \cd  \&  a_{1n} \\
        a_{21} \& \cd \&a_{2i}    \& \cd \&a_{2j}+ka_{2i}    \& \cd  \&  a_{jn} \\
        \vd    \&     \&\vd       \&     \&\vd       \&      \&  \vd \\
        a_{m1} \& \cd \&a_{mi}    \& \cd \&a_{mj}+ka_{mi}    \& \cd  \&  a_{mn} \\
      };
      \node[below=12pt  of MM-4-3, blue]  {第$i$列};
      \node[below=12pt  of MM-4-5, blue]  {第$j$列};
    \end{tikzpicture}
  \end{figure}

  相当于\red{把$\MA$的第$i$列乘以数$k$加到第$j$列上($c_j+c_i\times k$)}。
  
\end{frame}


\begin{frame}
  % 
  \begin{dingli}
    设$\MA$为一个$m\times n$矩阵,
    \begin{itemize}
    \item 
      对$\MA$施行一次初等行变换,相当于在$\MA$的左边乘以相应的$m$阶初等矩阵;
    \item
      对$\MA$施行一次初等列变换,相当于在$\MA$的右边乘以相应的$n$阶初等矩阵。
    \end{itemize}
  \end{dingli}
\end{frame}


\begin{frame}
  \begin{lianxi}
    请自行补充以下变换的具体含义:
    \begin{itemize}
    \item[] $\ME_i(k)\MA$:
    \item[] $\ME_{ij}(k)\MA$:
    \item[] $\ME_{ij}\MA$:
    \item[] $\MA\ME_i(k)$:
    \item[] $\MA\ME_{ij}(k)$:
    \item[] $\MA\ME_{ij}$:
    \end{itemize}
  \end{lianxi}
\end{frame}


\begin{frame}
  由初等变换可逆,可知初等矩阵可逆。  
  \begin{itemize}
  \item[(i)] 由\blue{变换$r_i\leftrightarrow r_j$的逆变换为其本身}可知
    $$
    \red{\ME_{ij}^{-1} = \ME_{ij}}
    $$ 
  \item[(ii)] 由\blue{变换$r_i\times k$的逆变换为$\ds r_i\div k$}可知
    $$
    \red{\ME_{i}(k)^{-1} = \ME_{i}(k^{-1})}
    $$ 
  \item[(iii)] 由\blue{变换$r_i+r_j\times k$的逆变换为$\ds r_i-r_j\times k$}可知
    $$
    \red{\ME_{ij}(k)^{-1} = \ME_{ij}(-k)}
    $$ 
  \end{itemize}
\end{frame}


\begin{frame}
  以上结论也可总结为
  $$ \red{
    \ME_{ij}\ME_{ij}=\MI, \quad
    \ME_{i}(k)\ME_{i}(k^{-1}) = \MI, \quad
    \ME_{ij}(k)\ME_{ij}(-k) = \MI.
  }
  $$      
\end{frame}


\begin{frame}

  \begin{li} 
    设初等矩阵
    $$
    \begin{aligned}
      \MP_1 = \left(
        \begin{array}{cccc}
          0&0&1&0 \\
          0&1&0&0 \\
          1&0&0&0 \\
          0&0&0&1
        \end{array}
      \right), ~~
      \MP_2 = \left(
        \begin{array}{cccc}
          1& & &  \\
          0&1& &  \\
          0&0&1& \\
          c&0&0&1
        \end{array}
      \right), ~~
      \MP_3 = \left(
        \begin{array}{cccc}
          1& & &  \\
           &k& &  \\
           & &1& \\
           & & &1
        \end{array}
      \right)
    \end{aligned}
    $$
    求$\MP_1\MP_2\MP_3$及$(\MP_1\MP_2\MP_3)^{-1}$
  \end{li}

\end{frame}


\begin{frame}

  \begin{jie}
    $$
    \begin{aligned}
      \MP_2\MP_3 =    \left(
        \begin{array}{cccc}
          1& & &  \\
          0&k& &  \\
          0&0&1& \\
          c&0&0&1
        \end{array}
      \right), ~~\pause
      \MP_1\MP_2\MP_3=\MP_1(\MP_2\MP_3) =\left(
        \begin{array}{cccc}
          0&0&1&0 \\
          0&k&0&0  \\
          1&0&0&0  \\
          c&0&0&1
        \end{array}
      \right)
    \end{aligned}
    $$
  \end{jie}

\end{frame}


\begin{frame}

  \begin{jie}[续]
    因
    $$
    (\MP_1\MP_2\MP_3)^{-1} = \MP_3^{-1}\MP_2^{-1}\MP_1^{-1}
    $$
    \pause
    而
    $$
    \MP_1^{-1} = \MP_1, ~~
    \MP_2^{-1} = \left(
      \begin{array}{cccc}
        1& & &  \\
        0&1& &  \\
        0&0&1& \\
        -c&0&0&1
      \end{array}
    \right), ~~
    \MP_3^{-1} = \left(
      \begin{array}{cccc}
        1& & &  \\
         &\frac1k& &  \\
         & &1& \\
         & & &1
      \end{array}
    \right)
    $$    
    \pause
    故
    $$
    \MP_2^{-1}\MP_1^{-1} =   \left(
      \begin{array}{cccc}
        0&0&1&0 \\
        0&1&0&0 \\
        1&0&0&0 \\
        0&0&-c&1
      \end{array}
    \right), \quad  
    \MP_3^{-1}\MP_2^{-1}\MP_1^{-1} =  \left(
      \begin{array}{cccc}
        0&0&1&0 \\
        0&\frac1k&0&0 \\
        1&0&0&0 \\
        0&0&-c&1
      \end{array}
    \right)
    $$
  \end{jie}

\end{frame}


\begin{frame}
  \begin{li}
    将三对角矩阵
    $
    \MA = \left(
      \begin{array}{cccc}
        2 & 1 & 0 & 0\\
        1 & 2 & 1 & 0\\
        0 & 1 & 2 & 1\\
        0 & 0 & 1 & 2
      \end{array}
    \right)
    $
    分解成主对角元为$1$的下三角矩阵$\mathbf{L}$和上三角阵$\mathbf{U}$的乘积$\MA=\mathbf{L}\mathbf{U}$(称为矩阵的LU分解)。
  \end{li}
\end{frame}


\begin{frame}
  \begin{small}
    \begin{jie}
      $$
      \left(
        \begin{array}{cccc}
          1 & && \\
          -\frac12 &1&&\\
            &&1&\\
            &&&1
        \end{array}
      \right)\underbrace{\left(
          \begin{array}{cccc}
            2 & 1 & 0 & 0\\
            1 & 2 & 1 & 0\\
            0 & 1 & 2 & 1\\
            0 & 0 & 1 & 2
          \end{array}
        \right)}_{\MA}=  \underbrace{\left(
          \begin{array}{cccc}
            2 & 1 & 0 & 0\\
            0 & \frac32 & 1 & 0\\
            0 & 1 & 2 & 1\\
            0 & 0 & 1 & 2
          \end{array}
        \right)}_{\MA_1}
      $$
      \pause
      $$
      \left(
        \begin{array}{cccc}
          1 & && \\
            &1&&\\
            &-\frac23&1&\\
            &&&1
        \end{array}
      \right)\underbrace{\left(
          \begin{array}{cccc}
            2 & 1 & 0 & 0\\
            0 & \frac32 & 1 & 0\\
            0 & 1 & 2 & 1\\
            0 & 0 & 1 & 2
          \end{array}
        \right)}_{\MA_1} = \underbrace{\left(
          \begin{array}{cccc}
            2 & 1 & 0 & 0\\
            0 & \frac32 & 1 & 0\\
            0 & 0 & \frac43 & 1\\
            0 & 0 & 1 & 2
          \end{array}
        \right)}_{\MA_2}
      $$
      \pause

      $$
      \left(
        \begin{array}{cccc}
          1 & && \\
            &1&&\\
            &&1&\\
            &&-\frac34&1
        \end{array}
      \right)\underbrace{\left(
          \begin{array}{cccc}
            2 & 1 & 0 & 0\\
            0 & \frac32 & 1 & 0\\
            0 & 0 & \frac43 & 1\\
            0 & 0 & 1 & 2
          \end{array}
        \right)}_{\MA_2} = \underbrace{\left(
          \begin{array}{cccc}
            2 & 1 & 0 & 0\\
            0 & \frac32 & 1 & 0\\
            0 & 0 & \frac43 & 1\\
            0 & 0 & 0  & \frac54
          \end{array}
        \right)}_{\mathbf{U}}
      $$ 
      
    \end{jie}
  \end{small}
\end{frame}


\begin{frame}
  \begin{jie}[续]
    将上面三个式子中左端的矩阵分别记为$\mathbf{L}_1,\mathbf{L}_2,\mathbf{L}_3$,则
    $$
    \mathbf{L}_3\mathbf{L}_2\mathbf{L}_1 \MA = \mathbf{U}
    $$\pause 
    于是
    $$
    \MA = (\mathbf{L}_3\mathbf{L}_2\mathbf{L}_1)^{-1}\mathbf{U} \triangleq \mathbf{L}\mathbf{U},
    $$\pause
    其中
    $$
    \begin{aligned}
      \mathbf{L}
      &=  (\mathbf{L}_3\mathbf{L}_2\mathbf{L}_1)^{-1} \\ =  \mathbf{L}_1^{-1}\mathbf{L}_2^{-1}\mathbf{L}_3^{-1}\\ \pause
      &=   
      \left(
        \begin{array}{cccc}
          1 & && \\
          \frac12 &1&&\\
            &&1&\\
            &&&1
        \end{array}
      \right)
      \left(
        \begin{array}{cccc}
          1 & && \\
            &1&&\\
            &\frac23&1&\\
            &&&1
        \end{array}
      \right)
      \left(
        \begin{array}{cccc}
          1 & && \\
            &1&&\\
            &&1&\\
            &&\frac34&1
        \end{array}
      \right) \\ \pause
      &=   \left(
        \begin{array}{cccc}
          1 & && \\
          \frac12 &1&&\\
            &\frac23&1&\\
            &&\frac34&1
        \end{array}
      \right)
    \end{aligned}    
    $$
  \end{jie}
  
\end{frame}



\begin{frame}
  
  \begin{dingli}
    可逆矩阵可以经过若干次初等行变换化为单位矩阵。
  \end{dingli}
\end{frame}



\begin{frame}
  \begin{small}
    \begin{proof}
      对于高斯消去法,其消去过程是对增广矩阵做$3$类初等行变换,并一定可以将其化为行简化阶梯形矩阵,即
      \begin{figure}
        \begin{tikzpicture}
          \matrix(MM) [matrix of math nodes,nodes in empty cells,
          column sep=0.6ex,row sep=.5ex,ampersand replacement=\&,left delimiter=(,right delimiter=)] {
            a_{11} \&  a_{12} \&  \cd \& a_{1n} \& \&  b_1\\
            a_{21} \&  a_{22} \&  \cd \& a_{2n} \& \&  b_2\\
            \vd   \&  \vd   \&      \&  \vd  \& \& \vd \\
            a_{m1} \&  a_{m2} \&  \cd \& a_{mn} \& \&  b_m\\
          };
          \draw[dashed] (MM-1-5.north) -- (MM-4-5);
          \matrix (M2) [right=.05in of MM,matrix of math nodes]  { 
            \Rightarrow\\
          };
          \matrix(MM) [right=.05in of M2,matrix of math nodes,nodes in empty cells,
          column sep=0.6ex,row sep=.5ex,ampersand replacement=\&,left delimiter=(,right delimiter=)] {
            c_{11} \&   0 \& \cd \&  0 \& c_{1,r+1} \& \cd \& c_{1n} \& \& d_1\\
            0   \& c_{22} \& \cd \&  0 \& c_{2,r+1} \& \cd \& c_{2n} \& \& d_2\\
            \vd \& \vd \& \dd \&\vd \& \vd     \&     \& \vd   \& \& \vd\\
            0   \&  0  \& \cd \& c_{rr} \& c_{r,r+1} \& \cd \& c_{rn} \& \& d_r\\
            0   \&  0  \& \cd \& 0 \& 0 \& \cd \& 0 \&  \& d_{r+1}\\
            0   \&  0  \& \cd \& 0 \& 0 \& \cd \& 0 \&  \& 0\\          
            \vd \& \vd \& \dd \&\vd \& \vd     \&     \& \vd   \& \& \vd\\
            0   \&  0  \& \cd \& 0 \& 0 \& \cd \& 0 \&  \& 0\\
          };
          \draw[dashed] (MM-1-8.north) -- (MM-8-8);
          \filldraw[opacity=0.5,red!50] (MM-5-9) circle (0.3cm);
        \end{tikzpicture}      
      \end{figure}  
      其中$c_{ii}=1~(i=1,2,\cd,r)$。
    \end{proof}
  \end{small}
\end{frame}

\begin{frame}
  \begin{small}
    \begin{proof}[续]
      
      因此,对于任何矩阵$\MA$,都可经过初等行变换将其化为行简化阶梯形矩阵,即存在初等矩阵$\MP_1,\MP_2,\cd,\MP_s$使得
      $$
      \MP_s \cd \MP_2 \MP_1 \MA = \mathbf{U}.
      $$
      \pause
      当$\MA$为$n$阶可逆矩阵时,行简化阶梯形矩阵也是可逆矩阵,从而$\mathbf{U}$必为单位矩阵$\MI$.
    \end{proof}
  \end{small}

\end{frame}


\begin{frame}
  \begin{tuilun}
    可逆矩阵$\MA$可以表示为若干个初等矩阵的乘积。
  \end{tuilun}
  \pause
  \begin{proof}
    由上述定理,必存在初等矩阵$\MP_1,\MP_2,\cd,\MP_s$使得
    $$
    \MP_s \cd \MP_2 \MP_1 \MA = \MI,
    $$\pause
    于是
    $$
    \MA = (\MP_s \cd \MP_2 \MP_1)^{-1} = \MP_1^{-1}\MP_2^{-1}\cd\MP_s^{-1},
    $$
    亦即
    $$
    \MA^{-1}=\MP_s \cd \MP_2 \MP_1.
    $$
  \end{proof}
\end{frame}


\begin{frame}
  \begin{tuilun}
    如果对可逆矩阵$\MA$与同阶单位矩阵$\MI$做同样的初等行变换,那么当$\MA$变为单位阵时,
    $\MI$就变为$\MA^{-1}$,即
    $$\red{
      \left(
        \begin{array}{cc}
          \MA & \MI
        \end{array}
      \right) \xrightarrow[]{\mbox{初等行变换}} \left(
        \begin{array}{cc}
          \MI & \MA^{-1}
        \end{array}
      \right)
    } 
    $$
    同理,
    $$\red{
      \left(
        \begin{array}{c}
          \MA\\
          \MI
        \end{array}
      \right) \xrightarrow[]{\mbox{初等列变换}} \left(
        \begin{array}{c}
          \MI \\
          \MA^{-1}
        \end{array}
      \right)
    } 
    $$
  \end{tuilun}
  \pause

  \begin{zhu}
    \blue{该推论给出了求可逆矩阵的逆的一种有效方法,请大家熟练掌握。}
  \end{zhu}
\end{frame}


\begin{frame}
  \begin{li}
    求$
    \MA=\left(
      \begin{array}{rrr}
        0&2&-1\\
        1&1&2\\
        -1&-1&-1
      \end{array}
    \right)
    $
    的逆矩阵。
  \end{li}
\end{frame}


\begin{frame}
  \begin{jie}

    $$
    \begin{aligned}
      &\left(
        \begin{array}{c|c}
          \MA & \MI
        \end{array}
      \right)=\left(
        \begin{array}{rrr|rrr}
          0 &  2 & -1 &  1 & 0 & 0\\
          1 &  1 &  2 &  0 & 1 & 0\\
          -1 & -1 & -1 &  0 & 0 & 1\\              
        \end{array}
      \right) \\ \pause
      &\xrightarrow[]{r_1\leftrightarrow r_2}\left(
        \begin{array}{rrr|rrr}
          1 &  1 &  2 &  0 & 1 & 0\\
          0 &  2 & -1 &  1 & 0 & 0\\
          -1 & -1 & -1 &  0 & 0 & 1\\          
        \end{array}
      \right) \pause
      \xrightarrow[]{r_3+ r_1}\left(
        \begin{array}{rrr|rrr}
          1 &  1 &  2 & 0 & 1 & 0\\
          0 &  2 & -1 & 1 & 0 & 0\\
          0 &  0 &  1 & 0 & 1 & 1\\          
        \end{array}
      \right)\\\pause
      &\xrightarrow[r_2+r_3]{r_1+ r_3\times(-2)}\left(
        \begin{array}{rrr|rrr}
          1 &  1 &  0  & 0 &-1 &-2\\
          0 &  2 &  0  & 1 & 1 & 1\\
          0 &  0 &  1  & 0 & 1 & 1\\    
        \end{array}
      \right)\\\pause
      &\xrightarrow[r_2\times \frac12]{r_1+ r_2\times(-\frac12)}\left(
        \begin{array}{rrr|rrr}
          1 &  0 &  0  & -\frac12 &-\frac32 &-\frac52\\[.1in]
          0 &  1 &  0  & \frac12 & \frac12 & \frac12 \\[.1in]
          0 &  0 &  1  & 0 & 1 & 1                   
        \end{array}
      \right)    
    \end{aligned}
    $$
  \end{jie}
\end{frame}


\begin{frame}
  \begin{li}
    已知$\MA\MB\MA^T=2\MB\MA^T+\MI$,求$\MB$,其中$
    \MA = \left(
      \begin{array}{ccc}
        1&0&0\\
        0&1&2\\
        0&0&1
      \end{array}
    \right)
    $
  \end{li}
\end{frame}


\begin{frame}
  \begin{jie}
    $$
    \MA\MB\MA^T=2\MB\MA^T+\MI ~~ \Rightarrow ~~ (\MA-2\MI)\MB\MA^T=\MI 
    ~~ \Rightarrow ~~ \MB\MA^T = (\MA-2\MI)^{-1}
    $$
    \pause
    故
    $$
    \MB = (\MA-2\MI)^{-1} (\MA^T)^{-1}  = [\MA^T(\MA-2\MI)]^{-1} 
    =(\MA^T\MA-2\MA^T)^{-1}
    $$\pause
    而
    $$
    \MA^T\MA-2\MA^T = \left(
      \begin{array}{rrr}
        -1&0&0\\
        0&-1&2\\
        0&-2&3
      \end{array}
    \right)
    $$ \pause
    可求得
    $$
    \MB = \left(
      \begin{array}{rrr}
        -1&0&0\\
        0&3&-2\\
        0&2&-1
      \end{array}
    \right)
    $$
  \end{jie}
\end{frame}


\begin{frame}

  \begin{tuilun}
    对于$n$个未知数$n$个方程的线性方程组
    $
    \MA\vx=\vb,
    $
    如果增广矩阵
    $$
    \red{(\MA,~\vb)~~\overset{r}{\sim}~~(\MI,\vx)},
    $$
    则$\MA$可逆,且$\vx=\MA^{-1}\vb$为惟一解。  
  \end{tuilun}
\end{frame}


\begin{frame}
  \begin{li}
    设
    $$
    \MA = \left(
      \begin{array}{rrr}
        2&1&-3\\
        1&2&-3\\
        -1&3&2
      \end{array}
    \right),
    ~~
    \vb_1=\left(
      \begin{array}{r}
        1\\
        2\\
        -2
      \end{array}
    \right),
    ~~
    \vb_2=\left(
      \begin{array}{r}
        -1\\
        0\\
        5
      \end{array}
    \right),
    $$
    求$\MA\vx=\vb_1$与$\MA\vx=\vb_2$的解。
  \end{li}
\end{frame}


\begin{frame}
  \begin{jie}
    $$
    \begin{aligned}
      & (\MA~~\red{\vb_1}~~\red{\vb_2})
      = \left(
        \begin{array}{rrrrr}
          2 & 1 & 3 &\red{ 1} & \red{-1}\\
          1 & 2 &-2 &\red{ 2} & \red{ 0}\\
          -1 & 3 & 2 &\red{-2} & \red{ 5}        
        \end{array}\right) \\ \pause
      & \overset{{r_1\leftrightarrow r_2 \atop r_2-2r_1}\atop  r_3+r_1}{\sim}
      \left(
        \begin{array}{rrrrr}
          1 & 2 &-2 & \red{ 2} & \red{ 0}\\
          0 &-3 & 1 & \red{-3} & \red{-1}\\
          0 & 5 & 0 & \red{ 0} & \red{ 5}        
        \end{array}
      \right) \\ \pause
      & \overset{{r_3\leftrightarrow r_2 \atop r_2\div5}\atop  r_3+3r_2}{\sim}
      \left(
        \begin{array}{rrrrr}
          1 & 2 &-2 &\red{ 2} &  \red{0}\\
          0 & 1 & 0 &\red{ 0} &  \red{1}\\
          0 & 0 & 1 &\red{-3} &  \red{2}        
        \end{array}
      \right)  \pause \overset{r_1-2r_2+2r_3}{\sim}
      \left(
        \begin{array}{rrrrr}
          1 & 0 & 0 & \red{-4} & \red{2}\\
          0 & 1 & 0 & \red{0} &  \red{1}\\
          0 & 0 & 1 & \red{-3} &  \red{2}        
        \end{array}
      \right) 
    \end{aligned}
    $$
  \end{jie}

\end{frame}


\begin{frame}

  \begin{li}
    求解矩阵方程$\MA\MX=\MA+\MX$,其中
    $
    \MA = \left(
      \begin{array}{ccc}
        2&2&0\\
        2&1&3\\
        0&1&0
      \end{array}
    \right)
    $
  \end{li}
\end{frame}


\begin{frame}
  \begin{jie}
    原方程等价于
    $$
    (\MA-\MI)\MX=\MA
    $$
    \pause
    而
    $$
    \begin{aligned}
      &  (\MA-\MI ~~\red{\MA}) =\left(
        \begin{array}{rrrrrr}
          1&2& 0&\red{2}&\red{2}&\red{0}\\
          2&0& 3&\red{2}&\red{1}&\red{3}\\
          0&1&-1&\red{0}&\red{1}&\red{0}
        \end{array}
      \right)\\\pause
      &\overset{r_2-2r_1\atop r_2\leftrightarrow r_3}{\sim}
      \left(
        \begin{array}{rrrrrr}
          1& 2& 0&\red{ 2}&\red{ 2}&\red{0}\\
          0& 1&-1&\red{ 0}&\red{ 1}&\red{0}\\
          0&-4& 3&\red{-2}&\red{-3}&\red{3}
        \end{array}
      \right)     \pause
      \overset{r_3+4r_2\atop r_3\div(-1)}{\sim}
      \left(
        \begin{array}{rrrrrr}
          1&2& 0&\red{2}&\red{ 2}&\red{ 0}\\
          0&1&-1&\red{0}&\red{ 1}&\red{ 0}\\
          0&0& 1&\red{2}&\red{-1}&\red{-3}
        \end{array}
      \right) \\ \pause
      &  
      \overset{r_3+4r_2\atop r_3\div(-1)}{\sim}
      \left(
        \begin{array}{rrrrrr}
          1&0&0&\red{-2}&\red{2}&\red{6}\\
          0&1&0&\red{2}&\red{0}&\red{-3}\\
          0&0&1&\red{2}&\red{-1}&\red{-3}
        \end{array}
      \right)
    \end{aligned}
    $$
  \end{jie}
\end{frame}


\begin{frame}
  \begin{li}
    当$a,b$满足什么条件时,矩阵$\MA=\left(
      \begin{array}{rrrr}
        0&1&2&3\\
        1&4&7&10\\
        -1&0&1&b\\
        a&2&3&4
      \end{array}
    \right)$
    不可逆。
  \end{li}
\end{frame}


\begin{frame}
  \begin{jie}
    $$
    \begin{aligned}
      \left(
        \begin{array}{rrrr}
          0&1&2&3\\
          1&4&7&10\\
          -1&0&1&b\\
          a&2&3&4
        \end{array}
      \right)\pause\xrightarrow[c_2\leftrightarrow c_3]{c_1\leftrightarrow c_2} &
      \left(
        \begin{array}{rrrr}
          1 & 2 & 0 & 3\\
          4 & 7 & 1 & 10\\
          0 & 1 & -1 & b\\
          2 & 3 & a & 4\\
        \end{array}
      \right)\\\pause
      \xrightarrow[r_4+r_1\times(-2)]{r_2+r_1\times(-4)}&
      \left(
        \begin{array}{rrrr}
          1 & 2 & 0 & 3\\
          0 & -1 & 1 & -2\\
          0 & 1 & -1 & b\\
          0 & -1 & a & -2\\
        \end{array}
      \right)\\\pause
      \xrightarrow[r_4+r_2\times(-1)]{r_3+r_2}&
      \left(
        \begin{array}{rrrr}      
          1 & 2 & 0 & 3\\
          0 & -1 & 1 & -2\\
          0 & 0 & -1 & b\\
          0 & 0 & a-1 & 0\\
        \end{array}
      \right)
    \end{aligned}
    $$\pause
    由此可知$\MA$不可逆的条件是$(a-1)b=0$。
  \end{jie}
\end{frame}

