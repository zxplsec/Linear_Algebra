\documentclass[10pt,a4paper%,twoside,openright,titlepage,fleqn,%
%headinclude,footinclude,BCOR5mm,%
%numbers=noenddot,cleardoublepage=empty,%
tablecaptionabove]{article}

\usepackage{geometry}
\geometry{left=2.5cm,right=2.5cm,top=2.5cm,bottom=2.5cm}

\usepackage{amsmath,amssymb,amsthm}

%% -----------------设置数学公式字体-------------------------
%% Font style 1
%% \newcommand\ibinom[2]{\genfrac\lbrace\rbrace{0pt}{}{#1}{#2}}
%% \usepackage{bm}

%% Font style 2
%% \newcommand\ibinom[2]{\genfrac\lbrace\rbrace{0pt}{}{#1}{#2}} 
%% \usepackage[boldsans]{ccfonts} 
%% \usepackage{bm} 

%% Font style 3
\newcommand\ibinom[2]{\genfrac\lbrace\rbrace{0pt}{}{#1}{#2}}
\usepackage[euler-digits]{eulervm}
\usepackage{bm}

%% Font style 4
%% \usepackage{fourier}
%% \newcommand\ibinom[2]{\genfrac\lbrace\rbrace{0pt}{}{#1}{#2}}
%% \usepackage{bm}

%% Font style 5
%% \newcommand\ibinom[2]{\genfrac\lbrace\rbrace{0pt}{}{#1}{#2}}
%% \usepackage{mathptmx}
%% \usepackage{bm} 


%% %% Font style 6
%% \newcommand\ibinom[2]{\genfrac\lbrace\rbrace{0pt}{}{#1}{#2}}
%% \usepackage{txfonts}
%% \usepackage{bm}
%% -----------------------------------------------------------

\usepackage{titlesec} %设置标题
\usepackage{titletoc}

\usepackage{extarrows}
\usepackage{verbatim,color,xcolor}
\usepackage{pgf}
\usepackage{tikz}
\usetikzlibrary{calc}
\usetikzlibrary{arrows,snakes,backgrounds,shapes,patterns}
\usetikzlibrary{matrix,fit,positioning,decorations.pathmorphing}
%% \usepackage{classicthesis}
\usepackage{CJK}
\usepackage{mathdots}

\usepackage{listings}
\lstset{
  keywordstyle=\color{blue!70},
  frame=single,
  basicstyle=\ttfamily\small,
  commentstyle=\small\color{red},
  breakindent=0pt,
  rulesepcolor=\color{red!20!green!20!blue!20},
  rulecolor=\color{black},
  tabsize=4,
  numbersep=5pt,
  breaklines=true,
  %% backgroundcolor=\color{red!10},
  showspaces=false,
  showtabs=false,
  extendedchars=false,
  escapeinside=``,
  frame=no,
}


\newcommand{\blue}{\textcolor{blue}}
\newcommand{\red}{\textcolor{red}}
\newcommand{\purple}{\textcolor{electricpurple}}
\newcommand{\ds}{\displaystyle}
\newcommand{\cd}{\cdots}
\newcommand{\dd}{\ddots}
\newcommand{\vd}{\vdots}
\newcommand{\id}{\iddots}

\newcommand{\R}{\mathbb R}
\def\nn{{\boldsymbol{n}}}
\def\xx{{\boldsymbol{x}}}
\def\F{{\boldsymbol{F}}}
\def\div{{\mathrm{div}}}


\begin{document}

\begin{CJK}{UTF8}{gkai}


 

\newtheorem{li}{例}
\newtheorem{jielun}{结论}
\newtheorem{dingli}{定理}
\newtheorem{mingti}{{命题}} 
\newtheorem{yinli}{{引理}} 
\newtheorem{tuilun}{{推论}}
\newtheorem{dingyi}{{定义}} 
\newtheorem{example}{{例}}
\newtheorem*{example*}{{例}}
\newtheorem*{jie}{{解}}
\newtheorem*{zhengming}{{证明}}
\newtheorem{zhu}{{注}}
\newtheorem*{zhu*}{{注}}
\newtheorem{xingzhi}{{性质}}
\newtheorem{wenti}{{问题}}
\newtheorem{rem}{{Remark}}
\newtheorem{lem}{{Lemma}}
\pagenumbering{roman}
\pagestyle{plain}

\pagenumbering{arabic}



\title{编译C程序的工作原理}
%\author{张晓平}
%\date{}                                           % Activate to display a given date or no date
\maketitle

C是一种高级语言,它需要编译器将其转换为可执行代码,以使得程序能在机器上运行。 

\section{如何编译和运行一个C程序?}

以下介绍在MAC或Linux上使用gcc编译器的几个步骤。

\begin{itemize}

\item 首先使用编辑器(如vi或emacs等)创建一个C程序,并将其保存为add2num.c。

\begin{lstlisting}[backgroundcolor=\color{red!10}]
$ emacs add2num.c
\end{lstlisting}
在编辑界面输入以下内容:
\lstinputlisting[language=c,frame=single]{Code/add2num.c}

\item 然后用以下命令编译,并查看当前目录下的文件。
\begin{lstlisting}[backgroundcolor=\color{red!10}]
$ gcc -Wall add2num.c –o add2num
$ ls
add2num    add2num.c
\end{lstlisting}
选项\red{\ttfamily -Wall}启动所有编译器的警告信息。建议使用该选项以生成更好的代码。 选项\red{\ttfamily -o}用来制定输出文件名。如果缺省该选项,则输出文件将默认为\red{\ttfamily a.out}。

\item 编译通过后,将会生成可执行文件,可使用以下命令来运行生成的可执行文件。


\begin{lstlisting}[backgroundcolor=\color{red!10}]
$ ./add2num
\end{lstlisting}

\end{itemize}


\section{编译过程中发生了什么?}

编译器将一个C程序转换为一个可执行文件,这经历了4个阶段:
\begin{itemize}
\item 预处理
\item 编译
\item 汇编
\item 链接
\end{itemize}
执行以下命令,在当前目录下会生成所有的中间文件以及可执行文件。
\begin{lstlisting}[backgroundcolor=\color{red!10}]
$gcc -Wall -save-temps filename.c –o filename 
$ls 
add2num		add2num.c	add2num.o
add2num.bc	add2num.i	add2num.s
\end{lstlisting}

接下来,让我们一个个地来看这些中间文件中的内容。
\begin{itemize}
\item 预处理
\item[] 这是源代码后的第一阶段,包括
\begin{itemize}
\item 去掉注释
\item 宏的展开
\item 头文件的展开
\end{itemize}
预处理的输出保存在文件add2num.i中,可用以下命令来查看其内容:
\begin{lstlisting}[backgroundcolor=\color{red!10}]
$less add2num.i
$ls 
...
int printf(const char * restrict, ...) __attribute__((__format__ (__printf__, 1, 2)));
...
extern int __vsnprintf_chk (char * restrict, size_t, int, size_t,
       const char * restrict, va_list);
# 499 "/usr/include/stdio.h" 2 3 4
# 3 "add2num.c" 2

int main(void)
{
  int a = 5, b = 4;
  printf("Addition is: %d\n", (a+b));
  return 0;
}
\end{lstlisting}
\red{分析:}在以上内容中,源文件被附加了很多信息,但在末尾代码仍被保留。{\ttfamily printf()}函数中包含了{\ttfamily a+b}而非{\ttfamily add(a,b)},因为宏已被展开。注释被去除掉。{\ttfamily\#include<stdio.h>}没有了,取而代之的是很多代码。因此,头文件已被展开,并且被包含到了源文件中。

\item 编译
\item[] 接下来就是编译{\ttfamily add2num.i},并生成一个编译过的输出文件{\ttfamily add2num.s}。该文件为汇编指令,可用以下命令查看:
\begin{lstlisting}[backgroundcolor=\color{red!10}]
$less add2num.s
...
Ltmp2:
        .cfi_def_cfa_register %rbp
        subq    $16, %rsp
        leaq    L_.str(%rip), %rdi
        movl    $0, -4(%rbp)
        movl    $5, -8(%rbp)
        movl    $4, -12(%rbp)
        movl    -8(%rbp), %eax
        addl    -12(%rbp), %eax
        movl    %eax, %esi
        movb    $0, %al
        callq   _printf
        xorl    %esi, %esi
...      
\end{lstlisting}
以上内容表明这是汇编语言,能为汇编器所识别。
\item 汇编
\item[] 这一阶段,将通过汇编器将{\ttfamily filename.s}转换成{\ttfamily filename.o}。该文件包含机器指令。需要注意的是,该阶段只会将现有代码转换成机器语言,而诸如{\ttfamily printf()}d的函数调用则不会。
\begin{lstlisting}[backgroundcolor=\color{red!10}]
$less add2num.o
<CF><FA><ED><FE>^G^@^@^A^C^@^@^@^A^@^@^@^D^@^@^@^@^B^@^@^@ ^@^@^@^@^@^@^Y^@^@^@<88>^A^@^@^@^@^@^@^@^
...      
\end{lstlisting}
\item 链接
\item[] 
这是最后一个阶段,将完成所有函数调用及其定义的链接工作。链接器知道所有这些函数在何处执行。链接器也会做一些额外的工作,以添加一些启动和结束程序所需的额外代码。 在命令行中输入以下命令,可看出从目标文件到可执行文件时文件大小的变化。这是因为链接器为我们的程序添加了额外的代码。
\begin{lstlisting}[backgroundcolor=\color{red!10}]
$ size add2num.o
__TEXT	__DATA	__OBJC	others	dec	hex
145	0	0	32	177	b1	
HXM:Code hxm$ size add2num
__TEXT	__DATA	__OBJC	others	dec	hex
4096	4096	0	4294971392	4294979584	100003000
\end{lstlisting}
\end{itemize}

\end{CJK}
\end{document}