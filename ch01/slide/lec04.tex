\section{克莱姆法则}
\begin{frame}
考察$n$元一次方程组
\begin{equation}\label{ls}
  \left\{
    \begin{array}{l}
      a_{11}x_1 + a_{12}x_2 + \cdots + a_{1n}x_n = b_1, \\[0.3cm]
      a_{21}x_1 + a_{22}x_2 + \cdots + a_{2n}x_n = b_2, \\[0.3cm]
      \cd \\[0.2cm]
      a_{n1}x_1 + a_{n2}x_2 + \cdots + a_{nn}x_n = b_n.
    \end{array}
  \right.
\end{equation}
与二、三元线性方程组相类似,它的解可以用$n$阶行列式表示。
\end{frame}

\begin{frame}
\begin{dingli}[克莱姆法则]
  如果线性方程组(\ref{ls})的系数行列式不等于0,即
  $$
  D = \left|
    \begin{array}{ccc}
      a_{11}  & \cd  & a_{1n} \\
      \vd    &      & \vd  \\
      a_{n1}  & \cd  & a_{nn}
    \end{array}
  \right|\ne 0
  $$
  则方程组(\ref{ls})存在唯一解
  $$
  x_1 = \frac{D_1}D, \ x_2 = \frac{D_2} D, \ \cdots, \ x_n = \frac{D_n}D,
  $$
  其中
  \begin{center}
    \begin{tikzpicture}
      \matrix (M) [matrix of math nodes]  { 
        D_j = \\
      };
      \matrix(MM) [right=2pt of M, matrix of math nodes,nodes in empty cells,
      ampersand replacement=\&,left delimiter=|,right delimiter=|] {
        a_{11} \& \cd \& a_{1,j-1} \&  b_1 \& a_{1, j+1} \& \cd \& a_{1n} \\
        \vd   \&     \& \vd \& \vd \& \vd \&  \&  \vd\\       
        a_{n1} \& \cd \&  a_{n,j-1} \&  b_n \& a_{n, j+1} \& \cd \& a_{nn} \\
      };
      \node[below=10pt  of MM-3-4, blue] (note)  {第$j$列};
    \end{tikzpicture}
  \end{center}
\end{dingli}
\end{frame}

\begin{frame}
\blue{\proofname.~}
  \red{先证存在性}: 
  将$x_i=\frac{D_i}D$代入第$i$个方程,则有
  $$
  \begin{array}{cl}
    & a_{i1}x_1 +\cd + a_{ii}x_i + \cd + a_{in}x_n \\[0.3cm]
    & =   \frac1D(a_{i1}\red{D_1} +\cd + a_{ii}\blue{D_i} + \cd + a_{in}\purple{D_n}) \\[0.3cm]
    & =   \frac1D \left[
      a_{i1}\red{(b_1 A_{11}+ \cd + b_nA_{n1})}
      +\cd+ a_{ii}\blue{(b_1 A_{1i} + \cd + b_nA_{ni})}  \right. \\[0.2cm]
    &  \left. \ \ \ \ \ \
      + \cd + a_{in}\purple{(b_1 A_{1n}+ \cd + b_nA_{nn})} \right]\\[0.3cm]
    & =   \frac1D \left[
      b_1(a_{i1} A_{11}+ a_{i2} A_{12}  \cd + a_{in}A_{1n}) 
      +\cd + b_i(a_{i1} A_{i1}+ a_{i2} A_{i2}  \cd + a_{in}A_{in})   \right. \\[0.2cm]
    &  \left. \ \ \ \ \ \ + \cd+ b_n(a_{i1} A_{n1}+ a_{i2} A_{n2}  \cd + a_{in}A_{nn}) \right] \\[0.3cm]
    & =   \frac 1D b_{i} D = b_i.
  \end{array}
  $$
\end{frame}

\begin{frame}
  \red{再证唯一性}:设还有一组解$\purple{y_i, i = 1, 2, \cd, n}$,以下证明$\purple{y_i =D_i/D}$。
  现构造一个新行列式
  $$
  \begin{aligned}
    y_1D & =  \left|
      \begin{array}{cccc}
        a_{11}y_1 & a_{12}    &  \cd   &   a_{1n}      \\[0.1cm] 
        a_{21}y_1 & a_{22}    &  \cd   &   a_{2n}      \\[0.1cm] 
        \vd    &    \vd     &        &     \vd      \\[0.1cm] 
        a_{n1}y_1 & a_{n2}    &  \cd   &   a_{nn}      
      \end{array}
    \right| \\
    & \xlongequal{c_1 + y_2 c_2 + \cd + y_n c_n} 
    \left|
      \begin{array}{cccc}
        \sum_{k=1}^n a_{1k}y_k & a_{12}    &  \cd   &   a_{1n}      \\[0.1cm] 
        \sum_{k=1}^n a_{2k}y_k & a_{22}    &  \cd   &   a_{2n}      \\[0.1cm] 
        \vd    &    \vd     &        &     \vd      \\[0.1cm] 
        \sum_{k=1}^n a_{2k}y_k & a_{n2}    &  \cd   &   a_{nn}      
      \end{array}
    \right|  \\
    & = \left|
      \begin{array}{cccc}
        b_1 & a_{12}    &  \cd   &   a_{1n}      \\[0.1cm] 
        b_2 & a_{22}    &  \cd   &   a_{2n}      \\[0.1cm] 
        \vd    &    \vd     &        &     \vd      \\[0.1cm] 
        b_n & a_{n2}    &  \cd   &   a_{nn}      
      \end{array}
    \right|  = D_1
  \end{aligned}
  $$
  所以$y_1 = D_1/D$。同理可证$y_i=D_i/D, i = 2, \cd, n$。
\end{frame}

\begin{frame}
\begin{li}
  $$\left\{
    \begin{array}{rcrcrcrcr}
      2x_1 &+ & x_2 &- & 5x_3&+ & x_4 &= & 8, \\[0.2cm]
      x_1 &- & 3x_2&  &     &- & 6x_4& = & 9, \\[0.2cm]
           &  & x_2 &- & x_3 &+ & 2x_4 &= & -5, \\[0.2cm]
      x_1 &+ & 4x_2&-& 7x_3 &+& 6x_4 &= & 0.
    \end{array}
  \right.
  $$
\end{li} \pause 

\begin{jie}
  $$
  \begin{array}{ll}
    D &= \left|
        \begin{array}{rrrr}
          2 &  1 & -5 &  1\\
          1 & -3 &  0 & -6\\
          0 &  2 & -1 &  2\\
          1 &  4 & -7 &  6
        \end{array}
                        \right|
                        \xlongequal[r_4-r_2]{r_1-2r_2}
                        \left|
                        \begin{array}{rrrr}
                          0 &  7 & -5 & 13\\
                          1 & -3 &  0 & -6\\
                          0 &  2 & -1 &  2\\
                          0 &  7 & -7 & 12
                        \end{array}
                                        \right|\\[0.8cm]
      & = - \left|
        \begin{array}{rrr}
          7  & -5 & 13\\
          2  & -1 &  2\\
          7  & -7 & 12
        \end{array}
                    \right|
                    \xlongequal[c_3+2c_2]{c_1+2c_2}
                    - \left|
                    \begin{array}{rrr}
                      -3  & -5 &  3\\
                      0  & -1 &  0\\
                      -7 & -7 & -2
                    \end{array}
                                \right|\\[0.8cm]
      & =\left| 
        \begin{array}{rr}
          -3 &  3\\
          -7 & -2
        \end{array}
               \right| = 27.
  \end{array}
  $$
\end{jie}
\end{frame}

\begin{frame}

  $$
  \begin{aligned}
    D_1 = \left|
      \begin{array}{rrrr}
        \red{8}  &  1 & -5 &  1\\
        \red{9}  & -3 &  0 & -6\\
        \red{-5} &  2 & -1 &  2\\
        \red{0}  &  4 & -7 &  6
      \end{array}
    \right|= 81, & \quad
    D_2 = \left|
      \begin{array}{rrrr}
        2 & \red{8}  & -5 &  1\\
        1 & \red{9}  &  0 & -6\\
        0 & \red{-5} & -1 &  2\\
        1 & \red{0}  & -7 &  6
      \end{array}
    \right|=-108,\\
    D_3 = \left|
      \begin{array}{rrrr}
        2  &  1 & \red{8}  &  1\\
        1  & -3 & \red{9}  & -6\\
        0  &  2 & \red{-5} &  2\\
        1  &  4 & \red{0}  &  6
      \end{array}
    \right|= -27, & \quad
    D_4 = \left|
      \begin{array}{rrrr}
        2 &  1 & -5 & \red{8} \\
        1 & -3 &  0 & \red{9} \\
        0 &  2 & -1 & \red{-5}\\
        1 &  4 & -7 & \red{0} 
      \end{array}
    \right|= 27
  \end{aligned}
  $$
  于是得
  $$
  x_1 = \frac{D_1}D = 3,  \ \ 
  x_2 = \frac{D_2}D = -4, \ \ 
  x_3 = \frac{D_3}D = -1, \ \ 
  x_4 = \frac{D_4}D = 1.
  $$

\end{frame}

\begin{frame}

\begin{li}
  设曲线$y=a_0+a_1x + a_2 x^2 + a_3 x^3$通过四点$(1,3), (2,4), (3,3), (4,-3)$,求系数$a_0,a_1,a_2,a_3$。
\end{li} \pause 
\begin{jie}
  依题意可得线性方程组

  $$
  \setlength{\arraycolsep}{1.0pt}
  \left\{
    \begin{array}{rcrcrcrcr}
      a_0 & + &  a_1 & + &  a_2 & + &   a_3 & = & 3, \\[0.2cm]
      a_0 & + & 2a_1 & + & 4a_2 & + &  8a_3 & = & 4, \\[0.2cm]
      a_0 & + & 3a_1 & + & 9a_3 & + & 27a_3 & = & 3, \\[0.2cm]
      a_0 & + & 4a_1 & + &16a_4 & + & 64a_3 & = & 3,
    \end{array}
  \right.
  $$
  其系数行列式为
  $$
  D = \left|
    \begin{array}{rrrr}
      1 & 1 &  1 &  1 \\
      1 & 2 &  4 &  8 \\
      1 & 3 &  9 & 27 \\
      1 & 4 & 16 & 64
    \end{array}
  \right|
  $$
  是一个范德蒙德行列式,其值为
  $$ 
  D = 1\cdot 2 \cdot 3 \cdot 1 \cdot 2 \cdot 1 = 12.
  $$
\end{jie}

\end{frame}

\begin{frame}
  而
  $$
  \begin{array}{ll}
    D_1 = \left|
    \begin{array}{rrrr}
      \red{3}  &  1 &  1 &  1\\
      \red{4}  &  2 &  4 &  8\\
      \red{3}  &  3 &  9 & 27\\
      \red{-3} &  4 & 16 & 64
    \end{array}
                           \right|= 36, &
                                          D_2 = \left|
                                          \begin{array}{rrrr}
                                            1 & \red{3}  & 1  &  1\\
                                            1 & \red{4}  & 4  &  8\\
                                            1 & \red{3}  & 8  &  27\\
                                            1 & \red{-3} & 16 &  64
                                          \end{array}
                                                                \right|=-18,\\[0.8cm]
    D_3 = \left|
    \begin{array}{rrrr}
      1 & 1 & \red{3}  &  1 \\
      1 & 2 & \red{4}  &  8 \\
      1 & 3 & \red{3}  & 27 \\
      1 & 4 & \red{-3} & 64
    \end{array}
                         \right|= 24, &
                                        D_4 = \left|
                                        \begin{array}{rrrr}
                                          1 & 1 &  1 & \red{3} \\
                                          1 & 2 &  4 & \red{4} \\
                                          1 & 3 &  9 & \red{3} \\
                                          1 & 4 & 16 & \red{-3}
                                        \end{array}
                                                       \right|= -6.
  \end{array}
  $$
  于是得
  $$
  a_0 = \frac{D_1}D = 3,  \ \ 
  a_1 = \frac{D_2}D = -3/2, \ \ 
  a_2 = \frac{D_3}D = 2, \ \ 
  a_3 = \frac{D_4}D = -1/2.
  $$
  即曲线方程为
  $$
  y = 3 - \frac 32 x + 2 x^2 - \frac 12 x^3.
  $$

\end{frame}
