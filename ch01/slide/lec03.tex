\section{行列式的计算}
\begin{frame}

\begin{testexample}
  计算
  $$
  D = \left|
    \begin{array}{rrrr}
      3   &  1  &  -1  &  2 \\
      -5  &  1  &   3  & -4 \\
      2   &  0  &   1  & -1 \\
      1   & -5  &   3  &  -3 
    \end{array}
  \right|
  $$
\end{testexample} \pause
\begin{jie}
  $$
  \begin{array}{ll}
    D&  \xlongequal[c_4+c_3]{c_1-2c_3}  \left|
       \begin{array}{rrrr}
         5  & 1 & -1 & 1  \\
         -11 & 1 &  3 & -1 \\
         0 & 0 &  1 & 0 \\
         -5 & -5 & 3 & 0 
       \end{array}
                       \right|\\[0.8cm]
     & = (-1)^{3+3} \left| 
       \begin{array}{rrr}
         5 & 1 & 1\\
         -11 & 1 & -1 \\
         -5 & -5 & 0
       \end{array}
                   \right| 
                   \xlongequal{r_2+r_1}
                   \left|
                   \begin{array}{rrr}
                     5 & 1 & 1\\
                     -6 & 2 & 0 \\
                     -5 & -5 & 0
                   \end{array}
                               \right|\\[0.8cm]
     &  = (-1)^{1+3} \left|
       \begin{array}{rr}
         -6 &  2\\
         -5 & -5
       \end{array}
              \right| = 40.
  \end{array}
  $$    
\end{jie}
%
\end{frame}

\begin{frame}
\begin{testexample}
  计算
  $$
  D = \left |
    \begin{array}{cccc}
      a &    b &       c &           d  \\
      a &  a+b &   a+b+c &     a+b+c+d  \\
      a & 2a+b & 3a+2b+c &  4a+3b+2c+d  \\
      a & 3a+b & 6a+3b+c & 10a+6b+3c+d   
    \end{array}
  \right|
  $$
\end{testexample}\pause 

\begin{jie}
  $$
  \begin{array}{ll}
    D &  \ds \xlongequal[r_2-r_1]{r_4-r_3\atop r_3-r_2}
        \left |
        \begin{array}{cccc}
          a &    b &     c &         d  \\
          0 &    a &   a+b &     a+b+c  \\
          0 &    a &  2a+b &   3a+2b+c  \\
          0 &    a &  3a+b &   6a+3b+c   
        \end{array}\right| \xlongequal[r_3-r_2]{r_4-r_3}\left |
                             \begin{array}{cccc}
                               a &    b &     c &         d  \\
                               0 &    a &   a+b &     a+b+c  \\
                               0 &    0 &     a &      2a+b  \\
                               0 &    0 &     a &      3a+b   
                             \end{array}
                                                  \right|
    \\[1.0cm]
      &  \ds \xlongequal{r_4-r_3} \left |
        \begin{array}{cccc}
          a &    b &     c &         d  \\
          0 &    a &   a+b &     a+b+c  \\
          0 &    0 &     a &      2a+b  \\
          0 &    0 &     0 &         a   
        \end{array}
                             \right| = a^4.      
  \end{array}
  $$
\end{jie}
%
\end{frame}

\begin{frame}
\begin{testexample}
  计算
  $$
  D = \left |
    \begin{array}{cccccc}
      1 &  2 &  3 & \cd &  n-1 & n\\
      2 &  3 &  4 & \cd &   n  & 1\\
      3 &  4 &  5 & \cd &   1  & 2\\
      \vd& \vd& \vd&     & \vd  & \vd \\
      n &  1 &  2 & \cd & n-2  & n-1
    \end{array}
  \right|
  $$
\end{testexample}
\end{frame}

\begin{frame}
\begin{jie}
  $$
  \begin{aligned}
    &D_n    
    \xlongequal[i=n,\cdots,2]{r_i-r_{i-1}} 
    \left|
      \begin{array}{cccccc}
        1   &  2 &  3 & \cd &  n-1 & n\\
        1   &  1 &  1 & \cd &   1  & 1-n \\
        1   &  1 &  1 & \cd &  1-n  & 1\\
        \vd & \vd & \vd&     & \vd  & \vd \\
        1   & 1-n &  1 & \cd &   1   & 1
      \end{array}
    \right|\\
    &  
    \xlongequal[i=2,\cdots,n]{c_i-c_1} 
    \left|
      \begin{array}{cccccc}
        1   &  1 &  2 & \cd &  n-2 & n-1\\
        1   &  0 &  0 & \cd &   0  & -n \\
        1   &  0 &  0 & \cd &  -n  & 0\\
        \vd & \vd & \vd&     & \vd  & \vd \\
        1   & -n &  0 & \cd &   0   & 0
      \end{array}
    \right| \xlongequal[i=2,\cd,n]{c_i\div n} n^{n-1} 
    \left|
      \begin{array}{cccccc}
        1   &  \frac 1n & \frac 2n & \cd &  \frac{n-2}n & \frac{n-1}n\\
        1   &  0 &  0 & \cd &   0  & -1 \\
        1   &  0 &  0 & \cd &  -1  & 0\\
        \vd & \vd & \vd&     & \vd  & \vd \\
        1   & -1 &  0 & \cd &   0   & 0
      \end{array}
    \right|\\
    &  \xlongequal{c_1+c_2+\cd+c_n} 
    n^{n-1} \left|
      \begin{array}{cccccc}
        1+\sum_{i-1}^{n-1}\frac in   &  \frac 1n & \frac 2n & \cd &  \frac{n-2}n & \frac{n-1}n\\
        0   &  0 &  0 & \cd &   0  & -1 \\
        0   &  0 &  0 & \cd &  -1  & 0\\
        \vd & \vd & \vd&     & \vd  & \vd \\
        0   & -1 &  0 & \cd &   0   & 0
      \end{array}       
    \right|\\
    &  = n^{n-1} \left[ 1 + \frac 1n \frac {n(n-1)}2\right] 
    (-1)^{\frac{(n-1)(n-2)}2}(-1)^{n-1} = (-1)^{\frac{(n-1)n}2} \frac{n+1}2 n^{n-1}.
  \end{aligned}
  $$
\end{jie}
%
\end{frame}

\begin{frame}
\begin{testexample}
  计算行列式
  $$
  D_{20} = \left|
    \begin{array}{rrrrrrr}
      1   & 2    & 3    & \cd  & 18    & 19    &  20 \\ 
      2   & 1    & 2    & \cd  & 17    & 18    &  19 \\
      3   & 2    & 1    & \cd  & 16    & 17    &  18 \\
      \vd & \vd  & \vd  & \cd  & \vd   & \vd   &  \vd \\
      20  & 19   & 18   & \cd  & 3     & 2     &  1
    \end{array}
  \right|
  $$    
\end{testexample} \pause 
\begin{jie}
  $$
  \begin{aligned}
    D_{20} &   \xlongequal[i=19,\cd,1]{c_{i+1}-c_i} 
             \left|
             \begin{array}{rrrrrrr}
               1   & 1    & 1    & \cd  & 1    & 1    &  1 \\ 
               2   & -1   & 1    & \cd  & 1    & 1    &  1 \\
               3   & -1   & -1   & \cd  & 1    & 1    &  1 \\
               \vd & \vd  & \vd  & \cd  & \vd  & \vd  &  \vd \\
               % 19  & -1   & -1   & \cd  & -1   & -1   &  1 \\
               20  & -1   & -1   & \cd  & -1   & -1   &  -1
             \end{array}
                                                        \right| \\ 
           & \xlongequal[i=2,\cd,20]{r_i+r_1} 
             \left|
             \begin{array}{rrrrrrr}
               1   & 1    & 1    & \cd  & 1    & 1    &  1 \\ 
               3   & 0    & 2    & \cd  & 2    & 2    &  2 \\
               4   & 0    & 0    & \cd  & 2    & 2    &  2 \\
               \vd & \vd  & \vd  & \cd  & \vd  & \vd  &  \vd \\
               % 20  & 0    & 0    & \cd  & 0    & 0    &  2\\
               21  & 0    & 0    & \cd  & 0    & 0    &  0
             \end{array}
                                                        \right|
                                                        = 21 \times (-1)^{20+1} \times 2^{18} = -21 \times 2^{18}.
  \end{aligned}
  $$

\end{jie}
%
\end{frame}

\begin{frame}
\begin{testexample}
  计算元素为$a_{ij}=|i-j|$的$n$阶行列式。
\end{testexample} \pause 
\begin{jie}
  $$
  \begin{aligned}
    D_{n} &  =   \left|
      \begin{array}{rrrrrrr}
        0   & 1   & 2    & \cd & n-2  & n-1 \\ 
        1   & 0   & 1    & \cd & n-3  & n-2 \\
        %% 2   & 1   & 0    & \cd & n-4  & n-3 \\
        \vd & \vd & \vd  &     & \vd  & \vd \\
        n-2 & n-3 & n-4  & \cd & 0     & 1 \\
        n-1 & n-2 & n-3  & \cd & 1  & 0 
      \end{array}
    \right| \\
    &   \xlongequal[i=n-1,\cd,1]{c_{i+1}-c_i}   
    \left|
      \begin{array}{rrrrrrr}
        0   & 1   & 1    & \cd & 1  & 1 \\ 
        1   & -1  & 1    & \cd & 1  & 1 \\
        %% 2   & -1  & -1   & \cd & 1  & 1 \\
        \vd & \vd & \vd  &     & \vd & \vd \\
        n-2 & -1  & -1   & \cd & -1 & 1 \\
        n-1 & -1  & -1   & \cd & -1 & -1 
      \end{array}
    \right| \\
    & \xlongequal[i=2,\cd,n]{r_{i}+r_1}  
    \left|
      \begin{array}{rrrrrrr}
        0   & 1   & 1   & \cd & 1   & 1   \\ 
        1   & 0   & 2   & \cd & 2   & 2   \\
        \vd & \vd & \vd &     & \vd & \vd \\
        n-2 & 0   & 0  & \cd  & 0   & 2 \\
        n-1 & 0   & 0  & \cd  & 0   & 0 
      \end{array}
    \right| = (-1)^{n-1}(n-1)2^{n-2}.
  \end{aligned}
  $$
\end{jie}
%
\end{frame}

\begin{frame}
\begin{testexample}
  计算
  $$
  D= \left|
    \begin{array}{ccccc}
      1 &  2  & 3   &\cd & n   \\
      2 &  2  & 0   &\cd & 0  \\
      3 &  0  & 3   &\cd & 0  \\
      \vd & \vd  \vd  &    & \\
      n &  0  & 0   &\cd & n
    \end{array}
  \right|
  $$
\end{testexample} \pause 
\begin{jie}
  $$
  \begin{aligned}
    D & \xlongequal[i=2,\cd,n]{r_1-r_i} & 
    \left|
      \begin{array}{ccccc}
        1-\sum_{i=2}^n i &  0  & 0   &\cd & 0   \\
        2 &  2  & 0   &\cd & 0  \\
        3 &  0  & 3   &\cd & 0  \\
        \vd & \vd & \vd  &    &\vd  \\
        n &  0  & 0   &\cd & n
      \end{array}
    \right|
    =  \left(1-\sum_{i=2}^n i\right) \cdot 2 \cdot 3 \cdot \cd \cdot n
    =  \left[2-\frac{(n+1)n}2\right] n!
  \end{aligned}
  $$
\end{jie}

\end{frame}

\begin{frame}
如何计算“爪形”行列式
\begin{figure}
  \centering
  \begin{tikzpicture}
    \matrix(A) [matrix of math nodes,nodes in empty cells,ampersand replacement=\&,left delimiter=|,right delimiter=|] {
    a_{11} \& a_{12} \& a_{13} \& \cd \& a_{1n} \\
    a_{21} \& a_{22} \& 0     \& \cd \&  0    \\
    a_{31} \&  0    \& a_{33} \& \cd \&  0    \\
    \vd  \&  \vd  \&  \vd  \&     \&  \vd  \\
    a_{n1} \&  0    \& 0 \& \cd \& a_{nn} \\
  };
    \draw[red] (A-1-1.center) -- (A-1-5.center) (A-1-1.center) -- (A-5-1.center) (A-1-1.center) -- (A-5-5.center);
  \end{tikzpicture}
\end{figure}
其解法固定,即从第二行开始,每行依次乘一个系数然后加到第一行,使得第一行除第一个元素外都为零,从而得到一个下三角行列式。
\end{frame}

\begin{frame}
请自行验证以下行列式(假定$a_i \ne 0$)
$$
D_{n+1} = 
\left|
\begin{array}{ccccc}
a_0 &  1  & 1   &\cd & 1   \\
1   & a_1 & 0   &\cd & 0  \\
1   & 0   & a_2 &\cd & 0  \\
\vd & \vd  \vd  &    & \\
1   & 0   & 0   &\cd & a_n
\end{array}
\right|  = \left(a_0 - \sum_{i=1}^n \frac1{a_i}\right) a_1 a_2 \cd a_n.
$$
%
\end{frame}

\begin{frame}
类似的方式还可用于求解如下形式的“爪型行列式”
\begin{figure}[htbp]
  \centering
  \subfigure[]{
  \begin{tikzpicture}[scale=3]
    \matrix(B) [matrix of math nodes,nodes in empty cells,ampersand replacement=\&,left delimiter=|,right delimiter=|] {
    \&  \& \\
    \&  \& \\
    \&  \& \\ 
  };
    \draw[red] (B-1-3.north east) -- (B-1-1.north west) 
    (B-1-3.north east) -- (B-3-1.south west) 
    (B-1-3.north east) -- (B-3-3.south east);
  \end{tikzpicture}
}
  \subfigure[]{
  \begin{tikzpicture}[scale=3]
    \matrix(B) [matrix of math nodes,nodes in empty cells,ampersand replacement=\&,left delimiter=|,right delimiter=|] {
    \&  \& \\
    \&  \& \\
    \&  \& \\
  };
    \draw[red]
    (B-3-1.south west) -- (B-1-1.north west) 
    (B-3-1.south west) -- (B-1-3.north east) 
    (B-3-1.south west) -- (B-3-3.south east);
  \end{tikzpicture}
}
  \subfigure[]{
  \begin{tikzpicture}[scale=3]
    \matrix(B) [matrix of math nodes,nodes in empty cells,ampersand replacement=\&,left delimiter=|,right delimiter=|] {
    \&  \& \\
    \&  \& \\
    \&  \& \\
  };
    \draw[red]
    (B-3-3.south east) -- (B-1-1.north west) 
    (B-3-3.south east) -- (B-1-3.north east) 
    (B-3-3.south east) -- (B-3-1.south west);
  \end{tikzpicture}
}      
\end{figure}
\end{frame}

\begin{frame}
\begin{testexample}
  $$
  \left|
    \begin{array}{ccccc}
      1 & 1 & \cd & 1 & 1 \\
      0 & 0 & \cd & 2 & 1 \\
      \vd & \vd & & \vd & \vd \\
      0 & n-1 & \cd & 0 & 1 \\
      n & 0 & \cd & 0 & 1
    \end{array}
  \right|  = (-1)^{\frac{n(n-1)}2} n! \left(1-\sum_{i=2}^n\frac1i\right)
  $$
\end{testexample}
\end{frame}

\begin{frame}
\begin{testexample}
  计算$n$阶行列式
  $$
  D_n = \left|
    \begin{array}{cccc}
      x & a & \cd & a \\
      a & x & \cd & a \\
      \vd & \vd & & \vd \\
      a & a &  \cd & x 
    \end{array}
  \right|
  $$
\end{testexample}
\end{frame}

\begin{frame}
 
\blue{解法1}
  $$
  \begin{aligned}
    D_n  
    &  \xlongequal{c_1+c_2+\cd +c_n}
    \left|
      \begin{array}{cccc}
        x+(n-1)a & a     & \cd & a \\
        x+(n-1)a & x     & \cd & a \\
        \vd &      \vd &  & \vd \\
        x+(n-1)a & a     & \cd & x 
      \end{array}
    \right|  \\
    &  \xlongequal{c_1\div [x+(n-1)a] } 
    \left[x+(n-1)a\right]\left|
      \begin{array}{cccc}
        1 & a     & \cd & a \\
        1 & x     & \cd & a \\
        \vd   & \vd &  & \vd \\
        1 & a     & \cd & x 
      \end{array}
    \right|  \\
    &  \xlongequal[i = 2, \cd, n]{r_i - r_1} 
    \left[x+(n-1)a\right]\left|
      \begin{array}{cccc}
        1 & a   & \cd & 0 \\
        0 & x-a & \cd & 0 \\
        \vd  & \vd &  & \vd \\
        0 & 0   & \cd & x-a 
      \end{array}
    \right| \\
    & =  \left[x+(n-1)a\right](x-a)^{n-1}.
  \end{aligned}
  $$

\end{frame}

\begin{frame}

\blue{解法2}:
  $$
  \begin{aligned}
    D_n 
    &  \xlongequal[i=2,\cd, n]{r_i - r_1} 
    \left|
      \begin{array}{ccccc}
        x   & a   & a    & \cd & a \\
        a-x & x-a & 0    & \cd & 0 \\
        a-x & 0   & x-a  & \cd & 0 \\
        \vd & \vd  & \vd &  & \vd \\
        a-x & 0   & 0    & \cd & x-a 
      \end{array}
    \right| \\
    & \xlongequal[i=2,\cd,n]{c_1+c_i} 
    \left|
      \begin{array}{ccccc}
        x+(n-1)a & a   & a    & \cd & a \\
        0    & x-a & 0    & \cd & 0 \\
        0    & 0   & x-a  & \cd & 0 \\
        \vd & \vd  & \vd &  & \vd \\
        0    & 0   & 0    & \cd & x-a 
      \end{array}
    \right|   \\
    &  =   \left[x+(n-1)a\right](x-a)^{n-1}.
  \end{aligned}
  $$
\end{frame}

\begin{frame}

\blue{解法3}
 
  $$
  D_n = \left|
    \begin{array}{ccccc}
      \red{1}   & \red{a}  & \red{a}  & \red{\cd} & \red{a}   \\
      \red{0}   & x  & a  & \cd & a   \\
      \red{0}   & a  & x  & \cd & a   \\
      \red{\vd} &\vd &\vd &     & \vd \\
      \red{0}   & a  & a  & \cd & x 
    \end{array}
  \right|_{n+1} 
  \xlongequal[i=2,\cdots,n+1]{r_i-r_1} 
  \left|
    \begin{array}{ccccc}
      \red{1}    & \red{a}  & \red{a} & \red{\cd} & \red{a}   \\
      \red{-1}   & x-a      &  0      & \cd & 0   \\
      \red{-1}   & 0        &  x-a    & \cd & 0   \\
      \red{\vd}  &\vd       & \vd     &     & \vd \\
      \red{-1}   & 0        &   0     & \cd & x-a 
    \end{array}
  \right|_{n+1}
  $$
  \begin{itemize}
  \item
    若$x=a$,则$D_n=0$。 
  \item 
    若$x\ne a$,则
    $$
    \begin{array}{rcl}
      D_n &  \xlongequal[j=2,\cd,n+1]{c_1+ \frac1{x-a}c_j} & 
                                                             \left|
                                                             \begin{array}{ccccc}
                                                               \red{1+\frac{a}{x-a}n}   & \red{a}  & \red{a}  & \red{\cd} & \red{a}   \\
                                                               \red{0}   & x-a  & 0  & \cd & 0   \\
                                                               \red{0}   & 0  & x-a  & \cd & 0   \\
                                                               \red{\vd} &\vd &\vd &     & \vd \\
                                                               \red{0}   & 0  & 0  & \cd & x-a
                                                             \end{array}
                                                                                           \right|_{n+1} \\[0.5in]
          &  = &   \left[x+(n-1)a\right](x-a)^{n-1}.
    \end{array}
    $$
  \end{itemize}
\end{frame}

\begin{frame}

\blue{解法4}:
  $$
  \begin{array}{ll}
    D_n &  = \left|
          \begin{array}{cccc}
            x-a  & a  & \cd & a   \\
            0    & x  & \cd & a   \\
            \vd  &\vd &     & \vd \\
            0    & a  & \cd & x 
          \end{array}
                              \right|
                              +\left|
                              \begin{array}{cccc}
                                a   & a  & \cd & a   \\
                                a   & x  & \cd & a   \\
                                \vd &\vd &     & \vd \\
                                a   & a  & \cd & x 
                              \end{array}
                                                 \right| \\[1.0cm]
        & = (x-a) D_{n-1} + a(x-a)^{n-1}.
  \end{array}
  $$ 
  于是
  $$
  \left\{
    \begin{array}{rcl}
      D_n           &=& (x-a)D_{n-1} + a(x-a)^{n-1} \\[0.2cm]
      (x-a)D_{n-1}      &=& (x-a)^2D_{n-2} + a(x-a)^{n-1} \\[0.2cm]
      \cd           && \\ [0.2cm]
      (x-a)^{n-4}D_4 &=& (x-a)^{n-3}D_{3} + a(x-a)^{n-1}\\ [0.2cm]
      (x-a)^{n-3}D_3 &=& (x-a)^{n-2}D_{2} + a(x-a)^{n-1}
    \end{array}
  \right.
  $$ 
  因此
  $$
  D_n = (x-a)^{n-2}(x^2-a^2) + (n-2)a(x-a)^{n-1} = [x+(n-1)a](x-a)^{n-1}.      
  $$
 
\end{frame}

\begin{frame}
 
  该行列式经常以不同方式出现,如
  \begin{itemize}
  \item
    $$
    \left|
      \begin{array}{cccc}
        0  &  1  & \cd & 1   \\
        1  &  0  & \cd & 1   \\
        \vd& \vd &     & \vd \\
        1  &  1  & \cd & 0 
      \end{array}
    \right|   = (-1)^{n-1}(n-1)
    $$ 
  \item
    $$
    \left|
      \begin{array}{cccc}
        1  &  a  & \cd & a   \\
        a  &  1  & \cd & a   \\
        \vd& \vd &     & \vd \\
        a  &  a  & \cd & 1
      \end{array}
    \right| 
    = [1+(n-1)a](1-a)^{n-1}
    $$ 
  \item
    $$
    \left|
      \begin{array}{cccc}
        1+\lambda  &  1  & \cd & 1   \\
        1  &  1+\lambda  & \cd & 1   \\
        \vd& \vd &     & \vd \\
        1  &  1  & \cd & 1+\lambda 
      \end{array}
    \right| 
    = (\lambda+n)\lambda^{n-1}
    $$
  \end{itemize}
\end{frame}

\begin{frame}

  升阶法适用于求以下形式的行列式:
  $$
  \blue{\left|
    \begin{array}{cccc}
      x_1 &  a  & \cd & a   \\
      a   & x_2 & \cd & a   \\
      \vd & \vd &     & \vd \\
      a   &  a  & \cd & x_n
    \end{array}
  \right|}=   \left(1+\sum_{i=1}^n \frac{a}{x_i-a}\right)\prod_{i=1}^n(x_i-a)
  $$
  或
  $$
  \blue{\left|
    \begin{array}{cccc}
      x_1 & a_2  & \cd & a_n   \\
      a_1 & x_2 & \cd  & a_n   \\
      \vd & \vd &     & \vd \\
      a_1 & a_2  & \cd & x_n
    \end{array}
  \right|}=  \left(1+\sum_{i=1}^n \frac{a_i}{x_i-a_i}\right)\prod_{i=1}^n(x_i-a_i).
  $$      
 
\end{frame}

\begin{frame} 
\begin{zhu}
  几种常见形式:
  $$
  \left|
    \begin{array}{cccc}
      1+a &  1  & \cd & 1   \\
      2   & 2+a & \cd & 2  \\
      \vd & \vd &     & \vd \\
      n   &  n  & \cd & n+a
    \end{array}
  \right|  = \left[a+ \frac{n(n+1)}2\right]a^{n-1}
  $$
  或
  $$
  \left|
    \begin{array}{cccc}
      a_1+b & a_1   & \cd & a_n   \\
      a_1   & a_2+b & \cd  & a_n   \\
      \vd   & \vd  &     & \vd \\
      a_1   & a_2   & \cd & a_n+b
    \end{array}
  \right|  = b^{n-1}(\sum_{i=1}^na_i+b)
  $$      

\end{zhu}
%
\end{frame}

\begin{frame} 
\begin{testexample}
  设
  $$
  \begin{aligned}
    D &= \left|
      \begin{array}{cccccc}
        a_{11} & \cd & a_{1k} &    &    &   \\
        \vd    &     &  \vd  &    &    &   \\
        a_{k1} & \cd & a_{kk} &    &    &   \\
        c_{11} & \cd & c_{1k} & b_{11}&  \cd & b_{1n}   \\
        \vd    &     & \vd   & \vd  &    & \vd \\
        c_{n1} & \cd & c_{nk} & b_{n1}&  \cd & b_{nn}
      \end{array}
    \right|, \\
    D_1 &%= det(a_{ij}) 
    = \left|
      \begin{array}{ccc}
        a_{11} & \cd & a_{1k} \\
        \vd    &     &  \vd  \\
        a_{k1} & \cd & a_{kk} \\
      \end{array}
    \right|, \\
    D_2 & %= det(b_{ij}) 
    = \left|
      \begin{array}{ccc}
        b_{11} & \cd & b_{1n} \\
        \vd    &     &  \vd  \\
        b_{n1} & \cd & b_{nn} \\
      \end{array}
    \right|.
  \end{aligned}
  $$
  证明:$D=D_1D_2$.
\end{testexample}
\end{frame}

\begin{frame} 
\begin{proof}
  对$D_1$做运算$r_i+\lambda r_j$将它转化成下三角行列式,设为
  $$
  D_1 =  \left|
    \begin{array}{ccc}
      p_{11} &       & \\
      \vd    & \dd  &  \\
      p_{k1} & \cd   & p_{kk} 
    \end{array}
  \right| = p_{11} \cd p_{kk}.
  $$ 
  对$D_2$做运算$c_i+\lambda c_j$将它转化成下三角行列式,设为
  $$
  D_2 =  \left|
    \begin{array}{ccc}
      q_{11} & \cd  &  q_{1n}\\
             & \dd  &  \vd \\
             &      & q_{nn} \\
    \end{array}
  \right| = q_{11}\cd q_{nn}.
  $$  
  于是,对$D$的前$k$行做运算$r_i+\lambda r_j$,对其后$n$列做运算$c_i+\lambda c_j$,把$D$转化为
  $$
  D = \left|
    \begin{array}{cccccc}
      p_{11} &      &  &    &    &   \\
      \vd    & \dd    &   &    &    &   \\
      p_{k1} & \cd & p_{kk} &    &    &   \\
      c_{11} & \cd & c_{1k} & q_{11}&  &    \\
      \vd    &     & \vd   & \vd  &  \dd  &  \\
      c_{n1} & \cd & c_{nk} & q_{n1}&  \cd & q_{nn}
    \end{array}
  \right|
  $$  
  故$D = p_{11}\cdots p_{kk} q_{11}\cdots q_{nn} = D_1 D_2$.
\end{proof}
%
\end{frame}

\begin{frame} 
\begin{testexample}
  计算$2n$阶行列式
  $$
  D_{2n} = \left|
    \begin{array}{cccccc}
      a &     & & & & b \\
        & \dd & & & \id & \\
        &   & a & b &  & \\
        &   & c & d &  &  \\
        & \id & & & \dd & \\
      c &     & & & & d
    \end{array}
  \right|
  $$
\end{testexample}
%
\end{frame}

\begin{frame} 
\begin{jie}
  把$D_{2n}$中的第$2n$行依次与第$2n-1$行、$\ldots$、第2行对调(共$2n-2$次相邻对换),
  在把第$2n$列依次与第$2n-1$列、$\ldots$、第2列对调,得
  $$
  D_{2n} = \left|
    \begin{array}{cccccccc}
      a & b & 0 &      & & & &0 \\
      c & d & 0 & & &  & &\\
      0 & 0 & a & & &  & & b \\
        &   &   & \dd &  &  & \id &   \\
        &   &   &     & a& b& & \\
        &   &   &     & c& d& & \\
        &   &   & \id &  &  & \dd &   \\
      0 & 0 & c & & &  & & d
    \end{array}
  \right|
  $$
  故
  $$
  \begin{aligned}
    D_{2n} & = D_2 D_{2(n-1)}  \\
    &= (ad-bc)D_{2(n-1)} \\
    &= (ad-bc)^2 D_{2(n-2)}\\
    &= \cdots \\
    &= (ad-bc)^{n-1}D_{2} \\
    &= (ad-bc)^n.      
  \end{aligned}
  $$
\end{jie}
%
\end{frame}

\begin{frame}
\begin{testexample}
  证明:范德蒙德(Vandermonde)行列式
  $$
  D_n = \left|
    \begin{array}{cccc}
      1        &  1        & \cd &    1     \\                    
      x_1      &  x_2      & \cd &    x_n    \\ 
      x_1^2    &  x_2^2     & \cd &   x_n^2   \\ 
      \vd      &  \vd      &     &    \vd      \\
      x_1^{n-1} & x_2^{n-1} &  \cd &  x_n^{n-1}
    \end{array}
  \right|
  = \prod_{n \ge i > j \ge 1}(x_i-x_j).
  $$
\end{testexample}
\end{frame}

\begin{frame}
\begin{proof}
  用数学归纳法证明。当$n=2$时,
  $$
  D_2 = \left|
    \begin{array}{cc}
      1 & 1 \\
      x_1 & x_2
    \end{array}
  \right|
  = x_2 - x_1 = \prod_{2 \ge i > j \ge 1} (x_i - x_j),
  $$
  结论成立。 \pause
  现假设结论对$n-1$阶范德蒙德行列式成立,以下证明结论对$n$阶范德蒙德行列式也成立。 
  $$
  D_n \xlongequal[i=n,\cdots, 2]{\red{r_i - x_1 r_{i-1}}}\left|
    \begin{array}{ccccc}  
      1     & 1                    & 1                       & \cd   & 1    \\
      0     & x_2 - x_1            & x_3 - x_1               &  \cd  & x_n - x_1 \\
      0     & x_2(x_2 - x_1)       & x_3(x_3 - x_1)          &  \cd  & x_n(x_n - x_1)\\
      \vd   & \vd                  & \vd                     &      & \vd   \\
      0     & x_2^{n-2}(x_2-x_1)    & x_3^{n-2}(x_3 - x_1)    &  \cd  & x_n^{n-2}(x_n - x_1) 
    \end{array}\right|
  $$  \pause
  按第1列展开,并把每列的公因子$(x_i-x_1)$提出,就有
  $$
  D_n = (x_2-x_1)(x_3-x_1)\cdots(x_n-x_1)\left|
    \begin{array}{cccc}  
      1            & 1          &  \cd  & 1 \\
      x_2          & x_3         &  \cd  & x_n\\
      \vd          & \vd         &      & \vd   \\
      x_2^{n-2}     & x_3^{n-2}    &  \cd  & x_n^{n-2}
    \end{array}\right|
  $$ \pause
  上式右端的行列式为$n-1$阶范德蒙德行列式,按归纳法假设,
  它等于所有$(x_i-x_j)$因子的乘积($n\ge i \ge j \ge 2$)。故
  $$
  D_n = (x_2-x_1)(x_3-x_1)\cdots(x_n-x_1) \prod_{n\ge i > j \ge 2}(x_i - x_j)
  = \prod_{n\ge i > j \ge 1}(x_i - x_j).
  $$
\end{proof}
%
\end{frame}

\begin{frame}
\begin{testexample}
  设$a,b,c$为互不相同的实数,证明:
  $$
  \left|
    \begin{array}{ccc}
      1   &   1   &   1\\
      a   &   b   &   c\\
      a^3 &   b^3 &   c^3
    \end{array}
  \right|=0
  $$
  的充要条件是$a+b+c=0$.
\end{testexample}
\end{frame}

\begin{frame}
\begin{proof}
  考察范德蒙德行列式
  $$
  \begin{array}{ll}
    D & = \left|
        \begin{array}{cccc}
          1   &   1   &   1   & 1\\
          a   &   b   &   c   & y\\
          a^2 &   b^2 &   c^2 & y^2\\
          a^3 &   b^3 &   c^3 & y^3\\
        \end{array}
    \right|
    = (a-b)(a-c)(b-c)(a-y)(b-y)(c-y) 
  \end{array}
  $$
  
  注意到行列式$D$可看成是关于$y$的多项式,比较包含$y^2$的项:
  $$
  \cd - \left|
    \begin{array}{cccc}
      1   &   1   &   1  \\ 
      a   &   b   &   c  \\
      a^3 &   b^3 &   c^3\\
    \end{array}
  \right| y^2 + \cd = 
  \cd -(a-b)(a-c)(b-c)(a+b+c)y^2 + \cd 
  $$
  于是
  $$
  (a-b)(a-c)(b-c)(a+b+c) = \left|
    \begin{array}{cccc}
      1   &   1   &   1  \\ 
      a   &   b   &   c  \\
      a^3 &   b^3 &   c^3\\
    \end{array}
  \right|
  = 0
  $$
  而$a,b,c$互不相同,故$a+b+c=0$.
\end{proof}
%
\end{frame}

\begin{frame}
\begin{testexample}
  计算三对角行列式
  $$
  D_n = \left|
    \begin{array}{cccccc}
      a & b & &&&\\
      c&a&b&&&\\
        &c&a&b&&\\
        &&\dd&\dd&\dd&\\
        &&&c&a&b\\
        &&&&c&a
    \end{array}
  \right|
  $$
\end{testexample}
\end{frame}

\begin{frame}
\begin{jie}
  对$D_n$按第一行展开
  $$
  D_n =  aD_{n-1} + (-1)^{1+2} b \left|
    \begin{array}{cccccc}
      c&b&&&&\\
      0&a&b&&&\\
      0&c&a&b&&\\
      \vd&\vd&\dd&\dd&\dd&\\
      0&0&\cd&c&a&b\\
      0&0&\cd&0&c&a
    \end{array}
  \right|  = a D_{n-1} - bc D_{n-2},
  $$
  其中$D_1=a, \quad D_2=a^2-bc$.
  将 
  $$D_n = a D_{n-1} - bc D_{n-2}$$
  改写成 
  $$
  D_n - k D_{n-1} = l(D_{n-1} - k D_{n-2})
  $$ 
  这里
  $$
  k+l=a, \quad kl = bc.
  $$    
\end{jie}
\end{frame}

\begin{frame}
  令$\Delta_n = D_n-kD_{n-1}$,它满足
  $$
  \left\{
    \begin{array}{l}
      \Delta_n = l\Delta_{n-1},  \\[0.05in] 
      \Delta_2 = D_2-kD_1 = a^2-bc - ka = (a-k)a-kl=la-lk=l^2.
    \end{array}    
  \right.
  $$ 
  由此可知
  $$
  \Delta_n = l^{n-2} \Delta_2 = l^2, 
  $$
  即
  $$
  \begin{array}{rl}
    D_n &  = l^n  + k D_{n-1}  = l^n  + k (l^{n-1}  + k D_{n-2}) 
          = l^n  + k l^{n-1}  + k^2 D_{n-2} \\[0.1in]
        & =  l^n  + k l^{n-1}  + k^2 (l^{n-2}  + k D_{n-3} )
          = l^n  + k l^{n-1}  + k^2 l^{n-2}  + k^3 D_{n-3} \\[0.1in]
        & = \cd  =  l^n  + k l^{n-1}  + k^2 l^{n-2}  + \cd + k^{n-2}l^2 + k^{n-1} D_1
  \end{array}
  $$ 

  而$D_1 = a = k+l$,故
  $$
  \red{
    D_n = l^n  + k l^{n-1}  + k^2 l^{n-2}  + \cd + k^{n-2}l^2 + k^{n-1}l + k^n.
  }
  $$


\end{frame}