\section{行列式的定义}
\subsection{二阶行列式}
%%%%% 

\begin{frame}
 
\begin{li}
  用消元法求解
  $$
  \left \lbrace
    \begin{array}{l}
      a_{11} x_1 + a_{12} x_2 = b_1, \\[0.2cm]
      a_{21} x_1 + a_{22} x_2 = b_2.
    \end{array}
  \right.
  $$
\end{li}  
\pause 

\begin{jie}
消去$x_2$得
$$
(a_{11}a_{22}-a_{12}a_{21})x_1 = b_1 a_{22} - b_2 a_{12},
$$
消去$x_1$得
$$
(a_{11}a_{22}-a_{12}a_{21})x_2 = b_2 a_{11} - b_1 a_{11}.
$$
\pause  

若$\boxed{\red{a_{11}a_{22}-a_{12}a_{21}\ne0}}$,则
$$
x_1 = \frac{b_1 a_{22} - b_2 a_{12}}{a_{11}a_{22}-a_{12}a_{21}}, \ \
x_2 = \frac{b_2 a_{11} - b_1 a_{11}}{a_{11}a_{22}-a_{12}a_{21}}.
$$
\end{jie}

\end{frame}

\begin{frame}
%
%%%% 
\begin{dingyi}[二阶行列式]
  由$2^2=4$个数,按下列形式排成2行2列的方形
  $$
  \left|
    \begin{array}{cc}
      a_{11} & a_{12} \\[0.2cm]
      a_{21} & a_{22} 
    \end{array}
  \right|,
  $$
  其被定义成一个数
  $$
  \left|
    \begin{array}{cc}
      a_{11} & a_{12} \\[0.2cm]
      a_{21} & a_{22} 
    \end{array}
  \right| = a_{11}a_{22} - a_{12}a_{21} \equiv D,
  $$
  该数称为由这四个数构成的二阶行列式。其中,
  \begin{itemize}
  	\item $\red{a_{ij}}$表示行列式的元素。
  	\item $i$为行标,表明该元素位于第$i$行;
  	\item $j$为列标,表明该元素位于第$j$列。
  \end{itemize}
  
\end{dingyi}
%%%% 
\end{frame}

\begin{frame}
\begin{figure}[htbp]
  \centering      
  \begin{tikzpicture}
    \matrix (A) [matrix of nodes,ampersand replacement=\&,row sep=15pt,column sep=15pt,left delimiter=|,
    right delimiter=|] {
      $a_{11}$ \& $a_{12}$  \\
      $a_{21}$ \& $a_{22}$  \\
    };
    \draw[blue, thick] (A-1-1.south east) -- (A-2-2.north west);
    \draw[red,  thick] (A-1-2.south west) -- (A-2-1.north east);
  \end{tikzpicture}
  \caption{对角线法则}
\end{figure}
\end{frame}

\begin{frame}
%%%%%% 

类似地,
$$
\begin{array}{l}
  b_1 a_{22} - b_2 a_{12} = \left|
  \begin{array}{cc}
    b_1 & a_{12} \\
    b_2 & a_{22} 
  \end{array}
          \right|  \equiv D_1\\[0.4cm]
  b_2 a_{11} - b_1 a_{21} = \left|
  \begin{array}{cc}
    a_{11} & b_1 \\
    a_{21} & b_2
  \end{array}
             \right|  \equiv D_2\\
\end{array}
$$      
则上述方程组的解可表示为
$$
x_1 = \frac{D_1}{D},\ \
x_2 = \frac{D_2}{D}.
$$

\end{frame}

\begin{frame}
\begin{li}
  求解二元线性方程组
  $$
  \left\{
    \begin{array}{l}
      3x_1 - 2x_2 = 12, \\[0.2cm]
      2x_1 + x_2  = 1.
    \end{array}
  \right.
  $$
\end{li} \pause
\begin{jie}
因为
$$
\begin{array}{l}
D = \left|
\begin{array}{cc}
3 & -2 \\
2 & 1 
\end{array}
\right| = 7 \ne 0,\\[0.4cm]
D_1 = \left|
\begin{array}{cc}
12 & -2 \\
1 & 1 
\end{array}
\right| = 14 , \\[0.4cm]
D_2 = \left|
\begin{array}{cc}
3 & 12 \\
2 & 1 
\end{array}
\right| = -21,
\end{array}
$$
因此,
$$
x_1=\frac{D_1}{D}=2, \ \ x_2 = \frac{D_2}{D} = -3.
$$
\end{jie}

\end{frame}


\subsection{三阶行列式}
\begin{frame}

\begin{dingyi}[三阶行列式]
  由$3^2=9$个数组成的3行3列的三阶行列式,则按如下形式定义一个数
  $$
  \begin{aligned}
    D_3 &= 
    \left|
      \begin{array}{ccc}
        a_{11} & a_{12} & a_{13} \\[0.2cm]
        a_{21} & a_{22} & a_{23} \\[0.2cm]
        a_{31} & a_{32} & a_{33} 
      \end{array}
    \right|
    \\[.1in]
    &=  a_{11}a_{22}a_{33} + a_{12}a_{23}a_{31} + a_{13}a_{21}a_{32} \\[.1in]
    & \qquad - a_{13}a_{22}a_{31} - a_{11}a_{23}a_{32} - a_{12}a_{21}a_{33}
  \end{aligned}
  $$
\end{dingyi} \pause 
%
\begin{figure}[htbp]
  \centering
  \begin{tikzpicture}               
    \matrix(A) [matrix of math nodes,nodes in empty cells,ampersand replacement=\&,row sep=15pt,column sep=15pt] {
      \&  a_{11} \& a_{12} \& a_{13}  \& \red{a_{11}} \& \red{a_{12}} \&\\
      \&  a_{21} \& a_{22} \& a_{23}  \& \red{a_{21}} \& \red{a_{22}} \&\\
      \&  a_{31} \& a_{32} \& a_{33}  \& \red{a_{31}} \& \red{a_{32}} \&\\
      -   \&    -        \&   -        \&               \&   +         \&   +        \& +\\
    }; \pause
    \draw[blue,thick] (A-1-2.center) -- (A-4-5.center); 
    \draw[blue,thick] (A-1-3.center)  -- (A-4-6.center); 
    \draw[blue,thick] (A-1-4.center) -- (A-4-7.center); \pause
    \draw[red,thick,->] (A-1-6.center) -- (A-4-3.center); 
    \draw[red,thick,->] (A-1-5.center) -- (A-4-2.center);
    \draw[red,thick,->] (A-1-4.center) -- (A-4-1.center); 
  \end{tikzpicture}
  \caption{沙路法}
\end{figure}
\end{frame}

\begin{frame}
\begin{li}
  计算
  $$
  D_3 = 
  \left |
    \begin{array}{rrr}
      1  & 2 & -4 \\ 
      -2 & 2 & 1  \\
      -3 & 4 & -2
    \end{array}
  \right|
  $$
\end{li} \pause 

\begin{jie}
  由沙路法可知,
  $$
  \begin{array}{ll}
    D_3 &=   1\times   2  \times (-2) +   2  \times 1 \times (-3) + (-2) \times 4 \times (-4)\\[0.2cm]
        & - 2\times (-2) \times (-2) - (-4) \times 2 \times (-3) +   1  \times 1 \times   4\\[0.2cm]
        & = -14.
  \end{array}
  $$
\end{jie}
\end{frame}

\begin{frame}
\begin{li}
  求方程
  $$
  \left |
    \begin{array}{ccc}
      1  & 1 & 1 \\
      2  & 3 & x  \\
      4  & 9 & x^2
    \end{array}
  \right| = 0
  $$        
\end{li} \pause 
\begin{jie}
  行列式
  $$ 
  D = 3x^2 + 18 + 4x - 2x^2 - 12 - 9x 
  = x^2 - 5x + 6
  $$
  由此可知$x=2$或$3$。
\end{jie}
%
\end{frame}

\begin{frame}
如果三元一次方程组
$$
\begin{array}{c}  
  a_{11}x_1 + a_{12}x_2 + a_{13}x_3 = b_1, \\
  a_{21}x_1 + a_{22}x_2 + a_{23}x_3 = b_2, \\
  a_{31}x_1 + a_{32}x_2 + a_{33}x_3 = b_3,
\end{array}
$$
的系数行列式
$$
D = \left|
  \begin{array}{ccc}
    a_{11} & a_{12} & a_{13}\\
    a_{21} & a_{22} & a_{23}\\
    a_{31} & a_{32} & a_{33}
  \end{array}
\right| \ne 0
$$
则用消元法求解可得
$$
x_1 = \frac{D_1}{D}, \quad
x_2 = \frac{D_2}{D}, \quad
x_3 = \frac{D_3}{D}, \quad
$$
其中
$$
D_1 = \left|
  \begin{array}{ccc}
    b_1 & a_{12} & a_{13}\\
    b_2 & a_{22} & a_{23}\\
    b_3 & a_{32} & a_{33}
  \end{array}
\right|, \
D_2 = \left|
  \begin{array}{ccc}
    a_{11} & b_1 & a_{13}\\
    a_{21} & b_2 & a_{23}\\
    a_{31} & b_3 & a_{33}
  \end{array}
\right|, \
D_3= \left|
  \begin{array}{ccc}
    a_{11} & a_{12} & b_1 \\
    a_{21} & a_{22} & b_2 \\
    a_{31} & a_{32} & b_3 
  \end{array}
\right|.
$$
\end{frame}

\begin{frame}
从二、三阶行列式的展开式中可发现:
$$
\begin{array}{l}
  D  =  \left|
  \begin{array}{ccc}
    a_{11} & a_{12} & a_{13}\\
    a_{21} & a_{22} & a_{23}\\
    a_{31} & a_{32} & a_{33}
  \end{array}
                      \right| \\[0.6cm]
  = 
  a_{11}a_{22}a_{33}+a_{12}a_{23}a_{31}+a_{13}a_{21}a_{32} \\[0.3cm] 
  \qquad -a_{11}a_{23}a_{32}-a_{12}a_{21}a_{33}-a_{13}a_{22}a_{31} \\[0.3cm] 
  = 
  a_{11}(a_{22}a_{33}-a_{23}a_{32})-
  a_{12}(a_{21}a_{33}-a_{23}a_{31})+
  a_{13}(a_{21}a_{32}-a_{22}a_{31}) \\[0.3cm] 
  = 
  a_{11} \underbrace{\left| \begin{array}{ccc} a_{22} & a_{33} \\ a_{23} & a_{32} \end{array} \right|}_{M_{11}} -
                                                                           a_{12} \underbrace{\left| \begin{array}{ccc} a_{21} & a_{23} \\ a_{31} & a_{33} \end{array} \right|}_{M_{12}} +
                                                                                                                                                    a_{13} \underbrace{\left| \begin{array}{ccc} a_{21} & a_{22} \\ a_{31} & a_{32} \end{array} \right|}_{M_{13}}
\end{array}
$$
这里,$M_{11},M_{12},M_{13}$分别称为$a_{11},a_{12},a_{13}$的\red{余子式},并称
$$
A_{11} = (-1)^{1+1} M_{11}, \quad
A_{12} = (-1)^{1+2} M_{12}, \quad
A_{13} = (-1)^{1+3} M_{13}
$$
分别称为$a_{11},a_{12},a_{13}$的\red{代数余子式}。 于是,$D$可表示为
$$
D= a_{11}A_{11} + a_{11}A_{13} + a_{13}A_{13}.
$$
\end{frame}

\begin{frame}



同样地,
$$
  \left| \begin{array}{ccc} a_{11} & a_{12} \\ a_{21} & a_{22} \end{array} \right|
= a_{11} A_{11} + a_{12} A_{12},
$$
其中
$$
A_{11} = (-1)^{1+1}|a_{22}| =  a_{22},\quad
A_{11} = (-1)^{1+2}|a_{21}| = -a_{21}.
$$
注意这里的$|a_{22}|,~|a_{21}|$是一阶行列式,而不是绝对值。
我们把一阶行列式$|a|$定义为$a$。
%
\end{frame}


\subsection{$n$阶行列式的定义}
\begin{frame}
\begin{dingyi}[$n$阶行列式]
  由$n^2$个数$a_{ij}(i,j=1,2,\cdots,n)$组成的$n$阶行列式
  \begin{equation}\label{Dn}
    D = \left|
      \begin{array}{cccc}
        a_{11}  &  a_{12} & \cdots & a_{1n} \\
        a_{21}  &  a_{22} & \cdots & a_{2n} \\
        \vdots & \vdots & \ddots & \vdots\\  
        a_{n1}  &  a_{n2} & \cdots & a_{nn} 
      \end{array}
    \right|
  \end{equation}
  是一个数。
  \begin{itemize}
  \item 当$n=1$时,定义$D=|a_{11}|=a_{11}$; 
  \item 当$n\ge2$时,定义
    \begin{equation}
      D = a_{11} A_{11} + a_{12} A_{12} + \cdots + a_{1n} A_{1n},
    \end{equation}
    其中
    $$A_{1j} = (-1)^{1+j} M_{1j}$$
    而$M_{1j}$是$D$中划去第$1$行第$j$列后,按原顺序排成的$n-1$阶行列式,即
    $$
    M_{1j} =   \left|
      \begin{array}{cccccc}
        a_{21}  & \cdots&  a_{2,j-1}  &  a_{2,j+1}  & \cdots & a_{2n} \\
        a_{31}  & \cdots&  a_{3,j-1}  &  a_{3,j+1}  & \cdots & a_{3n} \\
        \vdots &       &  \vdots &  \vdots &  & \vdots\\  
        a_{n1}  & \cdots&  a_{n,j-1}  &  a_{n,j+1}  & \cdots & a_{nn} 
      \end{array}
    \right| \quad (j = 1,2,\cdots, n),
    $$
    并称$M_{1j}$为$a_{1j}$的\red{余子式},$A_{1j}$为$a_{1j}$的\red{代数余子式}.
  \end{itemize}
\end{dingyi}
\end{frame}

\begin{frame}
\begin{zhu}
  需注意以下两点:
  \begin{enumerate}
  \item[1]
    在$D$中,$a_{11},a_{22},\cdots,a_{nn}$所在的对角线称为行列式的\red{主对角线},$a_{11},a_{22},\cdots,a_{nn}$称为\red{主对角元}。\\[.1in]
  \item[2]
    行列式$D$是由$n^2$个元素构成的$n$次齐次多项式:\\[.1in]
    \begin{itemize}
    \item 二阶行列式的展开式有$2!$项;\\[.1in]
    \item 三阶行列式的展开式有$3!$项;\\[.1in]
    \item $n$阶行列式的展开式有$n!$项,其中每一项都是不同行不同列的$n$个元素的乘积,带正号的项与带负号的项各占一半。
    \end{itemize}      
  \end{enumerate}    
\end{zhu}
\end{frame}

\begin{frame}
由行列式的定义可知,一个$n$阶行列式可以展开成$n$个$n$阶行列式之和,即
$$
\begin{aligned}
  \left|
    \begin{array}{cccc}
      a_{11}  &  a_{12} & \cdots & a_{1n} \\
      a_{21}  &  a_{22} & \cdots & a_{2n} \\
      \vdots & \vdots & \ddots & \vdots\\  
      a_{n1}  &  a_{n2} & \cdots & a_{nn} 
    \end{array}
  \right| &= 
  \left|
    \begin{array}{cccc}
      a_{11}  &  0 & \cdots & 0 \\
      0  &  a_{22} & \cdots & a_{2n} \\
      \vdots & \vdots & \ddots & \vdots\\  
      0  &  a_{n2} & \cdots & a_{nn} 
    \end{array}
  \right|  \\[0.1in]
  &+ \left|
    \begin{array}{ccccc}
      0  &  a_{12} & 0 & \cdots & 0 \\
      a_{21} & 0  &  a_{23} & \cdots & a_{2n} \\
      \vdots & \vdots & \vdots & \ddots & \vdots\\  
      a_{n1}  & 0&  a_{n3} & \cdots & a_{nn} 
    \end{array}
  \right| \\[0.1in]
  &+ \cdots  \\[0.1in]
  &+ 
  \left|
    \begin{array}{cccc}
      0 & \cdots & 0 & a_{1n} \\
      a_{21}  &   \cdots & a_{2,n-1} & 0 \\
      \vdots &  \ddots & \vdots & \vdots\\  
      a_{n1}  &   \cdots & a_{n,n-1} & 0
    \end{array}
  \right| 
\end{aligned}
$$
%
\end{frame}

\begin{frame}
\begin{li}
  证明:$n$阶下三角行列式
  $$
  D_n = \left|
    \begin{array}{cccc}
      a_{11}  &  0 & \cdots & 0 \\
      a_{21}  &  a_{22} & \cdots & 0 \\
      \vdots & \vdots & \ddots & \vdots\\  
      a_{n1}  &  a_{n2} & \cdots & a_{nn} 
    \end{array}
  \right| = a_{11} a_{22} \cdots a_{nn}.
  $$  
\end{li}
\end{frame}

\begin{frame}
\begin{proof}
  用数学归纳法证明。
  \begin{enumerate}
  \item 当$n=2$时,结论成立。 \pause 
  \item 假设结论对$n-1$阶下三角阵成立,则由定义
    $$
    \begin{array}{rcl}
      \left|
      \begin{array}{cccc}
        a_{11}  &  0 & \cdots & 0 \\
        a_{21}  &  a_{22} & \cdots & 0 \\
        \vdots & \vdots & \ddots & \vdots\\  
        a_{n1}  &  a_{n2} & \cdots & a_{nn} 
      \end{array}
                                     \right| &=&  a_{11} \cdot (-1)^{1+1} \left|
                                                 \begin{array}{cccc}
                                                   a_{22}  &  0 & \cdots & 0 \\
                                                   a_{31}  &  a_{33} & \cdots & 0 \\
                                                   \vdots & \vdots & \ddots & \vdots\\  
                                                   a_{n1}  &  a_{n2} & \cdots & a_{nn} 
                                                 \end{array}
                                                                                \right| \\[0.4in]
                &=&  a_{11} (a_{22} a_{33} \cdots a_{nn}).       
    \end{array}
    $$    
  \end{enumerate}
  综上所述,结论成立。
\end{proof}
\end{frame}

\begin{frame}
同理可证
$$
\left|
\begin{array}{cccc}
a_{11}  &  a_{12} & \cdots & a_{1n} \\
0  &  a_{22} & \cdots & a_{2n} \\
\vdots & \vdots & \ddots & \vdots\\  
0  &  0 & \cdots & a_{nn} 
\end{array}
\right| = a_{11}a_{22}\cdots a_{nn}
$$
和
$$
\left|
  \begin{array}{cccc}
    a_{11}  &  0 & \cdots & 0 \\
    0  &  a_{22} & \cdots & 0 \\
    \vdots & \vdots & \ddots & \vdots\\  
    0  &  0 & \cdots & a_{nn} 
  \end{array}
\right| = a_{11}a_{22}\cdots a_{nn}
$$
\end{frame}

\begin{frame}
\begin{li}
  计算$n$阶行列式
  $$
  \left|
    \begin{array}{ccccc}
      0  &  0 & \cdots & 0 & a_n \\
      0  &  0 & \cdots & a_{n-1} & * \\
      \vdots & \vdots & & \vdots & \vdots\\  
      0  &  a_2 & \cdots & * & * \\
      a_1 & * & \cdots & * & *
    \end{array}
  \right| 
  $$
\end{li}
\end{frame}

\begin{frame}
\begin{jie}
  由行列式定义,
  $$
  \begin{array}{rcl}
    D_n &=& \left|
            \begin{array}{ccccc}
              0  &  0 & \cdots & 0 & a_n \\
              0  &  0 & \cdots & a_{n-1} & * \\
              \vdots & \vdots & & \vdots & \vdots\\  
              0  &  a_2 & \cdots & * & * \\
              a_1 & * & \cdots & * & *
            \end{array}
                                     \right| =   (-1)^{1+n} a_n \left|
                                     \begin{array}{cccc}
                                       0  &  0 & \cdots &   a_{n-1} \\
                                       \vdots & \vdots &  & \vdots\\  
                                       0  &  a_2 & \cdots  & * \\
                                       a_1 & * & \cdots  & *
                                     \end{array} 
                                                           \right| \\[0.5in]
        &=&  (-1)^{n-1} a_n D_{n-1}.      
  \end{array}
  $$ \pause 
  同理递推,
  $$
  \begin{array}{rcl}
    D_n & =& (-1)^{n-1} a_n D_{n-1}  = (-1)^{n-1} a_n (-1)^{n-2} a_{n-1} D_{n-2} \\[0.1in]
        && \cdots\cdots \\[0.2cm]
        &=&   (-1)^{(n-1)+(n-2)+\cdots+2+1} a_n a_{n-1} \cdots a_2 a_1 \\[0.1in]
        &=&  (-1)^{\frac{n(n-1)}{2}} a_n a_{n-1} \cdots a_2 a_1.
  \end{array}
  $$
\end{jie} \pause 

  例如,
  $$
  D_2 = -a_1a_2, \quad
  D_3 = -a_1a_2a_3, \quad
  D_4 = a_1a_2a_3a_4, \quad
  D_5 = a_1a_2a_3a_4a_5.
  $$

\end{frame}