\section{关系运算符和表达式}

\begin{frame}[fragile]\ft{\secname}
\begin{table}
\centering
\caption{关系运算符}
\begin{tabular}{p{2cm}|p{3cm}}\hline
运算符& 含义\\\hline\hline
\lstinline|<|  & 小于\\[0.1in]
\lstinline|<=| & 小于或等于\\[0.1in]
\lstinline|>|  & 大于\\[0.1in]
\lstinline|>=| & 大于或等于\\[0.1in]
\lstinline|==| & 等于\\[0.1in]
\lstinline|!=| & 不等于\\\hline
\end{tabular}
\end{table}
\end{frame}

\begin{frame}[fragile]\ft{\secname}
关系运算符用来构成while语句和其它C语句中使用的关系表达式,这些语句检查表达式是真还是假。
\end{frame}

\begin{frame}[fragile]\ft{\secname}
\begin{lstlisting}
while(number < 6) {
  printf("Your number is too small.\n");
  scanf("%d", &number);
}
\end{lstlisting}
\begin{lstlisting}
while(ch != '*') {
  count++;
  scanf("%c", &ch);  
}
\end{lstlisting}

\begin{lstlisting}
while(scanf("%f", &num) == 1)
  sum = sum + num;
\end{lstlisting}

\end{frame}

\begin{frame}[fragile]\ft{\secname}
请注意,不能使用关系运算符来比较字符串。
\end{frame}

\begin{frame}[fragile]\ft{\secname}
关系运算符可用于浮点数。但要小心,在浮点数比较时只能使用 \lstinline|<| 和 \lstinline|>|,原因在于舍入误差可能造成两个逻辑上应该相等的数不相等。\vspace{0.1in}

比如,虽然从数学上看
$$
3*\frac13==1.0
$$
但若用6位小数表示 \lstinline|1/3|,其乘积为 \lstinline|0.999 999|。
\end{frame}

\begin{frame}[fragile]\ft{\secname}
请使用头文件 \lstinline|math.h| 中的 \lstinline|fabs| 函数来进行浮点数的判断,该函数返回一个浮点值的绝对值。
\end{frame}

\begin{frame}[fragile,allowframebreaks]\ft{\secname}
\lstinputlisting[numbers=left]{ch06/code/cmpflt.c}
\end{frame}

\begin{frame}[fragile]\ft{\secname}
\begin{lstlisting}
What is the value of pi?
3.14
Try again!
3.1416
Close enough!
\end{lstlisting}
\end{frame}

\begin{frame}[fragile]\ft{\secname:什么是真?}
\lstinputlisting[numbers=left]{ch06/code/t_and_f.c}
\pause

\begin{lstlisting}
true = 1; false = 0
\end{lstlisting}
\end{frame}

\begin{frame}[fragile]\ft{\secname:什么是真?}
对C来说,一个真表达式的值为1,而一个假表达式的值为0。
\end{frame}

\begin{frame}[fragile]\ft{\secname:什么是真?}

\begin{lstlisting}
while (1) {
  ...
}
\end{lstlisting}
\pause\vspace{.1in}

死循环
\end{frame}

\begin{frame}[fragile]\ft{\secname:还有什么是真?}
\begin{wenti}
既然可以使用1或0来作为while语句的判断表达式,那么还可以使用其他数字吗?
\end{wenti}
\end{frame}

\begin{frame}[fragile,allowframebreaks]\ft{\secname:还有什么是真?}
  \lstinputlisting[numbers=left]{ch06/code/truth.c}    
\end{frame}

\begin{frame}[fragile]\ft{\secname:还有什么是真?}
  \begin{lstlisting}
 3 is true
 2 is true
 1 is true
 0 is false
-3 is true
-2 is true
-1 is true
 0 is false
\end{lstlisting}    

\end{frame}

\begin{frame}[fragile]\ft{\secname:还有什么是真?}

\red{对C来说,所有非零值都被认为是真,只有0被认为是假。}
\end{frame}

\begin{frame}[fragile]\ft{\secname:还有什么是真?}
基于上述认识可知,如下两种方式等价: \vspace{.1in}

\begin{minipage}{.45\textwidth}
\begin{lstlisting}[language=c,backgroundcolor=\color{red!10}]
while (n != 0) {
  ...
}
\end{lstlisting}
\end{minipage}\hfill 
\begin{minipage}{.45\textwidth}
\begin{lstlisting}[language=c,backgroundcolor=\color{red!10}]
while (n) {
  ...
}
\end{lstlisting}
\end{minipage}
\end{frame}


\begin{frame}[fragile,allowframebreaks]\ft{\secname:真值的问题}
\lstinputlisting[numbers=left]{ch06/code/trouble.c}
\end{frame}

\begin{frame}[fragile]\ft{\secname:真值的问题}
\begin{lstlisting}
Enter an integer to be summed (q to quit): 
1
Enter next integer (q to quit): 
2
Enter next integer (q to quit): 
3
Enter next integer (q to quit): 
q
Enter next integer (q to quit): 
Enter next integer (q to quit): 
Enter next integer (q to quit):
... 
\end{lstlisting}
\end{frame}

\begin{frame}[fragile]\ft{\secname:真值的问题}
% 该例改变了while的判断条件,用 \lstinline|status=1| 代替了 \lstinline|status==1|。而\red{赋值表达式的值就是其左侧的值},故 \lstinline|status=1| 的值为1。
实际上,
\begin{lstlisting}
while (status = 1) 
\end{lstlisting}
等价于
\begin{lstlisting}
while (1) 
\end{lstlisting}
进入死循环。
\end{frame}

\begin{frame}[fragile]\ft{\secname:真值的问题}
\begin{itemize}
\item
当你输入q后,根本没有机会进行更多的输入。\\[0.1in]
\item
当 \lstinline|scanf| 未能读取指定形式的输入时,它就留下这个不相容的输入,以供下次进行读取。\\[0.1in]
\item
当 \lstinline|scanf| 试着把q当做整数读取并失败时,它就把q留在那里。下次循环继续读取这个q,\lstinline|scanf| 再次失败。\\[0.1in]
\item
该例不但建立了一个无限循环,更建立了一个无限失败的循环。
\end{itemize}
\end{frame}


\begin{frame}[fragile]\ft{\secname:真值的问题}
\begin{itemize}
\item
不要在应该使用 \lstinline|==| 的地方使用 \lstinline|=|。\\[0.1in]
\item
  赋值运算符 \lstinline|=| 把一个值赋给左边的变量,而关系运算符 \lstinline|==| 检查左右两边的值是否相等,它并不改变左边变量的值。
\begin{lstlisting}
i = 5   // `把i赋值为5`
i == 5  // `检查i的值是否为5`
\end{lstlisting}
\item
  在使用 \lstinline|==| 时,若比较双方有一个是常量,可以把它放在左侧,以便于发现错误。
\begin{lstlisting}
5 = i   // `语法错误`
5 == i  // `检查i的值是否为5`
\end{lstlisting}
\end{itemize}
\end{frame}


\begin{frame}[fragile]\ft{\secname:真值小结}
\begin{itemize}
\item 关系运算符用于构成关系表达式。关系表达式为真时值为1,为假时值为0。\\[0.1in]
\item 对于使用关系表达式作为判断条件的语句(while和if),可以使用任何表达式作为判断,非零值被认为是真,而零值被认为是假。
\end{itemize}
\end{frame}


\begin{frame}[fragile]\ft{\secname:\lstinline|_Bool|类型}
\begin{itemize}
\item 表示真/假的变量被称为布尔变量,在C中一直由int类型来表示。\\[0.1in]
\item C99添加了\lstinline|_Bool|类型,是布尔变量的类型名。
\\[0.1in]
\item 一个布尔变量只可以具有值1或0。若把一个布尔变量赋为非零数值,它就被设置为1。
  这说明C把任何非零值都认为是真。
\end{itemize}

\end{frame}


\begin{frame}[fragile,allowframebreaks]\ft{\secname:\lstinline|_Bool|类型}
\lstinputlisting[numbers=left]{ch06/code/boolean.c}
\end{frame}


\begin{frame}[fragile]\ft{\secname:\lstinline|_Bool|类型}
\begin{lstlisting}
  input_is_good = (scanf("%ld", &num) == 1);
\end{lstlisting}

\begin{itemize}
\item 该代码将比较的结果赋给布尔变量。\\[0.1in]
\item 外层括号不是必需的,因为 \lstinline|==| 的优先级比 \lstinline|=| 要高,但这样写可使代码更容易阅读。
\\[0.1in]
\item 还请注意布尔变量的命名方式:
\begin{lstlisting}
  while (input_is_good)
\end{lstlisting}
\end{itemize}

\end{frame}

\begin{frame}[fragile]\ft{\secname:\lstinline|_Bool|类型}
  C99还提供了一个头文件 \lstinline|stdbool.h|。
  使用它可以用\lstinline|bool|代替\lstinline|_Bool|,并把 \lstinline|true| 和 \lstinline|false|定义成值为1和0的常量。
\end{frame}

\begin{frame}[fragile,allowframebreaks]\ft{\secname:\lstinline|_Bool|类型}
\lstinputlisting[numbers=left]{ch06/code/boolean1.c}
\end{frame}


\begin{frame}[fragile]\ft{\secname:关系运算符的优先级}
关系运算符的优先级低于包括 \lstinline|+| 和 \lstinline|-| 在内的算术运算符,但要高于赋值运算符。 \pause \vspace{.15in}

\begin{minipage}{.4\textwidth}
\begin{lstlisting}
x > y+2
\end{lstlisting}
\end{minipage}$~~~\blue\Longleftrightarrow~~~$
\begin{minipage}{.4\textwidth}
\begin{lstlisting}
x > (y+2)
\end{lstlisting}
\end{minipage} \vspace{.1in}

\begin{minipage}{.4\textwidth}
\begin{lstlisting}
x = y > 2
\end{lstlisting}
\end{minipage}$~~~\blue\Longleftrightarrow~~~$
\begin{minipage}{.4\textwidth}
\begin{lstlisting}
x = (y > 2)
\end{lstlisting}
\end{minipage}\vspace{.1in}


\begin{minipage}{.4\textwidth}
\begin{lstlisting}
x_bigger = x > y
\end{lstlisting}
\end{minipage}$~~~\blue\Longleftrightarrow~~~$
\begin{minipage}{.4\textwidth}
\begin{lstlisting}
x_bigger = (x > y)
\end{lstlisting}
\end{minipage}
\end{frame}

\begin{frame}[fragile]\ft{\secname:关系运算符的优先级}
\begin{table}
\centering
\caption{关系运算符本身有两组不同的优先级}
\begin{tabular}{p{2cm}|p{4cm}}\hline
高优先级 & \lstinline| <   <=   >   >=|\\[0.1in]
低优先级 & \lstinline| ==   !=|\\\hline
\end{tabular}
\end{table}

\begin{center}
结合规则:从左到右。
\end{center}
\end{frame}


\begin{frame}[fragile]\ft{\secname:关系运算符的优先级}
 

\begin{minipage}{.4\textwidth}
\begin{lstlisting}
c != a == b
\end{lstlisting}
\end{minipage}$~~~\blue\Longleftrightarrow~~~$
\begin{minipage}{.4\textwidth}
\begin{lstlisting}
(c != a) == b
\end{lstlisting}
\end{minipage}

\end{frame}


\begin{frame}[fragile]\ft{\secname:运算符的优先级}
\begin{table}
\centering
%\caption{运算符优先级}
\begin{tabular}{p{7cm}|p{2cm}}\hline
运算符(从高到低) & 结合性 \\\hline\hline
\lstinline| ( )| & 从左到右\\[0.1in]
\lstinline| -   +   ++   --   sizeof   (type)|
& 从右到左\\[0.1in]
\lstinline| *   /   %| & 从左到右 \\[0.1in]
\lstinline| +   -| & 从左到右 \\[0.1in]
\lstinline| <   <=   >   >=| & 从左到右\\[0.1in]
\lstinline| ==   !=| & 从左到右\\[0.1in]
\lstinline| = |& 从右到左\\\hline
\end{tabular}
\end{table}

\end{frame}
