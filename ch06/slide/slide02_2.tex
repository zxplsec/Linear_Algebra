\subsection{配方法}


\begin{frame}
  
    \begin{li}
      用配方法把三元二次型
      $$
      f(x_1,x_2,x_3)=2x_1^2+3x_2^2+x_3^2+4x_1x_2-4x_1x_3-8x_2x_3
      $$
      化为标准型。
    \end{li}
    \pause
    \begin{jie}
    先按$x_1^2$和含$x_1$的混合项配成完全平方,即
    $$
    \begin{array}{rl}
      f(x_1,x_2,x_3)&=2[x_1^2+2x_1(x_2-x_3)+(x_2-x_3)^2]-2(x_2-x_3)^2+3x_2^2+x_3^2-8x_2x_3\\[0.1in]
      &=2(x_1+x_2-x_3)^2+x_2^2-x_3^2-4x_2x_3
    \end{array}
    $$\pause
    再按$x_2^2-4x_2x_3$配成完全平方,得
    $$
    f(x_1,x_2,x_3)=2(x_1+x_2-x_3)^2+(x_2-2x_3)^2-5x_3^2.
    $$
    \pause
    令
    $$
    \left\{
    \begin{array}{rcrcrcr}
      y_1&=&x_1&+&x_2&-&x_3\\
      y_2&=& &&x_2&-&2x_3\\
      y_3&=&&&&&x_3
    \end{array}
    \right. \pause ~~\Longrightarrow~~
    \left(
    \begin{array}{c}
      x_1\\
      x_2\\
      x_3
    \end{array}
    \right)=
    \left(
    \begin{array}{rrr}
      1&-1&-1\\
      0&1&2\\
      0&0&1
    \end{array}
    \right)
    \left(
    \begin{array}{c}
      y_1\\
      y_2\\
      y_3
    \end{array}
    \right)
    $$ \pause 
    则
    $$
    f(x_1,x_2,x_3)=2y_1^2+y_2^2-5y_3^2.
    $$
  \end{jie}
\end{frame}


\begin{frame}
  
    \begin{li}
      用配方法化二次型
      $$
      f(x_1,x_2,x_3)=2x_1x_2+4x_1x_3
      $$
      为标准型。
    \end{li}
    \pause
    \begin{jie}
    对$x_1x_2$利用平方差公式,令
    $$
    \left\{
    \begin{array}{l}
      x_1=y_1+y_2\\
      x_2=y_1-y_2\\
      x_3=y_3
    \end{array}
    \right.
    $$
    则
    $$
    f(x_1,x_2,x_3)=2(y_1+y_2)(y_1-y_2)+4(y_1+y_2)y_3=2y_1^2-2y_2^2+4y_1y_3+4y_2y_3
    $$
    \pause
    先对含$y_1$的项配完全平方,得
    $$
    f(x_1,x_2,x_3)=2(y_1^2+2y_1y_3+y_3^2)-2y_2^2-2y_3^2+4y_2y_3
    $$
    再对含$y_2$的项配完全平方,得
    $$
    f(x_1,x_2,x_3)=2(y_1+y_3)^2-2(y_2-y_3)^2
    $$
  \end{jie}
\end{frame}

\begin{frame}
  
    令
    $$
    \left\{
    \begin{array}{l}
      z_1=y_1+y_3\\
      z_2=y_2-y_3\\
      z_3=y_3
    \end{array}
    \right. ~~\Longleftrightarrow~~
    \left\{
    \begin{array}{l}
      y_1=z_1-z_3\\
      y_2=z_2+z_3\\
      y_3=z_3
    \end{array}
    \right.
    $$
    则
    $$
    f(x_1,x_2,x_3)=2z_1^2-2z_2^2.
    $$\pause
    坐标变换记为
    $$
    \vx=\MC_1\vy, \quad  \vy=\MC_2\vz, \quad \vx=\MC_1\MC_2\vz=\MC\vz
    $$
    其中
    $$
    \begin{array}{c}
      \MC_1=\left(
      \begin{array}{rrr}
        1&1&0\\
        1&-1&0\\
        0&0&1
      \end{array}
      \right),
      \quad\MC_2=\left(
      \begin{array}{rrr}
        1&0&-1\\
        0&1&1\\
        0&0&1
      \end{array}
      \right)
      \\[0.4in]
      \MC=\MC_1\MC_2=\left(
      \begin{array}{rrr}
        1&1&0\\
        1&-1&-2\\
        0&0&1
      \end{array}
      \right)      
    \end{array}
    $$
  
\end{frame}

\begin{frame}
  
    \begin{table}
      \caption{}
      \begin{tabular}{|c|c|}\hline
        二次型&对应矩阵\\\hline
        $2x_1x_2+4x_1x_3$ & $\MA=\left(
        \begin{array}{ccc}
          0&1&2\\
          1&0&0\\
          2&0&0
        \end{array}
        \right)$\\\hline
        $2z_1^2-2z_2^2$ & $\Lambdabd=\left(
        \begin{array}{ccc}
          2&&\\
          &-2&\\
          &&0
        \end{array}
        \right)$ \\\hline
      \end{tabular}      
    \end{table}
    易验证
    $$
    \MC^T\MA\MC=\diag(2,-2,0)    
    $$
  
\end{frame}

\begin{frame}
  
    任何$n$元二次型都可用配方法化为标准型,相应的变换矩阵为主对角元为1的上三角阵和对角块矩阵,或者是这两类矩阵的乘积。
  
\end{frame}

