\subsection{初等变换法}

\begin{frame}
  
    对于实对称矩阵,可通过一系列相同类型的初等行、列变换将其化为合同标准型。
    所谓相同类型的初等行、列变换,指的是
    \begin{itemize}
    \item [(1)]
      $$
      \MA ~~\xlongrightarrow[]{\ds c_i+kc_j} ~~ \MA\ME_{ij}(k)
      ~~\xlongrightarrow[]{\ds r_i+kr_j} ~~ \ME_{ij}(k)^T\MA\ME_{ij}(k)
      $$
      $\ME_{ij}(k)^T\MA\ME_{ij}(k)$仍为对称矩阵。
    \item [(2)]
      $$
      \MA ~~\xlongrightarrow[]{\ds c_i\times k} ~~ \MA\ME_{i}(k)
      ~~\xlongrightarrow[]{\ds r_i\times k} ~~ \ME_{i}(k)^T\MA\ME_{i}(k)
      $$
      $\ME_{i}(k)^T\MA\ME_{i}(k)$仍为对称矩阵。
    \item [(2)]
      $$
      \MA ~~\xlongrightarrow[]{\ds c_i \leftrightarrow c_j} ~~ \MA\ME_{ij}
      ~~\xlongrightarrow[]{\ds r_i\leftrightarrow  r_j} ~~ \ME_{ij}^T\MA\ME_{ij}
      $$
      $\ME_{ij}(k)^T\MA\ME_{ij}(k)$仍为对称矩阵。
    \end{itemize}
  
\end{frame}

\begin{frame}
  
    \begin{dingli}
      对于任一个$n$阶实对称矩阵$\MA$,都存在可逆矩阵$\MC$,使得
      $$
      \MC^T\MA\MC=\diag(d_1,d_2,\cd,d_n)
      $$
    \end{dingli}

    $$
    \MC=\MP_1\MP_2\cd\MP_k=\MI\MP_1\MP_2\cd\MP_k
    $$
    这说明,将施加于$\MA$的列变换同时施加于单位阵$\MI$,当$\MA$变为对角阵时,$\MI$就变为变换矩阵$\MC$。
  
\end{frame}

\begin{frame}
  
    \begin{li}
      用初等变换法将二次型
      $$
      f(x_1,x_2,x_3)=(x_1,x_2,x_3)\left(
      \begin{array}{rrr}
        2&2&-2\\
        2&5&-4\\
        -2&-4&5
      \end{array}
      \right)\left(
      \begin{array}{c}
        x_1\\x_2\\x_3
      \end{array}
      \right)
      $$
      化为标准型,并求坐标变换$\vx=\MC\vy$。
    \end{li}
  
\end{frame}

\begin{frame}
  \begin{jie}
    $$
    \begin{array}{rl}
      \left(
      \begin{array}{c}
        \MA\\
        \red{\MI}
      \end{array}
      \right)&=\left(
      \begin{array}{rrr}
        2&2&-2\\
        2&5&-4\\
        -2&-4&5\\
        \red{1}&\red{0}&\red{0}\\
        \red{0}&\red{1}&\red{0}\\
        \red{0}&\red{0}&\red{1}
      \end{array}
      \right)\xlongrightarrow[\ds c_2-c_1\atop \ds c_3+c_1]{\ds c_2-c_1\atop \ds c_3+c_1}
      \left(
      \begin{array}{rrr}
        2&0&0\\
        2&3&-2\\
        -2&-2&3\\
        \red{1}&\red{-1}&\red{1}\\
        \red{0}&\red{1}&\red{0}\\
        \red{0}&\red{0}&\red{1}
      \end{array}
      \right)\\[0.5in]
      &\xlongrightarrow[]{\ds r_2-r_1\atop \ds r_3+r_1}
      \left(
      \begin{array}{rrr}
        2&0&0\\
        0&3&-2\\
        0&-2&3\\
        \red{1}&\red{-1}&\red{1}\\
        \red{0}&\red{1}&\red{0}\\
        \red{0}&\red{0}&\red{1}
      \end{array}
      \right)\xlongrightarrow[\ds r_3+\frac23r_2]{\ds c_3+\frac23c_2}
      \left(
      \begin{array}{rrr}
        2&0&0\\
        0&3&0\\
        0&-2&\ds5/3\\
        \red{1}&\red{-1}&\red{1/3}\\
        \red{0}&\red{1}&\red{2/3}\\
        \red{0}&\red{0}&\red{1}
      \end{array}
      \right)\\[0.5in]
      &\xlongrightarrow[]{\ds r_3+\frac23r_2}
      \left(
      \begin{array}{rrr}
        2&0&0\\
        0&3&0\\
        0&0&\ds5/3\\
        \red{1}&\red{-1}&\red{1/3}\\
        \red{0}&\red{1}&\red{2/3}\\
        \red{0}&\red{0}&\red{1}
      \end{array}
      \right)=\left(
      \begin{array}{c}
        \Lambdabd\\
        \red{\MC}
      \end{array}
      \right)
    \end{array}
    $$
  \end{jie}
\end{frame}
