\section{正定二次型和正定矩阵}

\begin{frame}  
    \begin{dingyi}
      如果对于任意的非零向量$\vx=(x_1,x_2,\cd,x_n)^T$,恒有
      $$
      \vx^T\MA\vx=\sum_{i=1}^n\sum_{j=1}^na_{ij}x_ix_j>0,
      $$
      就称$\vx^T\MA\vx$为正定二次型,称$\MA$为正定矩阵。
    \end{dingyi}
    \pause\vspace{0.1in}

    
    注:正定矩阵是针对对称矩阵而言的。
    
  
\end{frame}

\begin{frame}
  
    \begin{jielun}
      二次型$f(y_1,y_2,\cd,y_n)=d_1y_1^2+d_2y_2^2+\cd+d_ny_n^2$正定
      $~~~\Longleftrightarrow~~~d_i>0~~(i=1,2,\cd,n)$
    \end{jielun}\pause

    \begin{proof}
    \begin{itemize}
    \item[$\Leftarrow$] 显然 \pause
    \item[$\Rightarrow$] 设$d_i\le 0$,取$y_i=1, y_j=0(j\ne i)$,代入二次型,得
      $$
      f(0,\cd,0,1,0,\cd,0)=d_i\le 0
      $$
      这与二次型$f(y_1,y_2,\cd,y_n)$正定矛盾。
    \end{itemize}
    \end{proof}
\end{frame}


\begin{frame}
  
    \begin{jielun}
      一个二次型$\vx^T\MA\vx$,经过非退化线性变换$\vx=\MC\vy$,化为$\vy^T(\MC^T\MA\MC)\vy$,其正定性保持不变。即当
      $$\vx^T\MA\vx~~~\xLongleftrightarrow[]{\ds \vx=\MC\vy}~~~\vy^T(\MC^T\MA\MC)\vy\quad (\MC\mbox{可逆})$$
      时,等式两端的二次型有相同的正定性。
    \end{jielun}\pause
    \begin{proof}
    $\forall \vy=(y_1,y_2,\cd,y_n)\ne\M0$,由于$\vx=\MC\vy(\MC\mbox{可逆})$,则$\vx\ne \M0$。若$\vx^T\MA\vx$正定,则$\vx^T\MA\vx>0$。
    从而有:$\forall \vy\ne\M0$,
    $$
    \vy^T(\MC^T\MA\MC)\vy=\vx^T\MA\vx>0
    $$
    故$\vy^T(\MC^T\MA\MC)\vy$是正定二次型。\pause 反之亦然。
    \end{proof}
\end{frame}


\begin{frame}
  由以上两个结论可知,一个二次型$\vx^T\MA\vx$通过坐标变换$\vx=\MC\vy$,将其化为标准型(或规范形),就容易判断其正定性。
\end{frame}


\begin{frame}
  
    \begin{dingli}
      若$\MA$是$n$阶实对称矩阵,则以下命题等价:
      \begin{itemize}
      \item[(1)]$\vx^T\MA\vx$是正定二次型($\MA$是正定矩阵);\\[.1in]
      \item[(2)]$\MA$的正惯性指数为$n$,即$\MA\simeq\MI$;\\[.1in]
      \item[(3)]存在可逆矩阵$\MP$使得$\MA=\MP^T\MP$;\\[.1in]
      \item[(4)]$\MA$的$n$个特征值$\lambda_1,\lambda_2,\cd,\lambda_n$全大于零。
      \end{itemize}
    \end{dingli}
\end{frame}


\begin{frame}
  \blue{\proofname} \red{(1)$\Rightarrow$(2)} \pause 
  对$\MA$,存在可逆阵$\MC$使得
  $$
  \MC^T\MA\MC=\diag(d_1,d_2,\cd,d_n)
  $$
  假设$\MA$的正惯性指数$<n$,则至少存在一个$d_i\le 0$,做变换$\vx=\MC\vy$,则
  $$
  \vx^T\MA\vx = \vy^T(\MC^T\MA\MC)\vy = d_1y_1^2+\cd+d_ny_n^2
  $$
  不恒大于零,与命题(1)矛盾,故$\MA$的正惯性指数为$n$,从而$\MA\simeq\MI$。
\end{frame}

\begin{frame}
  \blue{\proofname} \red{(2)$\Rightarrow$(3)} \pause
  由$\MC^T\MA\MC=\MI$($\MC$可逆),得$\MA=(\MC^T)^{-1}\MC^{-1}=(\MC^{-1})^T\MC^{-1}$,取$\MP=\MC^{-1}$,则有$\MA=\MP^T\MP$。 \pause 
  \vspace{.1in}

  \red{(3)$\Rightarrow$(4)} \pause
  设$\MA\vx=\lambda\vx$,即$(\MP^T\MP)\vx=\lambda\vx$,于是便有
  $$
  \vx^T\MP^T\MP\vx=\lambda\vx^T\vx, \mbox{ 即 } (\MP\vx,\MP\vx)=\lambda(\vx,\vx)
  $$
  由于特征向量$\vx\ne 0$,从而$\MP\vx\ne 0$,故
  $$
  \lambda = \frac{(\MP\vx,\MP\vx)}{(\vx,\vx)}>0.
  $$\pause 
  \vspace{.1in}

  \red{(4)$\Rightarrow$(1)} \pause
  对$n$阶实对称矩阵$\MA$,存在正交阵$\MQ$使得
  $$
  \MQ^T\MA\MQ=\diag(\lambda_1,\cd,\lambda_n)
  $$
  做正交变换$\vx=\MQ\vy$得
  $$
  \vx^T\MA\vx=\lambda_1y_1^2+\lambda_2y_2^2+\cd+\lambda_ny_n^2.
  $$
  因特征值均大于零,故$\vx^T\MA\vx$正定。
\end{frame}


\begin{frame}
  
    \begin{li}
      $\MA\mbox{正定} ~~\Longrightarrow~~ \MA^{-1}\mbox{正定}$
    \end{li}
    \pause 
    \begin{proof}
      \begin{itemize}
        \item 方法1:
          $$
          \vx^T\MA^{-1}\vx=(\vx,\MA^{-1}\vx) \xlongequal[]{\vx=\MA\vy} (\MA\vy,\MA^{-1}\MA\vy)
          =(\MA\vy,\vy)=\vy^T\MA\vy
          $$
          由于当$\vx=\MA\vy$时,$\vx\ne\M0\Leftrightarrow\vy\ne\M0$;又已知$\forall \vy\ne\M0$,恒有$\vy^T\MA\vy\ne\M0$,故$\forall \vx\ne\M0$,恒有$\vx^T\MA^{-1}\vx>0$,因此$\MA^{-1}$正定。 \pause \vspace{.1in}

        \item 方法2:因$\MA$正定,故存在可逆阵$\MC$使得$\MC^T\MA\MC=\MI$。
          将$\MC^T\MA\MC=\MI$两边求逆,得$\MC^{-1}\MA^{-1}(\MC^{-1})^T=\MI$,取$\MD=(\MC^{-1})^T$,
          则$\MD^T\MA^{-1}\MD=\MI$,故$\MA^{-1}$正定。
      \end{itemize}
    \end{proof}
\end{frame}

\begin{frame}
  
    \begin{li}
      判断二次型
      $$
      f(x_1,x_2,x_3)=x_1^2+2x_2^2+3x_3^2+2x_1x_2-2x_2x_3
      $$
      是否为正定二次型。
    \end{li} \pause 

    \begin{jie}
      用配方法得
      $$
      f(x_1,x_2,x_3)=(x_1+x_2)^2+(x_2-x_3)^2+2x_3^2\ge 0
      $$
      等号成立的充分必要条件是
      $$
      x_1+x_2=0, \quad x_2-x_3=0, \quad x_3=0
      $$
      即$x_1=x_2=x_3=0$,故$f(x_1,x_2,x_3)$正定。
    \end{jie}
    \pause \vspace{.15in}

    \blue{\bf 注:~~} 任何二次型都可用配方法判断其正定性。 
  
\end{frame}

\begin{frame}
  
    \begin{li}
      判断二次型
      $$
      f(x_1,x_2,x_3)=3x_1^2+x_2^2+3x_3^2-4x_1x_2-4x_1x_3+4x_2x_3
      $$
      是否为正定二次型。
    \end{li}
    \pause 

    \begin{jie}
      该二次型对应的矩阵为
      $$
      \MA=\left(
        \begin{array}{rrr}
          3&-2&-2\\
          -2&1&2\\
          -2&2&3
        \end{array}
      \right)
      $$
      由
      $$
      |\MA-\lambda\MI|=\left|
        \begin{array}{rrr}
          3-\lambda&-2&-2\\
          -2&1-\lambda&2\\
          -2&2&3-\lambda
        \end{array}
      \right|=-(\lambda-1)(\lambda^2-6\lambda-3)=0
      $$
      得$\MA$的特征值:$\lambda_1=1,\lambda_2=3+2\sqrt3,\lambda_3=3-2\sqrt3$,故$\MA$不是正定矩阵,从而二次型也不是正定的。
    \end{jie}
\end{frame}

\begin{frame}
  
    \begin{dingli}
      $$
      \MA\mbox{正定}~~\Longrightarrow~~
      a_{ii}>0(i=1,2,\cd,n) \mbox{~~且~~}
      |\MA|>0
      $$
    \end{dingli}
    \vspace{.1in}\pause 

    \begin{proof}
      \begin{itemize}
        \item 因$\vx^T\MA\vx$正定,故选$\vx_i=(0,\cd,0,1,0,\cd,0)^T$,则有$\vx_i^T\MA\vx_i=a_{ii}x_i^2=a_{ii}>0$。\\[0.1in]
        \item 因$\MA$正定,故存在可逆矩阵$\MP$,使得$\MA=\MP^T\MP$,因此$|\MA|=|\MP^T||\MP|=|\MP|^2>0$。
      \end{itemize}
    \end{proof}
\end{frame}

\begin{frame}
  
    \begin{dingli}
      $$\MA\mbox{正定} ~~\Longleftrightarrow~~ \MA\mbox{的$n$个顺序主子式全大于零。}$$
    \end{dingli}
    \vspace{.1in}\pause 

    \begin{dingyi}
      记$\MA=(a_{ij})_{n\times n}$,则
      $$
      |\MA_k|=\left|
        \begin{array}{cccc}
          a_{11}&a_{12}&\cd&a_{1k}\\
          a_{21}&a_{22}&\cd&a_{2k}\\
          \vd&\vd&&\vd\\
          a_{k1}&a_{k2}&\cd&a_{kk}
        \end{array}
      \right|
      $$
      称为$n$阶矩阵$\MA$的$k$阶顺序主子式。当$k$取$1,2,\cd,n$时,就得$\MA$的$n$个顺序主子式。 
    \end{dingyi}
    
\end{frame}

\begin{frame}
  \blue{\proofname~~}
    \blue{($\Rightarrow$)~~} 取$\vx_k=(x_1,x_2,\cd,x_k)^T\ne\M0,~\vx=(x_1,\cd,x_k,0,\cd,0)^T\ne\M0$,记$\vx=(\vx_k^T,\M0)^T$,则必有
    $$
    \vx^T\MA\vx=(\vx_k^T,\M0)\left(\begin{array}{cc}\MA_k&\star\\\star&\star\end{array}\right)\left(\begin{array}{c}\vx_k\\\M0\end{array}\right)=\vx_k^T\MA_k\vx_k>0
    $$
    $\forall \vx_k\ne \M0$都成立,故$x_1,x_2,\cd,x_k$的$k$元二次型$\vx_k^T\MA_k\vx_k$是正定的,从而有$|\MA_k|>0$。
\end{frame}

\begin{frame}
  \begin{proof}
    \blue{($\Leftarrow$)~~} 对$n$做数学归纳法。 \pause 
    \begin{itemize}
    \item 当$n=1$时,$a_{11}>0$,$\vx^T\MA\vx=a_{11}x_1^2>0(\forall x_1\ne 0)$,故充分性成立。\\[0.1in] \pause 

    \item 假设充分性对$n-1$元二次型成立,下证对$n$元二次型也成立。将$\MA$分块表示为
      $$
      \MA=\left(
        \begin{array}{cc}
          \MA_{n-1}&\alphabd\\
          \alphabd^T&a_{nn}
        \end{array}
      \right).
      $$
      下证$\MA$合同于单位矩阵。\pause 取
      $$
      \MC_1^T=\left(
        \begin{array}{cc}
          \MI_{n-1}&\M0\\
          -\alphabd^T\MA_{n-1}^{-1}&1
        \end{array}
      \right)
      $$则
      $$
      \MC_1=\left(
        \begin{array}{cc}
          \MI_{n-1}&-\MA_{n-1}^{-1}\alphabd\\
          \M0&1
        \end{array}
      \right), \quad \blue{
        \MC_1^T\MA\MC_1=\left(
        \begin{array}{cc}
          \MA_{n-1}&\M0\\[.1in]
          \M0&\underbrace{a_{nn}-\alphabd^T\MA_{n-1}^{-1}\alphabd}_{\red{a}}
        \end{array}
      \right)
      }
      $$\pause 
      根据充分性条件$|\MA|>0,|\MA_{n-1}|>0$,由上式易得$a>0$。由归纳假设$\MA_{n-1}$正定,故存在$n-1$阶可逆阵$\MG$使得$\MG^T\MA_{n-1}\MG=\MI_{n-1}$。故再取
      $$
      \MC_2^T=\left(
        \begin{array}{cc}
          \MG^T&\M0\\[.1in]
          \M0&\frac1{\sqrt a}
        \end{array}
      \right),\quad \MC_2=\left(
        \begin{array}{cc}
          \MG&\M0\\[.1in]
          \M0&\frac1{\sqrt a}
        \end{array}
      \right)
      $$
      便可得$\MC_2^T(\MC_1^T\MA\MC_1)\MC_2=\MI_n$,故$\MA$合同于单位矩阵,从而$\MA$正定。
    \end{itemize}
  \end{proof}
\end{frame}

\begin{frame}
  \begin{li}
    证明:若$\MA$为$n$阶正定矩阵,则存在正定矩阵$\MB$使得$\MA=\MB^2$。
  \end{li}

  \begin{proof}
    因正定矩阵$\MA$是实对称矩阵,且特征值全大于零,故存在正交阵$\MQ$使得
    $$
    \MA=\MQ\diag(\lambda_1,\cd,\lambda_n)\MQ^T
    $$
    其中$\lambda_i>0$。利用$\MQ^T\MQ=\MI$,及
    $$
    \diag(\lambda_1,\cd,\lambda_n)=[\diag(\sqrt{\lambda_1},\cd,\sqrt{\lambda_n})]^2
    $$
    于是
    $$
    \MA=[\MQ\diag(\sqrt{\lambda_1},\cd,\sqrt{\lambda_n})\MQ^T]^2
    $$
    记
    $$
    \MB=\MQ\diag(\sqrt{\lambda_1},\cd,\sqrt{\lambda_n})\MQ^T
    $$
    即得
    $$
    \MA=\MB^2.
    $$\pause 
    因$\sqrt{\lambda_i}>0$,故$\MB$为正定矩阵。
  \end{proof}
\end{frame}
