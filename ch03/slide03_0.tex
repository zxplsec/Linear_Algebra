\section{例子}

\begin{frame}[fragile]\ft{\secname}
  编制程序,实现华氏温度到摄氏温度的转换。转化公式为
  $$
  C = \frac59 (F-32).
  $$
  其中$F$表示华氏温度,$C$表示摄氏温度。
\end{frame}

\begin{frame}[fragile]\ft{\secname}

\lstinputlisting[language=C,numbers=left,frame=single]{ch03/code/fah2cel.c}
\end{frame}

\begin{frame}[fragile]\ft{\secname}

\begin{lstlisting}
$ gcc fah2cel.c 
$ ./a.out
Please input the Fahrenheit temperature:
78
78.00 F = 25.56 C
\end{lstlisting}
\end{frame}


\begin{frame}[fragile]\ft{\secname}
\begin{itemize}
\item 
\begin{lstlisting}
float fah, cel;
\end{lstlisting}
声明两个\lstinline|float|变量\lstinline|fah, cel|。\\[0.1in]
\item \lstinline|%f|用于输出浮点型数据;\lstinline|%.2f|可以精确控制输出格式,使浮点数显示到小数点后两位。\\[0.1in]
\item 使用\lstinline|scanf()|为程序提供键盘输入。\\[0.1in]
\item[] \lstinline|%f|指示\lstinline|scanf()|从键盘读取一个浮点数;\\[0.1in]
\item[] \lstinline|&fah|表示变量\lstinline|fah|的位置,指定将输入值赋给变量\lstinline|fah|。

\end{itemize}
\end{frame}


\begin{frame}[fragile]\ft{\secname}
\begin{itemize}
\item
该程序的最大特点是\red{交互性},交互性使得程序更加灵活。\\[0.1in]
\item[] 例如,该程序可以输入任意的华氏温度,而不必每次重写。\\[0.2in]
\item \lstinline|scanf()|和\lstinline|printf()|使得这种交互成为可能。\\[0.1in]
\item[] \lstinline|scanf()|从键盘读取数据并将其传递给程序;\\[0.1in]
\item[] \lstinline|printf()|则从程序读取数据并将其打印到屏幕。\\[0.1in]
\item[] 两者一起使用,就建立起了人机之间的双向通信。
\end{itemize}

\end{frame}




