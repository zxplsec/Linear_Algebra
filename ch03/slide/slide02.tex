\section{向量组的秩及其极大线性无关组}

\begin{frame}
\begin{dingyi}[向量组的秩]
  向量组$\alphabd_1,\alphabd_2,\cd,\alphabd_s$中,若
  \begin{itemize}
  \item 存在$r$个\blue{\underline{线性无关}}的向量,
  \item 且其中\blue{\underline{任一向量}}可由这$r$个线性无关的向量线性表示, 
  \end{itemize}
  则数$r$称为\red{向量组的秩(rank)},记作
  $$
  \rank(\alphabd_1,\alphabd_2,\cd,\alphabd_s)=r
  $$
  或
  $$
  \rank(\alphabd_1,\alphabd_2,\cd,\alphabd_s)=r
  $$
\end{dingyi}

\end{frame}

\begin{frame}
\begin{itemize}
\item 若$\alphabd_1,\alphabd_2,\cd,\alphabd_s$线性无关,
  则$\rank(\alphabd_1,\alphabd_2,\cd,\alphabd_s)=s$;
\item 只含零向量的向量组的秩为零。
\item 只含一个非零向量的向量组的秩为1。
\end{itemize}
\end{frame}

\begin{frame}
\begin{dingyi}
  若向量组$B:~\betabd_1,\betabd_2,\cd,\betabd_t$中每个向量可由向量组$A:~\alphabd_1,\alphabd_2,\cd,\alphabd_s$线性表示,
  就称\red{向量组$B:~\betabd_1,\betabd_2,\cd,\betabd_t$可由向量组$A:~\alphabd_1,\alphabd_2,\cd,\alphabd_s$线性表示}。 
  \vspace{0.1in}
  
  如果两个向量组可以互相线性表示,则称这两个向量组是\red{等价}的。
\end{dingyi}
\end{frame}

\begin{frame}
向量组的线性表示,具备
\begin{itemize}
\item \red{自反性}
\item[]向量组自己可以由自己线性表示  
\item \red{传递性}
\item[] 设向量组$A$可以被向量组$B$线性表示,向量组$B$又可以被向量组$C$线性表示,
  则向量组$A$可以被向量组$C$线性表示  
\item \red{不具备对称性}
\item[] 向量组$A$可以被向量组$B$线性表示,不一定有向量组$B$又可以被向量组$A$线性表示。  
\item[] \blue{如}:部分组总是可以由整体线性表示,但反之不成立
\end{itemize} 
\end{frame}

\begin{frame}
向量组的等价,具备
\begin{itemize}
\item \red{自反性}
\item[] 任一向量组和自身等价  
\item \red{对称性}
\item[] 向量组$A$与向量组$B$等价,当然向量组$B$与向量组$A$等价  
\item \red{传递性}
\item[] 设向量组$A$与向量组$B$等价,向量组$B$与向量组$C$等价,
  则向量组$A$与向量组$C$等价
\end{itemize} 
\end{frame}

\begin{frame}
\begin{dingli}
  若向量组$\blue{B:~\betabd_1,\betabd_2,\cd,\betabd_t}$可由向量组$\blue{A:~\alphabd_1,\alphabd_2,\cd,\alphabd_s}$线性表示,且$\blue{t>s}$,
  则$\blue{B:~\betabd_1,\betabd_2,\cd,\betabd_t}$线性相关。
\end{dingli}
\pause 
\begin{proof}
设
$
\ds \betabd_j=\sum_{i=1}^sk_{ij}\alphabd_i, \quad j=1,2,\cd,t.~~
$  
欲证$\betabd_1,\betabd_2,\cd,\betabd_t$线性相关,只需证:存在不全为零的数$x_1,x_2,\cd,x_t$使得
\begin{equation}\label{thm4-1}
  x_1\betabd_1+x_2\betabd_2+\cd+x_t\betabd_t=\M0,
\end{equation}     
即
$$
\sum_{j=1}^t x_j \betabd_j = \sum_{j=1}^t x_j\left(\sum_{i=1}^sk_{ij}\alphabd_i\right)
= \sum_{i=1}^s\left(\sum_{j=1}^t k_{ij} x_j \right)  \alphabd_i = \M0.
$$  
当其中$\alphabd_1,\alphabd_2,\cd,\alphabd_s$的系数
\begin{equation}\label{thm4-2}
  \sum_{j=1}^tk_{ij}x_j = 0, \quad i=1,2,\cd,s
\end{equation}
时,(\ref{thm4-1})显然成立。 
注意到\blue{齐次线性方程组(\ref{thm4-2})含$t$个未知量,$s$个方程,而$t>s$},故(\ref{thm4-2})有非零解。 即有不全为零的$x_1,x_2,\cd,x_t$使得(\ref{thm4-1})成立,故$\betabd_1,\betabd_2,\cd,\betabd_t$线性相关。
\end{proof}
\end{frame}

\begin{frame}
\begin{tuilun}
  若向量组$\blue{B:~\betabd_1,\betabd_2,\cd,\betabd_t}$可由向量组$\blue{A:~\alphabd_1,\alphabd_2,\cd,\alphabd_s}$线性表示,
  且$\blue{B:~\betabd_1,\betabd_2,\cd,\betabd_t}$线性无关,则
  $$\red{t\le s}.$$
\end{tuilun}
\end{frame}

\begin{frame}
\begin{tuilun}
  若$\blue{\rank(\alphabd_1,\alphabd_2,\cd,\alphabd_s)=r}$,
  则$\blue{\alphabd_1,\alphabd_2,\cd,\alphabd_s}$中任何$r+1$个向量都是线性相关的。
\end{tuilun}
\pause  
\begin{proof}
不妨设$\alphabd_1,\alphabd_2,\cd,\alphabd_r$是$\alphabd_1,\alphabd_2,\cd,\alphabd_s$中的$r$个线性无关的向量,由于该向量组中任一个向量可由$\alphabd_1,\alphabd_2,\cd,\alphabd_r$线性表示,由定理3.2.1可知,其中任意$r+1$个向量都线性相关。
\end{proof}
\end{frame}

\begin{frame}
\begin{dingyi}[向量组的秩的等价定义 \& 极大线性无关组]
  设有向量组$\alphabd_1,\alphabd_2,\cd,\alphabd_s$。
  如果能从其中选出$r$个向量$\alphabd_1,\alphabd_2,\cd,\alphabd_{\red{r}}$,满足
  \begin{itemize}
  \item 向量组$\alphabd_1,\alphabd_2,\cd,\alphabd_{\red{r}}$线性无关;
  \item 向量组$\alphabd_1,\alphabd_2,\cd,\alphabd_s$中任意$r+1$个向量都线性相关,
  \end{itemize}
  则称向量组$\alphabd_1,\alphabd_2,\cd,\alphabd_{\red{r}}$为原向量组的一个\red{极大线性无关组},简称\red{极大无关组}。 
  \vspace{0.1in}

  \blue{\underline{极大线性无关组所含向量的个数$\red{r}$}},称为原向量组的\red{秩}。
\end{dingyi}
\end{frame}

\begin{frame}
\begin{zhu}
  \begin{itemize}
  \item   秩为$r$的向量组中,任一个线性无关的部分组最多含有$r$个向量;
  \item 一般情况下,极大无关组不惟一;
  \item 不同的极大无关组所含向量个数相同;
  \item 极大无关组与原向量组是等价的;
  \item 极大无关组是原向量组的\red{全权代表}。
  \end{itemize}
\end{zhu}
\end{frame}

\begin{frame}
\begin{tuilun}
  设$\blue{\rank(\alphabd_1,\alphabd_2,\cd,\alphabd_s)=p,~~\rank(\betabd_1,\betabd_2,\cd,\betabd_t)=r}$,
  如果向量组$\blue{B:~\betabd_1,\betabd_2,\cd,\betabd_t}$可由$\blue{A:~\alphabd_1,\alphabd_2,\cd,\alphabd_s}$线性表示,则
  $$\red{r\le p.}$$
\end{tuilun}
\pause 
\begin{proof}
不妨设$\alphabd_1,\alphabd_2,\cd,\alphabd_p$与$\betabd_1,\betabd_2,\cd,\betabd_r$分别为两个向量组的极大线性无关组。  
$$
\begin{array}{rl}
  (1) & \betabd_1,\betabd_2,\cd,\betabd_r\mbox{等价于}\betabd_1,\betabd_2,\cd,\betabd_t\\[0.1in]
  \Rightarrow & 
                \blue{\betabd_1,\betabd_2,\cd,\betabd_r\mbox{可由}\betabd_1,\betabd_2,\cd,\betabd_t\mbox{线性表示}} \\[0.1in]  
  (2) & \blue{\betabd_1,\betabd_2,\cd,\betabd_t\mbox{可由}\alphabd_1,\alphabd_2,\cd,\alphabd_s\mbox{线性表示}}\\[0.1in]  
  (3)& \alphabd_1,\alphabd_2,\cd,\alphabd_s\mbox{等价于}\alphabd_1,\alphabd_2,\cd,\alphabd_p\\[0.1in]
  \Rightarrow & \blue{\alphabd_1,\alphabd_2,\cd,\alphabd_s\mbox{可由}\alphabd_1,\alphabd_2,\cd,\alphabd_p\mbox{线性表示}} \\[0.1in]  
  \red{\Longrightarrow} &
                          \red{\betabd_1,\betabd_2,\cd,\betabd_t\mbox{可由} \alphabd_1,\alphabd_2,\cd,\alphabd_p\mbox{线性表示}}
\end{array}    
$$  
由上述推论可知$r\le p$。
\end{proof}
\end{frame}
 

\begin{frame}
\begin{tuilun}
  \red{等价向量组的秩相等}。
\end{tuilun}
\end{frame}


