\section{齐次线性方程组有非零解的条件及解的结构}
\begin{frame}
设$\MA$为$m\times n$矩阵,考察
\begin{equation}\label{ax0}
  \MA\vx=\M0.
\end{equation}    
若将$\MA$按列分块为
$$
\MA = (\alphabd_1,~\alphabd_2,~\cd,~\alphabd_n),
$$
齐次方程组(\ref{ax0})可表示为
$$
x_1\alphabd_1+x_2\alphabd_2+\cd+x_n\alphabd_n=\M0.
$$
而齐次方程组(\ref{ax0})有\blue{非零解}的充分必要条件是$\alphabd_1,~\alphabd_2,~\cd,~\alphabd_n$\blue{线性相关},即
$$
\rank(\MA) = \rank(\alphabd_1,~\alphabd_2,~\cd,~\alphabd_n) < n.
$$
\end{frame}


\begin{frame}
\begin{dingli}
  设$\MA$为$m\times n$矩阵,则
  $$
  \blue{\underline{\MA\vx=\M0\mbox{有非零解}}} ~~\Longleftrightarrow~~
  \blue{\underline{\rank(\MA)<n}}.$$
\end{dingli}


\begin{dingli}[定理1的等价命题]
  设$\MA$为$m\times n$矩阵,则
  $$
  \blue{\underline{\MA\vx=\M0\mbox{只有零解}}} ~~\Longleftrightarrow~~
  \blue{\underline{\rank(\MA)=n=\MA\mbox{的列数}}}.
  $$
\end{dingli}
\end{frame}

\begin{frame}
\begin{li}
  设$\MA$为$n$阶矩阵,证明:存在$n\times s$矩阵$\MB\ne \M0$,使得$\MA\MB=\M0$的充分必要条件是
  $$
  |\MA|=0.
  $$      
\end{li}\pause 
\begin{proof}
$|\MA|=0~~\Longleftrightarrow~~\MA\vx=\M0$有非零解。 下证
$$
\purple{\mbox{存在}n\times s\mbox{矩阵}\MB\ne 0\mbox{使得}\MA\MB=\M0
  ~~\Longleftrightarrow~~\MA\vx=\M0\mbox{有非零解}}.
$$      \pause 
\begin{itemize}
\item[($\red{\Longrightarrow}$)]
  设$\MA\MB=\M0$,则$\MB$的解向量为$\MA\vx=\M0$的解。又$\MB\ne\M0$,则$\MB$至少有一个非零列向量,
  从而$\MA\vx=\M0$至少有一个非零解。\\[0.1in] \pause 
\item[($\red{\Longleftarrow}$)]
  设$\MA\vx=\M0$有非零解,任取一个非零解$\betabd$,令
  $$
  \MB=(\betabd,~\M0,~\cd,~\M0)
  $$
  则$\MB\ne\M0$,且$\MA\MB=\M0$。
\end{itemize}
\end{proof}
\end{frame}

\begin{frame}
\begin{dingli}
  若$\vx_1,~\vx_2$为齐次线性方程组$\MA\vx=\M0$的两个解,则
  $$
  k_1\vx_1+k_2\vx_2\quad(k_1,~k_2\mbox{为任意常数})
  $$
  也是它的解。
\end{dingli}\pause 
\begin{proof}
因为
$$
\MA(\red{k_1\vx_1+k_2\vx_2}) = k_1\MA\vx_1+k_2\MA\vx_2 = k_1\M0+k_2\M0=\M0,
$$
故$k_1\vx_1+k_2\vx_2$也为$\MA\vx=\M0$的解。
\end{proof}
\end{frame}

\begin{frame}
\begin{dingyi}[基础解系]
  设$\vx_1,~\vx_2,~\cd,~\vx_p$为$\MA\vx=\M0$的解向量,若
  \begin{itemize}
  \item[(1)] $\vx_1,~\vx_2,~\cd,~\vx_p$线性无关;
  \item[(2)] $\MA\vx=\M0$的任一解向量可由$\vx_1,~\vx_2,~\cd,~\vx_p$线性表示。
  \end{itemize}
  则称$\vx_1,~\vx_2,~\cd,~\vx_p$为$\MA\vx=\M0$的一个\blue{\underline{基础解系}}。
\end{dingyi}
\end{frame}

\begin{frame}
\begin{zhu}
  关于基础解系,请注意以下几点:
  \begin{itemize}
  \item[(1)] 基础解系即全部解向量的\blue{极大无关组}。\\[0.1in]  
  \item[(2)] 找到了基础解系,就找到了齐次线性方程组的全部解:
    $$
    k_1\vx_1+k_2\vx_2+\cd+k_p\vx_p \quad(k_1,k_2,\cd,k_p\mbox{为任意常数}).
    $$ 
  \item[(3)] 基础解系\blue{不唯一}。
  \end{itemize}
\end{zhu}
\end{frame}


\begin{frame}[allowframebreaks]
\begin{li}
  求方程组
  $$
  x+y+z=0
  $$
  的全部解。
\end{li}

\begin{jie}
\begin{itemize}
\item[(1)] 选取$y,z$为自由未知量,则
  $$
  \left\{
    \begin{array}{cccccc}
      x&=&-&y&-&z\\
      y&=&&y&&\\
      z&=&&&&z
    \end{array}
  \right.
  $$
  则方程组的全部解为
  $$
  \left(
    \begin{array}{r}
      x\\y\\z
    \end{array}
  \right) = c_1      \left(
    \begin{array}{r}
      -1\\1\\0
    \end{array}
  \right) + c_2      \left(
    \begin{array}{r}
      -1\\0\\1
    \end{array}
  \right) \quad (c_1,c_2\mbox{为任意常数})
  $$
\end{itemize}
\end{jie}
\end{frame}

\begin{frame}
\begin{jie}[续]
\begin{itemize}  
\item[(2)] 选取$x,z$为自由未知量,则
  $$
  \left\{
    \begin{array}{cccccc}
      x&=&&x&&\\
      y&=&-&x&-&z\\
      z&=&&&&z
    \end{array}
  \right.
  $$
  则方程组的全部解为
  $$
  \left(
    \begin{array}{r}
      x\\y\\z
    \end{array}
  \right) = c_1      \left(
    \begin{array}{r}
      1\\-1\\0
    \end{array}
  \right) + c_2      \left(
    \begin{array}{r}
      0\\-1\\1
    \end{array}
  \right) \quad (c_1,c_2\mbox{为任意常数})
  $$  
\end{itemize}
\end{jie}
\end{frame}

\begin{frame}
  \begin{jie}[续]
    \begin{itemize}
    \item[(3)] 选取$x,y$为自由未知量,则
      $$
      \left\{
        \begin{array}{cccccc}
          x&=&&x&&\\
          y&=&&&&y\\
          z&=&-&x&-&y
        \end{array}
      \right.
      $$
      则方程组的全部解为
      $$
      \left(
        \begin{array}{r}
          x\\y\\z
        \end{array}
      \right) = c_1      \left(
        \begin{array}{r}
          1\\0\\-1
        \end{array}
      \right) + c_2      \left(
        \begin{array}{r}
          0\\1\\-1
        \end{array}
      \right) \quad (c_1,c_2\mbox{为任意常数})
      $$
    \end{itemize}
  \end{jie}
\end{frame}

\begin{frame}
  三个不同的基础解系为
  $$
  \left\{
    \left(
      \begin{array}{r}
        -1\\1\\0
      \end{array}
    \right),~~
    \left(
      \begin{array}{r}
        -1\\0\\1
      \end{array}
    \right)
  \right\},
  $$
  $$
  \left\{
    \left(
      \begin{array}{r}
        1\\-1\\0
      \end{array}
    \right),~~
    \left(
      \begin{array}{r}
        0\\-1\\1
      \end{array}
    \right)
  \right\},
  $$
  $$
  \left\{
    \left(
      \begin{array}{r}
        1\\0\\-1
      \end{array}
    \right),~~
    \left(
      \begin{array}{r}
        0\\1\\-1
      \end{array}
    \right)
  \right\}.
  $$
\end{frame}


\begin{frame}
\begin{dingli}
  设$\MA$为$m\times n$矩阵,若$\rank(\MA)=r<n$,则齐次线性方程组$\MA\vx=\M0$存在基础解系,
  且基础解系含$n-r$个解向量。
\end{dingli} \vspace{0.2in} \pause 

\begin{zhu}
  注意以下两点:
  \begin{itemize}
  \item $r$为$\MA$的秩,也是$\MA$的行阶梯形矩阵的非零行行数,是非自由未知量的个数。 \\[0.1in]
  \item $n$为未知量的个数,故$n-r$为自由未知量的个数。 \\[0.1in]
  \item[] 有多少自由未知量,基础解系里就对应有多少个向量。
  \end{itemize}
\end{zhu}
\end{frame}

\begin{frame}
\begin{li}
  求齐次线性方程组$\MA\vx=\M0$的基础解系,其中
  $$
  \MA = \left(
    \begin{array}{rrrr}
      1&-8&10&2\\
      2&4&5&-1\\
      3&8&6&-2
    \end{array}
  \right).
  $$
\end{li} 
\end{frame}

\begin{frame}
\begin{jie}
$$
\begin{array}{l}
  \left(
  \begin{array}{rrrr}
    1&-8&10&2\\
    2&4&5&-1\\
    3&8&6&-2
  \end{array}
           \right) \xlongrightarrow[r_3-3r_1]{r_2-2r_1}
           \left(
           \begin{array}{rrrr}
             1&-8&10&2\\
             0&20&-15&-5\\
             0&32&24&-8
           \end{array}
                      \right)\\[0.3in]
  \xlongrightarrow[r_2\div4]{r_3\div8}
  \left(
  \begin{array}{rrrr}
    1&-8&10&2\\
    0&4&-3&-1\\
    0&4&-3&-1
  \end{array}
            \right) \xlongrightarrow[r_1+2r_2]{r_3-r_2}
            \left(
            \begin{array}{rrrr}
              1&0&4&0\\
              0&4&-3&-1\\
              0&0&0&0
            \end{array}
                     \right) \\[0.3in]
  \xlongrightarrow[]{r_2\div4}
  \left(
  \begin{array}{rrrr}
    1&0&4&0\\
    0&1&-3/4&-1/4\\
    0&0&0&0
  \end{array}
           \right)
\end{array}
$$
\end{jie}
\end{frame}

\begin{frame}
\begin{jie}[续]
原方程等价于
$$\left\{
  \begin{array}{rcrcrc}
    x_1&=&-4&x_3&&\\[0.1in]
    x_2&=&\frac34&x_3&+\frac14&x_4
  \end{array}
\right.  \Leftrightarrow
\left\{
  \begin{array}{rcrcrc}
    x_1&=&-4&x_3&&\\[0.1in]
    x_2&=&\frac34&x_3&+\frac14&x_4\\[0.1in]
    x_3&=&&x_3&&\\[0.1in]
    x_4&=&&&&x_4      
  \end{array}
\right.
$$
基础解系为
$$
\xibd_1 = \left(
  \begin{array}{r}
    -4\\[0.1in]
    \frac34\\[0.1in]
    1\\[0.1in]
    0
  \end{array}
\right), \quad \xibd_2 = \left(
  \begin{array}{r}
    0\\[0.1in]
    \frac14\\[0.1in]
    0\\[0.1in]
    1
  \end{array}
\right)
$$
\end{jie}
\end{frame}

\begin{frame}
\begin{li}
  求齐次线性方程组
  $$
  nx_1+(n-1)x_2+\cd+2x_{n-1}+x_n=0
  $$
  的基础解系。      
\end{li}
\pause 
\begin{jie}
原方程等价于$x_n=-nx_1-(n-1)x_2-\cd-2x_{n-1}$, 即
$$
\left\{
  \begin{array}{rcrrrr}
    x_1&=&x_1&&&\\
    x_2&=&&x_2&&\\
       &\vd&&&&\\
    x_{n-1}&=&&&&x_{n-1}\\      
    x_n&=&-nx_1&-(n-1)x_2&\cd&-2x_{n-1}
  \end{array}    
\right.
$$
基础解系为
$$
(\xibd_1,\xibd_2,\cd,\xibd_{n-1})=\left(
  \begin{array}{rrrr}
    1&0&\cd&0\\
    0&1&\cd&0\\
    \vd&\vd&&\vd\\
    0&0&\cd&1\\
    -n&-n+1&\cd&-2
  \end{array}
\right)
$$
\end{jie}
\end{frame}

\begin{frame}
\begin{li}
  设$\MA$与$\MB$分别为$m\times n$和$n\times s$矩阵,且$\MA\MB=\M0$。证明:
  $$
  \rank(\MA)+\rank(\MB)\le n.
  $$
\end{li}
\pause 
\begin{proof}
由$\MA\MB=\M0$知,$\MB$的列向量是$\MA\vx=\M0$的解。
故$\MB$的列向量组的秩,不超过$\MA\vx=\M0$的基础解系的秩,即
$$
\rank(\MB) \le n-\rank(\MA),
$$
即
$$
\rank(\MA)+\rank(\MB)\le n.
$$
\end{proof}
\end{frame}

\begin{frame}

\begin{li}
  设$n$元齐次线性方程组$\MA\vx=\M0$与$\MB\vx=\M0$同解,证明
  $$
  \rank(\MA)=\rank(\MB).
  $$
\end{li}
\pause 
\begin{jie}
$\MA\vx=\M0$与$\MB\vx=\M0$同解,故它们有相同的基础解系,而基础解系包含的向量个数相等,即
$$
n-\rank(\MA)=n-\rank(\MB),
$$
从而
$$
\rank(\MA)=\rank(\MB).
$$
\end{jie}
\end{frame}


\begin{frame}
\begin{li}
  设$\MA$为$m\times n$实矩阵,证明$\rank(\MA^T\MA)=\rank(\MA)$。    
\end{li}
\pause 
\begin{proof}
只需证明$\MA\vx=\M0$与$(\MA^T\MA)\vx=\M0$同解。
\begin{itemize}
\item[(1)] 若$\vx$满足$\MA\vx=\M0$,则有$(\MA^T\MA)\vx=\MA^T(\MA\vx)=\M0$。 
\item[(2)] 若$\vx$满足$\MA^T\MA\vx=\M0$,则
  $$
  \vx^T\MA^T\MA\vx=\M0,
  $$
  即
  $$
  (\MA\vx)^T\MA\vx=\M0,
  $$
  故$\MA\vx=\M0$。
\end{itemize}
\end{proof}
\end{frame}

