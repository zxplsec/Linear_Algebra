\subsection{实对称矩阵的特征值和特征向量}

\begin{frame}
  \begin{dingli}
      实对称矩阵$\MA$的任一特征值都是实数。
    \end{dingli}
    \pause
    \begin{proof}
    $$
    \begin{array}{rl}
      \MA\vx = \lambda \vx
      &~~\Longrightarrow~~
      \overline{(\MA\vx)}^T = \overline{\lambda \vx}^T\\[0.1in]
      &~~\Longrightarrow~~
      \overline{\vx}^T~\overline{\MA}^T~\vx = \overline{\lambda} ~\overline{\vx}^T~\vx\\[0.1in]
      &~~\Longrightarrow~~
      \overline{\vx}^T~\MA^T~\vx = \overline{\lambda} ~\overline{\vx}^T~\vx\\[0.1in]
      &~~\Longrightarrow~~
      \lambda \overline{\vx}^T~\vx = \overline{\lambda} ~\overline{\vx}^T~\vx\\[0.1in]
      &~~\Longrightarrow~~
      \lambda = \overline \lambda
    \end{array}
    $$
  \end{proof}
\end{frame}


\begin{frame}
  
    \begin{dingli}
      实对称矩阵$\MA$对应于不同特征值的特征向量是正交的。
    \end{dingli}
    \pause
    \begin{proof}
    设$\MA\vx_1=\lambda_1\vx_1,~~\MA\vx_1=\lambda_1\vx_1~ (\lambda_1\ne\lambda_2), ~~\MA^T=\MA$,则
    $$
    \lambda_1\vx_2^T\vx_1=\vx_2^T\MA\vx_1=\vx_2^T\MA^T\vx_1=(\MA\vx_2)^T\vx_1=(\lambda_2\vx_2)^T\vx_1=\lambda_2\vx_2^T\vx_1
    $$\pause
    由于$\lambda_1\ne\lambda_2$,所以
    $$
    \vx_2^T\vx_1=0.
    $$
  \end{proof}
\end{frame}
