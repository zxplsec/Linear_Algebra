\subsection{相似矩阵及其性质}

\begin{frame}
  \begin{dingyi}
    对于矩阵$\MA$和$\MB$,若存在可逆阵$\MP$,使得$\MP^{-1}\MA\MP=\MB$,就称\red{$\MA$相似于$\MB$,记为$\MA\sim \MB$}。
  \end{dingyi}
  \vspace{.1in} \pause

  矩阵的相似关系也是一种等价关系,即满足以下三条性质
  \begin{itemize}
    \item 反身性:$\MA\sim \MA$
    \item 对称性:$\MA \sim \MB ~\Rightarrow~ \MB \sim \MA$
    \item 传递性:$\MA \sim \MB,~\MB \sim \MC ~\Rightarrow~ \MA \sim \MC$
  \end{itemize}
\end{frame}

\begin{frame}\ft{相似矩阵的性质}
  \begin{xingzhi}
    $$
    \MP^{-1} (k_1\MA_1+k_2\MA_2)\MP = k_1\MP^{-1}\MA_1\MP + k_2\MP^{-1}\MA_2\MP.
    $$
  \end{xingzhi}

  \begin{xingzhi}
    $$
    \MP^{-1}\MA_1\MA_2\MP = (\MP^{-1}\MA_1\MP)(\MP^{-1}\MA_2\MP)
    $$
  \end{xingzhi}
\end{frame}

\begin{frame}\ft{相似矩阵的性质}
  \begin{xingzhi}
    $$
    \MA\sim\MB ~\Rightarrow~ \MA^m\sim\MB^m(m\in \mathbb Z^+).
    $$
  \end{xingzhi}\pause \vspace{.1in}

  \begin{proof}
    因$\MA\sim\MB$,故存在可逆阵$\MP$使得
    $$
    \MP^{-1}\MA\MP=\MB
    $$
    于是
    $$
    \MB^m=(\MP^{-1}\MA\MP)(\MP^{-1}\MA\MP)\cd(\MP^{-1}\MA\MP)=\MP^{-1}\MA^m\MP
    $$
    故$\MA^m\sim\MB^m$。
  \end{proof}
\end{frame}

\begin{frame}\ft{相似矩阵的性质}
  \begin{xingzhi}
    $$
    \MA\sim\MB ~\Rightarrow~ f(\MA)\sim f(\MB),
    $$
    其中
    $$
    f(x) = a_nx^n+a_{n-1}x^{n-1}+\cd+a_1x+a0.
    $$
  \end{xingzhi}
\end{frame}

\begin{frame}\ft{相似矩阵的性质}
  \begin{dingli}
    \red{相似矩阵的特征值相同。}
  \end{dingli}
  \vspace{.1in}\pause 
  \begin{proof}
  只需证明相似矩阵有相同的特征多项式。设$\MA\sim\MB$,则存在可逆阵$\MP$,使得
  $$
  \MP^{-1}\MA\MP = \MB
  $$
  于是
  $$
  \begin{aligned}
    |\lambda\MI-\MB|&=|\lambda\MI-\MP^{-1}\MA\MP|\\
    &=|\MP^{-1}(\lambda\MI-\MA)\MP|=|\MP^{-1}||\lambda\MI-\MA||\MP|\\
    &=|\lambda\MI-\MA|
  \end{aligned}
  $$
\end{proof}
\end{frame}

\begin{frame}
  \begin{zhu}
    上述定理的逆命题不成立。例如,
    $$
    \MI = \left(
      \begin{array}{cc}
        1&0\\
        0&1
      \end{array}
    \right),\quad
    \MA = \left(
      \begin{array}{cc}
        1&1\\
        0&1
      \end{array}
    \right)
    $$
    都以$1$为二重特征值,但对于任何可逆阵$\MP$,都有
    $$
    \MP^{-1}\MI\MP=\MI\ne \MA,
    $$
    故$\MA$与$\MI$不相似。
  \end{zhu}
\end{frame}

\begin{frame}

\end{frame}