\section{矩阵可对角化的条件}
\begin{frame}  
    矩阵可对角化,即矩阵与对角阵相似。    
\end{frame}

\begin{frame}
  
    \begin{dingli}
      $\mbox{矩阵可对角化} ~~\Longleftrightarrow~~
      \mbox{$n$阶矩阵有$n$个线性无关的特征向量}$ 
    \end{dingli}
    \pause
    \begin{proof}
      \begin{itemize}
      \item[$\Rightarrow$] 
        $$
        \MP^{-1}\MA\MP=\Lambdabd  ~~\Longrightarrow~~
        \MA\MP=\MP\Lambdabd
        $$\pause
        将$\MP$按列分块,即
        $$
        \MP=(\vx_1,~\vx_2,~\cd,~\vx_n),
        $$
        则
        $$
        \MA(\vx_1,~\vx_2,~\cd,~\vx_n)=(\vx_1,~\vx_2,~\cd,~\vx_n)\left(
          \begin{array}{cccc}
            \lambda_1&&&\\
                     &\lambda_2&&\\
                     &&\dd&\\
                     &&&\lambda_n
          \end{array}
        \right)
        $$\pause
        于是
        $$
        \MA\vx_i=\lambda_i\vx_i\quad(i=1,2,\cd,n).
        $$\pause
        故$\vx_1,~\vx_2,~\cd,~\vx_n$是$\MA$分别对应于$\lambda_1,~\lambda_2,~\cd,~\lambda_n$的特征向量。\pause
        由于$\MP$可逆,所以它们是线性无关的。
      \end{itemize}
    \end{proof}
\end{frame}


\begin{frame}
  
    若$\MA$与$\Lambdabd$相似,则$\Lambdabd$的主对角元都是$\MA$的特征值。
    若不计$\lambda_k$的排列次序,则$\Lambdabd$是唯一的,称$\Lambdabd$为$\MA$的相似标准型。
  
\end{frame}


\begin{frame}
  
    \begin{dingli}
      $\MA$的属于不同特征值的特征向量是线性无关的。
    \end{dingli}
    \pause
    \begin{proof}
      设$\MA$的$m$个互不相同的特征值为$\lambda_1,\lambda_2,\cd,\lambda_m$,其相应的特征向量为$\vx_1,~\vx_2,~\cd,~\vx_m$.
      对$m$做数学归纳法。
      \begin{itemize}
      \item[$1^o$] 当$m=1$时,结论显然成立。
      \item[$2^o$] 设$k$个不同特征值$\lambda_1,\lambda_2,\cd,\lambda_k$的特征向量$\vx_1,~\vx_2,~\cd,~\vx_k$线性无关。下面考虑$k+1$个不同特征值的特征向量。
        \pause
        设
        $$
        \begin{array}{rc}
          & a_1\vx_1+a_2\vx_2+\cd+a_k\vx_k+a_{k+1}\vx_{k+1}=\M0\qquad(1)\\[0.1cm]\pause
          \Longrightarrow&
                           \MA(a_1\vx_1+a_2\vx_2+\cd+a_k\vx_k+a_{k+1}\vx_{k+1}=\M0\\[0.1cm]\pause
          \Longrightarrow& 
                           a_1\lambda_1\vx_1+a_2\lambda_2\vx_2+\cd+a_k\lambda_k\vx_k+a_{k+1}\lambda_{k+1}\vx_{k+1}=\M0\quad(2)\\[0.1cm]\pause
          \xLongrightarrow[]{(2)-\lambda_{k+1}(1)}&
                                                    a_1(\lambda_{k+1}-\lambda_1)\vx_1+a_2(\lambda_{k+1}-\lambda_2)\vx_2+\cd+a_k(\lambda_{k+1}-\lambda_k)\vx_k=\M0\\[0.1cm]\pause
          \Longrightarrow&
                           a_i(\lambda_{k+1}-\lambda_i)=0, ~~i=1,2,\cd,k\\[0.1cm]\pause
          \Longrightarrow&
                           a_i=0, ~~i=1,2,\cd,k\\[0.1cm]\pause
          \Longrightarrow&
                           a_{k+1}\vx_{k+1}=0\\[0.1cm]\pause
          \Longrightarrow&
                           a_{k+1}=0\\[0.1cm]\pause
          \Longrightarrow&
                           \vx_1,~\vx_2,~\cd,~\vx_k,~~\vx_{k+1}\mbox{线性无关}
        \end{array}
        $$
      \end{itemize}
    \end{proof}
\end{frame}

\begin{frame}
  
    \begin{tuilun}
      若$\MA$有$n$个互不相同的特征值,则$\MA$与对角阵相似。
    \end{tuilun}
  
\end{frame}

\begin{frame}
  
    \begin{li}
      设实对称矩阵
      $$
      \MA=\left(
      \begin{array}{rrrr}
        1&-1&-1&-1\\
        -1&1&-1&-1\\
        -1&-1&1&-1\\
        -1&-1&-1&1
      \end{array}
      \right)
      $$
      问$\MA$是否可对角化?若可对角化,求对角阵$\Lambdabd$及可逆矩阵$\MP$使得$\MP^{-1}\MA\MP=\Lambdabd$,再求$\MA^k$。
    \end{li}
  
\end{frame}

\begin{frame}
  
    \begin{li}
      设$\MA=(a_{ij})_{n\times n}$是主对角元全为$2$的上三角矩阵,且存在$a_{ij}\ne 0(i<j)$,问$\MA$是否可对角化?
    \end{li}
  
\end{frame}
