\subsection{特征值与特征向量}
\begin{frame}  
    \begin{dingyi}[特征值与特征向量]
      设$\MA$为复数域$\C$上的$n$阶矩阵,如果存在数$\lambda\in\C$和非零的$n$维向量$\vx$使得
      $$
      \MA\vx=\lambda\vx
      $$
      则称$\lambda$为矩阵$\MA$的\blue{\underline{特征值}},$\vx$为$\MA$的对应于特征值$\lambda$的\blue{\underline{特征向量}}。
    \end{dingyi} \pause

    \begin{itemize}
    \item[(1)] 特征向量$\vx\ne\M0$;
    \item[(2)] 特征值问题是对方针而言的。 
    \end{itemize}
\end{frame}


\begin{frame}
  
    由定义,$n$阶矩阵$\MA$的特征值,就是使齐次线性方程组
    $$
    (\lambda\MI-\MA)\vx=\M0
    $$
    有非零解的$\lambda$值,即满足方程
    $$
    \det(\lambda\MI-\MA)=0
    $$
    的$\lambda$都是矩阵$\MA$的特征值。
    \pause

    \begin{jielun}
      特征值$\lambda$是关于$\lambda$的多项式$\det(\lambda\MI-\MA)$的根。
    \end{jielun}
    
  
\end{frame}

\begin{frame}
  
  \begin{dingyi}[特征多项式、特征矩阵、特征方程]
      设$n$阶矩阵$\MA=(a_{ij})$,则
      $$
      f(\lambda)=\det(\lambda\MI-\MA)
      =\left|
      \begin{array}{cccc}
        \lambda-a_{11}&-a_{12}&\cd&-a_{1n}\\[0.2cm]
        -a_{21}&\lambda-a_{22}&\cd&-a_{2n}\\[0.2cm]
        \vd&\vd&&\vd\\[0.2cm]
        -a_{n1}&-a_{n2}&\cd&\lambda-a_{nn}
      \end{array}
      \right|
      $$
      称为矩阵$\MA$的特征多项式,$\lambda\MI-\MA$称为$\MA$的特征矩阵,$\det(\lambda\MI-\MA)=0$称为$\MA$的特征方程。
    \end{dingyi}
    \pause
    \begin{itemize}
    \item[(1)]  $n$阶矩阵$\MA$的特征多项式是$\lambda$的$n$次多项式。\pause
    \item[(2)]  特征多项式的$k$重根称为$k$重特征值。
    \end{itemize}
  
\end{frame}

\begin{frame}
  
    \begin{li}
      求矩阵
      $$
      \MA=\left(
      \begin{array}{rrr}
        5&-1&-1\\
        3&1&-1\\
        4&-2&1
      \end{array}
      \right)
      $$
      的特征值与特征向量。
    \end{li}\pause 
    
    \begin{jie}
      $$
      \begin{array}{rl}
        \det(\MA-\lambda\MI)
        &=\left|
          \begin{array}{rrr}
            5-\lambda&-1&-1\\
            3&1-\lambda&-1\\
            4&-2&1-\lambda
          \end{array}
                  \right|
                  = (3-\lambda)
                  \left|\begin{array}{rrr}
                          1&-1&-1\\
                          1&1-\lambda&-1\\
                          1&-2&1-\lambda
                        \end{array}\right|\\[0.3in]
        &= (3-\lambda)
          \left|\begin{array}{rrr}
                  1&-1&-1\\
                  0&2-\lambda&0\\
                  0&-1&2-\lambda
                        \end{array}\right|
                        =-(\lambda-3)(\lambda-2)^2=0
      \end{array}
      $$
      故$\MA$的特征值为$\lambda_{1,2}=2,~\lambda_3=3$。
    \end{jie}
\end{frame}

\begin{frame}
  \begin{jie}[续]
    当$\lambda_{1,2}=2$时,由$(\MA-\lambda\MI)\vx=\M0$,即
        $$
        \left(\begin{array}{rrr}
                3&-1&-1\\
                3&-1&-1\\
                4&-2&-1
              \end{array}\right)\left(
              \begin{array}{c}
                x_1\\
                x_2\\
                x_3
              \end{array}
            \right)=\left(
              \begin{array}{c}
                0\\
                0\\
                0
              \end{array}
            \right)
            $$
            得其基础解系为$\vx_1=(1,1,2)^T$,因此$k_1\vx_1$($k_1$为非零任意常数)是$\MA$对应于$\lambda_{1,2}=2$的全部特征向量。
        \pause\vspace{0.1in}

        当$\lambda_3=3$时,由$(\MA-\lambda\MI)\vx=\M0$,即
        $$
        \left(
          \begin{array}{rrr}
            2&-1&-1\\
            3&-2&-1\\
            4&-2&-2
          \end{array}\right)\left(
          \begin{array}{c}
            x_1\\
            x_2\\
            x_3
          \end{array}
        \right)=\left(
          \begin{array}{c}
            0\\
            0\\
            0
          \end{array}
        \right)
        $$
        得其基础解系为$\vx_2=(1,1,1)^T$,因此$k_2\vx_2$($k_2$为非零任意常数)是$\MA$对应于$\lambda_3=3$的全部特征向量。
          \end{jie}
  
\end{frame}

\begin{frame}
  
    \begin{li}
      $$
      \left(
      \begin{array}{cccc}
        a_{11}&0&\cd&0\\
        0&a_{22}&\cd&0\\
        \vd&\vd&&\vd\\
        0&0&\cd&a_{nn}
      \end{array}
      \right), \left(
      \begin{array}{cccc}
        a_{11}&a_{12}&\cd&a_{1n}\\
        0&a_{22}&\cd&a_{2n}\\
        \vd&\vd&&\vd\\
        0&0&\cd&a_{nn}
      \end{array}
      \right),\left(
      \begin{array}{cccc}
        a_{11}&0&\cd&0\\
        a_{21}&a_{22}&\cd&0\\
        \vd&\vd&&\vd\\
        a_{n1}&a_{n2}&\cd&a_{nn}
      \end{array}
      \right)
      $$
      的特征多项式为
      $$
      (\lambda-a_{11})(\lambda-a_{22})\cd(\lambda-a_{nn})
      $$
      故其$n$个特征值为$n$个对角元。
    \end{li}
  
\end{frame}

