\begin{frame}[fragile]\ft{\secname}
\begin{itemize}
\item[1] 不使用任何运算符,求两个正整数的和(可使用{\tf printf()})。
\end{itemize}
\end{frame}



\begin{frame}[fragile]\ft{\secname}
\begin{itemize}
\item[2] 输入分钟,以小时和分钟的形式显示时间。\\[.1in]
  \begin{itemize}
  \item 使用 \lstinline|#define| 或 \lstinline|const| 来创建一个代表 \lstinline|60| 的符号常量。\\[.1in]
  \item 使用 \lstinline|while| 来允许用户重复键入值,并且键入一个小于等于0的时间来终止循环。
  \end{itemize}
\end{itemize}
\end{frame}


\begin{frame}[fragile]\ft{\secname}
\begin{itemize}
\item[3] 输入一个整数,打印从输入的值到比它大10的所有整数值。\\[.1in]
  \begin{itemize}
  \item 请在各个输出值之间用空格、制表符或换行符分开。
  \end{itemize}
\end{itemize}
\end{frame}



\begin{frame}[fragile]\ft{\secname}
\begin{itemize}
\item[4] 按厘米输入一个高度值,然后按英尺和英寸显示它。\\[.1in]
  \begin{itemize}
  \item 换算公式
\begin{lstlisting}
1 feet = 12 inches
1 cm = 0.0328084 feet
\end{lstlisting}    
  \item 允许厘米和英寸的值出现小数部分。
  \item 允许用户多次输入,直到用户输入一个非正的数值。
  \end{itemize}

\end{itemize}
% \begin{lstlisting}[backgroundcolor=\color{red!10}]
% Enter a height in centimeters: 182
% 182.0 cm = 5 feet, 11.7 inches
% Enter a height in centimeters (<=0 to quit): 168
% 168.0 cm = 5 feet, 6.1 inches
% Enter a height in centimeters (<=0 to quit): 0
% Bye!
% \end{lstlisting}

\end{frame}

\begin{frame}[fragile]\ft{\secname}
\begin{itemize}
\item[5] 输入$n$,求前$n$个整数的和。
\end{itemize}
\end{frame}

\begin{frame}[fragile]\ft{\secname}
\begin{itemize}
\item[6] 输入一个华氏温度,以double型读入该温度值,并将它作为一个参数传递给一个函数 \lstinline|temp|。\\[.1in]
  \begin{itemize}
  \item 该函数将计算相应的摄氏温度和绝对温度,并显示这三种温度;\\[.1in]
  \item 换算公式:
\begin{lstlisting}
Cel =  (Fah - 32.0) / 1.8 
Kel  = Cel + 273.16
\end{lstlisting}
\item
  允许重复输入温度,当用户输入q或其他非数字值时,循环结束。
\end{itemize}
\end{itemize}
\end{frame}
