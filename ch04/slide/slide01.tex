\section{$\mathbb R^n$的基与向量关于基的坐标}

\begin{frame}
    \begin{dingyi}
      设有序向量组$B=(\betabd_1,\betabd_2,\cd,\betabd_n)\subset\mathbb R^n$,如果$B$线性无关,
      则$\mathbb R^n$中任一向量$\alphabd$均可由$B$线性表示,即
      $$
      \alphabd=a_1\betabd_1+a_2\betabd_2+\cd+a_n\betabd_n,
      $$
      称$B$为$\mathbb R^n$的一组基,有序数组$(a_1,a_2,\cd,a_n)$是向量$\alphabd$在基$B$下的坐标,记作
      $$
      \alphabd_B=(a_1,a_2,\cd,a_n)\mbox{~~或~~}\alphabd_B=(a_1,a_2,\cd,a_n)^T
      $$
      并称之为$\alphabd$的坐标向量。
    \end{dingyi}
\end{frame}


\begin{frame}
  
    \begin{zhu}
      \begin{itemize}
      \item $\MR^n$的基不是唯一的;\\[0.2in]
      \item 基本向量组
        $$
        \epsilonbd_i=(0,\cd,0,1,0,\cd,0), \quad i=1,2,\cd,n
        $$
        称为$\MR^n$的自然基或标准基;\\[0.2in]
      \item 本课程对于向量及其坐标,采用列向量的形式,即
        $$
        \alphabd=(\betabd_1,\betabd_2,\cd,\betabd_n)\left(
        \begin{array}{c}
          a_1\\
          a_2\\
          \vd\\
          a_n
        \end{array}
        \right)
        $$
      \end{itemize}
    \end{zhu}
  
\end{frame}


\begin{frame}
  
    \begin{li}
      设$\MR^n$的两组基为自然基$B_1$和$B_2=\{\betabd_1,\betabd_2,\cd,\betabd_n\}$,其中
      \begin{equation}\label{twobase}
      \begin{array}{lrrrrrrrrr}
        \betabd_1&=(&1,&-1,& 0,&0,&\cd,&0,&0,&0)^T,\\[0.2cm]
        \betabd_2&=(&0,& 1,&-1,&0,&\cd,&0,&0,&0)^T,\\[0.2cm]
        &\vd&&&&&\\[0.2cm]
        \betabd_{n-1}&=(&0,&0,&0,&0,&\cd,&0,&1,&-1)^T,\\[0.2cm]
        \betabd_{n}&=(&0,&0,&0,&0,&\cd,&0,&0,&1)^T.
      \end{array}
      \end{equation}
      求向量组$\alphabd=(a_1,a_2,\cd,a_n)^T$分别在两组基下的坐标。
    \end{li}
  
\end{frame}


\begin{frame}
  
    \begin{dingli}
      设$B=\{\alphabd_1,\alphabd_2,\cd,\alphabd_n\}$是$\MR^n$的一组基,且
      $$
      \left\{
      \begin{array}{l}
        \etabd_1=a_{11}\alphabd_1+a_{21}\alphabd_2+\cd+a_{n1}\alphabd_n,\\[0.2cm]
        \etabd_2=a_{12}\alphabd_1+a_{22}\alphabd_2+\cd+a_{n2}\alphabd_n,\\[0.2cm]
        \cd\cd\\[0.2cm]
        \etabd_n=a_{1n}\alphabd_1+a_{2n}\alphabd_2+\cd+a_{nn}\alphabd_n.
      \end{array}
      \right.
      $$
      则$\etabd_1,\etabd_2,\cd,\etabd_n$线性无关的充要条件是
      $$
      \det\MA=\left|
      \begin{array}{cccc}
        a_{11}&a_{12}&\cd&a_{1n}\\
        a_{21}&a_{22}&\cd&a_{2n}\\
        \vd&\vd&&\vd\\
        a_{n1}&a_{n2}&\cd&a_{nn}
      \end{array}
      \right|\ne 0.
      $$
    \end{dingli}
  
\end{frame}


\begin{frame}
  
    设$\MR^n$的两组基$B_1=\{\alphabd_1,\alphabd_2,\cd,\alphabd_n\}$和$B_2=\{\etabd_1,\etabd_2,\cd,\etabd_n\}$满足关系式
    $$
    (\etabd_1,\etabd_2,\cd,\etabd_n)=(\alphabd_1,\alphabd_2,\cd,\alphabd_n)\left(
    \begin{array}{cccc}
      a_{11}&a_{12}&\cd&a_{1n}\\
      a_{21}&a_{22}&\cd&a_{2n}\\
      \vd&\vd&&\vd\\
      a_{n1}&a_{n2}&\cd&a_{nn}
    \end{array}
    \right)
    $$
    则矩阵
    $$
    \MA=\left(
    \begin{array}{cccc}
      a_{11}&a_{12}&\cd&a_{1n}\\
      a_{21}&a_{22}&\cd&a_{2n}\\
      \vd&\vd&&\vd\\
      a_{n1}&a_{n2}&\cd&a_{nn}
    \end{array}
    \right)
    $$
    称为由旧基$B_1$到新基$B_2$的过渡矩阵。
  
\end{frame}

\begin{frame}
  
    \begin{dingli}
      设$\alphabd$在两组基$B_1=\{\alphabd_1,\alphabd_2,\cd,\alphabd_n\}$与$B_2=\{\etabd_1,\etabd_2,\cd,\etabd_n\}$的坐标分别为
      $$
      \vx=(x_1,x_2,\cd,x_n)^T\mbox{~~和~~}\vy=(y_1,y_2,\cd,y_n)^T
      $$
      基$B_1$到$B_2$的过渡矩阵为$\MA$,则
      $$
      \MA\vy=\vx\mbox{~~或~~}\vy=\MA^{-1}\vx
      $$
    \end{dingli}
  
\end{frame}


\begin{frame}
  
    \begin{li}
      已知$\MR^3$的一组基为$B_2=\{\betabd_1,\betabd_2,\betabd_3\}$,其中
      $$\betabd_1=(1,2,1)^T,\betabd_2=(1,-1,0)^T,\betabd_3=(1,0,-1)^T,$$
      求自然基$B_1$到$B_2$的过渡矩阵。
    \end{li}
  
\end{frame}


\begin{frame}
  
    \begin{li}
      已知$\MR^3$的两组基为$B_1=\{\alphabd_1,\alphabd_2,\alphabd_3\}$和$B_2=\{\betabd_1,\betabd_2,\betabd_3\}$,
      其中
      $$
      \begin{array}{lll}
        \alphabd_1=(1,1,1)^T,&\alphabd_2=(0,1,1)^T,&\alphabd_3=(0,0,1)^T, \\[0.2cm]
        \betabd_1=(1,0,1)^T,&\betabd_2=(0,1,-1)^T,&\betabd_3=(1,2,0)^T.  
      \end{array}
      $$
      \begin{itemize}
      \item[(1)] 求基$B_1$到$B_2$的过渡矩阵。
      \item[(2)] 已知$\alpha$在基$B_1$的坐标为$(1,-2,-1)^T$,求$\alphabd$在基$B_2$下的坐标。
      \end{itemize}
      
    \end{li}
  
\end{frame}


