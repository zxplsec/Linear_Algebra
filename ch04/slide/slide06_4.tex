\subsection{线性变换的象(值域)与核}

\begin{frame}
  \begin{dingyi}
    设$\sigmabd$是线性空间$V(F)$的一个线性变换,
    \begin{itemize}
      \item 把$V$中所有元素在$\sigmabd$下的象所组成的集合
        $$
        \sigmabd(V)=\{\betabd|\betabd=\sigmabd(\alphabd), \alphabd\in V\}
        $$
        称为$\sigmabd$的\red{象或值域},记为$\Im\sigmabd$;
      \item
        $V$的零元$\M0$在$\sigmabd$下的完全原象
        $$
        \sigmabd^{-1}(\M0)=\{\alphabd|\sigmabd(\alphabd)=\M0, ~~\alphabd \in V\}
        $$
        称为$\sigmabd$的核,记为$\Ker\sigmabd$。
      \end{itemize}
  \end{dingyi}
\end{frame}

\begin{frame}
  \begin{li}
    $\R^2$上旋转矩阵$\MR_\theta$与镜像变换$\varphibd$的值域都是$\R^2$自身,它们的核都只含一个零向量$\{\M0\}$。
  \end{li}
\end{frame}

\begin{frame}
  \begin{itemize}
  \item[(1)] \blue{$\sigmabd(V)$(或$\Im \sigmabd$)}是线性空间$V(F)$的一个子空间。   
  \end{itemize} \pause 

  \begin{proof}
    由$\sigmabd(\M0)=\M0$可知$\sigmabd(V)$是一个非空集合,且$\forall \betabd_1,\betabd_2\in \sigmabd(V)$,$\exists \alphabd_1,\alphabd_2\in V$,使得$\sigmabd(\alphabd_1)=\betabd_1,\sigmabd(\alphabd_2)=\betabd_2$,于是$\forall \lambda_1,\lambda_2\in F$,有
    $$
    \lambda_1\betabd_1+\lambda_2\betabd_2=\lambda_1\sigmabd(\alphabd_1)+\lambda_2\sigmabd(\alphabd_2)=\sigmabd(\lambda_1\alphabd_1+\lambda_2\alphabd_2)\in \sigmabd(V)
    $$
    所以,$\sigmabd(V)$是$V(F)$的一个子空间。
  \end{proof}
\end{frame}

\begin{frame}
  \begin{itemize}
    \item[(2)] \blue{$\sigmabd^{-1}(\M0)$(或$\Ker \sigmabd$)}也是线性空间$V(F)$的一个子空间。   
  \end{itemize} \pause 

  \begin{proof}
    因$\sigmabd^{-1}(\M0)$不是空集,且$\forall \alphabd_1,\alphabd_2\in \sigmabd^{-1}(\M0)$和$\forall \lambda_1,\lambda_2\in F$,均有
    $$
    \sigmabd(\lambda_1\alphabd_1+\lambda_2\alphabd_2)=\lambda_1\sigmabd(\alphabd_1)+\lambda_2\sigmabd(\alphabd_2)=\lambda_1\M0+\lambda_2\M0=\M0
    $$
    即$\lambda_1\alphabd_1+\lambda_2\alphabd_2\in\sigmabd^{-1}(\M0)$,故$\sigmabd^{-1}(\M0)$是$V(F)$的子空间。
  \end{proof}
\end{frame}

\begin{frame}
  \begin{itemize}
    \item[(3)] 线性变换$\sigmabd$是单射的充分必要条件是$\sigmabd^{-1}(\M0) = \{\M0\}$。   
  \end{itemize}\pause 

  \begin{proof}
    \begin{itemize}
    \item[$\Rightarrow$] 因$\sigmabd$是单射,则$\forall \alphabd \in V$,若$\sigmabd(\alphabd)=\M0=\sigmabd(\M0)$,则$\alphabd=\M0$,故$\sigmabd^{-1}(\M0) = \{\M0\}$;\\[.1in] \pause 
    \item[$\Leftarrow$] 由$\sigmabd^{-1}(\M0) = \{\M0\}$可得$\forall \alphabd_1,\alphabd_2\in V$,若$\sigmabd(\alphabd_1)=\sigmabd(\alphabd_2)$,即$\sigmabd(\alphabd_1)-\sigmabd(\alphabd_2)=\sigmabd(\alphabd_1-\alphabd_2)=\M0$,则$\alphabd_1-\alphabd_2=0$,即$\alphabd_1=\alphabd_2$,故$\sigmabd$为单射。
    \end{itemize}
  \end{proof}
\end{frame}


\begin{frame}
  \begin{itemize}
    \item $\dim \sigmabd(V)$称为$\sigmabd$的秩,记作$\rank(\sigmabd)$;\\[.15in]
    \item $\dim \sigmabd^{-1}(\M0)$称为$\sigmabd$的零度,记作$\mathcal N(\sigmabd)$。
    \end{itemize}
\end{frame}

\begin{frame}
  \begin{dingli}
    设线性空间$V(F)$的维数为$n$,$\sigmabd$是$V(F)$的一个线性变换,则
    $$
    \dim \sigmabd(V)+\dim \sigmabd^{-1}(\M0)=n.
    $$
  \end{dingli}\vspace{.1in}\pause 

  \begin{proof}
    设$\dim \sigma^{-1}(\M0)=k, B_1=\{\alphabd_1,\cd,\alphabd_k\}$是核$\sigmabd^{-1}(\M0)$的一组基,把$B_1$扩充到$V$的基
    $$
    B=\{\alphabd_1,\cd,\alphabd_k,\red{\alphabd_{k+1},\cd,\alphabd_n}\}.
    $$\vspace{.1in}\pause 
    
    由于$\forall \alphabd=x_1\alphabd_1+\cd+x_n\alphabd_n\in V$,有$\sigmabd(\alphabd)=x_1\sigmabd(\alphabd_1)+\cd+x_n\sigmabd(\alphabd_n)$,故$\sigmabd$的值域是$\sigmabd$关于$V$的基象生成的子空间,即
    $$
    \sigmabd(V)=L(\sigmabd(\alphabd_1),\cd,\sigmabd(\alphabd_k),\sigmabd(\alphabd_{k+1}),\cd,\sigmabd(\alphabd_n))
    $$
    再由$\sigmabd(\alphabd_i)=\M0(i=1,\cd,k)$得
    $$
    \sigmabd(V)=L(\sigmabd(\alphabd_{k+1}),\cd,\sigmabd(\alphabd_n)).
    $$
    因此,只需证明\blue{$\dim \sigmabd(V)=n-k$,即$\{\sigmabd(\alphabd_{k+1}),\cd,\sigmabd(\alphabd_n)\}$线性无关。}
  \end{proof}
  
\end{frame}

\begin{frame}
  \begin{proof}[续]
    设
    $$
    c_{k+1}\sigmabd(\alphabd_{k+1})+\cd+c_n\sigmabd(\alphabd_n)=\M0
    $$
    即
    $$
    \sigmabd(c_{k+1}\alphabd_{k+1}+\cd+c_n\alphabd_n)=\M0
    $$
    故$c_{k+1}\alphabd_{k+1}+\cd+c_n\alphabd_n\in \sigmabd^{-1}(\M0)$,因此它可被$B_1$线性表示,于是
    $$
    c_1\alphabd_1+\cd+c_k\alphabd_k-c_{k+1}\alphabd_{k+1}-\cd-c_n\alphabd_n=\M0
    $$
    从而有
    $$
    c_1=\cd=c_k=c_{k+1}=\cd=c_n,
    $$
    故$\{\sigmabd(\alphabd_{k+1}),\cd,\sigmabd(\alphabd_n)\}$线性无关。

  \end{proof}
  
\end{frame}

\begin{frame}
  由于线性变换$\sigmabd$的值域$\sigmabd(V)$是$\sigmabd$关于$V$的基$\{\alphabd_1,\cd,\alphabd_n\}$的象$\sigmabd(\alphabd_1),\cd,\sigmabd(\alphabd_n)$的生成子空间,故
  $$
  \dim \sigmabd(V) = \rank\{\sigmabd(\alphabd_1),\cd,\sigmabd(\alphabd_n)\}.
  $$\pause 
  而
  $$
  (\sigmabd(\alphabd_1),\cd,\sigmabd(\alphabd_n))=(\alphabd_1,\cd,\alphabd_n)\MA,
  $$
  可以证明:\red{基象组$\{\sigmabd(\alphabd_1),\cd,\sigmabd(\alphabd_n)\}$与$\MA$的列向量组有相同的线性相关性。} \pause 

  于是有
  $$
  \rank\{\sigmabd(\alphabd_1),\cd,\sigmabd(\alphabd_n)\}=\rank(\MA),
  $$
  从而
  $$
  \dim \sigmabd(V) = \rank(\MA).
  $$

\end{frame}

\begin{frame}
  以下证明:\red{基象组$\{\sigmabd(\alphabd_1),\cd,\sigmabd(\alphabd_n)\}$与$\MA$的列向量组有相同的线性相关性。} \vspace{.1in}\pause

  设
  $$
  x_1\sigmabd(\alphabd_1)+\cd+x_n\sigmabd(\alphabd_n)=\M0
  $$
  即
  $$
  \sum_{j=1}^n x_j\sigmabd(\alphabd_j)=\sum_{j=1}^n x_j\sum_{i=1}^na_{ij}\alphabd_i=\sum_{i=1}^n\left(\sum_{j=1}^na_{ij}x_j\right)\alphabd_i=\M0
  $$
  由$\alphabd_1,\cd,\alphabd_n$线性无关可得
  $$
  \sum_{j=1}^na_{ij}x_j=0, ~~~i=1,2,\cd,n
  $$
  即
  $$
  \MA\vx=\M0.
  $$
\end{frame}


\begin{frame}
  \begin{itemize}
    \item 若$\sigmabd(\alphabd_1),\cd,\sigmabd(\alphabd_n)$线性无关,则$x_1=\cd=x_n=0$,即$\MA\vx=\M0$只有零解,故$\MA$的列向量组线性无关。
    \item[] 反之亦然。 \\[0.1in]

    \item 若$\sigmabd(\alphabd_1),\cd,\sigmabd(\alphabd_n)$线性无关,说明$\MA\vx=\M0$有非零解,即$\MA$的列向量组线性相关。
    \item[] 反之亦然。
  \end{itemize}
\end{frame}


\begin{frame}
  若线性变换$\sigmabd$在$B=\{\alphabd_1,\cd,\alphabd_n\}$下对应的矩阵为$\MA$,则核$\sigmabd^{-1}(\M0)$中任一向量$\alphabd=x_1\alphabd_1+\cd+x_n\alphabd_n$在基$B$下的坐标向量$(x_1,\cd,x_n)^T$,就是$\MA\vx=\M0$的解向量。因此,$\MA\vx=\M0$的解空间$\mathcal N(\MA)$的维数等于核$\sigmabd^{-1}(\M0)$的维数,即
  $$
  \red{\dim \sigmabd^{-1}(\M0)=\dim \mathcal N(\MA). }
  $$
\end{frame}




% \begin{frame}
  
% \end{frame}
