\section{向量空间的线性变换}

\begin{frame}
  \begin{dingyi}[映射]
    设$X,Y$是两个非空集合,如果有一个法则$\sigma$,它使$X$中每个元素$\alpha$都有$Y$中唯一确定的一个元素$\beta$与之对应,则称$\sigma$是$X$到$Y$的一个映射,记作
    $$
    \sigma: X \to Y,
    $$
    并称$\beta$为$\alpha$在$\sigma$下的象,$\alpha$为$\beta$在$\sigma$下一个原象,记作
    $$
    \sigma: \alpha \mapsto \beta \mbox{ 或 } \sigma(\alpha)=\beta.
    $$    
  \end{dingyi} \vspace{.1in} \pause 

  \begin{zhu}
    $\alpha$的象是唯一的,但$\beta$的原象不是唯一的。
  \end{zhu}\vspace{.1in} \pause 

  \blue{由$X$到自身的映射$\sigma$,常称之为\red{变换}。}
\end{frame}

\begin{frame}
  \begin{itemize}
    \item 若$\forall \alpha_1,\alpha_2\in X, \alpha_1\ne \alpha_2$,
      都有$\sigma(\alpha_1)\ne \sigma(\alpha_2)$,就称$\sigma$为\red{单射}。\\[0.15in]
    \item 若$\forall \beta\in Y$,都有$\alpha \in X$,使得$\sigma(\alpha)=\beta$,则称$\sigma$为\red{满射}。\\[0.15in]
    \item 若$\sigma$既是单射,也是满射,则称$\sigma$为\red{双射}(或称\red{一一对应})。
  \end{itemize}
\end{frame}

\begin{frame}
  \begin{li}
    \begin{itemize}
    \item $f(x)=\sin x$是$\R \to [-1,1]$的一个映射,它是满射,但不是单射。\\[0.1in]
    \item $f(x)=e^x$是$\R \to \R$的一个映射,它是单射而不是满射。
    \end{itemize}
  \end{li}
\end{frame}

\begin{frame}
  线性函数
  $$
  y = f(x) = ax
  $$
  是$\R\to \R$的映射。显然它是双射,且具有以下性质
  \begin{enumerate}
  \item $f(x_1+x_2)=f(x_1)+f(x_2)$\\[0.1in]
  \item $f(\lambda x) = \lambda f(x), \lambda$为常数。
  \end{enumerate}
\end{frame}

\begin{frame}
  现在把一元线性函数推广到$n$维向量空间,设$\MA\in \R^{n\times n}$,若对每一个列向量$\vx\in \R^n$,映射
  $$
  \sigma: \vx \to \MA\vx \mbox{  即  } \sigma(\vx) = \MA \vx
  $$
  是$\R^n\to\R^n$的一个映射,且满足以下性质
  $$
  \begin{array}{rl}
    \sigma(\vx_1+\vx_2)&=\MA(\vx_1+\vx_2)=\MA\vx_1+\MA\vx_2=\sigma(\vx_1)+\sigma(\vx_2), \\[.15in]
    \sigma(\lambda\vx)&=\MA(\lambda\vx) = \lambda\MA\vx=\lambda\sigma(\vx), ~\lambda\in R.
  \end{array}
  $$\pause 
  我们把这个映射$\sigma$称为$\R^n\to\R^n$的\red{线性映射}(也称\red{线性变换})。
\end{frame}

\begin{frame}
更一般地,若$\MA\in \R^{m\times n}, \vx \in \R^n$,则映射
$$
\sigma: \vx \to \MA\vx \in \R^m
$$
是$\R^n\to \R^m$的线性映射。 \vspace{.1in}\pause 

\begin{li}
  二元线性函数
  $$
  y = f(x_1,x_2)=a_1x_1+a_2x_2=(a_1,a_2)\left(\begin{array}{l}x_1\\x_2\end{array}\right)
  $$
  是$\R^2\to \R^1$的线性映射。
\vspace{.1in}\pause 

\red{本节主要讨论$\R^n\to \R^n$的线性映射(也称$\R^n$的线性变换)。}

\end{li}
\end{frame}

\subsection{线性变化的定义及其性质}
\begin{frame}
\begin{dingyi}[线性变换]
  设$V(F)$是一个向量空间,若$V(F)$的一个变换$\sigmabd$满足条件:$\forall \alpha,\beta\in V$和$\lambda\in F$,
  \begin{enumerate}
    \item $\sigmabd(\alphabd+\betabd) = \sigmabd(\alphabd) + \sigmabd(\betabd)$\\[0.1in]
    \item $\sigmabd(\lambda\alphabd) = \lambda\sigmabd(\alphabd)$
  \end{enumerate}
  就称$\sigmabd$是$V(F)$的一个\red{线性变换},并称$\sigmabd(\alphabd)$为$\alphabd$的象,$\alphabd$为$\sigmabd(\alphabd)$的原象。
\end{dingyi}
\vspace{.1in}\pause 

线性运算等价于:$\forall \alphabd,\betabd\in V$和$\lambda, \mu \in F$,有
$$
\sigmabd(\lambda\alphabd+\mu\betabd) = \lambda\sigmabd(\alphabd)+\mu\sigmabd(\betabd).
$$
\end{frame}

\begin{frame}
  \begin{li}[旋转变换]
    $\R^2$中每个向量绕原点按逆时针方向旋转$\theta$角的变换$\MR_\theta$是$\R^2$的一个线性变换。即$\forall\alphabd=(x,y)\in\R^2$,
    $$
    \MR_\theta(x,y)=\MR_\theta(\alphabd)=\alphabd^\prime=(x^\prime,y^\prime),
    $$
    其中$|\alphabd|=r$,而
    $$
    \left\{
      \begin{array}{ll}
        x^\prime&=r\cos(\beta+\theta)=r\cos\beta\cos\theta-r\sin\beta\sin\theta=x\cos\theta-y\sin\theta,\\
        y^\prime&=r\sin(\beta+\theta)=r\sin\beta\cos\theta+r\cos\beta\sin\theta=y\cos\theta+x\sin\theta.
      \end{array}
    \right.,
    $$
    于是,$\forall \alphabd_1=(x_1,y_1), \alphabd_2=(x_2,y_2)\in \R^2$和$\forall \lambda,\mu\in \R$,有
    $$
    \begin{aligned}
      &\MR_\theta(\lambda\alphabd_1+\mu\alphabd_2)\\
      =&\MR_\theta(\lambda x_1+\mu x_2, \lambda y_1+\mu y_2) \\
      =&(\blue{(\lambda x_1+\mu x_2)\cos\theta-(\lambda y_1+\mu y_2)\sin\theta}, \red{(\lambda x_1+\mu x_2)\sin\theta+(\lambda y_1+\mu y_2)\cos\theta})\\
      =&\lambda(x_1\cos\theta-y_1\sin\theta,x_1\sin\theta+y_1\cos\theta)+\mu(x_2\cos\theta-y_2\sin\theta,x_2\sin\theta+y_2\cos\theta)\\
      =&\lambda\MR_\theta(x_1,y_1)+\mu\MR_\theta(x_2,y_2)\\
      =&\lambda\MR_\theta(\alphabd_1)+\mu\MR_\theta(\alphabd_2),
    \end{aligned}
    $$
    故$\MR_\theta$是$\R^2$的一个线性变换。

  \end{li}
\end{frame}

\begin{frame}
  \begin{li}[镜像变换(镜像反射)]
    $\R^2$中每个向量关于过原点的直线$L$(看做镜面)相对称的变换$\phibd$也是$\R^2$的一个线性变换,即
    $$
    \phibd(\alphabd)=\alphabd^\prime.
    $$
  \end{li}
\end{frame}

\begin{frame}
  \begin{li}[投影变换]
    把$\R^3$中向量$\alphabd=(x_1,x_2,x_3)$投影到$xOy$平面上的向量$\betabd=(x_1,x_2,0)$的投影变换$P(\alphabd)=\betabd$,即
    $$
    \MP(x_1,x_2,x_3)=(x_1,x_2,0)
    $$
    是$\R^2$的一个线性变换。
  \end{li}
\end{frame}

\begin{frame}
  \begin{li}[恒等变换、零变换、数乘变换]
    \begin{itemize}
    \item 恒等变换$\sigmabd(\alphabd)=\alphabd, ~~\forall \alphabd\in\R^n$
    \item 零变换 $\sigmabd(\alphabd)=0, ~~\forall \alphabd\in\R^n$
    \item 数乘变换$\sigmabd(\alphabd)=\lambda\alphabd, ~~\forall \alphabd\in\R^n$
    \end{itemize}
  \end{li}
\end{frame}

\begin{frame}
  \begin{li}
    $\R^3$中定义变换
    $$
    \sigmabd(x_1,x_2,x_3)=(x_1+x_2,x_2-4x_3,2x_3),
    $$
    则$\sigmabd$是$\R^3$的一个线性变换。
  \end{li}
\end{frame}

\begin{frame}
  \begin{li}
    $\R^3$中定义变换
    $$
    \sigmabd(x_1,x_2,x_3)=(x_1^2,x_2+x_3,x_2),
    $$
    则$\sigmabd$不是$\R^3$的一个线性变换。
  \end{li}
\end{frame}

\begin{frame}
  对于$\R^n$的变换
  $$
  \sigmabd(x_1,x_2,\cd,x_n)=(y_1,y_2,\cd,y_n)
  $$
  \begin{itemize}
  \item 当$y_i$都是$x_1,x_2,\cd,x_n$的线性组合时,$\sigmabd$是$\R^n$的线性变换。\\[0.1in]
  \item 当$y_i$有一个不是$x_1,x_2,\cd,x_n$的线性组合时,$\sigmabd$不是$\R^n$的线性变换。 \\[0.1in]
  \item[] 上例中,$y_1=x_1^2$,故不是线性变换。
  \end{itemize}
\end{frame}


\begin{frame} \ft{线性变换的简单性质}
  对于数域$F$上的向量空间$V$中的线性变换$\sigma$
  
  \begin{itemize}
  \item $\sigmabd(\M0)=\M0, \quad \sigmabd(-\alphabd)=\sigmabd(\alphabd), \quad \forall\alphabd\in V$
  \end{itemize}
\end{frame}

\begin{frame} \ft{线性变换的简单性质}
    对于数域$F$上的向量空间$V$中的线性变换$\sigma$

  \begin{itemize}
  \item 若$\alphabd=k_1\alphabd_1+k_2\alphabd_2+\cd+k_n\alphabd_n, \quad k_i\in F, \quad \alphabd_i\in V$,则
    $$
    \sigma(\alphabd)=k_1\sigma(\alphabd_1)+k_2\sigma(\alphabd_2)+\cd+k_n\sigma(\alphabd_n).
    $$
  \end{itemize}
\end{frame}

\begin{frame} \ft{线性变换的简单性质}
    对于数域$F$上的向量空间$V$中的线性变换$\sigma$

  \begin{itemize}
  \item 若$\alphabd_1, \alphabd_2, \cd, \alphabd_n$线性相关,则其象向量组$\sigma(\alphabd_1),\sigma(\alphabd_n),\cd,\sigma(\alphabd_n)$也线性相关。
  \end{itemize}

  \vspace{.1in}
  \pause 

  \begin{zhu}
    但$\alphabd_1, \alphabd_2, \cd, \alphabd_n$线性无关,不能推导出$\sigma(\alphabd_1),\sigma(\alphabd_n),\cd,\sigma(\alphabd_n)$也线性无关。如
    $$
    \alphabd_1=(1,1,2)^T, \quad \alphabd_2=(2,2,2)^T
    $$
    线性无关,而
    $$
    \MP(\alphabd_1)=(1,1,0)^T, \quad \MP(\alphabd_2)=(2,2,0)^T
    $$
    线性相关。
  \end{zhu}
\end{frame}

\subsection{线性变换的矩阵表示}

\begin{frame}
  设$\{\alphabd_1,\alphabd_2,\cd,\alphabd_n\}$是$V(F)$的一组基,$\sigmabd$是$V(F)$的一个线性变换,若$\alphabd\in V(F)$,且
  $$
  \alphabd=x_1\alphabd_1+x_2\alphabd_2+\cd+x_n\alphabd_n,
  $$
  则
  $$
  \sigmabd(\alphabd)=x_1\sigmabd(\alphabd_1)+x_2\sigmabd(\alphabd_2)+\cd+x_n\sigmabd(\alphabd_n).
  $$
  这意味着,如果知道了$\sigmabd$关于$V(F)$的基的象$\sigmabd(\alphabd_1),\sigmabd(\alphabd_2),\cd,\sigmabd(\alphabd_n)$,则任一向量$\alphabd$的象$\sigmabd(\alphabd)$就知道了。
\end{frame}

\begin{frame}
  \begin{dingli}
    设$\{\alphabd_1,\alphabd_2,\cd,\alphabd_n\}$是$V(F)$的一组基,若$V(F)$的两个线性变换$\sigmabd$和$\taubd$关于这组基的象相同,即
    $$
    \sigmabd(\alphabd_i)=\taubd(\alphabd_i), \quad i=1,2,\cd,n,
    $$
    则$\sigmabd=\taubd$.
  \end{dingli}
  \vspace{.1in}\pause 

  \begin{proof}
    所谓$\sigmabd=\taubd$,即每个向量在它们的作用下的象相同,即对任意的$\alphabd\in V$,有$\sigmabd(\alphabd)=\taubd(\alphabd)$。\vspace{.1in}\pause 

    对任一的$\alphabd=x_1\alphabd_1+x_2\alphabd_2+\cd+x_n\alphabd_n$,则
    $$
    \begin{aligned}
      \sigmabd(\alphabd)
      &=x_1\sigmabd(\alphabd_1)+x_2\sigmabd(\alphabd_2)+\cd+x_n\sigmabd(\alphabd_n)\\
      &=x_1\taubd(\alphabd_1)+x_2\taubd(\alphabd_2)+\cd+x_n\taubd(\alphabd_n)\\
      &=\taubd(\alphabd)
  \end{aligned}
    $$
  \end{proof}
\end{frame}

\begin{frame}
  因$\sigmabd(\alphabd_i)\in V(F)$,故它们可由$V(F)$的基$\{\alphabd_1,\alphabd_2,\cd,\alphabd_n\}$线性表出,即有
  $$
  \left\{
    \begin{array}{c}
      \sigmabd(\alphabd_1)=a_{11}\alphabd_{1}+a_{21}\alphabd_{12}+\cd+a_{n1}\alphabd_{n}, \\[0.1in]
      \sigmabd(\alphabd_1)=a_{12}\alphabd_{1}+a_{22}\alphabd_{22}+\cd+a_{n2}\alphabd_{n}, \\[0.1in]
      \cd\cd\\[0.1in]
      \sigmabd(\alphabd_1)=a_{1n}\alphabd_{1}+a_{2n}\alphabd_{22}+\cd+a_{nn}\alphabd_{n}.
    \end{array}
  \right.
  $$
  记
  $$
  \sigmabd(\alphabd_1,\alphabd_2,\cd,\alphabd_n)=(\sigmabd(\alphabd_1),\sigmabd(\alphabd_2),\cd,\sigmabd(\alphabd_n))
  $$
  其矩阵形式为
  \begin{equation}\label{a_sigma}
  \sigmabd(\alphabd_1,\alphabd_2,\cd,\alphabd_n)=(\alphabd_1,\alphabd_2,\cd,\alphabd_n)\underbrace{\left[
    \begin{array}{cccc}
      a_{11}&a_{12}&\cd&a_{1n}\\
      a_{21}&a_{22}&\cd&a_{2n}\\
      \vdots&\vdots&&\vdots\\
      a_{n1}&a_{n2}&\cd&a_{nn}
    \end{array}
  \right]}_{\red{\MA}}.
  \end{equation}

\end{frame}

\begin{frame}
  \begin{dingyi}
    若$V(F)$中的线性变换$\sigmabd$,使得$V(F)$的基$\{\alphabd_1,\alphabd_2,\cd,\alphabd_n\}$和$\sigmabd$关于基的象$\sigmabd(\alphabd_1),\sigmabd(\alphabd_2),\cd,\sigmabd(\alphabd_n)$满足\eqref{a_sigma},就称\eqref{a_sigma}中的\red{$\MA$是$\sigma$在基$\{\alphabd_1,\alphabd_2,\cd,\alphabd_n\}$的矩阵表示},或称\red{$\MA$是$\sigmabd$在基$\{\alphabd_1,\alphabd_2,\cd,\alphabd_n\}$下对应的矩阵}。
  \end{dingyi}
\end{frame}

\begin{frame}
  \begin{dingli}
    设$V(F)$的线性变换$\sigmabd$在基$\{\alphabd_1,\cd,\alphabd_n\}$下的矩阵为$\MA$,向量$\alphabd$在基下的坐标向量为$\vx=(x_1,\cd,x_n)^T$,$\sigmabd(\alphabd)$在基下的坐标向量为$\vy=(y_1,\cd,y_n)^T$,则
    $$
    \red{\vy=\MA\vx.}
    $$
  \end{dingli} \vspace{.1in} \pause 

  \begin{proof}
    由
    $$
    \alphabd=x_1\alphabd_1+\cd+x_n\alphabd_n=(\alphabd_1,\cd,\alphabd_n)\left(\begin{array}{c}x_1\\\vdots\\x_n\end{array}\right)
    $$
    可得
    $$
    \begin{aligned}
      \sigmabd(\alphabd)&=x_1\sigmabd(\alphabd_1)+\cd+x_n\sigmabd(\alphabd_n)\\
      &=\blue{(\sigmabd(\alphabd_1),\cd,\sigmabd(\alphabd_n))}\left(\begin{array}{c}x_1\\\vdots\\x_n\end{array}\right)\\
      &=\blue{(\alphabd_1,\cd,\alphabd_n)\MA}\left(\begin{array}{c}x_1\\\vdots\\x_n\end{array}\right)
    \end{aligned}
    $$
    故$\sigmabd(\alphabd)$在基$\{\alphabd_1,\cd,\alphabd_n\}$下的坐标为
    $$
    \left(\begin{array}{c}y_1\\\vdots\\y_n\end{array}\right)=\MA\left(\begin{array}{c}x_1\\\vdots\\x_n\end{array}\right)
    $$
  \end{proof}
\end{frame}

\begin{frame}
   \begin{li}
    求旋转变换$\MR_\theta$在$\R^2$的标准正交基$\ve_1=(1,0)^T$和$\ve_2=(0,1)^T$的矩阵。
  \end{li} \pause 
  \begin{jie}
    $$
    \left\{
      \begin{array}{rcr}
        \MR_\theta(\ve_1)&=&\cos\theta \ve_1 + \sin\theta \ve_2, \\[0.1in]
        \MR_\theta(\ve_2)&=&-\sin\theta \ve_1 + \cos\theta \ve_2.
      \end{array}
    \right.
    $$
    即
    $$
    \MR_\theta(\ve_1,\ve_2)=(\MR_\theta(\ve_1),\MR_\theta(\ve_2))=(\ve_1,\ve_2)\left(
      \begin{array}{rr}
        \cos\theta&-\sin\theta\\
        \sin\theta& \cos\theta
      \end{array}
    \right)
    $$
    故初等旋转变换$\MR_\theta$在标准正交基$\{\ve_1,\ve_2\}$下的矩阵为
    $$
    \left(
      \begin{array}{rr}
        \cos\theta&-\sin\theta\\
        \sin\theta& \cos\theta
      \end{array}
    \right).
    $$
  \end{jie}
\end{frame}

\begin{frame}
  \begin{li}
    求镜像变换$\varphibd$在$\R^2$的标准正交基$\{\omegabd,\etabd\}$下所对应的矩阵$\MH$。
  \end{li}\vspace{.1in}\pause 

  \begin{jie}
    根据镜像变换的定义,有
    $$
    \left\{
      \begin{array}{rcr}
        \varphibd(\omegabd)&=& \omegabd,\\[.1in]
        \varphibd(\etabd)&=& -\etabd
      \end{array}
    \right. \mbox{  即  }
    \varphibd(\omegabd,\etabd)=(\omegabd,\etabd)\left(
      \begin{array}{rr}
        1&0\\
        0&-1
      \end{array}
    \right).   
    $$
    所以$\varphibd$在标准正交基$\{\omegabd,\etabd\}$下的矩阵为
    $$
    \left(
      \begin{array}{rr}
        1&0\\
        0&-1
      \end{array}
    \right).
    $$
  \end{jie}
\end{frame}

\begin{frame}
  \begin{li}
    $\R^n$的恒等变换、零变换和数乘变换在任何基下的矩阵分别都是$\MI_n, \M0_{n},\lambda \MI_n $。
  \end{li}
\end{frame}

\begin{frame}
  \begin{li}
    设$\sigmabd$是$\R^3$的一个线性变换,$B=\{\alphabd_1,\alphabd_2,\alphabd_3\}$是$\R^3$的一组基,已知
    $$
    \begin{array}{rrr}
      \alphabd_1=(1,0,0)^T,&\alphabd_2=(1,1,0)^T,&\alphabd_3=(1,1,1)^T,\\
      \sigmabd(\alphabd_1)=(1,-1,0)^T,&\sigmabd(\alphabd_2)=(-1,1,-1)^T,&\sigmabd(\alphabd_3)=(1,-1,2)^T.
    \end{array}
    $$
    \begin{enumerate}
      \item 求$\sigmabd$在基$B$下对应的矩阵;
      \item 求$\sigmabd^2(\alphabd_1),\sigmabd^2(\alphabd_2),\sigmabd^2(\alphabd_3)$;
      \item 已知$\sigmabd(\betabd)$在基$B$下的坐标为$(2,1,-2)^T$,问$\sigmabd(\betabd)$的原象$\betabd$是否唯一?并求$\betabd$在基$B$下的坐标。
    \end{enumerate}
  \end{li}

\end{frame}

\begin{frame}
  \begin{jie}
    1. 由$\sigmabd(\alphabd_1,\sigmabd_2,\sigmabd_3)=(\alphabd_1,\sigmabd_2,\sigmabd_3)\MA$可知
    $$
    \left(
      \begin{array}{rrr}
        1&-1&1\\
        -1&1&-1\\
        0&-1&2
      \end{array}
    \right)=\left(
      \begin{array}{rrr}
        1&1&1\\
        0&1&1\\
        0&0&1
      \end{array}
    \right)\MA
    $$
    可求得
    $$
    \MA=\left(
      \begin{array}{rrr}
        2&-2&2\\
        -1&2&-3\\
        0&1&2
      \end{array}
    \right)
    $$
  \end{jie}
\end{frame}

\begin{frame}
  \begin{jie}
    2. 由
    $$
    \sigmabd(\alphabd_1,\sigmabd_2,\sigmabd_3)=(\sigmabd(\alphabd_1),\sigmabd(\sigmabd_2),\sigmabd(\sigmabd_3))=(\alphabd_1,\sigmabd_2,\sigmabd_3)\MA
    $$
    可知
    $$
    \begin{aligned}
      \sigmabd(\sigmabd(\alphabd_1),\sigmabd(\sigmabd_2),\sigmabd(\sigmabd_3))
      &=\sigmabd((\alphabd_1,\sigmabd_2,\sigmabd_3)\MA)\\
      &=(\sigmabd(\alphabd_1,\sigmabd_2,\sigmabd_3))\MA
      =(\alphabd_1,\sigmabd_2,\sigmabd_3)\MA^2\\
      &=(\alphabd_1,\sigmabd_2,\sigmabd_3)\left(
      \begin{array}{rrr}
        6&-10&14\\
        -4&9&-14\\
        1&-4&7
      \end{array}
    \right)
    \end{aligned}
    $$
  \end{jie}
\end{frame}



\begin{frame}
\begin{jie}
  3. 设$(\betabd)_B=(x_1,x_2,x_3)^T$,则
  $$
  \left(
    \begin{array}{rrr}
      2&-2&2\\
      -1&2&-3\\
      0&1&2
    \end{array}
  \right)\left(
    \begin{array}{c}
      x_1\\
      x_2\\
      x_3
    \end{array}
  \right)=\left(
    \begin{array}{r}
      2\\
      1\\
      -2
    \end{array}
  \right)
  $$
  解得
  $$
  (x_1,x_2,x_3)=(3,2,0)+k(1,2,1), ~~k\in \R
  $$
  故$\sigmabd(\betabd)$的原象$\betabd$不唯一。
\end{jie}
\end{frame}


\begin{frame}
  \begin{dingli}
    设线性变换$\sigmabd$在基$B_1=\{\alphabd_1,\cd,\alphabd_n\}$和基$B_2=\{\betabd_1,\cd,\betabd_n\}$下的矩阵分别为$\MA$和$\MB$,且$B_1$到$B_2$的过渡矩阵为$\MC$,则
    $$
    \red{\MB = \MC^{-1}\MA\MC.}
    $$
  \end{dingli}\vspace{.1in}\pause 
  \begin{proof}
    由
    $$
    \begin{array}{rl}
      \sigmabd(\alphabd_1,\cd,\alphabd_n)&=(\alphabd_1,\cd,\alphabd_n)\MA, \\[.1in]
      (\betabd_1,\cd,\betabd_n)&=(\alphabd_1,\cd,\alphabd_n)\MC
    \end{array}
    $$
    知
    $$
    \begin{aligned}
      \sigmabd(\betabd_1,\cd,\betabd_n)&=\sigmabd(\alphabd_1,\cd,\alphabd_n)\MC\\
      &=(\alphabd_1,\cd,\alphabd_n)\MA\MC\\
      &=(\betabd_1,\cd,\betabd_n)\MC^{-1}\MA\MC,
    \end{aligned}
    $$
    由此即得$\MB=\MC^{-1}\MA\MC$。
  \end{proof}
\end{frame}


\begin{frame}
  \begin{li}
    设$\R^3$的线性变换$\sigmabd$在自然基$\{\ve_1,\ve_2,\ve_3\}$下的矩阵为
    $$
    \MA=\left(
      \begin{array}{rrr}
        2&-1&-1\\
        -1&2&-1\\
        -1&-1&2
      \end{array}
    \right)
    $$
    \begin{enumerate}
    \item 求$\sigmabd$在基$\{\betabd_1,\betabd_2,\betabd_3\}$下的矩阵,其中
      $$
      \betabd_1=(1,1,1)^T, ~~\betabd_2=(-1,1,0)^T, ~~\betabd_3=(-1,0,1)^T.
      $$
    \item $\alphabd=(1,2,3)^T$,求$\sigmabd$在基$\{\betabd_1,\betabd_2,\betabd_3\}$下的坐标向量$(y_1,y_2,y_3)^T$及$\sigmabd(\alphabd)$.
    \end{enumerate}
  \end{li}
\end{frame}


\begin{frame}
  \begin{jie}
    1. 由
    $$
    (\betabd_1,\betabd_2,\betabd_3)=(\alphabd_1,\alphabd_2,\alphabd_3)\MC
    $$
    知
    $$
    \MC=(\betabd_1,\betabd_2,\betabd_3)=\left(
      \begin{array}{rrr}
        1&-1&-1\\
        1&1&0\\
        1&0&1
      \end{array}
    \right), \pause~~ 
    \MC^{-1}=\frac13\left(
      \begin{array}{rrr}
        1&1&1\\
        -1&2&-1\\
        -1&-1&2
      \end{array}
    \right)
    $$
    于是$\sigmabd$在基$\{\betabd_1,\betabd_2,\betabd_3\}$下的矩阵为
    $$
    \MB=\MC^{-1}\MA\MC=
    \left(
      \begin{array}{rrr}
        0&0&0\\
        0&3&0\\
        0&0&3
      \end{array}
    \right).
    $$
  \end{jie}
\end{frame}


\begin{frame}
  \begin{jie}
    2. $\alphabd$在自然基下的坐标向量为其本身,即$(1,2,3)^T$,因此,由坐标变换公式得
    $$
    \left(
      \begin{array}{c}
        x_1\\x_2\\x_3
      \end{array}
    \right)=\MC^{-1}
    \left(
      \begin{array}{c}
        1\\2\\3
      \end{array}
    \right)=\left(
      \begin{array}{c}
        2\\0\\1
      \end{array}
    \right)
    $$\pause 

    $\sigmabd$在基$\{\betabd_1,\betabd_2,\betabd_3\}$下的坐标向量为
    $$
    \left(
      \begin{array}{c}
        y_1\\y_2\\y_3
      \end{array}
    \right)=\MB
    \left(
      \begin{array}{c}
        x_1\\x_2\\x_3
      \end{array}
    \right)=\left(
      \begin{array}{c}
        0\\0\\3
      \end{array}
    \right).
    $$
  \end{jie}
\end{frame}


\begin{frame}
  由
  $$
  \sigmabd(\alphabd_1,\cd,\alphabd_n)=(\alphabd_1,\cd,\alphabd_n)\left(
    \begin{array}{cccc}
      a_{11}&a_{12}&\cd&a_{1n}\\
      a_{21}&a_{22}&\cd&a_{2n}\\
      \vdots&\vdots&&\vdots\\
      a_{n1}&a_{n2}&\cd&a_{nn}
    \end{array}
  \right):=(\betabd_1,\cd,\betabd_n)
  $$
  知,给定$\R^n$中的一组基$\{\alphabd_1,\cd,\alphabd_n\}$,$\R^n$中任一向量组$\betabd_1,\cd,\betabd_n$就等价于任给上式中的一个矩阵$\MA$。\vspace{.1in}\pause 

  \blue{反过来,任给$n$个向量$\betabd_1,\cd,\betabd_n$,是否存在唯一的一个线性变换$\sigmabd$,使得$\sigmabd(\alphabd_j)=\betabd_j$?}
\end{frame}


\begin{frame}
  \begin{dingli}
    设$\{\alphabd_1,\cd,\alphabd_n\}$是$\R^n$的一组基,$\betabd_1,\cd,\betabd_n$是在$\R^n$中任意给定的$n$个向量,则一定存在唯一的线性变换$\sigmabd$,使得
    $$
    \red{\sigmabd(\alphabd_j)=\betabd_j, ~~ j=1,\cd,n.}
    $$
  \end{dingli} \vspace{.1in} \pause 

  \begin{proof}
    \blue{存在性} \vspace{.1in}

    设$\zetabd=x_1\alphabd_1+x_2\alphabd_2+\cd+x_n\alphabd_n$,定义变换
    $$
    \sigmabd(\zetabd)=x_1\betabd_1+x_2\betabd_2+\cd+x_n\betabd_n
    $$\pause 
    当$\zetabd=\alphabd_j$时,显然有
    $$
    \sigmabd(\alphabd_j)=\betabd_j.
    $$\vspace{.1in} \pause 
    \blue{下证$\sigmabd$为线性变换。}
    任给$\R^n$中的两个向量$\zetabd_1=\sum_{j=1}^na_j\alphabd_j$和$\zetabd_2=\sum_{j=1}^nb_j\alphabd_j$以及$k\in \R$,有
    $$
    \begin{aligned}
      \sigmabd(\zetabd_1+\zetabd_2)&=\sigmabd(\sum_{j=1}^n(a_j+b_j)\alphabd_j)
      =\sum_{j=1}^n(a_j+b_j)\betabd_j=\sum_{j=1}^na_j\betabd_j+\sum_{j=1}^nb_j\betabd_j=\sigmabd(\zetabd_1)+\sigmabd(\zetabd_2), \\
      \sigmabd(k\zetabd)&=\sigmabd(\sum_{j=1}^nkx_j\alphabd_j)=\sum_{j=1}^nkx_j\betabd_j=k\sigmabd(\zetabd).
    \end{aligned}
    $$ 
  \end{proof}
\end{frame}


\begin{frame}
  综上所述,可得重要结论:

  \blue{给定$\R^n$的一组基后,$\R^n$中的线性变换与$\R^{n\times n}$中的矩阵一一对应。}
\end{frame}
\subsection{线性变换的运算}
\begin{frame}
\begin{dingyi}
  设$\sigmabd$与$\taubd$是线性空间$V(F)$的两个线性变换,$\lambda\in F$,定义
  $$
  \begin{array}{rl}
    (\sigmabd+\taubd)(\alphabd)&=\sigmabd(\alphabd)+\taubd(\alpha),\\
    (\lambda\sigmabd)(\alphabd)&=\lambda\sigmabd(\alphabd), \\
    (\sigmabd\taubd)(\alphabd)&=\sigmabd(\taubd(\alphabd))
  \end{array}
  $$,
\end{dingyi}
\end{frame}

\begin{frame}
  可以验证上述定义的\blue{$\sigmabd+\taubd,\lambda\sigmabd,\sigmabd\taubd$仍是$V(F)$的线性变换。}\vspace{.1in} \pause 


  以$\sigmabd\taubd$为例,
  $\forall \alphabd_1,\alphabd_2\in V, k_1,k_2\in F$,
  $$
  \begin{array}{rl}
    (\sigmabd\taubd)(k_1\alphabd_1+k_2\alphabd_2)&=\sigmabd(\taubd(k_1\alphabd_1+k_2\alphabd_2))\\[.1in]
                                                 &=\sigmabd(k_1\taubd(\alphabd_1)+k_2\taubd(\alphabd_2))\\[.1in]
                                                 &=k_1\sigmabd(\taubd(\alphabd_1))+k_2\sigmabd(\taubd(\alphabd_2))\\[.1in]
                                                 &=k_1(\sigmabd\taubd)(\alphabd_1)+k_2(\sigmabd\taubd)(\alphabd_2).
  \end{array}
  $$
\end{frame}

\begin{frame}
  \begin{dingli}
    设线性空间$V(F)$的线性变换$\sigmabd$与$\taubd$在$V$的基$\{\alphabd_1,\cd,\alphabd_n\}$下对应的矩阵分别为$\MA$和$\MB$,则\blue{$\sigmabd+\taubd, \lambda\sigmabd$和$\sigmabd\taubd$}在该组基下对应的矩阵分别为\red{$\MA+\MB, \lambda \MA$和$\MA\MB$}。
  \end{dingli}
  \vspace{.1in}\pause 

  \begin{proof}
    由
    $$
    \sigmabd(\alphabd_1,\cd,\alphabd_n)=(\alphabd_1,\cd,\alphabd_n)\MA, ~~
    \taubd(\alphabd_1,\cd,\alphabd_n)=(\alphabd_1,\cd,\alphabd_n)\MB
    $$
    可知
    $$
    \sigmabd(\alphabd_j)=\sum_{j=1}^na_{ij}\alphabd_j, ~~
    \taubd(\alphabd_j)=\sum_{j=1}^nb_{ij}\alphabd_j,
    $$
    于是
    $$
    (\sigmabd+\taubd)(\alphabd_j)=\sigmabd(\alphabd_j)+\taubd(\alphabd_j)
    =\sum_{j=1}^na_{ij}\alphabd_j+\sum_{j=1}^nb_{ij}\alphabd_j
    =\sum_{j=1}^n\blue{(a_{ij}+b_{ij})}\alphabd_j.
    $$
    这表明\blue{$\sigmabd+\taubd$所对应的矩阵是为$\MA+\MB$}。
    $$
    \begin{aligned}
      (\sigmabd \taubd)(\alphabd_j)&=\sigmabd(\taubd(\alphabd_j))
      =\sigmabd(\sum_{j=1}^nb_{ij}\alphabd_j)
      =\sum_{j=1}^nb_{ij}\sigmabd(\alphabd_j)\\
      &=\sum_{j=1}^nb_{ij}\left(\sum_{k=1}^na_{ki}\alphabd_k\right)=\sum_{k=1}^n\blue{\left(\sum_{j=1}^na_{ki}b_{ij}\right)}\alphabd_k
    \end{aligned}
    $$
    这表明\blue{$\sigmabd\taubd$所对应的矩阵是为$\MA\MB$}。
  \end{proof}
\end{frame}

\begin{frame}
  \begin{dingyi}
    如果线性变换$\sigmabd$对应的矩阵$\MA$为可逆矩阵,则称$\sigmabd$是\red{可逆的线性变换}。$\sigmabd$可逆也可定义为:如果存在线性变换$\taubd$使得
    $$
    \sigmabd\taubd=\taubd\sigmabd=\MI
    $$
    则称$\sigmabd$为\red{可逆的线性变换}。
  \end{dingyi}
\end{frame}

% \begin{frame}

% \end{frame}

% \begin{frame}

% \end{frame}


% \begin{frame}

% \end{frame}

% \begin{frame}

% \end{frame}

% \begin{frame}

% \end{frame}


% \begin{frame}

% \end{frame}

% \begin{frame}

% \end{frame}

% \begin{frame}

% \end{frame}


% \begin{frame}

% \end{frame}

% \begin{frame}

% \end{frame}

% \begin{frame}

% \end{frame}

\subsection{线性变换的象(值域)与核}

\begin{frame}
  \begin{dingyi}
    设$\sigmabd$是线性空间$V(F)$的一个线性变换,
    \begin{itemize}
      \item 把$V$中所有元素在$\sigmabd$下的象所组成的集合
        $$
        \sigmabd(V)=\{\betabd|\betabd=\sigmabd(\alphabd), \alphabd\in V\}
        $$
        称为$\sigmabd$的\red{象或值域},记为$\Im\sigmabd$;
      \item
        $V$的零元$\M0$在$\sigmabd$下的完全原象
        $$
        \sigmabd^{-1}(\M0)=\{\alphabd|\sigmabd(\alphabd)=\M0, ~~\alphabd \in V\}
        $$
        称为$\sigmabd$的核,记为$\Ker\sigmabd$。
      \end{itemize}
  \end{dingyi}
\end{frame}

\begin{frame}
  \begin{li}
    $\R^2$上旋转矩阵$\MR_\theta$与镜像变换$\varphibd$的值域都是$\R^2$自身,它们的核都只含一个零向量$\{\M0\}$。
  \end{li}
\end{frame}

\begin{frame}
  \begin{itemize}
  \item[(1)] \blue{$\sigmabd(V)$(或$\Im \sigmabd$)}是线性空间$V(F)$的一个子空间。   
  \end{itemize} \pause 

  \begin{proof}
    由$\sigmabd(\M0)=\M0$可知$\sigmabd(V)$是一个非空集合,且$\forall \betabd_1,\betabd_2\in \sigmabd(V)$,$\exists \alphabd_1,\alphabd_2\in V$,使得$\sigmabd(\alphabd_1)=\betabd_1,\sigmabd(\alphabd_2)=\betabd_2$,于是$\forall \lambda_1,\lambda_2\in F$,有
    $$
    \lambda_1\betabd_1+\lambda_2\betabd_2=\lambda_1\sigmabd(\alphabd_1)+\lambda_2\sigmabd(\alphabd_2)=\sigmabd(\lambda_1\alphabd_1+\lambda_2\alphabd_2)\in \sigmabd(V)
    $$
    所以,$\sigmabd(V)$是$V(F)$的一个子空间。
  \end{proof}
\end{frame}

\begin{frame}
  \begin{itemize}
    \item[(2)] \blue{$\sigmabd^{-1}(\M0)$(或$\Ker \sigmabd$)}也是线性空间$V(F)$的一个子空间。   
  \end{itemize} \pause 

  \begin{proof}
    因$\sigmabd^{-1}(\M0)$不是空集,且$\forall \alphabd_1,\alphabd_2\in \sigmabd^{-1}(\M0)$和$\forall \lambda_1,\lambda_2\in F$,均有
    $$
    \sigmabd(\lambda_1\alphabd_1+\lambda_2\alphabd_2)=\lambda_1\sigmabd(\alphabd_1)+\lambda_2\sigmabd(\alphabd_2)=\lambda_1\M0+\lambda_2\M0=\M0
    $$
    即$\lambda_1\alphabd_1+\lambda_2\alphabd_2\in\sigmabd^{-1}(\M0)$,故$\sigmabd^{-1}(\M0)$是$V(F)$的子空间。
  \end{proof}
\end{frame}

\begin{frame}
  \begin{itemize}
    \item[(3)] 线性变换$\sigmabd$是单射的充分必要条件是$\sigmabd^{-1}(\M0) = \{\M0\}$。   
  \end{itemize}\pause 

  \begin{proof}
    \begin{itemize}
    \item[$\Rightarrow$] 因$\sigmabd$是单射,则$\forall \alphabd \in V$,若$\sigmabd(\alphabd)=\M0=\sigmabd(\M0)$,则$\alphabd=\M0$,故$\sigmabd^{-1}(\M0) = \{\M0\}$;\\[.1in] \pause 
    \item[$\Leftarrow$] 由$\sigmabd^{-1}(\M0) = \{\M0\}$可得$\forall \alphabd_1,\alphabd_2\in V$,若$\sigmabd(\alphabd_1)=\sigmabd(\alphabd_2)$,即$\sigmabd(\alphabd_1)-\sigmabd(\alphabd_2)=\sigmabd(\alphabd_1-\alphabd_2)=\M0$,则$\alphabd_1-\alphabd_2=0$,即$\alphabd_1=\alphabd_2$,故$\sigmabd$为单射。
    \end{itemize}
  \end{proof}
\end{frame}


\begin{frame}
  \begin{itemize}
    \item $\dim \sigmabd(V)$称为$\sigmabd$的秩,记作$\rank(\sigmabd)$;\\[.15in]
    \item $\dim \sigmabd^{-1}(\M0)$称为$\sigmabd$的零度,记作$\mathcal N(\sigmabd)$。
    \end{itemize}
\end{frame}

\begin{frame}
  \begin{dingli}
    设线性空间$V(F)$的维数为$n$,$\sigmabd$是$V(F)$的一个线性变换,则
    $$
    \dim \sigmabd(V)+\dim \sigmabd^{-1}(\M0)=n.
    $$
  \end{dingli}\vspace{.1in}\pause 

  \begin{proof}
    设$\dim \sigma^{-1}(\M0)=k, B_1=\{\alphabd_1,\cd,\alphabd_k\}$是核$\sigmabd^{-1}(\M0)$的一组基,把$B_1$扩充到$V$的基
    $$
    B=\{\alphabd_1,\cd,\alphabd_k,\red{\alphabd_{k+1},\cd,\alphabd_n}\}.
    $$\vspace{.1in}\pause 
    
    由于$\forall \alphabd=x_1\alphabd_1+\cd+x_n\alphabd_n\in V$,有$\sigmabd(\alphabd)=x_1\sigmabd(\alphabd_1)+\cd+x_n\sigmabd(\alphabd_n)$,故$\sigmabd$的值域是$\sigmabd$关于$V$的基象生成的子空间,即
    $$
    \sigmabd(V)=L(\sigmabd(\alphabd_1),\cd,\sigmabd(\alphabd_k),\sigmabd(\alphabd_{k+1}),\cd,\sigmabd(\alphabd_n))
    $$
    再由$\sigmabd(\alphabd_i)=\M0(i=1,\cd,k)$得
    $$
    \sigmabd(V)=L(\sigmabd(\alphabd_{k+1}),\cd,\sigmabd(\alphabd_n)).
    $$
    因此,只需证明\blue{$\dim \sigmabd(V)=n-k$,即$\{\sigmabd(\alphabd_{k+1}),\cd,\sigmabd(\alphabd_n)\}$线性无关。}
  \end{proof}
  
\end{frame}

\begin{frame}
  \begin{proof}[续]
    设
    $$
    c_{k+1}\sigmabd(\alphabd_{k+1})+\cd+c_n\sigmabd(\alphabd_n)=\M0
    $$
    即
    $$
    \sigmabd(c_{k+1}\alphabd_{k+1}+\cd+c_n\alphabd_n)=\M0
    $$
    故$c_{k+1}\alphabd_{k+1}+\cd+c_n\alphabd_n\in \sigmabd^{-1}(\M0)$,因此它可被$B_1$线性表示,于是
    $$
    c_1\alphabd_1+\cd+c_k\alphabd_k-c_{k+1}\alphabd_{k+1}-\cd-c_n\alphabd_n=\M0
    $$
    从而有
    $$
    c_1=\cd=c_k=c_{k+1}=\cd=c_n,
    $$
    故$\{\sigmabd(\alphabd_{k+1}),\cd,\sigmabd(\alphabd_n)\}$线性无关。

  \end{proof}
  
\end{frame}

\begin{frame}
  由于线性变换$\sigmabd$的值域$\sigmabd(V)$是$\sigmabd$关于$V$的基$\{\alphabd_1,\cd,\alphabd_n\}$的象$\sigmabd(\alphabd_1),\cd,\sigmabd(\alphabd_n)$的生成子空间,故
  $$
  \dim \sigmabd(V) = \rank\{\sigmabd(\alphabd_1),\cd,\sigmabd(\alphabd_n)\}.
  $$\pause 
  而
  $$
  (\sigmabd(\alphabd_1),\cd,\sigmabd(\alphabd_n))=(\alphabd_1,\cd,\alphabd_n)\MA,
  $$
  可以证明:\red{基象组$\{\sigmabd(\alphabd_1),\cd,\sigmabd(\alphabd_n)\}$与$\MA$的列向量组有相同的线性相关性。} \pause 

  于是有
  $$
  \rank\{\sigmabd(\alphabd_1),\cd,\sigmabd(\alphabd_n)\}=\rank(\MA),
  $$
  从而
  $$
  \dim \sigmabd(V) = \rank(\MA).
  $$

\end{frame}

\begin{frame}
  以下证明:\red{基象组$\{\sigmabd(\alphabd_1),\cd,\sigmabd(\alphabd_n)\}$与$\MA$的列向量组有相同的线性相关性。} \vspace{.1in}\pause

  设
  $$
  x_1\sigmabd(\alphabd_1)+\cd+x_n\sigmabd(\alphabd_n)=\M0
  $$
  即
  $$
  \sum_{j=1}^n x_j\sigmabd(\alphabd_j)=\sum_{j=1}^n x_j\sum_{i=1}^na_{ij}\alphabd_i=\sum_{i=1}^n\left(\sum_{j=1}^na_{ij}x_j\right)\alphabd_i=\M0
  $$
  由$\alphabd_1,\cd,\alphabd_n$线性无关可得
  $$
  \sum_{j=1}^na_{ij}x_j=0, ~~~i=1,2,\cd,n
  $$
  即
  $$
  \MA\vx=\M0.
  $$
\end{frame}


\begin{frame}
  \begin{itemize}
    \item 若$\sigmabd(\alphabd_1),\cd,\sigmabd(\alphabd_n)$线性无关,则$x_1=\cd=x_n=0$,即$\MA\vx=\M0$只有零解,故$\MA$的列向量组线性无关。
    \item[] 反之亦然。 \\[0.1in]

    \item 若$\sigmabd(\alphabd_1),\cd,\sigmabd(\alphabd_n)$线性无关,说明$\MA\vx=\M0$有非零解,即$\MA$的列向量组线性相关。
    \item[] 反之亦然。
  \end{itemize}
\end{frame}


\begin{frame}
  若线性变换$\sigmabd$在$B=\{\alphabd_1,\cd,\alphabd_n\}$下对应的矩阵为$\MA$,则核$\sigmabd^{-1}(\M0)$中任一向量$\alphabd=x_1\alphabd_1+\cd+x_n\alphabd_n$在基$B$下的坐标向量$(x_1,\cd,x_n)^T$,就是$\MA\vx=\M0$的解向量。因此,$\MA\vx=\M0$的解空间$\mathcal N(\MA)$的维数等于核$\sigmabd^{-1}(\M0)$的维数,即
  $$
  \red{\dim \sigmabd^{-1}(\M0)=\dim \mathcal N(\MA). }
  $$
\end{frame}




% \begin{frame}
  
% \end{frame}
