\subsection{正交矩阵及其性质}
\begin{frame}  
  \begin{dingyi}[正交矩阵]
    设$\MA\in\MR^{n\times n}$,如果
    $$
    \MA^T\MA=\MI
    $$
    则称$\MA$为正交矩阵。
  \end{dingyi}  
\end{frame}


\begin{frame}  
  \begin{dingli}
    $$
    \MA\mbox{为}\mbox{正交矩阵}
    ~~\Longleftrightarrow~~
    \MA\mbox{的列向量组为一组标准正交基。}
    $$
  \end{dingli}
  \pause
  \begin{proof}
    将$\MA$按列分块为$(\alphabd_1,\alphabd_2,\cd,\alphabd_n)$,则
    $$
    \MA^T\MA = \left(
      \begin{array}{c}
        \alphabd_1^T\\
        \alphabd_2^T\\
        \vd\\
        \alphabd_n^T
      \end{array}
    \right) (\alphabd_1,\alphabd_2,\cd,\alphabd_n) = \left(
      \begin{array}{cccc}
        \alphabd_1^T\alphabd_1&\alphabd_1^T\alphabd_2&\cd&\alphabd_1^T\alphabd_n\\
        \alphabd_2^T\alphabd_1&\alphabd_2^T\alphabd_2&\cd&\alphabd_2^T\alphabd_n\\
        \vd&\vd&&\vd\\
        \alphabd_n^T\alphabd_1&\alphabd_n^T\alphabd_2&\cd&\alphabd_n^T\alphabd_n
      \end{array}
    \right)
    $$
    \pause 
    因此
    $$
    \begin{array}{rl}
      \MA^T\MA=\MI &~~\Longleftrightarrow~~
                     \left\{
                     \begin{array}{ll}
                       \alphabd_i^T\alphabd_i=1,  &i=1,2,\cd,n\\[0.2cm]
                       \alphabd_i^T\alphabd_j=0,  &j\ne i, ~~i,j=1,2,\cd,n
                     \end{array}
                                                    \right.\\[0.3in]
                   &~~\Longleftrightarrow~~
                     \MA\mbox{的列向量组为一组标准正交基。}
    \end{array}
    $$
  \end{proof}
\end{frame}


\begin{frame}  
  \begin{dingli}
    设$\MA,\MB$皆为$n$阶正交矩阵,则
    \begin{itemize}
    \item[(1)] $|\MA|=1\mbox{~或~} -1$
    \item[(2)] $\MA^{-1}=\MA^T$
    \item[(3)] $\MA^T$也是正交矩阵
    \item[(4)] $\MA\MB$也是正交矩阵
    \end{itemize}
  \end{dingli}  
\end{frame}

\begin{frame}  
  \begin{dingli}
    若列向量$\vx,\vy\in\MR^n$在$n$阶正交矩阵$\MA$的作用下变换为$\MA\vx,\MA\vy\in\MR^n$,则向量的内积、长度与向量间的夹角都保持不变,即
    $$
    \begin{array}{c}
      (\MA\vx,\MA\vy)=(\vx,\vy),\\[0.1in]
      \|\MA\vx\|=\|\vx\|, ~~\|\MA\vy\|=\|\vy\|, \\[0.1in]
      <\MA\vx,\MA\vy>=<\vx,\vy>.
    \end{array}
    $$
  \end{dingli}  
\end{frame}
