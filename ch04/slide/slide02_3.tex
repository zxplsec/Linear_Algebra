\subsection{施密特(Schmidt)正交化方法}

\begin{frame}  
  \begin{block}{目标}
    从一组线性无关的向量$\alphabd_1,\alphabd_2,\cd,\alphabd_m$出发,构造一组标准正交向量组。
  \end{block}
\end{frame}


\begin{frame}
  
  \begin{center}
    \red{施密特(Schmidt)正交化过程}
  \end{center}

  给定$\MR^n$中的线性无关组$\alphabd_1,\alphabd_2,\cd,\alphabd_m$, \pause
  \begin{itemize}
  \item[(1)] 取$\betabd_1=\alphabd_1$\pause
  \item[(2)] 令$\betabd_2=\alphabd_2+k_{21}\betabd_1$,\pause
    $$
    \begin{array}{rl}
      &\betabd_2\perp\betabd_1\\[0.2cm]\pause
      \Longrightarrow&(\betabd_1,\betabd_2)=0\\[0.2cm] \pause
      \Longrightarrow& \ds k_{21}=-\frac{(\alphabd_2,\betabd_1)}{(\betabd_1,\betabd_1)} \pause
    \end{array}
    $$
    故
    $$
    \betabd_2=\alphabd_2-\frac{(\alphabd_2,\betabd_1)}{(\betabd_1,\betabd_1)}\betabd_1
    $$
  \end{itemize}
\end{frame}


\begin{frame}
  \begin{center}
    \red{施密特(Schmidt)正交化过程(续)}
  \end{center}  

  \begin{itemize}
  \item[(3)] 令$\betabd_3=\alphabd_3+k_{31}\betabd_1+k_{32}\betabd_2$,\pause
    $$
    \begin{array}{rll}
      &\betabd_3\perp\betabd_i, & (i=1,2)\\[0.2cm] \pause
      \Longrightarrow&(\betabd_3,\betabd_i)=0, & (i=1,2)\\[0.2cm] \pause
      \Longrightarrow& \ds k_{3i}=-\frac{(\alphabd_3,\betabd_i)}{(\betabd_i,\betabd_i)} & (i=1,2) \pause
    \end{array}
    $$
    故
    $$
    \betabd_3=\alphabd_3-\frac{(\alphabd_3,\betabd_1)}{(\betabd_1,\betabd_1)}\betabd_1
    -\frac{(\alphabd_3,\betabd_2)}{(\betabd_2,\betabd_2)}\betabd_2
    $$
  \end{itemize} 
\end{frame}


\begin{frame}
  \begin{center}
    \red{施密特(Schmidt)正交化过程(续)}
  \end{center}  

  \begin{itemize}
  \item[(4)] 继续以上步骤,假设已经求出两两正交的非零向量$\betabd_1,\betabd_2,\cd,\betabd_{j-1}$,\pause
    取
    $$
    \betabd_j=\alphabd_j+k_{j1}\betabd_1+k_{j2}\betabd_2+\cd++k_{j,j-1}\betabd_{j-1},
    $$\pause
    $$
    \begin{array}{rl}
      & \betabd_j \perp \betabd_i\quad (i=1,2,\cd,j-1)\\[0.4cm] \pause
      \Longrightarrow &(\betabd_j,\betabd_i)=0 \quad (i=1,2,\cd,j-1)\\[0.4cm] \pause
      \Longrightarrow &(\alphabd_j+k_{ji}\betabd_i,\betabd_i)=0 \quad (i=1,2,\cd,j-1)\\[0.4cm] \pause
      \Longrightarrow& \ds k_{ji}=-\frac{(\alphabd_j,\betabd_i)}{(\betabd_i,\betabd_i)} \pause
    \end{array}
    $$
    故
    $$
    \betabd_j=\alphabd_j-\frac{(\alphabd_j,\betabd_1)}{(\betabd_1,\betabd_1)}\betabd_1
    -\frac{(\alphabd_j,\betabd_2)}{(\betabd_2,\betabd_2)}\betabd_2
    -\cd
    -\frac{(\alphabd_j,\betabd_{j-1})}{(\betabd_{j-1},\betabd_{j-1})}\betabd_{j-1}.
    $$
  \end{itemize}  
\end{frame}


\begin{frame}  
  \begin{center}
    \red{施密特(Schmidt)正交化过程(续)}
  \end{center}  

  \begin{itemize}
  \item[(5)] 单位化
    $$
    \betabd_1, \betabd_2, \cd, \betabd_m \xlongrightarrow[]{\ds \eta_j=\frac{\betabd_j}{\|\betabd_j\|}}
    \etabd_1, \etabd_2, \cd, \etabd_m
    $$
  \end{itemize}  
\end{frame}


\begin{frame}
  \begin{li}
    已知$B=\{\alphabd_1,\alphabd_2,\alphabd_3\}$是$\MR^3$的一组基,其中
    $$
    \alphabd_1=(1,-1,0)^T,~~
    \alphabd_2=(1,0,1)^T,~~
    \alphabd_3=(1,-1,1)^T.
    $$
    试用施密特正交化方法,由$B$构造$\MR^3$的一组标准正交基。
  \end{li}
  \pause
  \begin{jie}
    $$
    \begin{array}{rl}
      \betabd_1&=\alphabd_1=(1,-1,0)^T, \\[0.2cm]\pause
      \betabd_2&\ds=\alphabd_2-\frac{(\alphabd_2,\betabd_1)}{(\betabd_1,\betabd_1)}\betabd_1\\[0.4cm]\pause
               &\ds=(1,0,1)^T-\frac12(1,-1,0)^T=\left(\frac12,\frac12,1\right),\\[0.4cm]\pause
      \betabd_3&\ds=\alphabd_3-\frac{(\alphabd_3,\betabd_1)}{(\betabd_1,\betabd_1)}\betabd_1
                 -\frac{(\alphabd_3,\betabd_2)}{(\betabd_2,\betabd_2)}\betabd_2\\[0.4cm]\pause
               &\ds=(1,-1,1)^T-\frac23\left(\frac12,\frac12,1\right)^T-\frac22(1,-1,0)^T=\left(-\frac13,-\frac13,\frac13\right).
    \end{array}
    $$
  \end{jie}
\end{frame}


\begin{frame}
  \begin{jie}[续]
    $$
    \begin{array}{rl}
      \etabd_1&\ds =\frac{\betabd_1}{\|\betabd_1\|}=\left(\frac1{\sqrt{2}},-\frac1{\sqrt{2}},0\right),\\[0.4cm]
      \etabd_2&\ds =\frac{\betabd_2}{\|\betabd_2\|}=\left(\frac1{\sqrt{6}},\frac1{\sqrt{6}},\frac2{\sqrt{6}}\right),\\[0.4cm]
      \etabd_3&\ds =\frac{\betabd_2}{\|\betabd_2\|}=\left(-\frac1{\sqrt{3}},-\frac1{\sqrt{3}},\frac1{\sqrt{3}}\right).
    \end{array}
    $$
  \end{jie}
\end{frame}
