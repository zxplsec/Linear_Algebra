\section{线性空间的基、维数和向量的坐标}

\begin{frame}
我们知道:\blue{$F^n$中任意$n$个线性无关的向量都是一组基,任何一个向量$\alphabd$都可以由$F^n$的基线性表示,其系数按序排成的向量就是$\alphabd$在这组基下的坐标。}

\vspace{.2in} \pause 

这里我们将在一般的线性空间$V(F)$中讨论类似的问题,为此先要讨论$V(F)$中元素(或称向量)的线性相关性。
\end{frame}

\begin{frame}

\begin{li}
证明:线性空间$\R[x]_n$中元素$f_0=1, f_1=x, f_2=x^2,\cdots,f_{n-1}=x^{n-1}$是线性无关。
\end{li}

\begin{proof}
设$k_0f_0+k_1f_1+k_2f_2+\cdots+k_{n-1}f_{n-1}=0(x)$,即
$$
k_0+k_1x+k_2x^2+\cdots+k_{n-1}x^{n-1}=0(x),
$$
其中$0(x)$是$\R[x]_n$的零元素,即零多项式。因此,要使$1,x,x^2,\cdots,x^{n-1}$的线性组合等于零多项式,仅当$k_0,k_1,k_2,\cdots,k_{n-1}$全为零才能成立,故$1,x,x^2,\cdots,x^{n-1}$是线性无关的。
\end{proof}
\end{frame}

\begin{frame}
\begin{li}
证明:线性空间$\R^{2\times 2}$中的元素
$$
\MA_1=\left(\begin{array}{cc} 1&0\\0&0 \end{array}\right), 
\MA_2=\left(\begin{array}{cc} 1&1\\0&0 \end{array}\right), 
\MA_3=\left(\begin{array}{cc} 1&1\\1&0 \end{array}\right), 
\MA_4=\left(\begin{array}{cc} 1&1\\1&1 \end{array}\right) 
$$
是线性无关的。
\end{li} \pause 

\begin{proof}
设
\begin{equation}\label{li2}
k_1\MA_1 + k_2\MA_2 + k_3\MA_3 + k_4\MA_4 = \M0_{2\times 2},
\end{equation}
即
$$
\left(\begin{array}{cc} k_1+k_2+k_3+k_4&k_2+k_3+k_4\\k_3+k_4&k_4 \end{array}\right)=
\left(\begin{array}{cc} 0&0\\0&0 \end{array}\right),
$$
即
$$
\left\{
\begin{array}{r}
k_1+k_2+k_3+k_4=0,\\
k_2+k_3+k_4=0,\\
k_3+k_4=0,\\
k_4=0.
\end{array}
\right.
$$
而此线性方程组只有零解,因此仅当$k_1=k_2=k_3=k_4=0$时,\eqref{li2}才成立,故$\MA_1,\MA_2,\MA_3,\MA_4$线性无关。
\end{proof}
\end{frame}

\begin{frame}
显然,在$\R^{2\times 2}$中,矩阵
$$
\ME_{11}=\left(\begin{array}{cc} 1&0\\0&0 \end{array}\right), 
\ME_{12}=\left(\begin{array}{cc} 0&1\\0&0 \end{array}\right), 
\ME_{21}=\left(\begin{array}{cc} 0&0\\1&0 \end{array}\right), 
\ME_{22}=\left(\begin{array}{cc} 0&0\\0&1 \end{array}\right) 
$$
是也线性无关的,且$\R^{2\times 2}$中任一矩阵
$$
\MA = \left(\begin{array}{cc} a&b\\c&d \end{array}\right)=a\ME_{11}+b\ME_{12}+c\ME_{21}+d\ME_{22}.
$$

\vspace{.1in}\pause 

不难证明:\blue{在$\R^{2\times 2}$中任意$5$个元素(二阶矩阵)$\MA,\MB,\MC,\MD,\MQ$是线性相关的,若$\MA,\MB,\MC,\MD$线性无关,则$\MQ$可由$\MA,\MB,\MC,\MD$线性表出,且表示法唯一。}
\vspace{.1in}\pause 


由此可以发现$\R^{2\times 2}$的这些属性与$\R^4$是类似的,我们可以把线性空间的这些属性抽象为基、维数与坐标的概念。
\end{frame}

\begin{frame}
\begin{dingyi}
	如果线性空间$V(F)$中存在线性无关的向量组$B=\{\alphabd_1, \alphabd_2, \cdots, \alphabd_n\}$,且任一$\alphabd\in V$都可以由$B$线性表示为
	$$
	\alphabd=x_1\alphabd_1+x_2\alphabd_2+\cdots+x_n\alphabd_n,
	$$
	则称
	\begin{itemize}
	\item $V$是$n$维线性空间(或者说\blue{$V$的维数为$n$,记作$dim V = n$});
	\item $B$是$V$的一个基;
	\item 有序数组$(x_1,x_2,\cdots,x_n)$为$\alphabd$关于基$B$的坐标(向量),记作
	$$
	\alphabd_B=(x_1,x_2,\cdots,x_n)^T\in F^n.
	$$
	\end{itemize}
    如果$V(F)$中有任意多个线性无关的向量,则称$V$是\red{无限维线性空间}。
\end{dingyi}
\vspace{.1in}\pause 

\begin{li}
在$F[x]$中,$1,x,x^2,\cdots,x^n$($n$为任意正整数)是线性无关的,故$F[x]$是无限维空间。
\end{li}

我们只讨论\red{有限维线性空间}。

\end{frame}

\begin{frame}

  \blue{在$n$维线性空间$V$中,任意$n+1$个元素$\betabd_1,\betabd_2,\cdots,\betabd_{n+1}$都可以由$V$的一个基$\alphabd_1,\alphabd_2,\cdots,\alphabd_n$线性表示,而$n$维线性空间中任意$n+1$个元素都是线性相关的,故$n$维线性空间$V$中,任何$n$个线性无关的向量都是$V$的一组基。}
  \vspace{.1in}\pause 
  \begin{li}
    \begin{itemize}
      \item $F[x]_n$是$n$维线性空间,$\{1,x,x^2,\cd,x^{n-1}\}$是它的一组基;\vspace{.1in}\pause 
      \item $\R^{2\times 2}$是$4$维线性空间,$\ME_{11},\ME_{12},\ME_{21},\ME_{22}$是它的一组基;\vspace{.1in}\pause 
      \item $F^{m\times n}$是$m\times n$维线性空间,$\{\ME_{ij}\}_{i=1,\cd,m; j=1,\cd,n}$是它的一组基。
    \end{itemize}
  \end{li}
\end{frame}

\begin{frame}
  在线性空间$V$中,由向量组$\alphabd_1,\alphabd_2,\cd,\alphabd_s$生成的子空间$L(\alphabd_1,\alphabd_2,\cd,\alphabd_s)$的维数等于向量组$\alphabd_1,\alphabd_2,\cd,\alphabd_s$的秩,向量组$\alphabd_1,\alphabd_2,\cd,\alphabd_s$的极大线性无关组是$L(\alphabd_1,\alphabd_2,\cd,\alphabd_s)$的基。

  \begin{li}
    矩阵$\MA$的列空间$\mathcal R(\MA)$和行空间$\mathcal R(\MA^T)$的维数都等于$\MA$的秩。$V$的零子空间$\{\M0\}$的维数为零。
  \end{li}
\end{frame}

\begin{frame}
  $\MA\vx = \M0$的基础解系是其解空间$\mathcal N(\MA)$的基,如果$\MA$是$m\times n$矩阵,$\rank(\MA)=r$,则解空间$\mathcal N(\MA)$的维数为$n-r$,所以
  $$
  \dim(\mathcal R(\MA^T)) + \dim(\mathcal N(\MA)) = n. 
  $$
\end{frame}

\begin{frame}
  \begin{dingli}
    设$V$是$n$维线性空间,$W$是$V$的$m$维子空间,且$B_1=\{\alphabd_1,\alphabd_2,\cd,\alphabd_m\}$是$W$中的一组基,则$B_1$可以扩充为$V$的基,即\blue{在$B_1$的基础上可以添加$n-m$个向量而成为$V$的一组基}.
  \end{dingli}
  \vspace{.1in}\pause 
  
  \begin{proof}
    \begin{itemize}
    \item 若$m=n$,$B_1$就是$V$的基。 \\[0.1in] \pause 
    \item 若$m<n$,则必存在$\alphabd_{m+1}\in V$,使得$\alphabd_1,\alphabd_2,\cd,\alphabd_m,,\alphabd_{m+1}$线性无关,否则$\dim V=m$,这与$\dim V=m$矛盾。\\[0.1in] \pause
      \begin{itemize}
      \item 若$n=m+1$,则定理得证;\\[0.1in] \pause
      \item 若$n>m+1$,重复以上步骤,必存在$\alphabd_{m+2},\cd,\alphabd_n\in V$,使得$\{\alphabd_1,\alphabd_2,\cd,\alphabd_m,,\alphabd_{m+1},\cd,\alphabd_n\}$线性无关,即$V$的基。
      \end{itemize}
    \end{itemize}
  \end{proof}
\end{frame}

\begin{frame}
  \begin{dingli}[子空间的维数公式]
    设$W_1,W_2$是线性空间$V(F)$的子空间,则
    $$
    \dim W_1 + \dim W_2 = \dim(W_1+W_2) + \dim(W_1\cap W_2).
    $$
  \end{dingli}
  \vspace{.1in}\pause 

  \blue{证明:}
    设$\dim W_1 = s, \dim W_2 = t, \dim(W_1\cap W_2) = r$,则
    $$
    \begin{aligned}
    W_1\cap W_2 = L(\alphabd_1,\alphabd_2,\cd,\alphabd_r), \\
    W_1=L(\alphabd_1,\cd,\alphabd_r,\betabd_1,\cd,\betabd_{s-r}), \\
    W_2=L(\alphabd_1,\cd,\alphabd_r,\gammabd_1,\cd,\gammabd_{t-r}).
    \end{aligned}
    $$
    于是
    $$
    W_1+W_2 = L(\alphabd_1,\cd,\alphabd_r,\betabd_1,\cd,\betabd_{s-r},\gammabd_1,\cd,\gammabd_{t-r}).
    $$
    只需证明$\dim(W_1+W_2)=s+t-r$,即上述生成$W_1+W_2$的$s+t-r$个向量是线性无关的。

\end{frame}

\begin{frame}
    为此,设
    \begin{equation}\label{subspace_dim1}
    a_1\alphabd_1+\cd+a_r\alphabd_r + b_1\betabd_1+\cd+b_{s-r}\betabd_{s-r} + c_1\gammabd_1+\cd+c_{t-r}\gammabd_{t-r}=\M0,
    \end{equation}
    于是
    \begin{equation}\label{subspace_dim2}
    a_1\alphabd_1+\cd+a_r\alphabd_r + b_1\betabd_1+\cd+b_{s-r}\betabd_{s-r} = -c_1\gammabd_1-\cd-c_{t-r}\gammabd_{t-r}.
    \end{equation}
    因上式两端的向量分别属于$W_1$和$W_2$,故它们都属于$W_1\cap W_2$,因此
    $$
    -c_1\gammabd_1-\cd-c_{t-r}\gammabd_{t-r} = d_1\alphabd_1+\cd+d_r\alphabd_r,
    $$
    即
    $$
    d_1\alphabd_1+\cd+d_r\alphabd_r+c_1\gammabd_1+\cd+c_{t-r}\gammabd_{t-r}=\M0,
    $$
    其中$\alphabd_1,\cd,\alphabd_r,\gammabd_1,\cd,\gammabd_{t-r}$为$W_2$的基,故其系数全为零。
    将其代入\eqref{subspace_dim2}右端,又得\eqref{subspace_dim2}的左端系数全为零,故\eqref{subspace_dim1}中的向量组线性无关。 \qed
\end{frame}  
  
\begin{frame}
  $n$维线性空间$V(F)$中向量在基$B$下的坐标,与$F^n$中向量关于基$B$的坐标是完全类似的,主要有以下几个结论:
  \begin{itemize}
    \item 向量在给定基下的坐标是唯一的;\\[0.1in]
    \item 由基$B_1$到基$B_2$的过渡矩阵是可逆的;\\[0.1in]
    \item 基变换与坐标变换的公式
  \end{itemize}
在这里都是适用的。
\end{frame}

\begin{frame}
给定$V(F)$中的一组基$B=\{\betabd_1,\betabd_2,\cd,\betabd_n\}$,$V(F)$中的向量及其坐标($F^n$中的向量)不仅是一一对应的,而且这种对应保持线性运算关系不变,即
$$
\red{
\begin{array}{l}
  \mbox{$V(F)$中$\alphabd+\gammabd$对应于$F^n$中$\alphabd_B+\gammabd_B$}\\[0.1in]
  \mbox{$V(F)$中$\lambda\alphabd$对应于$F^n$中$\lambda\alphabd_B$}
\end{array}
}
$$\pause 

事实上,若$\alpha=x_1\betabd_1+x_2\betabd_2+\cd+x_n\betabd_n,\gammabd=y_1\betabd_1+y_2\betabd_2+\cd+y_n\betabd_n,\lambda\in F$,则有
$$
\begin{array}{rl}
  (\alphabd+\betabd) & = (x_1+y_1)\betabd_1+(x_2+y_2)\betabd_2+\cd+(x_n+y_n)\betabd_n, \\[0.1in]
  \lambda\alphabd & = (\lambda x_1)\betabd_1+(\lambda x_2)\betabd_2+\cd+(\lambda x_n)\betabd_n
\end{array}
$$
故
$$
(\alphabd+\betabd)_B = \alphabd_B+\betabd_B, \quad
(\lambda\alphabd)_B = \lambda\alphabd_B.
$$
\end{frame}

\begin{frame}
具有上述对应关系的两个线性空间$V(F)$和$F^n$,称它们是\red{同构}的。 \vspace{.1in}\pause

也就是说,研究任何$n$维线性空间$V(F)$,都可以通过基和坐标,转化为研究$n$维向量空间$F^n$。\vspace{.1in}\pause

这样,\blue{我们对不同的$n$维线性空间就有了统一的研究方法,统一到研究$F^n$。}
\vspace{.1in}\pause

因此,\red{通常把线性空间也成为向量空间,线性空间中的元素也称为向量。}
\end{frame}

\begin{frame}i
  \begin{li}
    证明:$B=\{1,x,\cd,x^{n-1}\}$是$\R[x]_n$的一组基,并求$p(x)=a_0+a_1x+\cd+a_{n-1}x^{n-1}$在基$B$下的坐标。
  \end{li}\pause 
  \begin{proof}
    前面我们已经证明$B$是线性无关的,且$\forall p(x) \in \R[x]_n$均可表示成
    $$
    p(x)= a_0+a_1x+\cd+a_{n-1}x^{n-1},
    $$
    故$B$是$\R[x]_n$的一组基(自然基),因此$\R[x]_n$是$n$维实线性空间。 
    $p(x)$在基$B$下的坐标为
    $$
    (p(x))_B=(a_0,a_1,\cd,a_{n-1})^T.
    $$  
    \vspace{.1in}\pause 
    
    $$
    \red{
    p(x) = (1, x, \cd, x^{n-1}) 
    \left( \begin{array}{c} a_0\\ a_1\\ \vd\\ a_{n-1} \end{array}\right).
    }
    $$
    
  \end{proof}
\end{frame}

\begin{frame}
  \begin{li}
    设$B_1=(g_1, g_2, g_3),B_2=(h_1, h_2, h_3)$,其中
    $$
    \left\{
      \begin{array}{l}
        g_1 = 1,\\
        g_2 = -1+x, \\
        g_3 = 1-x+x^2,
      \end{array}   
      \right., \quad
      \left\{
      \begin{array}{l}
        h_1 = 1-x-x^2,\\
        h_2 = 3x-2x^2, \\
        h_3 = 1-2x^2,
      \end{array}
    \right.
    $$
    \begin{enumerate}
    \item 证明$B_1,B_2$是$\R[x]_3$的基
    \item 求$B_1$到$B_2$的过渡矩阵
    \item 已知$[p(x)]_{B_1} = (1,4,3)^T$,求$[p(x)]_{B_2}$.
    \end{enumerate}
  \end{li}
\end{frame}

