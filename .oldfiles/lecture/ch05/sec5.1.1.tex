\subsection{特征值与特征向量}


\begin{dingyi}[特征值与特征向量]
  设$\A$为复数域$\mathbb C$上的$n$阶矩阵,如果存在数$\lambda\in\mathbb C$和非零的$n$维向量$\xx$使得
  $$
  \A\xx=\lambda\xx
  $$
  则称$\lambda$为矩阵$\A$的\blue{\underline{特征值}},$\xx$为$\A$的对应于特征值$\lambda$的\blue{\underline{特征向量}}。
\end{dingyi} 

\begin{itemize}
\item[(1)] 特征向量$\xx\ne\zero$;
\item[(2)] 特征值问题是对方针而言的。 
\end{itemize}


由定义,$n$阶矩阵$\A$的特征值,就是使齐次线性方程组
$$
(\A-\lambda\II)\xx=\zero
$$
有非零解的$\lambda$值,即满足方程
$$
\det(\A-\lambda\II)=0
$$
的$\lambda$都是矩阵$\A$的特征值。


\begin{jielun}
  特征值$\lambda$是关于$\lambda$的多项式$\det(\A-\lambda\II)$的根。
\end{jielun}


\begin{dingyi}[特征多项式、特征矩阵、特征方程]
  设$n$阶矩阵$\A=(a_{ij})$,则
  $$
  f(\lambda)=\det(\A-\lambda\II)
  =\left|
    \begin{array}{cccc}
      a_{11}-\lambda&a_{12}&\cd&a_{1n}\\[0.2cm]
      a_{21}&a_{22}-\lambda&\cd&a_{2n}\\[0.2cm]
      \vd&\vd&&\vd\\[0.2cm]
      a_{n1}&a_{n2}&\cd&a_{nn}-\lambda
    \end{array}
  \right|
  $$
  称为矩阵$\A$的特征多项式,$\A-\lambda\II$称为$\A$的特征矩阵,$\det(\A-\lambda\II)=0$称为$\A$的特征方程。
\end{dingyi}

\begin{itemize}
\item[(1)]  $n$阶矩阵$\A$的特征多项式是$\lambda$的$n$次多项式。
\item[(2)]  特征多项式的$k$重根称为$k$重特征值。
\end{itemize}

\begin{li}
  求矩阵
  $$
  \A=\left(
    \begin{array}{rrr}
      5&-1&-1\\
      3&1&-1\\
      4&-2&1
    \end{array}
  \right)
  $$
  的特征值与特征向量。
\end{li}
\begin{jie}
$$
\begin{array}{rl}
  \det(\A-\lambda\II)&=\left|
    \begin{array}{rrr}
      5-\lambda&-1&-1\\
      3&1-\lambda&-1\\
      4&-2&1-\lambda
    \end{array}
  \right| = \left|
    \begin{array}{rrr}
      3-\lambda&-1&-1\\
      3-\lambda&1-\lambda&-1\\
      3-\lambda&-2&1-\lambda
    \end{array}
                    \right|\\[0.3in]
  &= (3-\lambda)\left|
    \begin{array}{rrr}
      1&-1&-1\\
      1&1-\lambda&-1\\
      1&-2&1-\lambda
    \end{array}
  \right|= (3-\lambda)\left|
    \begin{array}{rrr}
      1&-1&-1\\
      0&2-\lambda&0\\
      0&-1&2-\lambda
    \end{array}
  \right|\\
                     &=(3-\lambda)(\lambda-2)^2=0
\end{array}
$$
故$\A$的特征值为$\lambda_1=3,~\lambda_2=2\mbox{(二重特征值)}$。

当$\lambda_1=3$时,由$(\A-\lambda\II)\xx=\zero$,即
$$
\left(\begin{array}{rrr}
        2&-1&-1\\
        3&-2&-1\\
        4&-2&-2
      \end{array}\right)\left(
      \begin{array}{c}
        x_1\\
        x_2\\
        x_3
      \end{array}
    \right)=\left(
      \begin{array}{c}
        0\\
        0\\
        0
      \end{array}
    \right)
    $$
    得其基础解系为$\xx_1=(1,1,1)^T$,因此$k_1\xx_1$($k_1$为非零任意常数)是$\A$对应于$\lambda_1=3$的全部特征向量。
    \vspace{0.1in}

    当$\lambda_1=2$时,由$(\A-\lambda\II)\xx=\zero$,即
    $$
    \left(\begin{array}{rrr}
            3&-1&-1\\
            3&-1&-1\\
            4&-2&-1
          \end{array}\right)\left(
          \begin{array}{c}
            x_1\\
            x_2\\
            x_3
          \end{array}
        \right)=\left(
          \begin{array}{c}
            0\\
            0\\
            0
          \end{array}
        \right)
        $$
        得其基础解系为$\xx_2=(1,1,2)^T$,因此$k_2\xx_2$($k_1$为非零任意常数)是$\A$对应于$\lambda_1=2$的全部特征向量。
\end{jie}
      
        
\begin{li}
  $$
  \left(
    \begin{array}{cccc}
                a_{11}&0&\cd&0\\
                0&a_{22}&\cd&0\\
                \vd&\vd&&\vd\\
                0&0&\cd&a_{nn}
                                 \end{array}
                                                      \right), \left(
                                                      \begin{array}{cccc}
                                                      a_{11}&a_{12}&\cd&a_{1n}\\
                0&a_{22}&\cd&a_{2n}\\
                \vd&\vd&&\vd\\
                0&0&\cd&a_{nn}
                                 \end{array}
                                                      \right),\left(
                                                      \begin{array}{cccc}
                                                      a_{11}&0&\cd&0\\
                a_{21}&a_{22}&\cd&0\\
                \vd&\vd&&\vd\\
                a_{n1}&a_{n2}&\cd&a_{nn}
                                           \end{array}
                                                                          \right)
                                                                          $$
                                                                          的特征多项式为
                                                                          $$
                                                                          (\lambda-a_{11})(\lambda-a_{22})\cd(\lambda-a_{nn})
                                                                          $$
                                                                          故其$n$个特征值为$n$个对角元。
                                                                          \end{li}
        


