\section{矩阵的初等变换与初等矩阵}

用高斯消去法求解线性方程组,其步骤是对增广矩阵做以下三种行变换:
\begin{itemize}
\item[(i)] 对调两行;
\item[(ii)] 以非零常数$k$乘矩阵的某一行;
\item[(iii)] 将矩阵的某一行乘以常数$k$并加到另一行。
\end{itemize}
这三类行变换统称为\red{矩阵的初等行变换},其中
\begin{itemize}
\item[(i)] \red{对换变换}  $\quad r_i \leftrightarrow r_j$;
\item[(ii)] \red{倍乘变换}       $\quad r_i \times k$;
\item[(iii)] \red{倍加变换} $\quad r_i + r_j \times k $。
\end{itemize}
对应的还有\red{初等列变换}。 \blue{初等行变换与初等列变换统称为初等变换。}


三种初等变换都是可逆的,
\begin{table}[htbp]
  \centering
  \begin{tabular}{|c|c|} \hline
    初等变换 &  逆变换 \\\hline
    $r_i \leftrightarrow r_j$ & $r_i \leftrightarrow r_j$ \\[0.2cm]\hline
    $r_i \times k$ & $\ds r_i \times \frac1k$ \\[0.2cm]\hline
    $r_i + r_j \times k$ & $r_i - r_j\times k$ \\[0.2cm]\hline
  \end{tabular}
  \caption{初等变换及其逆变换}
\end{table}

\begin{dingyi}[矩阵的等价]
  
  \begin{itemize}
  \item[(i)] 如果$\A$经过有限次初等行变换变成$\B$,就称\blue{$\A$与$\B$行等价},记为$\red{\A\overset{r}{\sim} \B}$;
  \item[(ii)] 如果$\A$经过有限次初等列变换变成$\B$,就称\blue{$\A$与$\B$列等价},记为$\red{\A\overset{c}{\sim} \B}$;
  \item[(iii)] 如果$\A$经过有限次初等变换变成$\B$,就称\blue{$\A$与$\B$等价},记为$\red{\A\sim \B}$。
  \end{itemize}
\end{dingyi}

\begin{xingzhi}
  矩阵的等价满足以下三条性质:
  \begin{itemize}
  \item[(i)] \red{反身性}:$\A \sim \A$;
  \item[(ii)] \red{对称性}:若$\A \sim \B$,则$\B \sim \A$;
  \item[(iii)] \red{传递性}:若$\A \sim \B, ~\B \sim \C$,则$\A \sim \C$。
  \end{itemize}
\end{xingzhi}

\begin{dingyi}[初等矩阵]
  将单位矩阵$\II$做一次初等变换所得的矩阵称为\red{初等矩阵}。
  对应于$3$类初等行、列变换,有$3$种类型的初等矩阵。
\end{dingyi}

以下介绍三种初等矩阵:

1、 对调单位矩阵的两行或两列(\red{初等对调矩阵})
  \begin{figure}[htbp]
    \centering
    \begin{tikzpicture}
      \matrix (M) [matrix of math nodes]  { 
        \E_{ij} = \\
      };
      \matrix(MM) [right=2pt of M, matrix of math nodes,nodes in empty cells,
      ampersand replacement=\&,left delimiter=(,right delimiter=)] {
        1 \&     \&   \&   \&     \&   \&   \& \& \\
        \& \dd \&   \&   \&     \&   \&   \& \& \\
        \&     \& 0 \&   \& \cd \&   \& 1 \& \& \\
        \&     \&   \& 1 \&     \&   \&   \& \& \\
        \&     \&\vd\&   \& \dd \&   \&\vd\& \& \\
        \&     \&   \&   \&     \& 1 \&   \& \& \\
        \&     \& 1 \&   \& \cd \&   \& 0 \& \& \\
        \&     \&   \&   \&     \&   \&   \& \dd \& \\
        \&     \&   \&   \&     \&   \&   \& \& 1\\
      };
      \node[right=12pt  of MM-3-9, blue]  {第$i$行};
      \node[right=12pt  of MM-7-9, blue]  {第$j$行};
      \node[below=12pt  of MM-9-3, blue]  {第$i$列};
      \node[below=12pt  of MM-9-7, blue]  {第$j$列};
    \end{tikzpicture}
  \end{figure}
\newpage
  \begin{itemize}
  \item[a.]
    用$m$阶初等矩阵$\E_{ij}$左乘$\A=(a_{ij})_{m\times n}$,得
    \begin{figure}[htbp]
      \centering
      \begin{tikzpicture}
        \matrix (M) [matrix of math nodes]  { 
          \E_{ij}\A = \\
        };
        \matrix(MM) [right=2pt of M, matrix of math nodes,nodes in empty cells,
        ampersand replacement=\&,left delimiter=(,right delimiter=)] {
          a_{11} \& a_{12}    \& \cd   \&  a_{1n} \\
          \vd   \& \vd      \&   \&  \vd \\          
          a_{j1} \& a_{j2}    \& \cd  \&  a_{jn} \\
          \vd   \& \vd      \&   \&  \vd \\
          a_{i1} \& a_{i2}    \& \cd   \&  a_{in} \\
          \vd   \& \vd      \&   \&  \vd \\
          a_{m1} \& a_{m2}    \& \cd  \&  a_{mn} \\
        };
        \node[right=12pt  of MM-3-4, blue]  {第$i$行};
        \node[right=12pt  of MM-5-4, blue]  {第$j$行};
      \end{tikzpicture}
    \end{figure}

    其结果相当于:
    \red{把$\A$的第$i$行与第$j$行对调($r_i \leftrightarrow r_j$).}
  \item[b.]
    用$n$阶初等矩阵$\E_{ij}$右乘$\A$,得
    \begin{figure}[htbp]
      \centering
      \begin{tikzpicture}
        \matrix (M) [matrix of math nodes]  { 
          \A \E_{ij}= \\
        };
        \matrix(MM) [right=2pt of M, matrix of math nodes,nodes in empty cells,
        ampersand replacement=\&,left delimiter=(,right delimiter=)] {
          a_{11} \& \cd \&a_{1j}    \& \cd \&a_{1i}    \& \cd  \&  a_{1n} \\
          a_{21} \& \cd \&a_{2j}    \& \cd \&a_{2i}    \& \cd  \&  a_{jn} \\
          \vd    \&     \&\vd       \&     \&\vd       \&      \&  \vd \\
          a_{m1} \& \cd \&a_{mj}    \& \cd \&a_{mi}    \& \cd  \&  a_{mn} \\
        };
        \node[below=12pt  of MM-4-3, blue]  {第$i$列};
        \node[below=12pt  of MM-4-5, blue]  {第$j$列};
      \end{tikzpicture}
    \end{figure}

    其结果相当于\red{把$\A$的第$i$列与第$j$列对调($c_i \leftrightarrow c_j$).}
    \end{itemize}
\newpage

2、 以非零常数$k$乘单位矩阵的某行或某列(初等倍乘矩阵)
\begin{figure}[htbp]
  \centering
    \begin{tikzpicture}
      \matrix (M) [matrix of math nodes]  { 
        \E_{i}(k) = \\
      };
      \matrix(MM) [right=2pt of M, matrix of math nodes,nodes in empty cells,
      ampersand replacement=\&,left delimiter=(,right delimiter=)] {
        1 \&     \&   \&   \&     \&   \& \\
        \& \dd \&   \&   \&     \&   \& \\
        \&     \& 1 \&   \&     \&   \& \\
        \&     \&   \& k \&     \&   \& \\
        \&     \&   \&   \& 1   \&   \& \\
        \&     \&   \&   \&     \& \dd \& \\
        \&     \&   \&   \&     \&   \& 1\\
      };
      \node[right=12pt  of MM-4-7, blue]  {第$i$行};
      \node[below=12pt  of MM-7-4, blue]  {第$i$列};
    \end{tikzpicture}
  \end{figure}

\begin{itemize}
\item[a.] 以$m$阶初等矩阵$\E_i(k)$左乘$\A$,得
    \begin{figure}[htbp]
      \centering
      \begin{tikzpicture}
        \matrix (M) [matrix of math nodes]  { 
          \E_{i}(k)\A = \\
        };
        \matrix(MM) [right=2pt of M, matrix of math nodes,nodes in empty cells,
        ampersand replacement=\&,left delimiter=(,right delimiter=)] {
          a_{11} \& a_{12}    \& \cd   \&  a_{1n} \\
          \vd   \& \vd      \&   \&  \vd \\          
          ka_{i1} \& ka_{i2}    \& \cd  \&  ka_{in} \\
          \vd   \& \vd      \&   \&  \vd \\
          a_{m1} \& a_{m2}    \& \cd  \&  a_{mn} \\
        };
        \node[right=12pt  of MM-3-4, blue]  {第$i$行};
      \end{tikzpicture}
    \end{figure}
    
  其结果相当于以数$k$乘$\A$的第$i$行($r_i\times k$);
\item[b.] 以$n$阶初等矩阵$\E_i(k)$右乘$\A$,得
    \begin{figure}[htbp]
      \centering
      \begin{tikzpicture}
        \matrix (M) [matrix of math nodes]  { 
          \A \E_{i}(k)= \\
        };
        \matrix(MM) [right=2pt of M, matrix of math nodes,nodes in empty cells,
        ampersand replacement=\&,left delimiter=(,right delimiter=)] {
          a_{11} \& \cd \&ka_{1i}      \& \cd  \&  a_{1n} \\
          a_{21} \& \cd \&ka_{2i}      \& \cd  \&  a_{jn} \\
          \vd    \&     \&\vd         \&      \&  \vd \\
          a_{m1} \& \cd \&ka_{mi}      \& \cd  \&  a_{mn} \\
        };
        \node[below=12pt  of MM-4-3, blue]  {第$i$列};
      \end{tikzpicture}
    \end{figure}
    
  其结果相当于以数$k$乘$\A$的第$i$列($c_i\times k$)。
\end{itemize}

\newpage
3、以非零常数$k$乘单位矩阵的某行再加到另一行上(初等倍加矩阵)
\begin{figure}[htbp]
  \centering
  \begin{tikzpicture}
    \matrix (M) [matrix of math nodes]  { 
      \E_{ij}(k) = \\
    };
    \matrix(MM) [right=2pt of M, matrix of math nodes,nodes in empty cells,
    ampersand replacement=\&,left delimiter=(,right delimiter=)] {
      1 \&     \&   \&   \&     \&   \& \\
      \& \dd \&   \&   \&     \&   \& \\
      \&     \& 1 \&\cd\& k    \&   \& \\
      \&     \&   \&\dd\& \vd  \&   \& \\
      \&     \&   \&   \& 1   \&   \& \\
      \&     \&   \&   \&     \& \dd \& \\
      \&     \&   \&   \&     \&   \& 1\\
    };
    \node[right=12pt  of MM-3-7, blue]  {第$i$行};
    \node[right=12pt  of MM-5-7, blue]  {第$j$行};
  \end{tikzpicture}
\end{figure} 

\begin{itemize}
\item[a.] 以$m$阶初等矩阵$\E_{ij}(k)$左乘$\A$,得
    \begin{figure}[htbp]
      \centering
      \begin{tikzpicture}
        \matrix (M) [matrix of math nodes]  { 
          \E_{ij}\A = \\
        };
        \matrix(MM) [right=2pt of M, matrix of math nodes,nodes in empty cells,
        ampersand replacement=\&,left delimiter=(,right delimiter=)] {
          a_{11} \& a_{12}    \& \cd   \&  a_{1n} \\
          \vd   \& \vd      \&   \&  \vd \\          
          a_{i1}+ka_{j1} \& a_{i2}+ka_{j2}    \& \cd  \&  a_{in}+ka_{jn} \\
          \vd   \& \vd      \&   \&  \vd \\
          a_{j1} \& a_{j2}    \& \cd   \&  a_{jn} \\
          \vd   \& \vd      \&   \&  \vd \\
          a_{m1} \& a_{m2}    \& \cd  \&  a_{mn} \\
        };
        \node[right=12pt  of MM-3-4, blue]  {第$i$行};
        \node[right=28pt  of MM-5-4, blue]  {第$j$行};
      \end{tikzpicture}
    \end{figure}

  其结果相当于把$\A$的第$j$行乘以数$k$加到第$i$行上\red{($r_i+r_j\times k$)};
\item[b.] 以$n$阶初等矩阵$\E_{ij}(k)$右乘$\A$,得
    \begin{figure}[htbp]
      \centering
      \begin{tikzpicture}
        \matrix (M) [matrix of math nodes]  { 
          \A \E_{ij}= \\
        };
        \matrix(MM) [right=2pt of M, matrix of math nodes,nodes in empty cells,
        ampersand replacement=\&,left delimiter=(,right delimiter=)] {
          a_{11} \& \cd \&a_{1i}    \& \cd \&a_{1j}+ka_{1i}  \& \cd  \&  a_{1n} \\
          a_{21} \& \cd \&a_{2i}    \& \cd \&a_{2j}+ka_{2i}    \& \cd  \&  a_{jn} \\
          \vd    \&     \&\vd       \&     \&\vd       \&      \&  \vd \\
          a_{m1} \& \cd \&a_{mi}    \& \cd \&a_{mj}+ka_{mi}    \& \cd  \&  a_{mn} \\
        };
        \node[below=12pt  of MM-4-3, blue]  {第$i$列};
        \node[below=12pt  of MM-4-5, blue]  {第$j$列};
      \end{tikzpicture}
    \end{figure}

  其结果相当于把$\A$的第$i$列乘以数$k$加到第$j$列上\red{($c_j+c_i\times k$)}。
\end{itemize}


\begin{dingli}
  设$\A$为一个$m\times n$矩阵,
  \begin{itemize}
  \item 
    对$\A$施行一次初等行变换,相当于在$\A$的左边乘以相应的$m$阶初等矩阵;
  \item
    对$\A$施行一次初等列变换,相当于在$\A$的右边乘以相应的$n$阶初等矩阵。
  \end{itemize}
\end{dingli}

\begin{lianxi}
  请自行补充以下变换的具体含义:
  \begin{itemize}
  \item[] $\E_i(k)\A$:
  \item[] $\E_{ij}(k)\A$:
  \item[] $\E_{ij}\A$:
  \item[] $\A\E_i(k)$:
  \item[] $\A\E_{ij}(k)$:
  \item[] $\A\E_{ij}$:
  \end{itemize}
\end{lianxi}


由初等变换可逆,可知初等矩阵可逆。  
\begin{itemize}
\item[(i)] 由\blue{变换$r_i\leftrightarrow r_j$的逆变换为其本身}可知
  $$
  \red{\E_{ij}^{-1} = \E_{ij}}
  $$ 
\item[(ii)] 由\blue{变换$r_i\times k$的逆变换为$\ds r_i\times \frac1k$}可知
  $$
  \red{\E_{i}(k)^{-1} = \E_{i}(\frac1k)}
  $$ 
\item[(iii)] 由\blue{变换$r_i+r_j\times k$的逆变换为$\ds r_i-r_j\times k$}可知
  $$
  \red{\E_{ij}(k)^{-1} = \E_{ij}(-k)}
  $$ 
\end{itemize}

  $$ \red{
    \E_{ij}\E_{ij}=\II, \quad
    \E_{i}(k)\E_{i}(\frac1k) = \II, \quad
    \E_{ij}(k)\E_{ij}(-k) = \II.
  }
  $$      


\begin{li}{例1}
  设初等矩阵
  $$
  \PP_1 = \left(
    \begin{array}{cccc}
      0&0&1&0 \\
      0&1&0&0 \\
      1&0&0&0 \\
      0&0&0&1
    \end{array}
  \right), ~~
  \PP_2 = \left(
    \begin{array}{cccc}
      1& & &  \\
      0&1& &  \\
      0&0&1& \\
      c&0&0&1
    \end{array}
  \right), ~~
  \PP_3 = \left(
    \begin{array}{cccc}
      1& & &  \\
       &k& &  \\
       & &1& \\
       & & &1
    \end{array}
  \right)
  $$
  求$\PP_1\PP_2\PP_3$及$(\PP_1\PP_2\PP_3)^{-1}$
\end{li}
\begin{jie}
$$
\PP_2\PP_3 =    \left(
  \begin{array}{cccc}
    1& & &  \\
    0&k& &  \\
    0&0&1& \\
    c&0&0&1
  \end{array}
\right), \quad   
\PP_1\PP_2\PP_3=\PP_1(\PP_2\PP_3) =\left(
  \begin{array}{cccc}
    0&0&1&0 \\
    0&k&0&0  \\
    1&0&0&0  \\
    c&0&0&1
  \end{array}
\right)
$$

$$
(\PP_1\PP_2\PP_3)^{-1} = \PP_3^{-1}\PP_2^{-1}\PP_1^{-1}
$$

$$
\PP_1^{-1} = \PP_1, ~~
\PP_2^{-1} = \left(
  \begin{array}{cccc}
    1& & &  \\
    0&1& &  \\
    0&0&1& \\
    -c&0&0&1
  \end{array}
\right), ~~
\PP_3^{-1} = \left(
  \begin{array}{cccc}
    1& & &  \\
     &\frac1k& &  \\
     & &1& \\
     & & &1
  \end{array}
\right)
$$    

$$
\PP_2^{-1}\PP_1^{-1} =   \left(
  \begin{array}{cccc}
    0&0&1&0 \\
    0&1&0&0 \\
    1&0&0&0 \\
    0&0&-c&1
  \end{array}
\right), \quad  
\PP_3^{-1}\PP_2^{-1}\PP_1^{-1} =  \left(
  \begin{array}{cccc}
    0&0&1&0 \\
    0&\frac1k&0&0 \\
    1&0&0&0 \\
    0&0&-c&1
  \end{array}
\right)
$$
\end{jie}


\begin{li}
  将三对角矩阵
  $
  \A = \left(
    \begin{array}{cccc}
      2 & 1 & 0 & 0\\
      1 & 2 & 1 & 0\\
      0 & 1 & 2 & 1\\
      0 & 0 & 1 & 2
    \end{array}
  \right)
  $
  分解成主对角元为$1$的下三角矩阵$\mathbf{L}$和上三角阵$\mathbf{U}$的乘积$\A=\mathbf{L}\mathbf{U}$(称为矩阵的LU分解)。
\end{li}
\begin{jie}
  $$
  \left(
    \begin{array}{cccc}
      1 & && \\
      -\frac12 &1&&\\
        &&1&\\
        &&&1
    \end{array}
  \right)\A = \left(
    \begin{array}{cccc}
      2 & 1 & 0 & 0\\
      0 & \frac32 & 1 & 0\\
      0 & 1 & 2 & 1\\
      0 & 0 & 1 & 2
    \end{array}
  \right) \triangleq \A_1
  $$

  $$
  \left(
    \begin{array}{cccc}
      1 & && \\
        &1&&\\
        &-\frac23&1&\\
        &&&1
    \end{array}
  \right)\A_1 = \left(
    \begin{array}{cccc}
      2 & 1 & 0 & 0\\
      0 & \frac32 & 1 & 0\\
      0 & 0 & \frac43 & 1\\
      0 & 0 & 1 & 2
    \end{array}
  \right) \triangleq \A_2
  $$


  $$
  \left(
    \begin{array}{cccc}
      1 & && \\
        &1&&\\
        &&1&\\
        &&-\frac34&1
    \end{array}
  \right)\A_2 = \left(
    \begin{array}{cccc}
      2 & 1 & 0 & 0\\
      0 & \frac32 & 1 & 0\\
      0 & 0 & \frac43 & 1\\
      0 & 0 & 0  & \frac54
    \end{array}
  \right) \triangleq \mathbf{U}
  $$
  将上面三个式子中左端的矩阵分别记为$\mathbf{L}_1,\mathbf{L}_2,\mathbf{L}_3$,则
  $$
  \mathbf{L}_3\mathbf{L}_2\mathbf{L}_1 \A = \mathbf{U}
  $$
  于是
  $$
  \A = (\mathbf{L}_3\mathbf{L}_2\mathbf{L}_1)^{-1}\mathbf{U} \triangleq \mathbf{L}\mathbf{U},
  $$
  其中
  $$
  \begin{array}{rcl}
    \mathbf{L}
    &=& (\mathbf{L}_3\mathbf{L}_2\mathbf{L}_1)^{-1} 
        =  \mathbf{L}_1^{-1}\mathbf{L}_2^{-1}\mathbf{L}_3^{-1}\\[0.2cm]
    &=&  
        \left(
        \begin{array}{cccc}
          1 & && \\
          \frac12 &1&&\\
            &&1&\\
            &&&1
        \end{array}
                \right)
                \left(
                \begin{array}{cccc}
                  1 & && \\
                    &1&&\\
                    &\frac23&1&\\
                    &&&1
                \end{array}
                        \right)
                        \left(
                        \begin{array}{cccc}
                          1 & && \\
                            &1&&\\
                            &&1&\\
                            &&\frac34&1
                        \end{array}
                                       \right) \\[0.6cm]
    &=&   \left(
        \begin{array}{cccc}
          1 & && \\
          \frac12 &1&&\\
            &\frac23&1&\\
            &&\frac34&1
        \end{array}
                       \right)
  \end{array}    
  $$
\end{jie}




\begin{dingli}
  可逆矩阵可以经过若干次初等行变换化为单位矩阵。
\end{dingli}
\begin{proof}
对于高斯消去法,其消去过程是对增广矩阵做$3$类初等行变换,并一定可以将其化为行简化阶梯形矩阵。 
因此,对于任何矩阵$\A$,都可经过初等行变换将其化为行简化阶梯形矩阵,即存在初等矩阵$\PP_1,\PP_2,\cd,\PP_s$使得
$$
\PP_s \cd \PP_2 \PP_1 \A = \mathbf{U}
$$

当$\A$为$n$阶可逆矩阵时,行简化阶梯形矩阵也是可逆矩阵,从而$\mathbf{U}$必为单位矩阵$\II$.
\end{proof}

\begin{tuilun}
  可逆矩阵$\A$可以表示为若干个初等矩阵的乘积。
\end{tuilun}

\begin{proof}
由上述定理,必存在初等矩阵$\PP_1,\PP_2,\cd,\PP_s$使得
$$
\PP_s \cd \PP_2 \PP_1 \A = \II,
$$
于是
$$
\A = (\PP_s \cd \PP_2 \PP_1)^{-1} = \PP_1^{-1}\PP_2^{-1}\cd\PP_s^{-1}
$$
\end{proof}


\begin{tuilun}
  如果对可逆矩阵$\A$与同阶单位矩阵$\II$做同样的初等行变换,那么当$\A$变为单位阵时,
  $\II$就变为$\A^{-1}$,即
  $$\red{
    \left(
      \begin{array}{cc}
        \A & \II
      \end{array}
    \right) \xrightarrow[]{\mbox{初等行变换}} \left(
      \begin{array}{cc}
        \II & \A^{-1}
      \end{array}
    \right)
  } 
  $$
  同理,
$$\red{
  \left(
    \begin{array}{c}
      \A\\
      \II
    \end{array}
  \right) \xrightarrow[]{\mbox{初等列变换}} \left(
    \begin{array}{c}
      \II \\
      \A^{-1}
    \end{array}
  \right)
} 
$$
\end{tuilun}


\begin{li}
  求$
  \A=\left(
    \begin{array}{rrr}
      0&2&-1\\
      1&1&2\\
      -1&-1&-1
    \end{array}
  \right)
  $
  的逆矩阵。
\end{li}
\begin{jie}

$$
\begin{aligned}
\left(
  \begin{array}{c|c}
    \A & \II
  \end{array}
\right)=&\left(
  \begin{array}{rrr|rrr}
    0 &  2 & -1 &  1 & 0 & 0\\
    1 &  1 &  2 &  0 & 1 & 0\\
    -1 & -1 & -1 &  0 & 0 & 1\\              
  \end{array}
\right)\\
\xrightarrow[]{r_1\leftrightarrow r_2}&\left(
  \begin{array}{rrr|rrr}
    1 &  1 &  2 &  0 & 1 & 0\\
    0 &  2 & -1 &  1 & 0 & 0\\
    -1 & -1 & -1 &  0 & 0 & 1\\          
  \end{array}
\right)\\
\xrightarrow[]{r_3+ r_1}&\left(
  \begin{array}{rrr|rrr}
    1 &  1 &  2 & 0 & 1 & 0\\
    0 &  2 & -1 & 1 & 0 & 0\\
    0 &  0 &  1 & 0 & 1 & 1\\          
  \end{array}
\right)\\
\xrightarrow[r_2+r_3]{r_1+ r_3\times(-2)}&\left(
  \begin{array}{rrr|rrr}
    1 &  1 &  0  & 0 &-1 &-2\\
    0 &  2 &  0  & 1 & 1 & 1\\
    0 &  0 &  1  & 0 & 1 & 1\\    
  \end{array}
\right)\\
\xrightarrow[r_2\times \frac12]{r_1+ r_2\times(-\frac12)}&\left(
  \begin{array}{rrr|rrr}
    1 &  0 &  0  & -\frac12 &-\frac32 &-\frac52\\[.1in]
    0 &  1 &  0  & \frac12 & \frac12 & \frac12 \\[.1in]
    0 &  0 &  1  & 0 & 1 & 1                   
  \end{array}
\right)    
\end{aligned}
$$
\end{jie}

\begin{li}
  已知$\A\B\A^T=2\B\A^T+\II$,求$\B$,其中$
  \A = \left(
    \begin{array}{ccc}
      1&0&0\\
      0&1&2\\
      0&0&1
    \end{array}
  \right)
  $
\end{li}
\begin{jie}
$$
\A\B\A^T=2\B\A^T+\II \Rightarrow (\A-2\II)\B\A^T=\II 
\Rightarrow \B\A^T = (\A-2\II)^{-1}
$$

故
$$
\B = (\A-2\II)^{-1} (\A^T)^{-1}  = [\A^T(\A-2\II)]^{-1} 
=(\A^T\A-2\A^T)^{-1}
$$

$$
\A^T\A-2\A^T = \left(
  \begin{array}{rrr}
    -1&0&0\\
    0&-1&2\\
    0&-2&3
  \end{array}
\right)
$$ 
可求得
$$
\B = \left(
  \begin{array}{rrr}
    -1&0&0\\
    0&3&-2\\
    0&2&-1
  \end{array}
\right)
$$
\end{jie}




\begin{tuilun}
  对于$n$个未知数$n$个方程的线性方程组
  $$
  \A\xx=\bb,
  $$
  如果增广矩阵
  $$
  \red{(\A,~\bb)~~\overset{r}{\sim}~~(\II,\xx)},
  $$
  则$\A$可逆,且$\xx=\A^{-1}\bb$为惟一解。  
\end{tuilun}


\begin{li}
  设
  $$
  \A = \left(
    \begin{array}{rrr}
      2&1&-3\\
      1&2&-3\\
      -1&3&2
    \end{array}
  \right),
  ~~
  \bb_1=\left(
    \begin{array}{r}
      1\\
      2\\
      -2
    \end{array}
  \right),
  ~~
  \bb_2=\left(
    \begin{array}{r}
      -1\\
      0\\
      5
    \end{array}
  \right),
  $$
  求$\A\xx=\bb_1$与$\A\xx=\bb_2$的解。
\end{li}
\begin{jie}
  $$
  \begin{array}{rl}
    (\A~~\red{\bb_1}~~\red{\bb_2})
    &= \left(
      \begin{array}{rrrrr}
        2 & 1 & 3 &\red{ 1} & \red{-1}\\
        1 & 2 &-2 &\red{ 2} & \red{ 0}\\
        -1 & 3 & 2 &\red{-2} & \red{ 5}        
      \end{array}\right) \overset{{r_1\leftrightarrow r_2 \atop r_2-2r_1}\atop  r_3+r_1}{\sim}
                               \left(
                               \begin{array}{rrrrr}
                                 1 & 2 &-2 & \red{ 2} & \red{ 0}\\
                                 0 &-3 & 1 & \red{-3} & \red{-1}\\
                                 0 & 5 & 0 & \red{ 0} & \red{ 5}        
                               \end{array}
                                                        \right) \\[0.3in]
    & \overset{{r_3\leftrightarrow r_2 \atop r_2\div5}\atop  r_3+3r_2}{\sim}
      \left(
      \begin{array}{rrrrr}
        1 & 2 &-2 &\red{ 2} &  \red{0}\\
        0 & 1 & 0 &\red{ 0} &  \red{1}\\
        0 & 0 & 1 &\red{-3} &  \red{2}        
      \end{array}
                              \right)   \overset{r_1-2r_2+2r_3}{\sim}
                              \left(
                              \begin{array}{rrrrr}
                                1 & 0 & 0 & \red{-4} & \red{2}\\
                                0 & 1 & 0 & \red{0} &  \red{1}\\
                                0 & 0 & 1 & \red{-3} &  \red{2}        
                              \end{array}
                                                       \right) 
  \end{array}
  $$
\end{jie}




\begin{li}
  求解矩阵方程$\A\XX=\A+\XX$,其中
  $
  \A = \left(
    \begin{array}{ccc}
      2&2&0\\
      2&1&3\\
      0&1&0
    \end{array}
  \right)
  $
\end{li}
\begin{jie}
原方程等价于
$$
(\A-\II)\XX=\A
$$

$$
\begin{array}{ll}
  (\A-\II ~~\red{\A})& =\left(
                       \begin{array}{rrrrrr}
                         1&2& 0&\red{2}&\red{2}&\red{0}\\
                         2&0& 3&\red{2}&\red{1}&\red{3}\\
                         0&1&-1&\red{0}&\red{1}&\red{0}
                       \end{array}
                                                 \right)
                                                 
                                                 \overset{r_2-2r_1\atop r_2\leftrightarrow r_3}{\sim}
                                                 \left(
                                                 \begin{array}{rrrrrr}
                                                   1& 2& 0&\red{ 2}&\red{ 2}&\red{0}\\
                                                   0& 1&-1&\red{ 0}&\red{ 1}&\red{0}\\
                                                   0&-4& 3&\red{-2}&\red{-3}&\red{3}
                                                 \end{array}
                                                                              \right)    \\[0.2in]
                     &
                       \overset{r_3+4r_2\atop r_3\div(-1)}{\sim}
                       \left(
                       \begin{array}{rrrrrr}
                         1&2& 0&\red{2}&\red{ 2}&\red{ 0}\\
                         0&1&-1&\red{0}&\red{ 1}&\red{ 0}\\
                         0&0& 1&\red{2}&\red{-1}&\red{-3}
                       \end{array}
                                                  \right) \\[0.2in]
                     &  
                       \overset{r_3+4r_2\atop r_3\div(-1)}{\sim}
                       \left(
                       \begin{array}{rrrrrr}
                         1&0&0&\red{-2}&\red{2}&\red{6}\\
                         0&1&0&\red{2}&\red{0}&\red{-3}\\
                         0&0&1&\red{2}&\red{-1}&\red{-3}
                       \end{array}
                                                 \right)
\end{array}
$$
\end{jie}




\begin{li}
  当$a,b$满足什么条件时,矩阵$\A=\left(
    \begin{array}{rrrr}
      0&1&2&3\\
      1&4&7&10\\
      -1&0&1&b\\
      a&2&3&4
    \end{array}
  \right)$
  不可逆。
\end{li}
\begin{jie}
$$
\begin{aligned}
  \A\xrightarrow[c_2\leftrightarrow c_3]{c_1\leftrightarrow c_2} &
  \left(
    \begin{array}{rrrr}
      1 & 2 & 0 & 3\\
      4 & 7 & 1 & 10\\
      0 & 1 & -1 & b\\
      2 & 3 & a & 4\\
    \end{array}
  \right)
  \xrightarrow[r_4+r_1\times(-2)]{r_2+r_1\times(-4)}
  \left(
    \begin{array}{rrrr}
      1 & 2 & 0 & 3\\
      0 & -1 & 1 & -2\\
      0 & 1 & -1 & b\\
      0 & -1 & a & -2\\
    \end{array}
  \right)\\
  \xrightarrow[r_4+r_2\times(-1)]{r_3+r_2}&
  \left(
    \begin{array}{rrrr}      
      1 & 2 & 0 & 3\\
      0 & -1 & 1 & -2\\
      0 & 0 & -1 & b\\
      0 & 0 & a-1 & 0\\
    \end{array}
  \right)
\end{aligned}
$$
\end{jie}


