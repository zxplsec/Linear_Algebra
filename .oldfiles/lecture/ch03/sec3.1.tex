%%%%%
\section{$n$维向量及其线性相关性}

考察三元齐次线性方程组
\begin{equation}\label{ls}
  \left\{
  \begin{array}{l}
    a_{11}x_1+a_{12}x_2+a_{13}x_3=0,\\[0.1in]
    a_{21}x_1+a_{22}x_2+a_{23}x_3=0,\\[0.1in]
    a_{31}x_1+a_{32}x_2+a_{33}x_3=0
  \end{array}
  \right.
\end{equation}
我们用向量工具给出其几何解释。
记
$$
\ii = \left(
\begin{array}{ccc}
  1&
  0&
  0
\end{array}
\right), ~~
\jj = \left(
\begin{array}{ccc}
  0&
  1&
  0
\end{array}
\right), ~~
\kk = \left(
\begin{array}{ccc}
  0&
  0&
  1
\end{array}
\right),
$$
则
$$
\alphabd_i = \left(
\begin{array}{ccc}
  a_{i1}&
  a_{i2}&
  a_{i3}
\end{array}
\right) = a_{i1} \ii + a_{i2} \jj + a_{i3} \kk, \quad i=1,2,3,
$$
且
$$
\xx = \left(
\begin{array}{ccc}
  x_1&
  x_2&
  x_3
\end{array}
\right) = x_1\ii + x_2\jj + x_3\kk
$$


\begin{dingyi}[向量的内积]
  两个向量$\uu=(u_1,u_2,u_3), ~~\vv=(v_1,v_2,v_3)$的内积定义为
  $$
  (\uu,\vv)=u_1v_1+u_2v_2+u_3v_3.
  $$
\end{dingyi}

\begin{dingyi}[向量的垂直]
  两个向量$\uu=(u_1,u_2,u_3), ~~\vv=(v_1,v_2,v_3)$垂直的充分必要条件是
  $$
  (\uu,\vv)=0.
  $$
\end{dingyi}


由以上方程组可看出,解向量$\xx$与$\alphabd_1,\alphabd_2,\alphabd_3$都垂直。 故
\begin{itemize}
\item[(1)] 若$\alphabd_1,\alphabd_2,\alphabd_3$不共面,只有零向量与三者都垂直,即线性方程组(\ref{ls})只有零解;
\item[(2)] 若$\alphabd_1,\alphabd_2,\alphabd_3$共面但不共线,则与该平面垂直的向量都是线性方程组(\ref{ls})的解,
  故(\ref{ls})有无穷多个彼此平行的解向量;
\item[(3)] 若$\alphabd_1,\alphabd_2,\alphabd_3$共线,则过原点且与该直线垂直的平面上的全体向量都是(\ref{ls})的解向量,
  此时任一解向量均可表示为
  $$
  \xx = k_1 \xx^{(1)} + k_2 \xx^{(2)},
  $$
  其中$\xx^{(1)}, \xx^{(2)}$为(\ref{ls})的某两个不共线的非零解向量,$k_1,k_2$为任意常数。
\end{itemize}


\begin{dingyi}[$n$维向量]
  数域$F$上的$n$个数$a_1,a_2,\cd,a_n$构成的有序数组称为数域$F$上的一个$n$维向量,记为
  \begin{equation}\label{vec}
    \alphabd = (a_1, a_2, \cd, a_n)
  \end{equation}
  其中$a_i$称为$\alphabd$的第$i$个分量。
\end{dingyi}


\begin{itemize}
\item 形如(\ref{vec})的向量称为\red{行向量};
\item 形如
  $$
  \alphabd = (a_1, a_2, \cd, a_n)^T = \left(
  \begin{array}{c}
    a_1\\
    a_2\\
    \vd\\
    a_n
  \end{array}
  \right)
  $$
  的向量称为\red{列向量}。
\end{itemize}


数域$F$上全体$n$维向量组成的集合,记作$F^n$。 设$\alphabd\in F^n$,则
\begin{itemize}
\item 当$F$取为$\mathbb R$时,$\alphabd$为实向量;
\item 当$F$取为$\mathbb C$时,$\alphabd$为复向量。
\end{itemize}






\begin{dingyi}{向量运算}
  设$\alphabd=(a_1,a_2,\cd,a_n),~~\betabd=(b_1,b_2,\cd,b_n)\in F^n$,$k\in F$,定义
  \begin{enumerate}
  \item[(i)]
    $\alphabd=\betabd$当且仅当$a_i=b_i(i=1,2,\cd,n)$;
  \item[(ii)] 向量加法
    $$
    \alphabd+\betabd=(a_1+b_1,a_2+b_2,\cd,a_n+b_n)
    $$
  \item[(iii)] 向量数乘
    $$
    k\alphabd=(ka_1,ka_2,\cd,ka_n)
    $$
  \end{enumerate}
\end{dingyi}


\begin{itemize}
\item 在$(iii)$中取$k=-1$,得
  $$
  (-1)\alphabd = (-a_1,-a_2,\cd,-a_n)
  $$
  右端的向量称为$\alphabd$的负向量,记为$-\alphabd$. 
\item 向量的减法定义为
  $$
  \betabd-\alphabd = \betabd + (-\alphabd)
  $$
\end{itemize}

\begin{dingyi}[向量的8条运算规则]
  设$\alphabd,\betabd,\gammabd\in F^n, 1,k,l\in F$,则
  \begin{itemize}
  \item[(1)] $\alphabd+\betabd=\betabd+\alphabd$
  \item[(2)] $(\alphabd+\betabd)+\gammabd=\alphabd+(\betabd+\gammabd)$
  \item[(3)] 对任一向量$\alphabd$,有$\alphabd+\zero=\alphabd$
  \item[(4)] 对任一向量$\alphabd$,存在负向量$-\alphabd$,使得$\alphabd+(-\alphabd)=\zero$
  \item[(5)] $1\alphabd=\alphabd$
  \item[(6)] $k(l\alphabd)=(kl)\alphabd$
  \item[(7)] $k(\alphabd+\betabd)=k\alphabd+k\betabd$
  \item[(8)] $(k+l)\alphabd=k\alphabd+l\alphabd$
  \end{itemize}
\end{dingyi}

\begin{dingyi}[向量空间]
  数域$F$上的$n$维向量,在其中定义了上述加法与数乘运算,就称之为$F$上的$n$维向量空间,仍记为$F^n$。
  当$F=\mathbb R$时,叫做$n$维实向量空间,记作$\mathbb R^n$。
\end{dingyi}

\begin{dingyi}[线性表示]
  设$\alphabd_i\in F^n, k_i \in F (i=1,2,\cd,m)$,则向量
  $$
  \sum_{i=1}^m k_i\alphabd_i = k_1\alphabd_1 + k_2\alphabd_2+\cd + k_m\alphabd_m
  $$
  称为向量组$\alphabd_1,\alphabd_2,\cd,\alphabd_m$在数域$F$上的一个\red{线性组合}。  如果记
  $$\betabd=\sum_{i=1}^m k_i\alphabd_i,$$
  则称$\betabd$可由$\alphabd_1,\alphabd_2,\cd,\alphabd_m$\red{线性表示}(或\red{线性表出})。
\end{dingyi}


设有线性方程组$\A\xx=\bb$,其中$\A$为$m\times n$矩阵。记
$$\A=(\alphabd_1,\alphabd_2,\cd,\alphabd_n),$$
即
$$
(\alphabd_1,\alphabd_2,\cd,\alphabd_n) \left(
\begin{array}{c}
  x_1\\
  x_2\\
  \vd\\
  x_n
\end{array}
\right)=\bb
$$
于是线性方程组可等价的表述为
$$
x_1\alphabd_1+x_2\alphabd_2+\cd+x_n\alphabd_n=\bb
$$
\begin{zhu*}
  向量$\bb$可由$\alphabd_1,\alphabd_2,\cd,\alphabd_n$线性表示,等价于方程组
  $$
  x_1\alphabd_1+x_2\alphabd_2+\cd+x_n\alphabd_n=\bb
  $$
  有解。
\end{zhu*}

\begin{dingyi}[线性相关与线性无关]
  若对$m$个向量$\alphabd_1,\alphabd_2,\cd,\alphabd_m\in F^n$,有$m$个不全为零的数$k_1,k_2,\cd,k_m\in F$,使
  \begin{equation}\label{def1}
    k_1\alphabd_1 + k_2\alphabd_2+\cd + k_m\alphabd_m = \zero        
  \end{equation}
  成立,则称\red{$\alphabd_1,\alphabd_2,\cd,\alphabd_m$线性相关};
  否则,称\red{$\alphabd_1,\alphabd_2,\cd,\alphabd_m$线性无关}。
\end{dingyi}


\begin{zhu*}
  向量组$\alphabd_1,\alphabd_2,\cd,\alphabd_m$线性无关,指的是
  \begin{itemize}
  \item 没有不全为零的数$k_1,k_2,\cd,k_m$使(\ref{def1})成立 
  \item 只有当$k_1,k_2,\cd,k_m$全为零时,才使(\ref{def1})成立 
  \item 若(\ref{def1})成立,则$k_1,k_2,\cd,k_m$必须全为零
  \end{itemize}
\end{zhu*}



\begin{dingli}
  以下两组等价关系成立:
  \begin{itemize}
  \item  向量组$\alphabd_1,\alphabd_2,\cd,\alphabd_m$线性相关,等价于齐次方程组
    $$
    x_1\alphabd_1+x_2\alphabd_2+\cd+x_m\alphabd_m=\zero
    $$
    有非零解。
  \item  向量组$\alphabd_1,\alphabd_2,\cd,\alphabd_m$线性无关,等价于齐次方程组
    $$
    x_1\alphabd_1+x_2\alphabd_2+\cd+x_m\alphabd_m=\zero
    $$
    只有零解。
  \end{itemize}

\end{dingli}

对于只含有一个向量$\alphabd$的向量组,若存在不为零的数$k$使得
$$
k \alphabd = \zero,
$$
则
$$
\Rightarrow \alphabd = \frac1k \zero = \zero
$$  
若$\alphabd\ne \zero$,要使
$$
k \alphabd = \zero,
$$
必须$k=0$.

\begin{itemize}
\item 当$\alphabd=\zero$时,向量组$\alphabd$线性相关
\item 当$\alphabd\ne \zero$时,向量组$\alphabd$线性无关
\end{itemize}



\begin{dingli}
  向量组$\alphabd_1,\alphabd_2,\cd,\alphabd_m(m\ge 2)$线性相关的充分必要条件是$\alphabd_1,\alphabd_2,\cd,\alphabd_m$中\red{至少有一个向量}可由其余$m-1$个向量线性表出。
\end{dingli}
\begin{proof}
\red{($\Rightarrow$)} \quad
若向量组$\alphabd_1,\alphabd_2,\cd,\alphabd_m(m\ge 2)$线性相关,则必存在不全为零的数$k_1,k_2,\cd,k_m$使得
$$
k_1\alphabd_1 + k_2\alphabd_2+\cd + k_m\alphabd_m = \zero,
$$  
不妨设$k_1\ne 0$,则
$$
\alphabd_1 =  -\frac{k_2}{k_1}\alphabd_2-\cd - \frac{k_m}{k_1}\alphabd_m,
$$
必要性得证。
\vspace{0.1in}

\red{($\Leftarrow$)} \quad
不妨设$\alphabd_1$可由$\alphabd_2,\cd,\alphabd_m$线性表示,即
$$
\alphabd_1 = l_2\alphabd_2+\cd+l_m\alphabd_m    
$$  
于是有
$$
\alphabd_1 - l_2\alphabd_2-\cd-l_m\alphabd_m=\zero,
$$  
显然$1,-l_2,\cd,-l_m$不全为零,故$\alphabd_1,\alphabd_2,\cd,\alphabd_m$线性相关。
  
\end{proof}

证明向量组$\alphabd_1,\alphabd_2,\cd,\alphabd_m$线性无关的基本方法为:
说明齐次方程组
$$
x_1\alphabd_1+x_2\alphabd_2+\cd+x_m\alphabd_m=\zero
$$
只有零解。
也常常表述为:设
$$
x_1\alphabd_1+x_2\alphabd_2+\cd+x_m\alphabd_m=\zero
$$
然后说明上式成立,只能有唯一选择:
$$
x_1=x_2=\cd=x_m=0.
$$

\begin{li}
  设$n$维向量$\ee_i=(0,\cd,0,1,0,\cd,0)$,则$\ee_1,\ee_2,\cd,\ee_n$线性无关。
\end{li}

\begin{jie}
  设存在$k_1,k_2,\cd,k_n$使得
$$
k_1\ee_1+k_2\ee_2+\cd+k_n\ee_n=\zero,
$$
即
$$
(k_1,k_2,\cd,k_n)=\zero,
$$
则必有$k_1=k_2=\cd=k_n=0$,故$\ee_1,\ee_2,\cd,\ee_n$线性无关。

\end{jie}



\begin{zhu*}
  $n$维向量$\ee_1,\ee_2,\cd,\ee_n$称为\red{基本向量}。$F^n$中任何向量$\alphabd=(a_1,a_2,\cd,a_n)$都可以由$\ee_1,\ee_2,\cd,\ee_n$线性表示,即
  $$
  \alphabd = a_1\ee_1+a_2\ee_2+\cd+a_n\ee_n.
  $$
\end{zhu*}


\begin{li}
  包含零向量的向量组是线性相关的。
\end{li}
\begin{jie}
设该向量组为$\alphabd_1,\alphabd_2,\cd,\alphabd_m$,其中$\alphabd_1=\zero$。则存在$m$个不全为零的数$1,0,\cd,0$使得
$$
1 \alphabd_1+0\alphabd_2+\cd+0\alphabd_m=\zero,
$$
故该向量组线性相关。
  
\end{jie}


\begin{zhu*}
  \begin{itemize}
  \item 单个向量$\alphabd$线性相关,当且仅当$\alphabd$为零向量;
  \item 单个向量$\alphabd$线性无关,当且仅当$\alphabd$为非零向量。        
  \end{itemize}
\end{zhu*}


\begin{li}
  如果向量组$\alphabd_1,\alphabd_2,\cd,\alphabd_m$中有一部分向量线性相关,则整个向量组也线性相关。
\end{li}

\begin{proof}
不妨设$\alphabd_1,\alphabd_2,\cd,\alphabd_r(r<m)$线性相关,则存在$r$个不全为零的数$k_1,k_2,\cd,k_r$使得
$$
k_1\alphabd_1+k_2\alphabd_2+\cd+k_r\alphabd_r=\zero,
$$
从而有$m$个不全为零的数$k_1,k_2,\cd,k_r,0,\cd,0$,使得
$$
k_1\alphabd_1+k_2\alphabd_2+\cd+k_r\alphabd_r+0\alphabd_{r+1}+\cd+0\alphabd_m=\zero,
$$
故$\alphabd_1,\alphabd_2,\cd,\alphabd_m$线性相关。
  
\end{proof}

\begin{zhu*}
  \begin{itemize}
  \item 如果$\alphabd_1,\alphabd_2,\cd,\alphabd_m$线性无关,则其中任一部分向量组也线性无关。              
  \item     \red{部分相关,则整体相关;整体无关,则部分无关。}
  \end{itemize}
  
\end{zhu*}


\begin{zhu*}
  该定理不能理解为:\blue{线性相关的向量组中,每一个向量都能由其余向量线性表示。}  

  如$\alphabd_1=(0,1), ~~\alphabd_2=(0,-2), ~~\alphabd_3=(1,1)$线性相关(因为$\alphabd_1,~~\alphabd_2$线性相关),
  但$\alphabd_3$不能由$\alphabd_1,~~\alphabd_2$线性表示。
  
\end{zhu*}





  
\begin{dingli}
  设$\alphabd_1,\alphabd_2,\cd,\alphabd_r\in F^n$,其中
  $$
  \begin{array}{c}
    \alphabd_1 = (a_{11},~a_{21},~\cd,~a_{n1})^T,~
    \alphabd_2 = (a_{12},~a_{22},~\cd,~a_{n2})^T,~
    \cd,~
    \alphabd_r = (a_{1r},~a_{2r},~\cd,~a_{nr})^T,
  \end{array}
  $$
  则向量组$\alphabd_1,\alphabd_2,\cd,\alphabd_r$线性相关的充分必要条件是齐次线性方程组
  \begin{equation}\label{ax}
    \A \xx = \zero
  \end{equation}
  有非零解,其中
  $$
  \A = (\alphabd_1,~\alphabd_2,~\cd,~\alphabd_r) = \left(
    \begin{array}{cccc}
      a_{11}&a_{12}&\cd&a_{1r}\\[0.05in]
      a_{21}&a_{22}&\cd&a_{2r}\\[0.05in]
      \vd&\vd&&\vd\\[0.05in]
      a_{n1}&a_{n2}&\cd&a_{nr}.
    \end{array}\right), \xx = \left(
    \begin{array}{c}
      x_1\\[0.05in]
      x_2\\[0.05in]
      \vd\\[0.05in]
      x_r
    \end{array}
  \right)
  $$
\end{dingli}
\begin{proof}
设
\begin{equation}\label{th3.1}
  x_1 \alphabd_1 + x_2 \alphabd_2 + \cd + x_r \alphabd_r = \zero,      
\end{equation}
即
$$
x_1 \left(
  \begin{array}{c}
    a_{11}\\
    a_{21}\\
    \vd \\
    a_{n1}
  \end{array}
\right) + x_2 \left(
  \begin{array}{c}
    a_{12}\\
    a_{22}\\
    \vd \\
    a_{n2} 
  \end{array}
\right)+ \cd + x_r \left(
  \begin{array}{c}
    a_{1r}\\
    a_{2r}\\
    \vd \\
    a_{nr}
  \end{array}
\right) = \zero.
$$
此即齐次线性方程组(\ref{ax})。

\begin{itemize}
\item[\red{($\Rightarrow$)}]     若向量组$\alphabd_1,\alphabd_2,\cd,\alphabd_r$线性相关,
  则必有不全为零的数$x_1,x_2,\cd,x_r$使得(\ref{th3.1})成立,
  即齐次线性方程组(\ref{ax})有非零解。  
\item[\red{($\Leftarrow$)}]     若方程组(\ref{ax})有非零解,就是说有不全为零的数$x_1,x_2,\cd,x_r$使得(\ref{th3.1})成立,故向量组$\alphabd_1,\alphabd_2,\cd,\alphabd_r$线性相关。

\end{itemize}
\end{proof}

\begin{jielun}
  对于齐次线性方程组,如果
  $$
  \red{\mbox{未知量个数} ~~>~~ \mbox{方程个数},}
  $$
  则它必有无穷多解,从而必有非零解。
\end{jielun}   





\begin{dingli}
  任意$n+1$个$n$维向量都是线性相关的。
\end{dingli}

\begin{proof}
对向量组$\alphabd_1,\alphabd_2,\cd,\alphabd_n,\alphabd_{n+1}\in F^n$,设
$$
x_1 \alphabd_1 + x_2 \alphabd_2 + \cd + x_n \alphabd_n + x_{n+1} \alphabd_{n+1} = \zero,  
$$
注意到此齐次线性方程组中,未知量个数为$n+1$,而方程个数为$n$,故方程组一定有无穷多个解,从而必有非零解。
得证$\alphabd_1,\alphabd_2,\cd,\alphabd_n,\alphabd_{n+1}$线性相关。
\end{proof}


\begin{zhu*}
  \begin{itemize}
  \item    向量个数$~>~$向量维数 $~~\red{\Rightarrow}~~$ 向量组必线性相关。 
  \item     在$\mathbb R^n$中,任意一组线性无关的向量最多只能含$n$个向量。
  \end{itemize}
\end{zhu*}

\begin{dingli}
  若向量组$\alphabd_1,\alphabd_2,\cd,\alphabd_r$线性无关,而$\red{\betabd},\alphabd_1,\alphabd_2,\cd,\alphabd_r$线性相关,则$\red{\betabd}$可由$\alphabd_1,\alphabd_2,\cd,\alphabd_r$线性表示,并且表示法惟一。
\end{dingli}
\begin{proof}
因为$\betabd,\alphabd_1,\alphabd_2,\cd,\alphabd_r$线性相关,故存在不全为零的数$k,k_1,k_2,\cd,k_r$使得
$$
k\betabd + k_1\alphabd_1+k_2\alphabd_2+\cd+k_r\alphabd_r=\zero,
$$
其中$k\ne 0 $ \red{(若$k=0$,则由$\alphabd_1,\alphabd_2,\cd,\alphabd_r$线性无关可知$k_1,k_2,\cd,k_r$全为零,这与$k,k_1,k_2,\cd,k_r$不全为零矛盾)}。  
于是$\betabd$可由$\alphabd_1,\alphabd_2,\cd,\alphabd_r$线性表示为
$$
\betabd=-\frac{k_1}k\alphabd_1-\frac{k_2}k\alphabd_2-\cd-\frac{k_r}k\alphabd_r.
$$

\red{再证唯一性} \quad 设有两种表示法
$$
\betabd=l_1\alphabd_1+l_2\alphabd_2+\cd+l_r\alphabd_r,\quad
\betabd=h_1\alphabd_1+h_2\alphabd_2+\cd+h_r\alphabd_r.
$$ 
于是
$$
(l_1-h_1)\alphabd_1+(l_2-h_2)\alphabd_2+\cd+(l_r-h_r)\alphabd_1=\zero,
$$ 
由$\alphabd_1,\alphabd_2,\cd,\alphabd_r$线性无关可知
\red{$l_i-h_i=0, \quad \mbox{即} l_i=h_i.$}
故$\betabd$由$\alphabd_1,\alphabd_2,\cd,\alphabd_r$线性表示的表示法惟一。

\end{proof}

\begin{tuilun}
  如果$F^n$中的$n$个向量$\alphabd_1,\alphabd_2,\cd,\alphabd_n$线性无关,则$F^n$中的任一向量$\alphabd$可由$\alphabd_1,\alphabd_2,\cd,\alphabd_n$线性表示,且表示法惟一。
\end{tuilun}
\begin{proof}
由"任意$n+1$个$n$维向量线性相关''知,$\alphabd,\alphabd_1,\alphabd_2,\cd,\alphabd_n$线性相关,由前述定理可得结论成立。
\end{proof}



\begin{li}
  设$\alphabd_1=(1,-1,1),\alphabd_2=(1,2,0),\alphabd_3=(1,0,3),\alphabd_4=(2,-3,7)$.
  问:
  \begin{itemize}
  \item[(1)]$\alphabd_1,\alphabd_2,\alphabd_3$是否线性相关?
  \item[(2)]$\alphabd_4$可否由$\alphabd_1,\alphabd_2,\alphabd_3$线性表示?如能表示求出其表示式。
  \end{itemize}
\end{li}
\begin{jie}
\begin{itemize}
\item[(1)]    考察
  $
  \A = (\alphabd_1^T, \alphabd_2^T, \alphabd_3^T) = \left(
  \begin{array}{rrr}
    1&1&1\\
    -1&2&0\\
    1&0&3
  \end{array}
  \right). 
  $ \quad
  由$|\A|=7$可知$\A$可逆,故$\A\xx=\zero$只有零解,从而$\alphabd_1,\alphabd_2,\alphabd_3$线性无关。  
\item[(2)] 根据推论,$\alphabd_4$可由$\alphabd_1,\alphabd_2,\alphabd_3$线性表示,且表示法惟一。  设
  $$
  x_1\alphabd_1+x_2\alphabd_2+x_3\alphabd_3=\alphabd_4   \Rightarrow
  x_1\alphabd_1^T+x_2\alphabd_2^T+x_3\alphabd_3^T=\alphabd_4^T       
  $$
  即$$
  \left(
  \begin{array}{ccc}
    \alphabd_1^T &\alphabd_2^T& \alphabd_3^T  
  \end{array}
  \right) \left(
  \begin{array}{c}
    x_1\\
    x_2\\
    x_3
  \end{array}
  \right)= 
  \left(
  \begin{array}{rrr}
    1&1&1\\
    -1&2&0\\
    1&0&3
  \end{array}
  \right) \left(
  \begin{array}{c}
    x_1\\
    x_2\\
    x_3
  \end{array}
  \right) =  \left(
  \begin{array}{r}
    2\\
    -3\\
    7
  \end{array}
  \right)
  $$  
  解此方程组得惟一解$x_1=1,x_2=-1,x_3=2$,故
  $
  \red{\alphabd_4=\alphabd_1-\alphabd_2+2\alphabd_3.}
  $
\end{itemize}
\end{jie}






\begin{li}
  设向量组$\alphabd_1,\alphabd_2,\alphabd_3$线性无关,又$\betabd=\alphabd_1+\alphabd_2+2\alphabd_3$,
  $\betabd_2=\alphabd_1-\alphabd_2$,$\betabd_3=\alphabd_1+\alphabd_3$,证明$\betabd_1,\betabd_2,\betabd_3$线性相关。       
\end{li}
\begin{jie}
设有数$x_1,x_2,x_3$使得
\begin{equation}\label{li5}
  x_1\betabd_1+x_2\betabd_2+x_3\betabd_3=\zero
\end{equation}    

即
$$
x_1(\alphabd_1+\alphabd_2+2\alphabd_3)+x_2(\alphabd_1-2\alphabd_2)+x_3(\alphabd_1+\alphabd_3)=\zero
$$
亦即
$$
(x_1+x_2)\alphabd_1+(x_1-2x_2)\alphabd_2+(x_1+x_3)\alphabd_3=\zero
$$

因为$\alphabd_1,\alphabd_2,\alphabd_3$线性无关,故
$$
\left\{
\begin{array}{rcrcrcrcr}
  x_1&+&x_2&&&=&0\\
  x_1&-&x_2&&&=&0\\
  2x_1&&&+&x_3&=&0.
\end{array}
\right.
$$
求解该方程组可得非零解$(-1,-1,2)$。因此,有不全为零的数$x_1,x_2,x_3$使得(\ref{li5})成立,从而$\betabd_1,\betabd_2,\betabd_3$线性相关。

\end{jie}

\begin{li}
  证明:$\alphabd_1+\alphabd_2,\alphabd_2+\alphabd_3,\alphabd_3+\alphabd_1$线性无关的充要条件是$\alphabd_1,\alphabd_2,\alphabd_3$线性无关。
\end{li}
\begin{proof}
\red{($\Rightarrow$)} \quad
假设$\alphabd_1,\alphabd_2,\alphabd_3$线性相关,则有不全为零的数$x_1+x_2,x_2+x_3,x_3+x_1$使得
$$
(x_1+x_2)\alphabd_1+(x_2+x_3)\alphabd_2+(x_3+x_1)\alphabd_3=\zero
$$
即
$$
x_1(\alphabd_1+\alphabd_2)+x_2(\alphabd_2+\alphabd_3)+x_3(\alphabd_3+\alphabd_1)=\zero
$$

\vspace{0.1in}

\red{($\Leftarrow$)} \quad
设有$x_1,x_2,x_3$使得
\begin{equation}\label{li6-1}
  x_1(\alphabd_1+\alphabd_2)+x_2(\alphabd_2+\alphabd_3)+x_3(\alphabd_3+\alphabd_1)=\zero
\end{equation}
即
$$
(x_1+x_3)\alphabd_1+(x_1+x_2)\alphabd_2+(x_2+x_3)\alphabd_3=\zero
$$
因为$\alphabd_1,\alphabd_2,\alphabd_3$线性无关,故
$$
x_1+x_3=0, \quad x_1+x_2=0, \quad x_2+x_3=0,
$$
该方程组只有零解。这说明若使(\ref{li6-1}),必有$x_1=x_2=x_3=0$,从而$\alphabd_1+\alphabd_2,\alphabd_2+\alphabd_3,\alphabd_3+\alphabd_1$线性无关。

\end{proof}


\begin{dingli}
  \begin{itemize}
  \item[(1)] 如果一组$n$维向量$\alphabd_1,\alphabd_2,\cd,\alphabd_s$线性无关,那么把这些向量各任意添加$m$个分量所得的向量(\red{$n+m$维})组$\alphabd^*_1,\alphabd^*_2,\cd,\alphabd^*_s$也线性无关。亦即
    $$
\left(
\begin{array}{c}
  a_{11}\\
  a_{21}\\
  \vd\\
  a_{n1}\\
\end{array}
\right)
\cd,
\left(
\begin{array}{c}
  a_{1s}\\
  a_{2s}\\
  \vd\\
  a_{ns}\\
\end{array}
\right) \mbox{线性无关}  ~~~\blue{\Rightarrow}~~~
\left(
\begin{array}{c}
  a_{11}\\
  a_{21}\\
  \vd\\
  a_{n1}\\
  \red{a_{n+1,1}}\\
  \vd\\
  \red{a_{n+m,1}}
\end{array}
\right),
\cd,
\left(
\begin{array}{c}
  a_{1s}\\
  a_{2s}\\
  \vd\\
  a_{ns}\\
  \red{a_{n+1,s}}\\
  \vd\\
  \red{a_{n+m,s}}
\end{array}
\right) \mbox{线性无关}
$$
\item[(2)] 如果$\alphabd_1,\alphabd_2,\cd,\alphabd_s$线性相关,那么它们各去掉相同的若干个分量所得到的新向量也线性相关,亦即
  $$  
\left(
\begin{array}{c}
  a_{11}\\
  a_{21}\\
  \vd\\
  a_{n1}\\
  \red{a_{n+1,1}}\\
  \vd\\
  \red{a_{n+m,1}}
\end{array}
\right),
\cd,
\left(
\begin{array}{c}
  a_{1s}\\
  a_{2s}\\
  \vd\\
  a_{ns}\\
  \red{a_{n+1,s}}\\
  \vd\\
  \red{a_{n+m,s}}
\end{array}
\right) \mbox{线性相关}  ~~~\blue{\Rightarrow}~~~
\left(
\begin{array}{c}
  a_{11}\\
  a_{21}\\
  \vd\\
  a_{n1}\\
\end{array}
\right)
\cd,
\left(
\begin{array}{c}
  a_{1s}\\
  a_{2s}\\
  \vd\\
  a_{ns}\\
\end{array}
\right) \mbox{线性相关}
$$
  \end{itemize}
\end{dingli}
\begin{proof}
两者互为逆否命题,证明第一个即可。 
向量组$\alphabd_1,\alphabd_2,\cd,\alphabd_s$线性无关,则方程组
$$
x_1\alphabd_1+x_2\alphabd_2+\cd+x_s\alphabd_s=\zero
$$
只有零解。 设$\alphabd_i=(a_{1i},a_{2i},\cd,a_{ni})^T,~~ i=1,2,\cd,s$,即
\begin{equation}\label{ls_ns}
  \left\{
  \begin{array}{rcrcrcrcr}
    a_{11}x_1&+&a_{12}x_2&+&\cd&+&a_{1s}x_s&=&0,\\[0.05in]
    a_{21}x_1&+&a_{22}x_2&+&\cd&+&a_{2s}x_s&=&0,\\[0.05in]
    &&&&\cd&&&&\\[0.05in]
    a_{n1}x_1&+&a_{n2}x_2&+&\cd&+&a_{ns}x_s&=&0.
  \end{array}
  \right.
\end{equation}
只有零解。 不妨设每个向量增加了一个分量,即
$$
\alphabd_i^*= (a_{1i},a_{2i},\cd,a_{ni},\red{a_{n+1,i}})^T, ~~ ii=1,2,\cd,s.
$$ 
设
$$
x_1\alphabd_1^*+x_2\alphabd_2^*+\cd+x_s\alphabd_s^*=\zero
$$
即
\begin{equation}\label{ls_ns1}
  \left\{
  \begin{array}{rcrcrcrcr}
    a_{11}x_1&+&a_{12}x_2&+&\cd&+&a_{1s}x_s&=&0,\\[0.05in]
    a_{21}x_1&+&a_{22}x_2&+&\cd&+&a_{2s}x_s&=&0,\\[0.05in]
    &&&&\cd&&&&\\[0.05in]
    a_{n1}x_1&+&a_{n2}x_2&+&\cd&+&a_{ns}x_s&=&0,\\[0.05in]
    \red{a_{n+1,1}x_1}&\red{+}&\red{a_{n+1,2}x_2}&\red{+}&\red{\cd}&\red{+}&\red{a_{n+1,s}x_s}&\red{=}&\red{0}.
  \end{array}
  \right.
\end{equation}
方程组(\ref{ls_ns1})的解全是方程组(\ref{ls_ns})的解。 而方程组(\ref{ls_ns})只有零解,故方程组(\ref{ls_ns1})也只有零解。故向量组$\alphabd^*_1,\alphabd^*_2,\cd,\alphabd^*_s$线性无关。
\end{proof}

\begin{tuilun}
  设向量组线性相关,若增加的分量全为零,则得到的新向量组也线性相关。
\end{tuilun}
\begin{proof}
设$\alphabd_1,\alphabd_2,\cd,\alphabd_s$线性相关,把这些向量各任意添加$m$个全为零的分量,
所得到的新向量组记为$\alphabd^*_1,\alphabd^*_2,\cd,\alphabd^*_s$。 
此时方程组
$$
x_1\alphabd_1+x_2\alphabd_2+\cd+x_s\alphabd_s=\zero
$$ 
与方程组
$$
x_1\alphabd_1^*+x_2\alphabd_2^*+\cd+x_s\alphabd_s^*=\zero
$$
完全相同。所以新向量组$\alphabd^*_1,\alphabd^*_2,\cd,\alphabd^*_s$也线性相关。
\end{proof}
\purple{对应位置全为零的向量,不影响向量组的线性相关性。}

如
$$  
\left(
\begin{array}{c}
  a_{11}\\
  0\\
  a_{21}\\
  \vd\\
  a_{n1}\\
  0\\
  \vd\\
  0
\end{array}
\right),
\cd,
\left(
\begin{array}{c}
  a_{1s}\\
  0\\
  a_{2s}\\
  \vd\\
  a_{ns}\\
  0\\
  \vd\\
  0
\end{array}
\right) \mbox{~~与~~}
\left(
\begin{array}{c}
  a_{11}\\
  a_{21}\\
  \vd\\
  a_{n1}\\
\end{array}
\right),
\cd,
\left(
\begin{array}{c}
  a_{1s}\\
  a_{2s}\\
  \vd\\
  a_{ns}\\
\end{array}
\right) 
$$
线性相关性一致。

\begin{li}
  考察以下向量组的线性相关性:
  $$
  \left(
  \begin{array}{c}
    1\\
    0\\
    0\\
    2\\
    5
  \end{array}
  \right), \quad
  \left(
  \begin{array}{c}
    0\\
    1\\
    0\\
    6\\
    9
  \end{array}
  \right), \quad
  \left(
  \begin{array}{c}
    0\\
    0\\
    1\\
    4\\
    3
  \end{array}
  \right)
  $$
\end{li}
\begin{jie}
去掉最后两个分量所得的向量组
$$
\left(
\begin{array}{c}
  1\\
  0\\
  0
\end{array}
\right), \quad
\left(
\begin{array}{c}
  0\\
  1\\
  0
\end{array}
\right), \quad
\left(
\begin{array}{c}
  0\\
  0\\
  1
\end{array}
\right)
$$
线性无关,故原向量组线性无关。
\end{jie}


