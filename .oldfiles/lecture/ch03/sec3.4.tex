\section{齐次线性方程组有非零解的条件及解的结构}

设$\mathbf A$为$m\times n$矩阵,考察以$\A$为系数矩阵的齐次线性方程组
\begin{equation}\label{ax0}
  \A\xx=\zero.
\end{equation}    
若将$\A$按列分块为
$$
\A = (\alphabd_1,~\alphabd_2,~\cd,~\alphabd_n),
$$
齐次方程组(\ref{ax0})可表示为
$$
x_1\alphabd_1+x_2\alphabd_2+\cd+x_n\alphabd_n=\zero.
$$
而齐次方程组(\ref{ax0})有\blue{非零解}的充分必要条件是$\alphabd_1,~\alphabd_2,~\cd,~\alphabd_n$\blue{线性相关},即
$$
\rank(\A) = \rank(\alphabd_1,~\alphabd_2,~\cd,~\alphabd_n) < n.
$$



\begin{dingli}
  设$\A$为$m\times n$矩阵,则
  $$
  \blue{\underline{\A\xx=\zero\mbox{有非零解}}} ~~\Longleftrightarrow~~
  \blue{\underline{\rank(\A)<n}}.$$
\end{dingli}


\begin{dingli}[上述定理的等价命题]
  设$\A$为$m\times n$矩阵,则
  $$
  \blue{\underline{\A\xx=\zero\mbox{只有零解}}} ~~\Longleftrightarrow~~
  \blue{\underline{\rank(\A)=n=\A\mbox{的列数}}}.
  $$
\end{dingli}


\begin{li}
  设$\A$为$n$阶矩阵,证明:存在$n\times s$矩阵$\B\ne \zero$,使得$\A\B=\zero$的充分必要条件是
  $$
  |\A|=0.
  $$      
\end{li}
\begin{proof}
$|\A|=0~~\Longleftrightarrow~~\A\xx=\zero$有非零解。 下证
$$
\purple{\mbox{存在}n\times s\mbox{矩阵}\B\ne 0\mbox{使得}\A\B=\zero
  ~~\Longleftrightarrow~~\A\xx=\zero\mbox{有非零解}}.
$$     
\begin{itemize}
\item[($\red{\Longrightarrow}$)]
  设$\A\B=\zero$,则$\B$的解向量为$\A\xx=\zero$的解。又$\B\ne\zero$,则$\B$至少有一个非零列向量,
  从而$\A\xx=\zero$至少有一个非零解。
\item[($\red{\Longleftarrow}$)]
  设$\A\xx=\zero$有非零解,任取一个非零解$\betabd$,令
  $$
  \B=(\betabd,~\zero,~\cd,~\zero)
  $$
  则$\B\ne\zero$,且$\A\B=\zero$。
\end{itemize}
\end{proof}

\begin{dingli}
  若$\xx_1,~\xx_2$为齐次线性方程组$\A\xx=\zero$的两个解,则
  $$
  k_1\xx_1+k_2\xx_2\quad(k_1,~k_2\mbox{为任意常数})
  $$
  也是它的解。
\end{dingli}
\begin{proof}
因为
$$
\A(\red{k_1\xx_1+k_2\xx_2}) = k_1\A\xx_1+k_2\A\xx_2 = k_1\zero+k_2\zero=\zero,
$$
故$k_1\xx_1+k_2\xx_2$也为$\A\xx=\zero$的解。
\end{proof}




\begin{dingyi}[基础解系]
  设$\xx_1,~\xx_2,~\cd,~\xx_p$为$\A\xx=\zero$的解向量,若
  \begin{itemize}
  \item[(1)] $\xx_1,~\xx_2,~\cd,~\xx_p$线性无关
  \item[(2)] $\A\xx=\zero$的任一解向量可由$\xx_1,~\xx_2,~\cd,~\xx_p$线性表示。
  \end{itemize}
  则称$\xx_1,~\xx_2,~\cd,~\xx_p$为$\A\xx=\zero$的一个\blue{\underline{基础解系}}。
\end{dingyi}


\begin{zhu}
  关于基础解系,请注意以下几点:
  \begin{itemize}
  \item[(1)] 基础解系即全部解向量的\blue{极大无关组}。\\[0.1in]  
  \item[(2)] 找到了基础解系,就找到了齐次线性方程组的全部解:
    $$
    k_1\xx_1+k_2\xx_2+\cd+k_p\xx_p \quad(k_1,k_2,\cd,k_p\mbox{为任意常数}).
    $$ 
  \item[(3)] 基础解系\blue{不唯一}。
  \end{itemize}
\end{zhu}





\begin{li}
  求方程组
  $$
  x+y+z=0
  $$
  的全部解。
\end{li}

\begin{jie}
\begin{itemize}
\item[(1)] 选取$y,z$为自由未知量,则
  $$
  \left\{
    \begin{array}{cccccc}
      x&=&-&y&-&z\\
      y&=&&y&&\\
      z&=&&&&z
    \end{array}
  \right.
  $$
  则方程组的全部解为
  $$
  \left(
    \begin{array}{r}
      x\\y\\z
    \end{array}
  \right) = c_1      \left(
    \begin{array}{r}
      -1\\1\\0
    \end{array}
  \right) + c_2      \left(
    \begin{array}{r}
      -1\\0\\1
    \end{array}
  \right) \quad (c_1,c_2\mbox{为任意常数})
  $$
\end{itemize}





\begin{itemize}
\item[(2)] 选取$x,z$为自由未知量,则
  $$
  \left\{
    \begin{array}{cccccc}
      x&=&&x&&\\
      y&=&-&x&-&z\\
      z&=&&&&z
    \end{array}
  \right.
  $$
  则方程组的全部解为
  $$
  \left(
    \begin{array}{r}
      x\\y\\z
    \end{array}
  \right) = c_1      \left(
    \begin{array}{r}
      1\\-1\\0
    \end{array}
  \right) + c_2      \left(
    \begin{array}{r}
      0\\-1\\1
    \end{array}
  \right) \quad (c_1,c_2\mbox{为任意常数})
  $$  
\item[(3)] 选取$x,y$为自由未知量,则
  $$
  \left\{
    \begin{array}{cccccc}
      x&=&&x&&\\
      y&=&&&&y\\
      z&=&-&x&-&y
    \end{array}
  \right.
  $$
  则方程组的全部解为
  $$
  \left(
    \begin{array}{r}
      x\\y\\z
    \end{array}
  \right) = c_1      \left(
    \begin{array}{r}
      1\\0\\-1
    \end{array}
  \right) + c_2      \left(
    \begin{array}{r}
      0\\1\\-1
    \end{array}
  \right) \quad (c_1,c_2\mbox{为任意常数})
  $$
\end{itemize}






三个不同的基础解系为
$$
\left\{
  \left(
    \begin{array}{r}
      -1\\1\\0
    \end{array}
  \right),~~
  \left(
    \begin{array}{r}
      -1\\0\\1
    \end{array}
  \right)
\right\},
$$
$$
\left\{
  \left(
    \begin{array}{r}
      1\\-1\\0
    \end{array}
  \right),~~
  \left(
    \begin{array}{r}
      0\\-1\\1
    \end{array}
  \right)
\right\},
$$
$$
\left\{
  \left(
    \begin{array}{r}
      1\\0\\-1
    \end{array}
  \right),~~
  \left(
    \begin{array}{r}
      0\\1\\-1
    \end{array}
  \right)
\right\}.
$$
\end{jie}





\begin{dingli}
  设$\A$为$m\times n$矩阵,若$\rank(\A)=r<n$,则齐次线性方程组$\A\xx=\zero$存在基础解系,
  且基础解系含$n-r$个解向量。
\end{dingli}

\begin{zhu}
  注意以下两点:
  \begin{itemize}
  \item $r$为$\A$的秩,也是$\A$的行阶梯形矩阵的非零行行数,是非自由未知量的个数。 
  \item $n$为未知量的个数,故$n-r$为自由未知量的个数。 有多少自由未知量,基础解系里就对应有多少个向量。
  \end{itemize}
\end{zhu}





\begin{li}
  求齐次线性方程组$\A\xx=\zero$的基础解系,其中
  $$
  \A = \left(
    \begin{array}{rrrr}
      1&-8&10&2\\
      2&4&5&-1\\
      3&8&6&-2
    \end{array}
  \right).
  $$
\end{li}
\begin{jie}
$$
\begin{array}{l}
  \left(
  \begin{array}{rrrr}
    1&-8&10&2\\
    2&4&5&-1\\
    3&8&6&-2
  \end{array}
           \right) \xlongrightarrow[r_3-3r_1]{r_2-2r_1}
           \left(
           \begin{array}{rrrr}
             1&-8&10&2\\
             0&20&-15&-5\\
             0&32&24&-8
           \end{array}
                      \right)\\[0.3in]
  \xlongrightarrow[r_2\div4]{r_3\div8}
  \left(
  \begin{array}{rrrr}
    1&-8&10&2\\
    0&4&-3&-1\\
    0&4&-3&-1
  \end{array}
            \right) \xlongrightarrow[r_1+2r_2]{r_3-r_2}
            \left(
            \begin{array}{rrrr}
              1&0&4&0\\
              0&4&-3&-1\\
              0&0&0&0
            \end{array}
                     \right) \\[0.3in]
  \xlongrightarrow[]{r_2\div4}
  \left(
  \begin{array}{rrrr}
    1&0&4&0\\
    0&1&-3/4&-1/4\\
    0&0&0&0
  \end{array}
           \right)
\end{array}
$$
原方程等价于
$$\left\{
  \begin{array}{rcrcrc}
    x_1&=&-4&x_3&&\\[0.1in]
    x_2&=&\frac34&x_3&+\frac14&x_4
  \end{array}
\right.  \Leftrightarrow
\left\{
  \begin{array}{rcrcrc}
    x_1&=&-4&x_3&&\\[0.1in]
    x_2&=&\frac34&x_3&+\frac14&x_4\\[0.1in]
    x_3&=&&x_3&&\\[0.1in]
    x_4&=&&&&x_4      
  \end{array}
\right.
$$
基础解系为
$$
\xibd_1 = \left(
  \begin{array}{r}
    -4\\[0.1in]
    \frac34\\[0.1in]
    1\\[0.1in]
    0
  \end{array}
\right), \quad \xibd_2 = \left(
  \begin{array}{r}
    0\\[0.1in]
    \frac14\\[0.1in]
    0\\[0.1in]
    1
  \end{array}
\right)
$$
\end{jie}


\begin{li}
  求齐次线性方程组
  $$
  nx_1+(n-1)x_2+\cd+2x_{n-1}+x_n=0
  $$
  的基础解系。      
\end{li}

\begin{jie}
原方程等价于$x_n=-nx_1-(n-1)x_2-\cd-2x_{n-1}$, 即
$$
\left\{
  \begin{array}{rcrrrr}
    x_1&=&x_1&&&\\
    x_2&=&&x_2&&\\
       &\vd&&&&\\
    x_{n-1}&=&&&&x_{n-1}\\      
    x_n&=&-nx_1&-(n-1)x_2&\cd&-2x_{n-1}
  \end{array}    
\right.
$$
基础解系为
$$
(\xibd_1,\xibd_2,\cd,\xibd_{n-1})=\left(
  \begin{array}{rrrr}
    1&0&\cd&0\\
    0&1&\cd&0\\
    \vd&\vd&&\vd\\
    0&0&\cd&1\\
    -n&-n+1&\cd&-2
  \end{array}
\right)
$$
\end{jie}




\begin{li}
  设$\A$与$\B$分别为$m\times n$和$n\times s$矩阵,且$\A\B=\zero$。证明:
  $$
  \rank(\A)+\rank(\B)\le n.
  $$
\end{li}

\begin{proof}
由$\A\B=\zero$知,$\B$的列向量是$\A\xx=\zero$的解。
故$\B$的列向量组的秩,不超过$\A\xx=\zero$的基础解系的秩,即
$$
\rank(\B) \le n-\rank(\A),
$$
即
$$
\rank(\A)+\rank(\B)\le n.
$$
\end{proof}

\begin{li}
  设$n$元齐次线性方程组$\A\xx=\zero$与$\B\xx=\zero$同解,证明
  $$
  \rank(\A)=\rank(\B).
  $$
\end{li}

\begin{jie}
$\A\xx=\zero$与$\B\xx=\zero$同解,故它们有相同的基础解系,而基础解系包含的向量个数相等,即
$$
n-\rank(\A)=n-\rank(\B),
$$
从而
$$
\rank(\A)=\rank(\B).
$$
\end{jie}





\begin{li}
  设$\A$为$m\times n$实矩阵,证明$\rank(\A^T\A)=\rank(\A)$。    
\end{li}
\begin{proof}
只需证明$\A\xx=\zero$与$(\A^T\A)\xx=\zero$同解。
\begin{itemize}
\item[(1)] 若$\xx$满足$\A\xx=\zero$,则有$(\A^T\A)\xx=\A^T(\A\xx)=\zero$。 
\item[(2)] 若$\xx$满足$\A^T\A\xx=\zero$,则
  $$
  \xx^T\A^T\A\xx=\zero,
  $$
  即
  $$
  (\A\xx)^T\A\xx=\zero,
  $$
  故$\A\xx=\zero$。
\end{itemize}
\end{proof}





