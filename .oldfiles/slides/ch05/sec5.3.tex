\section{矩阵可对角化的条件}
\begin{frame}[fragile]\ft{\secname}
矩阵可对角化,即矩阵与对角阵相似。    
\end{frame}

\begin{frame}[fragile]\ft{\secname}

\begin{dingli}
  $\mbox{矩阵可对角化} ~~\Longleftrightarrow~~
  \mbox{$n$阶矩阵有$n$个线性无关的特征向量}$ 
\end{dingli}
\end{frame}

\begin{frame}[fragile]\ft{\secname}
\begin{proof}
\begin{itemize}
\item[\blue{($\Rightarrow$)}] 设
  $$
  \PP^{-1}\A\PP=\Lambdabd  ~~\Longrightarrow~~
  \A\PP=\PP\Lambdabd
  $$
  将$\PP$按列分块,即
  $
  \PP=(\xx_1,~\xx_2,~\cd,~\xx_n),
  $
  则
  $$
  \A(\xx_1,~\xx_2,~\cd,~\xx_n)=(\xx_1,~\xx_2,~\cd,~\xx_n)\left(
    \begin{array}{cccc}
      \lambda_1&&&\\
               &\lambda_2&&\\
               &&\dd&\\
               &&&\lambda_n
    \end{array}
  \right)
  $$
  于是
  $$
  \A\xx_i=\lambda_i\xx_i\quad(i=1,2,\cd,n).
  $$
  故$\xx_1,~\xx_2,~\cd,~\xx_n$是$\A$分别对应于$\lambda_1,~\lambda_2,~\cd,~\lambda_n$的特征向量。由于$\PP$可逆,所以它们是线性无关的。
  \item[\blue{($\Leftarrow$)}] 上述步骤显然可逆,故充分性也成立。
\end{itemize}
\end{proof}





\end{frame}

\begin{frame}[fragile]\ft{\secname}

若$\A$与$\Lambdabd$相似,则$\Lambdabd$的主对角元都是$\A$的特征值。
若不计$\lambda_k$的排列次序,则$\Lambdabd$是唯一的,称$\Lambdabd$为$\A$的相似标准型。




\end{frame}

\begin{frame}[fragile]\ft{\secname}


\begin{dingli}
  $\A$的属于不同特征值的特征向量是线性无关的。
\end{dingli} 
\end{frame}

\begin{frame}[fragile,allowframebreaks]\ft{\secname}

  \begin{proof}
设$\A$的$m$个互不相同的特征值为$\lambda_1,\lambda_2,\cd,\lambda_m$,其相应的特征向量为$\xx_1,~\xx_2,~\cd,~\xx_m$.
对$m$做数学归纳法。
\begin{itemize}
\item[$1^o$] 当$m=1$时,结论显然成立。
\item[$2^o$] 设$k$个不同特征值$\lambda_1,\lambda_2,\cd,\lambda_k$的特征向量$\xx_1,~\xx_2,~\cd,~\xx_k$。下面考虑$k+1$个不同特征值的特征向量。
  
  设
  $$
  \begin{array}{rc}
    & a_1\xx_1+a_2\xx_2+\cd+a_k\xx_k+a_{k+1}\xx_{k+1}=\zero\qquad(1)\\[0.1cm]
    \Longrightarrow&
                     \A(a_1\xx_1+a_2\xx_2+\cd+a_k\xx_k+a_{k+1}\xx_{k+1})=\zero\\[0.1cm]
    \Longrightarrow& 
                     a_1\lambda_1\xx_1+a_2\lambda_2\xx_2+\cd+a_k\lambda_k\xx_k+a_{k+1}\lambda_{k+1}\xx_{k+1}=\zero\quad(2)\\[0.1cm]
    \Longrightarrow&
                                              a_1(\lambda_{k+1}-\lambda_1)\xx_1+a_2(\lambda_{k+1}-\lambda_2)\xx_2+\cd+a_k(\lambda_{k+1}-\lambda_k)\xx_k=\zero\\[0.1cm]
    \Longrightarrow&
                     a_i(\lambda_{k+1}-\lambda_i)=0, ~~i=1,2,\cd,k\\[0.1cm]
    \Longrightarrow&
                     a_i=0, ~~i=1,2,\cd,k\\[0.1cm]
    \Longrightarrow&
                     a_{k+1}\xx_{k+1}=0\\[0.1cm]
    \Longrightarrow&
                     a_{k+1}=0\\[0.1cm]
    \Longrightarrow&
                     \xx_1,~\xx_2,~\cd,~\xx_k,~~\xx_{k+1}\mbox{线性无关}
  \end{array}
  $$
\end{itemize}
\end{proof}


\end{frame}

\begin{frame}[fragile]\ft{\secname}



\begin{tuilun}
  若$\A$有$n$个互不相同的特征值,则$\A$与对角阵相似。
\end{tuilun}


\end{frame}

\begin{frame}[fragile]\ft{\secname}



\begin{li}
  设实对称矩阵
  $$
  \A=\left(
    \begin{array}{rrrr}
      1&-1&-1&-1\\
      -1&1&-1&-1\\
      -1&-1&1&-1\\
      -1&-1&-1&1
    \end{array}
  \right)
  $$
  问$\A$是否可对角化?若可对角化,求对角阵$\Lambdabd$及可逆矩阵$\PP$使得$\PP^{-1}\A\PP=\Lambdabd$,再求$\A^k$。
\end{li}
\end{frame}

\begin{frame}[fragile,allowframebreaks]\ft{\secname}
\begin{jie}
  由
  $$
  \begin{aligned}
  |\A-\lambda\II|&=
  \left|
    \begin{array}{rrrr}
      1-\lambda&-1&-1&-1\\
      -1&1-\lambda&-1&-1\\
      -1&-1&1-\lambda&-1\\
      -1&-1&-1&1-\lambda
    \end{array}
  \right|\\
  &=-(\lambda+2)
  \left|
    \begin{array}{rrrr}
      1&-1&-1&-1\\
      1&1-\lambda&-1&-1\\
      1&-1&1-\lambda&-1\\
      1&-1&-1&1-\lambda
    \end{array}
  \right|\\
  &=-(\lambda+2)
  \left|
    \begin{array}{rrrr}
      1&-1&-1&-1\\
      0&2-\lambda&0&0\\
      0&0&2-\lambda&0\\
      0&0&0&2-\lambda
    \end{array}
  \right|=(\lambda+2)(\lambda-2)^3,
  \end{aligned}
  $$
  故$\A$的特征值为$\lambda_1=-2$(单根),$\lambda_2=2$(三重根)。

  由$(\A-\lambda_1\II)\xx=0$,即
  $$
  \left(
    \begin{array}{rrrr}
      3&-1&-1&-1\\
      -1&3&-1&-1\\
      -1&-1&3&-1\\
      -1&-1&-1&3
    \end{array}
  \right)
  \left(
    \begin{array}{c}
      x_1\\x_2\\x_3\\x_4
    \end{array}
  \right)=
  \left(
    \begin{array}{c}
      0\\0\\0\\0
    \end{array}
  \right)
  $$
  得$\lambda_1$对应的特征向量为$\{k_1\xx_1|\xx_1=(1,1,1,1)^T, k_1\ne 0\}$。

  由$(\A-\lambda_2\II)\xx=0$,即
  $$
  \left(
    \begin{array}{rrrr}
      -1&-1&-1&-1\\
      -1&-1&-1&-1\\
      -1&-1&-1&-1\\
      -1&-1&-1&-1
    \end{array}
  \right)
  \left(
    \begin{array}{c}
      x_1\\x_2\\x_3\\x_4
    \end{array}
  \right)=
  \left(
    \begin{array}{c}
      0\\0\\0\\0
    \end{array}
  \right)
  $$
  得基础解系:
  $$
  \xx_{21}=(1,-1,0,0)^T, \quad
  \xx_{22}=(1,0,-1,0)^T, \quad
  \xx_{23}=(1,0,0,-1)^T.
  $$

  因$\A$有$4$个线性无关的特征向量,故$\A\sim \Lambdabd$。

  取
  $$
  \PP=(\xx_1,\xx_{21},\xx_{22},\xx_{23})=
  \left(
    \begin{array}{rrrr}
      1& 1& 1&-1\\
      1&-1& 0& 0\\
      1& 0&-1& 0\\
      1& 0& 0&-1
    \end{array}
  \right),
  $$
  则
  $$
  \PP^{-1}\A\PP=\left(
    \begin{array}{rrrr}
      -2&&&\\
        &2&&\\
        &&2\\
        &&&2
    \end{array}
  \right)=\Lambdabd.
  $$
  再由$\A=\PP\Lambdabd\PP^{-1}$得
  $$
  \begin{aligned}
    \A^k&=(\PP\Lambdabd\PP^{-1})^k=\PP\Lambdabd^k\PP^{-1}\\
    &=\left(
    \begin{array}{rrrr}
      1& 1& 1&-1\\
      1&-1& 0& 0\\
      1& 0&-1& 0\\
      1& 0& 0&-1
    \end{array}
  \right)
  \left(
    \begin{array}{rrrr}
      (-2)^k&&&\\
        &2^k&&\\
        &&2^k\\
        &&&2^k
    \end{array}
  \right)
  \frac14
  \left(
    \begin{array}{rrrr}
      1& 1& 1& 1\\
      1&-3& 1& 1\\
      1& 1&-3& 1\\
      1& 1& 1&-3
    \end{array}
  \right)\\
  &=\left\{
    \begin{array}{ll}
      2^k \II_4, & k\mbox{ even}, \\
      2^{k-1} \A, & k\mbox{ odd}.
    \end{array}
  \right.
  \end{aligned}
  $$
\end{jie}

\end{frame}

\begin{frame}[fragile]\ft{\secname}

\begin{li}
  设
  $$
  \A=\left(
    \begin{array}{ccc}
      0&0&1\\
      1&1&x\\
      1&0&0
    \end{array}
  \right)
  $$
  问$x$为何值时,矩阵$\A$能对角化?
\end{li}
\end{frame}

\begin{frame}[fragile]\ft{\secname}

\begin{jie}
  由
  $$
  |\A-\lambda\II|=
  \left|
    \begin{array}{ccc}
      -\lambda&0&1\\
      1&1-\lambda&x\\
      1&0&-\lambda
    \end{array}
  \right|
  =(1-\lambda)
  \left|
    \begin{array}{ccc}
      -\lambda&1\\
      1&-\lambda
    \end{array}
  \right|=-(\lambda-1)^2(\lambda+1),
  $$
  即$\lambda_1=-1,\lambda_2=\lambda_3=1$。

  对应于单根$\lambda_1=-1$,可求得线性无关的特征向量恰有$1$个,故$\A$可对角化的充分必要条件是对应重根$\lambda_2=\lambda_3=1$,有$2$个线性无关的特征向量,即$(\A-\II)\xx=0$有两个线性无关的解,亦即$\A-\II$的秩$\rank(\A-\II)=1$。由
  $$
  \A-\II=\left(
    \begin{array}{ccc}
      -1&0&1\\
      1&0&x\\
      1&0&-1
    \end{array}
  \right)\sim \left(
    \begin{array}{ccc}
      1&0&-1\\
      0&0&x+1\\
      0&0&0
    \end{array}
  \right)
  $$
  欲使$\rank(\A-\II)=1$,须有$x+1=0$,即$x=-1$。因此当$x=-1$时,矩阵$\A$能对角化。
\end{jie}

\end{frame}

\begin{frame}[fragile]\ft{\secname}

\begin{li}
  设$\A=(a_{ij})_{n\times n}$是主对角元全为$2$的上三角矩阵,且存在$a_{ij}\ne 0(i<j)$,问$\A$是否可对角化?
\end{li}

\end{frame}

\begin{frame}[fragile]\ft{\secname}
\begin{jie}
  设
  $$
  A=\left(
    \begin{array}{cccc}
      2&*&\cd&*\\
       &2&\cd&*\\
       &&\dd&\vd\\
       &&&2
    \end{array}
  \right)
  $$
  其中$*$为不全为零的任意常数,则
  $$
  |\A-\lambda\II|=(2-\lambda)^n,
  $$
  即$\lambda=2$为$\A$的$n$重特征根,而$\rank(\A-2\II)\ge 1$,故$(\A-2\II)\xx=0$的基础解系所含向量个数$\le n-1$个,即$\A$的线性无关的特征向量的个数$\le n-1$个,因此$\A$不与对角阵相似。
\end{jie}
\end{frame}
