\subsection{特征值与特征值的性质}



\begin{dingli}
  若$\xx_1$和$\xx_2$都是$\A$的对应于特征值$\lambda_0$的特征向量,则$k_1\xx_1+k_2\xx_2$也是$\A$的对应于特征值$\lambda_0$的特征向量(其中$k_1,k_2$为任意常数,但$k_1\xx_1+k_2\xx_2\ne 0$)。
\end{dingli}

\begin{proof}
  由于$\xx_1$和$\xx_2$是齐次线性方程组
  $$
  (\A-\lambda_0\II)\xx=0
  $$
  的解,因此$k_1\xx_1+k_2\xx_2$也是上式的解,故当$k_1\xx_1+k_2\xx_2\ne0$时,是$\A$的属于$\lambda_0$的特征向量。
\end{proof}

在$(\A-\lambda\II)\xx=0$的解空间中,除零向量以外的全体解向量就是$\A$的属于特征值$\lambda$的全体特征向量。因此,$(\A-\lambda\II)\xx=0$的解空间也称为$\A$关于特征值$\lambda$的特征子空间,记作$V_\lambda$。$n$阶矩阵$\A$的特征子空间就是$n$维向量空间的子空间,其维数为
$$
dim V_\lambda = n - rank(\A-\lambda \II).
$$
需要注意的是,$n$维实矩阵的特征值可能是复数,所以特征子空间一般是$n$维复向量空间$\mathbb C^n$的子空间。

上例中,矩阵$\A=\left(
    \begin{array}{rrr}
      5&-1&-1\\
      3&1&-1\\
      4&-2&1
    \end{array}
  \right)$的两个特征子空间为
  $$
  \begin{aligned}
    V_{\lambda_1}=\{k\xx | \xx = (1,1,1)^T, k \in \mathbb C\},\\
    V_{\lambda_2}=\{k\xx | \xx = (1,1,2)^T, k \in \mathbb C\}.
  \end{aligned}
  $$

\begin{dingli}
  设$n$阶矩阵$\A=(a_{ij})$的$n$个特征值为$\lambda_1,\lambda_2,\cd,\lambda_n$,则
  \begin{itemize}
  \item[(1)] $\ds \sum_{i=1}^n\lambda_i=\sum_{i=1}^na_{ii}$;
  \item[(2)] $\ds \prod_{i=1}^n\lambda_i=\det(\A)$,         
  \end{itemize}
  其中$\sum_{i=1}^na_{ii}$是$\A$的主对角元之和,称为$\A$的迹(trace),记为$tr(\A)$。
\end{dingli}
\begin{proof}
  设
  $$
  \det(\A-\lambda\II)=\left|
    \begin{array}{rrrr}
      a_{11}-\lambda&a_{12}&\cd&a_{1n}\\
      a_{21}&a_{22}-\lambda&\cd&a_{2n}\\
      \vd&\vd&&\vd\\
      a_{n1}&a_{n2}&\cd&a_{nn}-\lambda
    \end{array}
    \right|=\lambda^n+c_1\lambda^{n-1}+c_2\lambda^{n-2}+\cd+c_{n-1}\lambda+c_n,
    $$
    展开后含$\lambda^{n-1}$项的行列式有下面$n$个
    $$
    \left|
    \begin{array}{ccccc}
      a_{11}&&&&\\
      a_{21}&-\lambda&&&\\
      a_{31}&&-\lambda&&\\
      \vd&&&\dd&\\
      a_{n1}&&&&-\lambda
    \end{array}
  \right|,
  \left|
    \begin{array}{ccccc}
      -\lambda&a_{12}&&&\\
              &a_{22}&&&\\
              &a_{32}&-\lambda&&\\
              &\vd&&\dd&\\
              &a_{n2}&&&-\lambda
    \end{array}
  \right|,\cd,
  \left|
    \begin{array}{ccccc}
      -\lambda&&&&a_{1n}\\
             &-\lambda&&&a_{2n}\\
             &&-\lambda&&a_{3n}\\
             &&&\dd&\vd\\
             &&&&a_{nn}
    \end{array}
  \right|
  $$
  它们之和等于
  $$
  (a_{11}+a_{22}+\cd+a_{nn})(-\lambda)^{n-1}=(-1)^{n-1}\sum_{i=1}^na_{ii}\lambda^{n-1},
  $$
  即$c_1=(-1)^{n-1}\sum_{i=1}^na_{ii}$。

  展开后常数项为
  $$
  \left|
    \begin{array}{rrrr}
      a_{11}&a_{12}&\cd&a_{1n}\\
      a_{21}&a_{22}&\cd&a_{2n}\\
      \vd&\vd&&\vd\\
      a_{n1}&a_{n2}&\cd&a_{nn}
    \end{array}
  \right|=\det(\A),
  $$
  即$c_n=\det(\A)$。

  假设$\A$的$n$个特征值为$\lambda_1,\lambda_2,\cd,\lambda_n$,根据$n$次多项式的根与系数的关系,得
  $$
  \begin{aligned}
    (-1)^{n-1}\sum_{i=1}^n\lambda_i = c_1 = (-1)^{n-1}\sum_{i=1}^na_{ii},\\
    \prod_{i=1}^n\lambda_i=c_n=\det(\A),
  \end{aligned}
  $$
  故
  $$
  \det(\A)=\prod_{i=1}^n\lambda_i.
  $$
  
\end{proof}

\begin{zhu}
\begin{itemize}
\item 当$\det(\A)\ne 0$,即$\A$为可逆矩阵时,其特征值全为非零数;
\item 奇异矩阵$\A$至少有一个零特征值。      
\end{itemize}
\end{zhu}





\begin{dingli}
  一个特征向量不能属于不同的特征值。
\end{dingli}
\begin{proof}

  若$\xx$是$\A$的属于特征值$\lambda_1,\lambda_2(\lambda_1\ne\lambda_2)$的特征向量,
  即有
  $$
  \A\xx=\lambda_1\xx, ~~ \A\xx=\lambda_2\xx
  ~~\Rightarrow~~ (\lambda_1-\lambda_2)\xx=\zero
  ~~\Rightarrow~~ \xx=\zero
  $$
  这与$\xx\ne\zero$矛盾。
\end{proof}





\begin{xingzhi}
  若$\lambda$是矩阵$\A$的特征值,$\xx$是$\A$属于$\lambda$的特征向量,则
  \begin{itemize}
  \item[(i)] $k\lambda$是$k\A$的特征值;
  \item[(ii)] $\lambda^m$是$\A^m$的特征值;
  \item[(iii)] 当$\A$可逆时,$\lambda^{-1}$是$\A^{-1}$的特征值;
  \end{itemize}
  且$\xx$仍是矩阵$k\A,\A^m,\A^{-1}$分别对应于$k\lambda,\lambda^m,\lambda^{-1}$的特征向量。
\end{xingzhi}
\begin{proof}
  \begin{itemize}
  \item[(i)]
  \item[(ii)]
  \item[(iii)] 当$\A$可逆时,$\lambda\ne0$,由$\A\xx=\lambda\xx$可得
    $$
    \A^{-1}(\A\xx)=\A^{-1}(\lambda\xx)=\lambda\A^{-1}\xx,
    $$
    因此
    $$
    \A^{-1}\xx=\lambda^{-1}\xx,
    $$
    故$\lambda^{-1}$是$\A^{-1}$的特征值,且$\xx$也是$\A^{-1}$对应于$\lambda^{-1}$的特征向量。
  \end{itemize}
\end{proof}

\begin{zhu*}
  若$\lambda$是$\A$的特征值,则$\varphi(\lambda)$是$\varphi(\A)$的特征值,其中$\varphi(\lambda)=a_0+a_1\lambda+\cd+a_m\lambda^m, \varphi(\A)=a_0\II+a_1\A+\cd+a_m\A^m$.
\end{zhu*}

\begin{li}
  设$3$阶矩阵$\A$的特征值为$1,-1,2$,求$|\A^*+3\A-2\II|$.
\end{li}
\begin{jie}
  因$\A$的特征值全不为零,故$\A$可逆,从而$\A^*=|\A|\A^{-1}$。又因$|\A|=\lambda_1\lambda_2\lambda_3=-2$,故
  $$
  \A^*+3\A-2\II=-2\A^{-1}+3\A-2\II.
  $$
  令$\varphi(\lambda)=-\frac2\lambda+3\lambda-2$,则$\varphi(\lambda)$为上述矩阵的特征值,分别为$\varphi(1)=-1,\varphi(-1)=-3,\varphi(2)=3$,于是
  $$
  |\A^*+3\A-2\II|=(-1)\cdot(-3)\cdot3=9.
  $$
\end{jie}

\begin{xingzhi}
  矩阵$\A$与$\A^T$的特征值相同。
\end{xingzhi}
\begin{proof}
  因$(\A-\lambda\II)^T=\A^T-(\lambda\II)^T=\A^T-\lambda\II$,故
  $$
  \det(\A-\lambda\II)=\det(\A^T-\lambda\II),
  $$
  故$\A$与$\A^T$有完全相同的特征值。
\end{proof}




% \begin{dingli}
%   设$\A=(a_{ij})$是$n$阶矩阵,若
%   \begin{itemize}
%   \item[(1)] $\ds \sum_{j=1}^n|a_{ij}|<1, ~~(i=1,2,\cd,n)$
%   \item[(2)] $\ds \sum_{i=1}^n|a_{ij}|<1, ~~(j=1,2,\cd,n)$
%   \end{itemize}
%   有一个成立,则$\A$的所有特征值的模都小于$1$。
% \end{dingli}


\begin{li}
  设$\A=\left(
    \begin{array}{rrr}
      1&-1&1\\
      2&-2&2\\
      -1&1&-1
    \end{array}
  \right)$
  \begin{itemize}
  \item[(i)]求$\A$的特征值与特征向量
  \item[(ii)] 求可逆矩阵$\PP$,使得$\PP^{-1}\A\PP$为对角阵。 
  \end{itemize}
\end{li}
\begin{jie}
  由
  $$
  \begin{aligned}
    |\A-\lambda\II|&=\left|
      \begin{array}{rrr}
        1-\lambda&-1&1\\
        2&-2-\lambda&2\\
        -1&1&-1-\lambda
      \end{array}
  \right|=\left|
    \begin{array}{rrr}
      1-\lambda&0&1\\
        2&-\lambda&2\\
        -1&-\lambda&-1-\lambda
    \end{array}
  \right|\\
  &=\left|
    \begin{array}{rrr}
      1-\lambda&0&1\\
      2&-\lambda&2\\
      -3&0&-3-\lambda
    \end{array}
  \right|=-\lambda[(\lambda-1)(\lambda+3)+3]=-\lambda^2(\lambda+2),
\end{aligned}
$$
知$\A$的特征值为$\lambda_1=\lambda_2=0$和$\lambda_3=-2$.

当$\lambda_{1,2}=0$时,由$(\A-0\II)\xx=0$,即$\A\xx=0$得基础解系
$$
\xx_1=(1,1,0)^T, \quad \xx_2=(-1,0,1)^T,
$$
故$\A$对应于$\lambda_{1,2}=0$的全体特征向量为$k_1\xx_1+k_2\xx_2=k_1(1,1,0)^T+k_2(-1,0,1)^T$,其中$k_1,k_2$为不全为零的任意常数。


当$\lambda_{3}=-2$时,由$(\A-\lambda_{3}\II)\xx=0$,即
$$
\left(
  \begin{array}{rrr}
    3&-1&1\\
    2&0&2\\
    -1&1&1
  \end{array}
\right)\left(
  \begin{array}{c}
    x_1\\x_2\\x_3
  \end{array}
\right)=\left(
  \begin{array}{c}
    0\\0\\0
  \end{array}
\right)
$$得基础解系
$$
\xx_3=(-1,-2,1)^T,
$$
故$\A$对应于$\lambda_{3}=-2$的全体特征向量为$k_3\xx_3=k_3(-1,-2,1)^T$,其中$k_3$为非零的任意常数。


将$\A\xx_i=\lambda_i\xx_i(i=1,2,3)$表示成
$$
\A(\xx_1,\xx_2,\xx_3)=(\xx_1,\xx_2,\xx_3)\left(
  \begin{array}{ccc}
    \lambda_1&&\\
    &\lambda_2\\
    &&\lambda_3\\
  \end{array}
\right)
$$
取
$$
\PP=(\xx_1,\xx_2,\xx_3)=\left(
  \begin{array}{rrr}
    1&-1&-1\\
    1& 0&-2\\
    0& 1& 1
  \end{array}
\right), \quad \Lambdabd=\left(
  \begin{array}{rrr}
    0&&\\
     &0&\\
     & &-2
  \end{array}
\right),
$$
则$\A\PP=\PP\Lambdabd$,且$|\PP|=2\ne 0$,故得
$$
\PP^{-1}\A\PP=\Lambdabd
$$
为对角阵。
\end{jie}


\begin{dingli}
  设$\lambda_1,\lambda_2,\cd,\lambda_m$是方阵$\A$的$m$个特征值,$\xx_1,\xx_2,\cd,\xx_m$依次是与之对应的特征向量,若$\lambda_1,\lambda_2,\cd,\lambda_m$互不相等,则$\xx_1,\xx_2,\cd,\xx_m$线性无关。
\end{dingli}
\begin{proof}
  设有常数$k_1,k_2,\cd,k_m$使得
  $$
  k_1\xx_1+k_2\xx_2+\cd+k_m\xx_m=0,
  $$
  则$\A(k_1\xx_1+k_2\xx_2+\cd+k_m\xx_m)=0$,即
  $$
  \lambda_1k_1\xx_1+\lambda_2k_2\xx_2+\cd+\lambda_mk_m\xx_m=0,
  $$
  以此类推,有
  $$
  \lambda_1^lk_1\xx_1+\lambda_2^lk_2\xx_2+\cd+\lambda_m^lk_m\xx_m=0, \quad l=1,2,\cd,m-1.
  $$
  写成矩阵形式即为
  $$
  (k_1\xx_1,k_2\xx_2,\cd,k_m\xx_m)\left(
    \begin{array}{cccc}
      1&\lambda_1&\cd&\lambda_1^{m-1}\\
      1&\lambda_2&\cd&\lambda_2^{m-1}\\
      \vd&\vd&&\vd\\
      1&\lambda_m&\cd&\lambda_m^{m-1}\\
    \end{array}
  \right)=(0,0,\cd,0).
  $$
  上式左边的第二个矩阵的行列式为范德蒙德行列式,当$\lambda_i$互不相等时该行列式不为零,从而该矩阵可逆。于是有
  $$
  (k_1\xx_1,k_2\xx_2,\cd,k_m\xx_m)=(0,0,\cd,0),
  $$
  即$k_j\xx_j=0, ~~j=1,2,\cd,m$。但$\xx_j\ne 0$,故$k_j=0,~~j=1,2,\cd,m$,从而$\xx_1,\xx_2,\cd,\xx_m$线性无关。
\end{proof}

\begin{li}
设$\lambda_1$和$\lambda_2$是矩阵$\A$的两个不同特征值,对应的特征向量依次为$\xx_1,\xx_2$,证明$\xx_1+\xx_2$不是$\A$的特征向量。
\end{li}
\begin{proof}
  按题设,有$\A\xx_1=k_1\xx_1,~\A\xx_2=k_2\xx_2$,故
  $$
  A(\xx_1+\xx_2)=\lambda_1\xx_1+\lambda_2\xx_2.
  $$
  (反证法)假设$\xx_1+\xx_2$是$\A$的特征向量,则应存在$\lambda$使得$\A(\xx_1+\xx_2)=\lambda(\xx_1+\xx_2)$,于是
  $$
  \lambda(\xx_1+\xx_2)=\lambda_1\xx_1+\lambda_2\xx_2,
  $$
  即
  $$
  (\lambda_1-\lambda)\xx_1+(\lambda_2-\lambda)\xx_2=0.
  $$
  因$\lambda_1\ne\lambda_2$,由上述定理知$\xx_1,\xx_2$线性无关,从而有
  $$
  \lambda_1-\lambda=\lambda_2-\lambda,
  $$
  即$\lambda_1=\lambda_2$,与题设矛盾,从而$\xx_1+\xx_2$不是$\A$的特征向量。
\end{proof}