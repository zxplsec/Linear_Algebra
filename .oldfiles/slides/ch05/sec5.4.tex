\section{实对称矩阵的对角化}

\begin{frame}
  
  \begin{li}
    实对称矩阵$\A$的任一特征值都是实数。
  \end{li}
  \begin{proof}
    $$
    \begin{array}{rl}
      \A\xx = \lambda \xx
      &~~\Longrightarrow~~
        \overline{(\A\xx)}^T = \overline{\lambda \xx}^T\\[0.1in]
      &~~\Longrightarrow~~
        \overline{\xx}^T~\overline{\A}^T~\xx = \overline{\lambda} ~\overline{\xx}^T~\xx\\[0.1in]
      &~~\Longrightarrow~~
        \overline{\xx}^T~\A^T~\xx = \overline{\lambda} ~\overline{\xx}^T~\xx\\[0.1in]
      &~~\Longrightarrow~~
        \lambda \overline{\xx}^T~\xx = \overline{\lambda} ~\overline{\xx}^T~\xx\\[0.1in]
      &~~\Longrightarrow~~
        \lambda = \overline \lambda
    \end{array}
    $$

  \end{proof}
  
\end{frame}


\begin{frame}
  
  \begin{li}
    实对称矩阵$\A$对应于不同特征值的特征向量是正交的。
  \end{li}
  \begin{proof}
  设$\A\xx_1=\lambda_1\xx_1,~~\A\xx_1=\lambda_1\xx_1~ (\lambda_1\ne\lambda_2), ~~\A^T=\A$,则
  $$
  \lambda_1\xx_2^T\xx_1=\xx_2^T\A\xx_1=\xx_2^T\A^T\xx_1=(\A\xx_2)^T\xx_1=(\lambda_2\xx_2)^T\xx_1=\lambda_2\xx_2^T\xx_1
  $$\pause
  由于$\lambda_1\ne\lambda_2$,所以
  $$
  \xx_2^T\xx_1=0.
  $$
  \end{proof}
  
\end{frame}


\begin{frame}
  \begin{dingli}
  设$\A$为$n$阶对称阵,则必有正交阵$\QQ$,使得$\QQ^{-1}\A\QQ=\QQ^T\A\QQ=\Lambdabd$,其中$\Lambdabd$是以$\A$的$n$个特征值为对角元的对角阵。
  \end{dingli}
\end{frame}


\begin{frame}
\begin{tuilun}
  设$\A$为$n$阶对称阵,$\lambda$为$\A$的特征方程的$k$重根,则矩阵$\A-\lambda\II$的秩$\rank(\A-\lambda\II)=n-k$,从而对应特征值$\lambda$恰有$k$个u线性无关的特征向量。
\end{tuilun} \pause

\begin{proof}
  
  由上述定理知,
  对称阵$\A$与对角阵$\Lambdabd=\diag(\lambda_1,\cd,\lambda_n)$相似,从而$\A-\lambda\II$与$\Lambdabd-\lambda\II=\diag(\lambda_1-\lambda,\cd,\lambda_n-\lambda)$相似。当$\lambda$是$\A$的$k$重特征根时,对角阵$\Lambdabd-\lambda\II$的对角元恰有$k$个等于$0$,于是$\rank(\Lambdabd-\lambda\II)=n-k$。而$\rank(\A-\lambda\II)=\rank(\Lambdabd-\lambda\II)$,故$\rank(\A-\lambda\II)=n-k$。
\end{proof}
\end{frame}

\begin{frame}
  将对称阵$\A$对角化的步骤:
  \begin{enumerate}
  \item 求出$\A$的全部互不相等的特征值$\lambda_1,\cd,\lambda_s$,它们的重数依次为$k_1,\cd,k_s(k_1+\cd+k_s=n)$;\\[0.1in]
  \item 对每个$k_i$重特征值$\lambda_i$,求$(\A-\lambda_i\II)\xx=0$的基础解系,得$k_i$个线性无关的特征向量;\\[0.1in]
  \item 再把它们正交化、单位化,得$k_i$个两两正交的单位特征向量。因$k_1+\cd+k_s=n$,故总共可得$n$个两两正交的单位特征向量;\\[0.1in]
  \item 将这$n$个两两正交的单位特征向量构成正交阵$\QQ$,便有$\QQ^{-1}\A\QQ=\Lambdabd$。
  \end{enumerate}
\end{frame}