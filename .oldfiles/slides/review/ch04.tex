\section{向量空间}

\subsection{向量空间的基、维数与向量关于基的坐标}

\begin{frame}\ft{\subsecname}

\begin{dingyi}
  设$V$是向量空间,如果
  \begin{itemize}
  \item[(1)] 在$V$中有$r$个向量$\alphabd_1,\cd,\alphabd_r$线性无关;
  \item[(2)] $V$中任一向量$\alphabd$可由向量组$\alphabd_1,\cd,\alphabd_r$线性表示,
  \end{itemize}
  则称$\alphabd_1,\cd,\alphabd_r$是向量空间$V$的一组基,$r$称为向量空间$V$的维数,并称$V$为$r$维向量空间。
\end{dingyi}
\end{frame}

\begin{frame}\ft{\subsecname}
\begin{zhu}
  只含一个零向量的集合$\{0\}$也是一个向量空间,该向量空间没有基,规定它的维数为$0$,并称之为$0$维向量空间。
\end{zhu}

\begin{zhu}
  如果把向量空间$V$看做是一个向量组,则$V$的基就是它的一个极大无关组,$V$的维数就是向量组的秩。于是,$V$的基不唯一,但它的维数是唯一确定的。设$V$是$r$维向量空间,则$V$中任意$r$个线性无关的向量就是$V$的一个基。
\end{zhu}
\end{frame}

\begin{frame}\ft{\subsecname}

\begin{li}
  在向量空间$\R^3$中,基本单位向量组
  $$
  \epsilonbd_1=(1,0,0)^T,~~
  \epsilonbd_2=(0,1,0)^T,~~
  \epsilonbd_3=(0,0,1)^T
  $$
  线性无关,且任一向量$\alphabd\in \R^3$可由$\epsilonbd_1,\epsilonbd_2,\epsilonbd_3$表示为
  $$
  \alphabd=x_1\epsilonbd_1+x_2\epsilonbd_2+x_3\epsilonbd_3.
  $$
  此时,$\R^3$可表示为
  $$
  \R^3=\{x=x_1\epsilonbd_1+x_2\epsilonbd_2+x_3\epsilonbd_3|x_1,x_2,x_3\in\R\}.
  $$
\end{li}

\end{frame}

\begin{frame}\ft{\subsecname}

  事实上,在$\R^3$中,任一组向量$\alphabd_1,\alphabd_2,\alphabd_3$,只要它们线性无关,就构成$\R^3$的一组基。\pause 


  例如,在$\R^3$中,向量组
$$
\epsilonbd_1=(1,0,0)^T,~~
\epsilonbd_2=(1,1,0)^T,~~
\epsilonbd_3=(1,1,1)^T
$$
线性无关,构成$\R^3$中的一组基。对任一向量$\alphabd\in \R^3$,
$$
\alphabd=(x_1-x_2)\alphabd_1+(x_2-x_3)\alphabd_2+x_3\alphabd_3.
$$
此时,
$\R^3$可表示为
$$
\R^3=\{x=(x_1-x_2)\alphabd_1+(x_2-x_3)\alphabd_2+x_3\alphabd_3|x_1,x_2,x_3\in\R\}.
$$

\end{frame}

\begin{frame}\ft{\subsecname}

\begin{dingyi}
  $\mathbb R^n$中任一向量$\alphabd$均可由线性无关向量组$\betabd_1,\betabd_2,\cd,\betabd_n$线性表示,即
  $$
  \alphabd=a_1\betabd_1+a_2\betabd_2+\cd+a_n\betabd_n,
  $$
  称$\betabd_1,\betabd_2,\cd,\betabd_n$为$\mathbb R^n$的一组基,有序数组$(a_1,a_2,\cd,a_n)$是向量$\alphabd$在基$\betabd_1,\betabd_2,\cd,\betabd_n$下的坐标,记作
  $$
  \alphabd_B=(a_1,a_2,\cd,a_n)\mbox{~~或~~}\alphabd_B=(a_1,a_2,\cd,a_n)^T
  $$
  并称之为$\alphabd$的坐标向量。
\end{dingyi}
\end{frame}

\begin{frame}\ft{\subsecname}

\begin{zhu}
  \begin{itemize}
  \item $\R^n$的基不是唯一的
  \item 基本向量组
    $$
    \epsilonbd_i=(0,\cd,0,1,0,\cd,0), \quad i=1,2,\cd,n
    $$
    称为$\R^n$的自然基或标准基。
  \item 一般来说,对于向量及其坐标,都采用列向量的形式,即
    $$
    \alphabd=(\betabd_1,\betabd_2,\cd,\betabd_n)\left(
      \begin{array}{c}
        a_1\\
        a_2\\
        \vd\\
        a_n
      \end{array}
    \right)
    $$
  \end{itemize}
\end{zhu}
\end{frame}

\begin{frame}\ft{\subsecname}

\begin{li}
  设$\R^n$的两组基为自然基$B_1$和$B_2=\{\betabd_1,\betabd_2,\cd,\betabd_n\}$,其中
  \begin{equation}\label{twobase}
    \begin{array}{lrrrrrrrrr}
      \betabd_1&=(&1,&-1,& 0,&0,&\cd,&0,&0,&0)^T,\\[0.2cm]
      \betabd_2&=(&0,& 1,&-1,&0,&\cd,&0,&0,&0)^T,\\[0.2cm]
               &\vd&&&&&\\[0.2cm]
      \betabd_{n-1}&=(&0,&0,&0,&0,&\cd,&0,&1,&-1)^T,\\[0.2cm]
      \betabd_{n}&=(&0,&0,&0,&0,&\cd,&0,&0,&1)^T.
    \end{array}
  \end{equation}
  求向量组$\alphabd=(a_1,a_2,\cd,a_n)^T$分别在两组基下的坐标。
\end{li}

\end{frame}

\begin{frame}\ft{\subsecname}


\begin{dingli}
  设$\alphabd_1,\alphabd_2,\cd,\alphabd_n$是$\R^n$的一组基,且
  $$
  \left\{
    \begin{array}{l}
      \etabd_1=a_{11}\alphabd_1+a_{21}\alphabd_2+\cd+a_{n1}\alphabd_n,\\[0.2cm]
      \etabd_2=a_{12}\alphabd_1+a_{22}\alphabd_2+\cd+a_{n2}\alphabd_n,\\[0.2cm]
      \cd\cd\\[0.2cm]
      \etabd_n=a_{1n}\alphabd_1+a_{2n}\alphabd_2+\cd+a_{nn}\alphabd_n.
    \end{array}
  \right.
  $$
  则$\etabd_1,\etabd_2,\cd,\etabd_n$线性无关的充要条件是
  $$
  \mathrm{det}\A=\left|
    \begin{array}{cccc}
      a_{11}&a_{12}&\cd&a_{1n}\\
      a_{21}&a_{22}&\cd&a_{2n}\\
      \vd&\vd&&\vd\\
      a_{n1}&a_{n2}&\cd&a_{nn}
    \end{array}
  \right|\ne 0.
  $$
\end{dingli}
\end{frame}

\begin{frame}\ft{\subsecname}
  \begin{dingyi}
    设$\R^n$的两组基$B_1=\{\alphabd_1,\alphabd_2,\cd,\alphabd_n\}$和$B_2=\{\etabd_1,\etabd_2,\cd,\etabd_n\}$满足关系式
    $$
    (\etabd_1,\etabd_2,\cd,\etabd_n)=(\alphabd_1,\alphabd_2,\cd,\alphabd_n)\left(
      \begin{array}{cccc}
        a_{11}&a_{12}&\cd&a_{1n}\\
        a_{21}&a_{22}&\cd&a_{2n}\\
        \vd&\vd&&\vd\\
        a_{n1}&a_{n2}&\cd&a_{nn}
      \end{array}
    \right)
    $$
    则矩阵
    $$
    \A=\left(
      \begin{array}{cccc}
        a_{11}&a_{12}&\cd&a_{1n}\\
        a_{21}&a_{22}&\cd&a_{2n}\\
        \vd&\vd&&\vd\\
        a_{n1}&a_{n2}&\cd&a_{nn}
      \end{array}
    \right)
    $$
    称为\red{由旧基$B_1$到新基$B_2$的过渡矩阵}。
  \end{dingyi}
\end{frame}

\begin{frame}\ft{\subsecname}

\begin{dingli}
  设$\alphabd$在两组基$B_1=\{\alphabd_1,\alphabd_2,\cd,\alphabd_n\}$与$B_2=\{\etabd_1,\etabd_2,\cd,\etabd_n\}$的坐标分别为
  $$
  \xx=(x_1,x_2,\cd,x_n)^T\mbox{~~和~~}\yy=(y_1,y_2,\cd,y_n)^T,
  $$
  由基$B_1$到$B_2$的过渡矩阵为$\A$,则
  $$
  \A\yy=\xx\mbox{~~或~~}\yy=\A^{-1}\xx
  $$
\end{dingli}

\end{frame}

\begin{frame}\ft{\subsecname}

\begin{li}
  已知$\R^3$的一组基为$B_2=\{\betabd_1,\betabd_2,\betabd_3\}$,其中
  $$\betabd_1=(1,2,1)^T,\betabd_2=(1,-1,0)^T,\betabd_3=(1,0,-1)^T,$$
  求自然基$B_1$到$B_2$的过渡矩阵。
\end{li}
\end{frame}

\begin{frame}\ft{\subsecname}

\begin{li}
  已知$\R^3$的两组基为$B_1=\{\alphabd_1,\alphabd_2,\alphabd_3\}$和$B_2=\{\betabd_1,\betabd_2,\betabd_3\}$,
  其中
  $$
  \begin{array}{lll}
    \alphabd_1=(1,1,1)^T,&\alphabd_2=(0,1,1)^T,&\alphabd_3=(0,0,1)^T, \\[0.2cm]
    \betabd_1=(1,0,1)^T,&\betabd_2=(0,1,-1)^T,&\betabd_3=(1,2,0)^T.  
  \end{array}
  $$
  \begin{itemize}
  \item[(1)] 求基$B_1$到$B_2$的过渡矩阵。
  \item[(2)] 已知$\alpha$在基$B_1$的坐标为$(1,-2,-1)^T$,求$\alphabd$在基$B_2$下的坐标。
  \end{itemize}
  
\end{li}

\end{frame}

\subsection{$\R^n$中向量的内积,欧式空间}
\begin{frame}\ft{\subsecname}
\begin{dingyi}
  在$\R^n$中,对于$\alphabd=(a_1,a_2,\cd,a_n)^T$和$\betabd=(b_1,b_2,\cd,b_n)^T$,规定$\alphabd$和$\betabd$的内积为
  $$
  (\alphabd,\betabd)=a_1b_1+a_2b_2+\cd+a_nb_n.
  $$
\end{dingyi}
当$\alphabd$和$\betabd$为列向量时,
$$
(\alphabd,\betabd)=\alphabd^T\betabd=\betabd^T\alphabd.
$$
\end{frame}

\begin{frame}\ft{\subsecname}

\begin{xingzhi}[内积的运算性质]
  对于$\alphabd,\betabd,\gammabd\in\R^n$和$k\in\R$,
  \begin{itemize}
  \item[(i)]   $(\alphabd,\betabd)=(\betabd,\alphabd)$
  \item[(ii)]  $(\alphabd+\betabd,\gammabd)=(\alphabd,\gammabd)+(\betabd,\gammabd)$
  \item[(iii)] $(k\alphabd,\betabd)=k(\alphabd,\betabd)$
  \item[(iv)]  $(\alphabd,\alphabd)\ge0$, 等号成立当且仅当$\alphabd=\zero$.
  \end{itemize}
\end{xingzhi}

\end{frame}

\begin{frame}\ft{\subsecname}

\begin{dingyi}[向量长度]
  向量$\alphabd$的长度定义为
  $$
  \|\alphabd\|=\sqrt{(\alphabd,\alphabd)}
  $$
\end{dingyi}

\begin{dingli}[柯西-施瓦茨(Cauchy-Schwarz)不等式]
  $$
  |(\alphabd,\betabd)|\le\|\alphabd\|\|\betabd\|
  $$
\end{dingli}
\end{frame}

\begin{frame}\ft{\subsecname}

\begin{dingyi}[向量之间的夹角]
  向量$\alphabd,\betabd$之间的夹角定义为
  $$
  <\alphabd,\betabd>=\arccos\frac{(\alphabd,\betabd)}{\|\alphabd\|\|\betabd\||}
  $$
\end{dingyi}

\begin{dingli}
  $$\alphabd\perp\betabd ~~\Longleftrightarrow~~
  (\alphabd,\betabd)=0
  $$
\end{dingli}

\begin{zhu}
零向量与任何向量的内积为零,从而零向量与任何向量正交。
\end{zhu}
\end{frame}

\begin{frame}\ft{\subsecname}

\begin{dingli}[三角不等式]
  $$
  \|\alphabd+\betabd\|\le\|\alphabd\|+\|\betabd\|.
  $$
\end{dingli}
\begin{zhu}
当$\alphabd\perp\betabd$时,$\|\alphabd+\betabd\|^2=\|\alphabd\|^2+\|\betabd\|^2$。
\end{zhu}

\begin{dingyi}[欧几里得空间]
  定义了内积运算的$n$维实向量空间,称为$n$维欧几里得空间(简称欧氏空间),仍记为$\R^n$。
\end{dingyi}
\end{frame}

\subsection{标准正交基}

\begin{frame}\ft{\subsecname}

\begin{dingli}
  $\R^n$中两两正交且不含零向量的向量组(称为非零正交向量组)$\alphabd_1,\alphabd_2,\cd,\alphabd_s$是线性无关的。
\end{dingli} 
\end{frame}

\begin{frame}\ft{\subsecname}

\begin{dingyi}[标准正交基]
  设$\alphabd_1,\alphabd_2,\cd,\alphabd_n\in \R^n$,若
  $$
  (\alphabd_i,\alphabd_j)=\delta_{ij}=\left\{
    \begin{array}{ll}
      1,& i=j,\\
      0,& i\ne j.
    \end{array}
  \right. \quad i,j=1,2,\cd,n.
  $$
  则称$\{\alphabd_1,\alphabd_2,\cd,\alphabd_n\}$是$\R^n$中的一组标准正交基。
\end{dingyi}

\end{frame}

\begin{frame}\ft{\subsecname}

\begin{li}
  设$B=(\alphabd_1,\alphabd_2,\cd,\alphabd_n)$是$\R^n$中的一组标准正交基,求$\R^n$中向量$\betabd$在基$B$下的坐标。
\end{li}
\begin{jie}
$$
\begin{array}{rl}
  & \betabd=x_1\alphabd_1+x_2\alphabd_2+\cd+x_n\alphabd_n\\[0.1in]
  \Longrightarrow&   (\betabd,\alphabd_j)=(x_1\alphabd_1+x_2\alphabd_2+\cd+x_n\alphabd_n,\alphabd_j)=x_j(\alphabd_j,\alphabd_j)\\[0.1in]
  \Longrightarrow& \ds x_j =\frac{ (\betabd,\alphabd_j)}{(\alphabd_j,\alphabd_j)} = (\betabd,\alphabd_j).
\end{array}
$$
\end{jie}

\end{frame}

\subsection{施密特(Schmidt)正交化方法}

\begin{frame}\ft{\subsecname}
本小节的目标是: 从一组线性无关的向量$\alphabd_1,\alphabd_2,\cd,\alphabd_m$出发,构造一组标准正交向量组。
\end{frame}


\begin{frame}[allowframebreaks]\ft{\subsecname}

  给定$\R^n$中的线性无关组$\alphabd_1,\alphabd_2,\cd,\alphabd_m$, 
  \begin{itemize}
  \item[(1)] 取$\betabd_1=\alphabd_1$;\\[0.3in]
  \item[(2)] 令$\betabd_2=\alphabd_2+k_{21}\betabd_1$,
    $$
    \begin{array}{rl}
      &\betabd_2\perp\betabd_1\\[0.2cm]
      \Longrightarrow&(\betabd_1,\betabd_2)=0\\[0.2cm] 
      \Longrightarrow& \ds k_{21}=-\frac{(\alphabd_2,\betabd_1)}{(\betabd_1,\betabd_1)} 
    \end{array}
    $$
    故
    $$
    \betabd_2=\alphabd_2-\frac{(\alphabd_2,\betabd_1)}{(\betabd_1,\betabd_1)}\betabd_1
    $$
  \item[(3)] 令$\betabd_3=\alphabd_3+k_{31}\betabd_1+k_{32}\betabd_2$,
    $$
    \begin{array}{rll}
      &\betabd_3\perp\betabd_i, & (i=1,2)\\[0.2cm] 
      \Longrightarrow&(\betabd_3,\betabd_i)=0, & (i=1,2)\\[0.2cm] 
      \Longrightarrow& \ds k_{3i}=-\frac{(\alphabd_3,\betabd_i)}{(\betabd_i,\betabd_i)} & (i=1,2) 
    \end{array}
    $$
    故
    $$
    \betabd_3=\alphabd_3-\frac{(\alphabd_3,\betabd_1)}{(\betabd_1,\betabd_1)}\betabd_1
    -\frac{(\alphabd_3,\betabd_2)}{(\betabd_2,\betabd_2)}\betabd_2
    $$ \\[1in]
  \item[(4)] 继续以上步骤,假设已经求出两两正交的非零向量$\betabd_1,\betabd_2,\cd,\betabd_{j-1}$,
    取
    $$
    \betabd_j=\alphabd_j+k_{j1}\betabd_1+k_{j2}\betabd_2+\cd++k_{j,j-1}\betabd_{j-1},
    $$
    $$
    \begin{array}{rl}
      & \betabd_j \perp \betabd_i\quad (i=1,2,\cd,j-1)\\[0.4cm] 
      \Longrightarrow &(\betabd_j,\betabd_i)=0 \quad (i=1,2,\cd,j-1)\\[0.4cm] 
      \Longrightarrow &(\alphabd_j+k_{ji}\betabd_i,\betabd_i)=0 \quad (i=1,2,\cd,j-1)\\[0.4cm] 
      \Longrightarrow& \ds k_{ji}=-\frac{(\alphabd_j,\betabd_i)}{(\betabd_i,\betabd_i)} 
    \end{array}
    $$
    故
    $$
    \betabd_j=\alphabd_j-\frac{(\alphabd_j,\betabd_1)}{(\betabd_1,\betabd_1)}\betabd_1
    -\frac{(\alphabd_j,\betabd_2)}{(\betabd_2,\betabd_2)}\betabd_2
    -\cd
    -\frac{(\alphabd_j,\betabd_{j-1})}{(\betabd_{j-1},\betabd_{j-1})}\betabd_{j-1}.
    $$
  \item[(5)] 单位化
    $$
    \betabd_1, \betabd_2, \cd, \betabd_m \xlongrightarrow[]{\ds \eta_j=\frac{\betabd_j}{\|\betabd_j\|}}
    \etabd_1, \etabd_2, \cd, \etabd_m
    $$
  \end{itemize}

\end{frame}


\begin{frame}[allowframebreaks]\ft{\subsecname}

\begin{li}
  已知$B=\{\alphabd_1,\alphabd_2,\alphabd_3\}$是$\R^3$的一组基,其中
  $$
  \alphabd_1=(1,-1,0)^T,~~
  \alphabd_2=(1,0,1)^T,~~
  \alphabd_3=(1,-1,1)^T.
  $$
  试用施密特正交化方法,由$B$构造$\R^3$的一组标准正交基。
\end{li}
\end{frame}


\begin{frame}[allowframebreaks]\ft{\subsecname}

  \begin{jie}
    1、正交化过程:
$$
\begin{array}{rl}
  \betabd_1&=\alphabd_1=(1,-1,0)^T, \\[0.2cm]
  \betabd_2&\ds=\alphabd_2-\frac{(\alphabd_2,\betabd_1)}{(\betabd_1,\betabd_1)}\betabd_1\\[0.4cm]
           &\ds=(1,0,1)^T-\frac12(1,-1,0)^T=\left(\frac12,\frac12,1\right),\\[0.4cm]
  \betabd_3&\ds=\alphabd_3-\frac{(\alphabd_3,\betabd_1)}{(\betabd_1,\betabd_1)}\betabd_1
             -\frac{(\alphabd_3,\betabd_2)}{(\betabd_2,\betabd_2)}\betabd_2\\[0.4cm]
           &\ds=(1,-1,1)^T-\frac23\left(\frac12,\frac12,1\right)^T-\frac22(1,-1,0)^T=\left(-\frac13,-\frac13,\frac13\right).
\end{array}
$$


2、单位化过程:
$$
\begin{array}{rl}
  \etabd_1&\ds =\frac{\betabd_1}{\|\betabd_1\|}=\left(\frac1{\sqrt{2}},-\frac1{\sqrt{2}},0\right),\\[0.4cm]
  \etabd_2&\ds =\frac{\betabd_2}{\|\betabd_2\|}=\left(\frac1{\sqrt{6}},\frac1{\sqrt{6}},\frac2{\sqrt{6}}\right),\\[0.4cm]
  \etabd_3&\ds =\frac{\betabd_2}{\|\betabd_2\|}=\left(-\frac1{\sqrt{3}},-\frac1{\sqrt{3}},\frac1{\sqrt{3}}\right).
\end{array}
$$
\end{jie}

\end{frame}


\subsection{正交矩阵及其性质}

\begin{frame}\ft{\subsecname}
\begin{dingyi}
  设$\A\in\R^{n\times n}$,如果
  $$
  \A^T\A=\II
  $$
  则称$\A$为正交矩阵。
\end{dingyi}
\end{frame}

\begin{frame}\ft{\subsecname}
\begin{dingli}
  $$
  \A\mbox{为}\mbox{正交矩阵}
  ~~\Longleftrightarrow~~
  \A\mbox{的列向量组为一组标准正交基。}
  $$
\end{dingli} 
\end{frame}

\begin{frame}\ft{\subsecname}

\begin{dingli}
  设$\A,\B$皆为$n$阶正交矩阵,则
  \begin{itemize}
  \item[(1)] $|\A|=1\mbox{~或~} -1$
  \item[(2)] $\A^{-1}=\A^T$
  \item[(3)] $\A^T$也是正交矩阵
  \item[(4)] $\A\B$也是正交矩阵
  \end{itemize}
\end{dingli}
\end{frame}

\begin{frame}\ft{\subsecname}


\begin{dingli}
  若列向量$\xx,\yy\in\R^n$在$n$阶正交矩阵$\A$的作用下变换为$\A\xx,\A\yy\in\R^n$,则向量的内积、长度与向量间的夹角都保持不变,即
  $$
  \begin{array}{c}
    (\A\xx,\A\yy)=(\xx,\yy),\\[0.1in]
    \|\A\xx\|=\|\xx\|, ~~\|\A\yy\|=\|\yy\|, \\[0.1in]
    \langle\A\xx,\A\yy\rangle=\langle\xx,\yy\rangle.
  \end{array}
  $$
\end{dingli}
\end{frame}

\begin{frame}
\begin{li}[\red{$\bigstar$}]
  给正交矩阵$\A$的某一行(或某一列)乘以$-1$后所得的矩阵$\B$是否仍为正交矩阵?
\end{li}\pause
\begin{jie}
  设$\A=(a_{ij})$为正交矩阵,给第$i$行乘以$-1$后得$\B=\left[
    \begin{array}{rrr}
      a_{11}&\cd&a_{1n}\\
      \vd& &\vd\\
      -a_{i1}&\cd&-a_{in}\\
      \vd&&\vd\\
      a_{n1}&\cd&a_{nn}
    \end{array}
  \right]$,
  
  因$a_{k1}^2+\cd+a_{kn}^2=1 ~~\forall k$,故当$k=i$时,$(-a_{i1})^2+\cd+(-a_{in})^2=1$

  因$a_{k1}a_{p1}+\cd+a_{kn}a_{pn}=1(k\ne p)$,故当$k=i$时,$(-a_{i1}a_{p1})+\cd+(-a_{in}a_{pn})=0$,于是$\B$为正交矩阵.
\end{jie}
\end{frame}