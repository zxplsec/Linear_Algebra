\section{逆矩阵}
给定一个从$\xx$到$\yy$的线性变换
\begin{equation}\label{yax}
  \yy = \A \xx  
\end{equation}       
即
$$
\left\{
  \begin{array}{c}
    y_1 = a_{11}x_1 + a_{12}x_2 + \cd + a_{1n}x_n\\[0.1cm]
    y_2 = a_{21}x_1 + a_{22}x_2 + \cd + a_{2n}x_n\\[0.1cm]
    \cd\cd \\[0.1cm]
    y_n = a_{n1}x_1 + a_{n2}x_2 + \cd + a_{nn}x_n
  \end{array}
\right.
$$
用$\A$的伴随阵$\A^*$左乘(\ref{yax}),得
$$
\A^* \yy = \A^* \A \xx  = |\A|\xx.
$$
当$|\A|\ne 0$时,有
$$
\xx = \frac{1}{|\A|}\A^* \yy.
$$

记
$$
\red{\B = \frac{1}{|\A|}\A^*,}
$$
则上式可记为
\begin{equation}\label{xby}
  \xx = \B \yy,
\end{equation}
它表示一个从$\yy$到$\xx$的线性变换,称为线性变换(\ref{yax})的逆变换。


\begin{zhu*}$\A$与$\B$的关系:
  \begin{enumerate}
  \item 将(\ref{xby})代入(\ref{yax})
    $$
    \yy = \A(\B\yy) = (\A\B)\yy
    $$
    可见$\A\B$为恒等变换对应的矩阵,故
    $$\A\B=\II.$$    
  \item 将(\ref{yax})代入(\ref{xby})
    $$
    \xx = \B(\A\xx) = (\B\A)\xx
    $$
    可见$\B\A$为恒等变换对应的矩阵,故
    $$\B\A=\II.$$
  \end{enumerate}
\end{zhu*}

$$
\red{
  \A\B = \B\A = \II.
}
$$

\begin{dingyi}[逆矩阵]
  对于$n$阶矩阵$\A$,如果有一个$n$阶矩阵$\B$,使
  $$
  \red{
    \A\B = \B\A = \II.
  }
  $$
  则称$\A$是\red{可逆}的,并把$\B$称为$\A$的\red{逆矩阵}。
\end{dingyi}


\begin{zhu*}
  \begin{enumerate}
  \item 可逆矩阵与其逆矩阵为同阶方阵。
  \item $\A$与$\B$地位相等,也可称$\A$为$\B$的逆矩阵。      
  \end{enumerate}
\end{zhu*}


\begin{dingli}
  若$\A$可逆,则$\A$的逆阵惟一。
\end{dingli}

\begin{proof}
\end{proof}

\red{
  $\A$的矩阵记作$\A^{-1}$,即
  $$
  \A\B = \B\A = \II ~ \Rightarrow ~ \B = \A^{-1}.
  $$
}




\begin{dingli}
  若$\A$可逆,则$|\A|\ne 0$.
\end{dingli}
\begin{proof}

\end{proof}

\begin{dingyi}{代数余子式矩阵,伴随矩阵}
  设$\A=(a_{ij})_{n\times n}$,$A_{ij}$为行列式$|\A|$中元素$a_{ij}$的代数余子式,称
  $$
  \mathrm{coef} \A = (A_{ij})_{n\times n}
  $$
  为$\A$的\red{代数余子式矩阵},并称$\mathrm{coef} \A$的转置矩阵为$\A$的\red{伴随矩阵},记为$\A^*$,
  即
  $$\red{
    \A^* = (\mathrm{coef}\A)^T = \left(
      \begin{array}{cccc}
        A_{11} & A_{21} & \cd & A_{n1} \\
        A_{12} & A_{22} & \cd & A_{n2} \\
        \vd   & \vd   &     & \vd   \\
        A_{1n} & A_{2n} & \cd & A_{nn} \\
      \end{array}
    \right)
  }
  $$
\end{dingyi}

之前已证
$$ \red{
  \A\A^* = |\A|\II
}
$$
同理可证
$$ \red{
  \A^*\A = |\A|\II
}
$$



\begin{dingli}
  若$|\A|\ne 0$,则$\A$可逆,且
  $$
  \A^{-1} = \frac1{|\A|} \A^*
  $$
\end{dingli}

\begin{proof}

\end{proof}

\red{
  该定理提供了求$\A^{-1}$的一种方法。
}


\begin{tuilun}
  若$\A\B = \II$(或$\B\A=\II$),则
  $$
  \B = \A^{-1}.
  $$
\end{tuilun}
\begin{proof}

\end{proof}

\red{
  该推论告诉我们,判断$\B$是否为$\A$的逆,只需验证$\A\B=\II$或$\B\A=\II$的一个等式成立即可。
}

\begin{dingyi}[奇异阵与非奇异阵]
  当$|\A|=0$时,$\A$称为\red{奇异矩阵},否则称为\red{非奇异矩阵}。
\end{dingyi}


\begin{zhu*}
  \red{可逆矩阵就是非奇异矩阵。}
\end{zhu*}


\begin{dingli}可逆矩阵有如下运算规律:
  \begin{enumerate}
  \item[1] 若$\A$可逆,则$\A^{-1}$亦可逆,且
    $$(\A^{-1})^{-1}=\A.$$
  \item[2] 若$\A$可逆,$k\ne 0$,则$k\A$可逆,且
    $$(k\A)^{-1}= k^{-1}A^{-1}.$$
  \item[3] 若$\A, ~\B$为同阶矩阵且均可逆,则$\A\B$可逆,且
    $$(\A\B)^{-1} = \B^{-1}\A^{-1}.$$
  \item[] 若$\A_1,\A_2,\cd,\A_m$皆可逆,则
    $$
    (\A_1\A_2\cd\A_m)^{-1}=\A_m^{-1}\cd\A_2^{-1}\A_1^{-1}
    $$
  \item[4] 若$\A$可逆,则$\A^T$亦可逆,且
    $$(\A^T)^{-1}=(\A^{-1})^T.$$ 
  \item[5] 若$\A$可逆,则
    $$|\A^{-1}|=|\A|^{-1}.$$
  \end{enumerate}
\end{dingli}


\begin{li}
  已知$\A = \left(
    \begin{array}{cc}
      a & b \\
      c & d
    \end{array}
  \right)$,求$\A^{-1}$。
\end{li}
\begin{jie}

$$
|\A| = ad-bc, \quad
|\A^*| = \left(
  \begin{array}{rr}
    d & -b \\
    -c & a
  \end{array}
\right)
$$

\begin{itemize}
\item[1] 当$|\A|=ad-bc=0$时,逆阵不存在; 
\item[2] 当$|\A|=ad-bc\ne0$时,
  $$
  \A^{-1} = \frac1{|\A|} \A^* = \frac1{ad-bc}\left(
    \begin{array}{rr}
      d & -b \\
      -c & a
    \end{array}
  \right)
  $$
\end{itemize}
\end{jie}

\begin{li}
  求方阵
  $
  \A = \left(
    \begin{array}{ccc}
      1 & 2 & 3\\
      2 & 2 & 1\\
      3 & 4 & 3
    \end{array}
  \right)
  $
  的逆阵。
\end{li}
\begin{jie}
$|\A| = 2$,故$\A$可逆。 计算$\A$的余子式
$$
\begin{array}{lll}
  M_{11}=2 & M_{12}=3 & M_{13}=2\\
  M_{21}=-6 & M_{22}=-6 & M_{23}=-2\\
  M_{31}=-4 & M_{32}=-5 & M_{33}=-2
\end{array}
$$

$$
\mathrm{coef} \A = \left(
  \begin{array}{rrr}
    M_{11} & -M_{12} &  M_{13}\\
    -M_{21} &  M_{22} & -M_{23}\\
    M_{31} & -M_{32} &  M_{33}
  \end{array}
\right)  = \left(
  \begin{array}{rrr}
    2 & -3 &  2\\
    6 & -6 &  2 \\
    -4 & 5 & -2
  \end{array}
\right)
$$

$$
\A^* =  \left(
  \begin{array}{rrr}
    2 & 6 &  -4\\
    -3 & -6 & 5 \\
    2 & 2 & -2
  \end{array}
\right)
$$

故
$$
\A^{-1} = \frac1{|A|}\A^* = \left(
  \begin{array}{rrr}
    1 & 3 &  -2\\
    -\frac32 & -3 & \frac52 \\
    1 & 1 & -1
  \end{array}
\right)
$$
\end{jie}

\begin{li}
  设方阵$\A$满足方程
  $$
  \A^2 - 3\A - 10 \II = \zero,
  $$
  证明:$\A, \A-4\II$都可逆,并求它们的逆矩阵。      
\end{li}
\begin{proof}
$$
\A^2-3\A-10\II=\zero ~\Rightarrow~ \A(\A-3\II) = 10\II 
~\Rightarrow~ \A\left[\frac1{10}(\A-3\II)\right] = \II
$$  
故$\A$可逆,且\red{$\ds \A^{-1} = \frac1{10}(\A-3\II)$}.

$$
\A^2-3\A-10\II=\zero ~\Rightarrow~ (\A+\II)(\A-4\II) = 6\II 
~\Rightarrow~ \frac1{6}(\A+\II)(\A-4\II) = \II
$$     
故$\A-4\II$可逆,且\red{$\ds (\A-4\II)^{-1} = \frac1{6}(\A+\II)$}.

\end{proof}

\begin{li}
  证明:可逆对称矩阵的逆矩阵仍为对称矩阵;可逆反对称矩阵的逆矩阵仍为反对称矩阵。
\end{li}

\begin{li}
  设$\A=(a_{ij})_{n\times n}$为非零实矩阵,证明:若$\A^*=\A^T$,则$\A$可逆。
\end{li}
\begin{proof}
欲证$\A$可逆,只需证$|\A|\ne 0$。

由$\A^* = \A^T$及$\A^*$的定义可知,$\A$的元素$a_{ij}$等于自身的代数余子式$A_{ij}$。

再根据行列式的按行展开定义,有
$$
|\A| = \sum_{j=1}^n a_{ij} A_{ij} = \sum_{j=1}^n a_{ij}^2.
$$

由于$\A$为非零实矩阵,故$|\A|\ne 0$,即$\A$可逆。
\end{proof}

\begin{li}
  设$\A$可逆,且$\A^*\B = \A^{-1}+\B$,证明$\B$可逆,当$\A=\left(
    \begin{array}{ccc}
      2 & 6 & 0 \\
      0 & 2 & 6\\
      0 & 0 & 2
    \end{array}
  \right)$时,求$\B$.
\end{li}
\begin{jie}
$$
\A^*\B = \A^{-1}+\B  \Rightarrow (\A^*-\II)\B = \A^{-1}
\Rightarrow |\A^*-\II|\cdot |\B| = |\A^{-1}|\ne 0 
$$
故$\B$与$\A^*-\II$可逆。

$$
\B = (\A^*-\II)^{-1} \A^{-1} = [\A(\A^*-\II)]^{-1} = (\A\A^*-\A)^{-1} = (|\A|\II-\A)^{-1}.
$$
其中
$$
|\A|\II-\A = \left(
  \begin{array}{ccc}
    8 & &\\
      & 8 &\\
      & & 8
  \end{array}
\right) - \left(
  \begin{array}{ccc}
    2 & 6 & 0 \\
    0 & 2 & 6\\
    0 & 0 & 2
  \end{array}
\right) = 6 \left(
  \begin{array}{rrr}
    1 & -1 & 0 \\
    0 &  1 & -1\\
    0 & 0 & 1
  \end{array}
\right)
$$

易算得
$$
\B = \frac16 \left(
  \begin{array}{rrr}
    1 &  1 & 1 \\
    0 &  1 & 1\\
    0 & 0 & 1
  \end{array}
\right)
$$
\end{jie}

\begin{li}
  设$\A,~\B$均为$n$阶可逆矩阵,证明:
  \begin{itemize}
  \item[(1).] $(\A\B)^*=\B^*\A^*$
  \item[(2).] $(\A^*)^*=|\A|^{n-2}\A$ 
  \end{itemize}
\end{li}
\begin{proof}
% \begin{block}{知识点}
%   $$
%   \red{\A^{-1}=\frac1{|\A|}\A^* ~~\Rightarrow~~ \A^*=|\A|\A^{-1}}
%   $$
% \end{block}

% \proofname
(1) 由$|\A\B| = |\A||\B| \ne 0$可知$\A\B$可逆,且有$(\A\B)(\A\B)^*=|\A\B|\II$。故
$$
\begin{array}{cl}
  (\A\B)^* &  =|\A\B| (\A\B)^{-1}
             = |\A||\B| \B^{-1}\A^{-1} \\[0.2cm]
           & \ds = |\B| \B^{-1} |\A| \A^{-1}   = \B^*\A^*.      
\end{array}
$$


(2) 由$(\A^*)^* \A^* = |\A^*|\II$,得 
$$(\A^*)^* \red{|\A|\A^{-1}} = |\A|^{n-1}\II$$  
两边同时右乘$\A$得
$$
(\A^*)^*=|\A|^{n-2}\A.
$$
\end{proof}

\begin{li}
  设$\PP = \left(
    \begin{array}{cc}
      1 & 2\\
      1 & 4
    \end{array}
  \right), ~~ \LLambda=\left(
    \begin{array}{cc}
      1 & \\
        & 2
    \end{array}
  \right), ~~ \A\PP=\PP\LLambda$,求$\A^n$.
\end{li}
\begin{jie}
$$
|\PP|=2, \quad \PP^{-1} = \frac12 \left(
  \begin{array}{rr}
    4 & -2\\
    -1 & 1
  \end{array}
\right).
$$

$$
\A = \PP\LLambda\PP^{-1}, \quad 
\A^2 = \PP\LLambda\PP^{-1}\PP\LLambda\PP^{-1} = \PP\LLambda^2\PP^{-1}, \quad 
\cd, \quad 
\A^n = \PP\LLambda^n\PP^{-1}.
$$

$$
\LLambda^n = \left(
  \begin{array}{cc}
    1 & \\
      & 2^n
  \end{array}
\right).
$$

$$
\A^n =  \left(
  \begin{array}{cc}
    1 & 2\\
    1 & 4
  \end{array}
\right) \cdot \left(
  \begin{array}{cc}
    1 & \\
      & 2^n
  \end{array}
\right) \cdot \frac12 \left(
  \begin{array}{rr}
    4 & -2\\
    -1 & 1
  \end{array}
\right) = \left(
  \begin{array}{cc}
    2-2^n & 2^n-1\\
    2-2^{n+1} & 2^{n+1}-1
  \end{array}
\right).
$$

\end{jie}




\begin{jielun}
  令
  $$
  \varphi(\A) = a_0 \II + a_1 \A + \cd + a_m \A^m.
  $$
  \begin{itemize}
  \item[(i)]
    若$\A = \PP \LLambda \PP^{-1}$,则$\A^k = \PP \LLambda^k \PP^{-1}$,从而
    $$
    \begin{array}{rcl}
      \varphi(\A) &=& a_0 \II + a_1 \A + \cd + a_m \A^m \\[0.2cm]
                  &=& \PP a_0 \II \PP^{-1} + \PP a_1 \LLambda\PP^{-1} + \cd + \PP a_m \LLambda^m \PP^{-1} \\[0.2cm]
                  &=& \PP \varphi(\LLambda) \PP^{-1}.
    \end{array}
    $$
  \item[(ii)] 若$\LLambda=\mathrm{diag}(\lambda_1,\lambda_2,\cd,\lambda_n)$为对角阵,则$\LLambda^k=\mathrm{diag}(\lambda_1^k,\lambda_2^k,\cd,\lambda_n^k)$,从而
    $$
    \begin{array}{l}
      \varphi(\LLambda) = a_0 \II + a_1 \LLambda + \cd + a_m \LLambda^m \\[0.2cm]
      =  a_0 \left(
      \begin{array}{cccc}
        1 & & &\\
          & 1 & & \\
          & & \dd & \\
          & & & 1
      \end{array}
                \right)
                + a_1 \left(
                \begin{array}{cccc}
                  \lambda_1 & & &\\
                            & \lambda_2 & & \\
                            & & \dd & \\
                            & & & \lambda_n
                \end{array}
                                  \right) + \cd +  a_m \left(
                                  \begin{array}{cccc}
                                    \lambda_1^m & & &\\
                                                & \lambda_2^m & & \\
                                                & & \dd & \\
                                                & & & \lambda_n^m
                                  \end{array}
                                                      \right)  \\[0.6cm]
      =\left(
      \begin{array}{cccc}
        \varphi(\lambda_1) & & &\\
                           & \varphi(\lambda_2) & & \\
                           & & \dd & \\
                           & & & \varphi(\lambda_n)
      \end{array}
                                 \right)
    \end{array}
    $$
  \end{itemize}


\end{jielun}



% 

