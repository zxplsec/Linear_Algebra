\section{矩阵分块}
\begin{frame}\ft{\secname}

矩阵
\begin{figure}[htbp]
  \centering
  \begin{tikzpicture}
    \matrix(A) [matrix of math nodes,nodes in empty cells,ampersand replacement=\&,left delimiter=(,right delimiter=)] {
      a_{11} \& a_{12} \& a_{13}  \& a_{14} \\
      a_{21} \& a_{22} \& a_{23}  \& a_{24} \\
      a_{31} \& a_{32} \& a_{33}  \& a_{34} \\
    };
    \draw[red, dashed, very thick] (A-1-2.north east) -- (A-3-2.south east) (A-2-1.south west) -- (A-2-4.south east);;
  \end{tikzpicture}
\end{figure}
可记为
$$
\left(
  \begin{array}{cc}
    \A_{11} &  \A_{12}\\
    \A_{21} &  \A_{22}
  \end{array}
\right)
$$
其中
$$
\begin{array}{ll}
  \A_{11} = 
  \left(
  \begin{array}{cc}
    a_{11} &  a_{12}\\
    a_{21} &  a_{22}
  \end{array} \right),
           &
             \A_{12} = 
             \left(
             \begin{array}{cc}
               a_{13} &  a_{14}\\
               a_{23} &  a_{24}
             \end{array}
                        \right)\\ [0.3cm]
  \A_{21} = 
  \left(
  \begin{array}{cc}
    a_{31} &  a_{32}
  \end{array}\right) ,
           &
             \A_{22} = 
             \left(
             \begin{array}{cc}
               a_{33} &  a_{34}
             \end{array}
                        \right)    
\end{array}
$$
\end{frame}

\begin{frame}\ft{\secname}
\begin{dingyi}[矩阵的按行分块]
  $$
  \A = \left(
    \begin{array}{cccc}
      a_{11} & a_{12} & \cd & a_{1n}\\
      a_{21} & a_{22} & \cd & a_{2n}\\
      \vd & \vd & & \vd \\
      a_{m1} & a_{m2} & \cd & a_{mn}
    \end{array}
  \right)
  = \left(
    \begin{array}{c}
      \abd_1\\
      \abd_2\\
      \vd \\
      \abd_m
    \end{array}
  \right)
  $$
  其中
  $$
  \abd_i = (a_{i1}, ~a_{i2}, ~\cd, ~a_{in})
  $$
\end{dingyi}
\end{frame}

\begin{frame}\ft{\secname}
\begin{dingyi}[矩阵的按列分块]
  $$
  \B = \left(
    \begin{array}{cccc}
      b_{11} & b_{12} & \cd & b_{1s}\\
      b_{21} & b_{22} & \cd & b_{2s}\\
      \vd & \vd & & \vd \\
      b_{n1} & b_{n2} & \cd & b_{ns}
    \end{array}
  \right)
  = \left(
    \begin{array}{c}
      \bb_1, ~ \bb_2, ~ \cd, \bb_s
    \end{array}
  \right)
  $$
  其中
  $$
  \bb_j = \left(
    \begin{array}{c}
      b_{1j}\\
      b_{2j}\\
      \vd\\
      b_{nj}
    \end{array}
  \right)
  $$
\end{dingyi}
\end{frame}

\begin{frame}\ft{\secname}
当$n$阶矩阵$\A$中非零元素都集中在主对角线附近,有时可分块成如下\textcolor{acolor3}{对角块矩阵}
$$
\A = \left(
  \begin{array}{cccc}
    \A_1 & & &\\
    &\A_2&&\\
    &&\dd&\\
    &&&\A_m
  \end{array}
\right)
$$
其中$\A_i$为$r_i$阶方阵($i=1,2,\cd,m$),且
$$
\sum_{i=1}^m r_i = n.
$$

\end{frame}

\begin{frame}\ft{\secname}
如
\begin{figure}
  \centering
  \begin{tikzpicture}       
    [column 1/.style={anchor=base east},
    column 2/.style={anchor=base east},
    column 3/.style={anchor=base east},
    column 4/.style={anchor=base east},
    column 5/.style={anchor=base east},
    column 6/.style={anchor=base east}]
    \matrix(A) [matrix of math nodes,nodes in empty cells,ampersand replacement=\&,left delimiter=(,right delimiter=)] {
    0 \& -1 \& 0 \& 0 \& 0 \& 0\\
    1 \&  2 \& 0 \& 0 \& 0 \& 0\\
    0 \&  0 \& 1 \&-1 \& 0 \& 0\\
    0 \&  0 \&-1 \& 1 \& 2 \& 0\\
    0 \&  0 \& 0 \& 2 \&-2 \& 0\\
    0 \&  0 \& 0 \& 0 \& 0 \& 3\\
  };
    \draw[red, dashed, very thick] 
    (A-1-2.north east) -- (A-6-2.south east)
    (A-1-5.north east) -- (A-6-5.south east) 
    (A-2-1.south west) -- (A-2-6.south east)
    (A-5-1.south west) -- (A-5-6.south east);

  \end{tikzpicture}
\end{figure}

\end{frame}

\begin{frame}\ft{\secname}



\begin{dingyi}[分块矩阵的加法]
  设$\A, \B$为同型矩阵,采用相同的分块法,有
  $$
  \A = \left(
    \begin{array}{ccc}
      \A_{11} & \cd & \A_{1r} \\
      \vd   &     & \vd   \\
      \A_{s1} & \cd & \A_{sr}
    \end{array}
  \right), \ \ 
  \B = \left(
    \begin{array}{ccc}
      \B_{11} & \cd & \B_{1r} \\
      \vd   &     & \vd   \\
      \B_{s1} & \cd & \B_{sr}
    \end{array}
  \right),
  $$
  其中$\A_{ij}$与$\B_{ij}$为同型矩阵,则
  $$
  A = \left(
    \begin{array}{ccc}
      \A_{11} + \B_{11}  & \cd & \A_{1r} + \B_{1r} \\
      \vd   &     & \vd   \\
      \A_{s1} + \B_{s1}  & \cd & \A_{sr} + \B_{sr}
    \end{array}
  \right).
  $$
\end{dingyi}
\end{frame}

\begin{frame}\ft{\secname}

\begin{dingyi}[分块矩阵的数乘]
  $$
  \lambda \A = \left(
    \begin{array}{ccc}
      \lambda \A_{11} & \cd & \lambda \A_{1r} \\
      \vd   &     & \vd   \\
      \lambda \A_{s1} & \cd & \lambda \A_{sr}
    \end{array}
  \right)
  $$    
\end{dingyi}
\end{frame}

\begin{frame}\ft{\secname}

\begin{dingyi}[分块矩阵的乘法]
  设$\A$为$m\times n$矩阵, $\B$为$n \times s$矩阵,
  $$
  \A = \left(
    \begin{array}{ccc}
      \A_{11} & \cd & \A_{1s} \\
      \vd   &     & \vd   \\
      \A_{r1} & \cd & \A_{rs}
    \end{array}
  \right), \ \ 
  \B = \left(
    \begin{array}{ccc}
      \B_{11} & \cd & \B_{1t} \\
      \vd   &     & \vd   \\
      \B_{s1} & \cd & \B_{st}
    \end{array}
  \right),
  $$
  其中\textcolor{acolor3}{$\A_{i1}, \A_{i2}, \cd, A_{is}$的列数分别等于$\B_{1j}, \B_{2j}, \cd, \B_{sj}$的行数},则
  $$
  \A \B = \left(
    \begin{array}{ccc}
      \C_{11}   & \cd & \C_{1t}  \\
      \vd   &     & \vd   \\
      \C_{r1}   & \cd & \C_{rt}
    \end{array}
  \right),
  $$
  其中
  $$
  \C_{ij} = \sum_{k=1}^s \A_{ik} \B_{kj}.
  $$
\end{dingyi}
\end{frame}

\begin{frame}\ft{\secname}


\begin{li} 
  用分块矩阵的乘法计算$\A\B$,其中
  $$
  \A = \left(
    \begin{array}{rrrrr}
      1&0&0&0&0\\
      0&1&0&0&0\\
      -1&2&1&0&0\\
      1&1&0&1&0\\
      -2&0&0&0&1
    \end{array}
  \right), \quad
  \B = \left(
    \begin{array}{rrrrr}
      3&2&0&1&0\\
      1&3&0&0&1\\
      -1&0&0&0&0\\
      0&-1&0&0&0\\
      0&0&-1&0&0
    \end{array}
  \right)
  $$
\end{li}
\end{frame}

\begin{frame}\ft{\secname}
\begin{center}
  \begin{tikzpicture}       
    [column 1/.style={anchor=base east},
    column 2/.style={anchor=base east},
    column 3/.style={anchor=base east},
    column 4/.style={anchor=base east},
    column 5/.style={anchor=base east}]
    \matrix(A1) [matrix of math nodes]{
    \A = \\
  };
  \matrix(A2) [right=.1in of A1,matrix of math nodes,nodes in empty cells,inner sep=0.2cm,ampersand replacement=\&,left delimiter=(,right delimiter=)] {
    1 \& 0 \& 0 \& 0 \& 0 \\
    0 \& 1 \& 0 \& 0 \& 0 \\
    -1 \& 2 \& 1 \& 0 \& 0 \\
    1 \& 1 \& 0 \& 1 \& 0 \\
    -2 \& 0 \& 0 \& 0 \& 1 \\
  };
    \draw[red, dashed, very thick] 
    (A2-1-2.north east) -- (A2-5-2.south east)
    (A2-2-1.south west) -- (A2-2-5.south east);
    \matrix(A3) [right=.1in of A2,matrix of math nodes]{
    = \\
  };
    \matrix(A4) [right=.1in of A3,matrix of math nodes,nodes in empty cells,inner sep=0.2cm,ampersand replacement=\&,left delimiter=(,right delimiter=)] {
    \II_2 \& \zero_{2\times 3} \\
    \A_1 \& \II_3\\
  };
  \end{tikzpicture}
\end{center}


\begin{center}
  \begin{tikzpicture}       
    [column 1/.style={anchor=base east},
    column 2/.style={anchor=base east},
    column 3/.style={anchor=base east},
    column 4/.style={anchor=base east},
    column 5/.style={anchor=base east}]
    \matrix(A1) [matrix of math nodes]{
    \B = \\
  };
    \matrix(A2) [right=.1in of A1,matrix of math nodes,nodes in empty cells,inner sep=0.2cm,ampersand replacement=\&,left delimiter=(,right delimiter=)] {
    3\&2\&0\&1\&0\\
    1\&3\&0\&0\&1\\
    -1\&0\&0\&0\&0\\
    0\&-1\&0\&0\&0\\
    0\&0\&-1\&0\&0\\
  };
    \draw[red, dashed, very thick] 
    (A2-1-3.north east) -- (A2-5-3.south east)
    (A2-2-1.south west) -- (A2-2-5.south east);
    \matrix(A3) [right=.1in of A2,matrix of math nodes]{
    = \\
  };
    \matrix(A4) [right=.1in of A3,matrix of math nodes,nodes in empty cells,inner sep=0.2cm,ampersand replacement=\&,left delimiter=(,right delimiter=)] {
    \B_1 \& \II_2 \\
    -\II_3 \& \zero_{3\times 2}\\
  };
  \end{tikzpicture}
\end{center}
\end{frame}

\begin{frame}\ft{\secname}
则
$$
\A\B = \left(
  \begin{array}{cc}
    \II_2 & \zero\\
    \A_1 & \II_3
  \end{array}
\right)\left(
  \begin{array}{cc}
    \B_1 & \II_2\\
    -\II_3 & \zero
  \end{array}
\right) = \left(
  \begin{array}{cc}
    \B_1 & \II_2\\
    \A_1\B_1-\II_3 & \A_1
  \end{array}
\right)
$$
其中
$$
\A_1\B_1-\II_3 = \left(
  \begin{array}{rr}
    -1&2\\
    1&1\\
    -2&0
  \end{array}
\right)\left(
  \begin{array}{rrr}
    3&2&0\\
    1&3&0
  \end{array}
\right)-\left(
  \begin{array}{ccc}
    1&0&0\\
    0&1&0\\
    0&0&1
  \end{array}
\right)=\left(
  \begin{array}{rrr}
    -2&4&0\\
    4&4&0\\
    -6&-4&-1
  \end{array}
\right)
$$
\end{frame}

\begin{frame}\ft{\secname}

\begin{li}
  设$\A$为$m\times n$矩阵,$\B$为$n\times s$矩阵,$\B$按列分块成$1\times s$分块矩阵,
  将$\A$看成$1\times 1$分块矩阵,则
  $$
  \A\B=\A(\bb_1,\bb_2,\cd,\bb_s)=(\A\bb_1,\A\bb_2,\cd,\A\bb_s)      
  $$
  若已知$\A\B=0$,则显然
  $$
  \A\bb_j=0, \quad j=1,2,\cd,s.
  $$
  因此,$\B$的每一列$\bb_j$都是线性方程组$\A\xx=0$的解。
\end{li}    
\end{frame}

\begin{frame}\ft{\secname}
\begin{li}
  设$\A^T\A=\zero$,证明$\A=\zero$.
\end{li}
\pause
\begin{proof}
设$\A=(a_{ij})_{m\times n}$,把$\A$用列向量表示为$\A=(\abd_1, ~\abd_2,~\cd,~\abd_n)$,则
$$
\A^T\A = \left(
  \begin{array}{c}
    \abd_1^T\\
    \abd_2^T\\
    \cd \\
    \abd_n^T
  \end{array}
\right) (\abd_1, ~\abd_2,~\cd,~\abd_n) = \left(
  \begin{array}{cccc}
    \abd_1^T\abd_1 & \abd_1^T\abd_2 & \cd & \abd_1^T\abd_n\\
    \abd_2^T\abd_1 & \abd_2^T\abd_2 & \cd & \abd_2^T\abd_n\\
    \vd & \vd & & \vd \\
    \abd_n^T\abd_1 & \abd_n^T\abd_2 & \cd & \abd_n^T\abd_n
  \end{array}
\right)
$$
\pause
因为$\A^T\A=\zero$,故
$$
\abd_i^T \abd_j = 0, \quad i,j=1,2,\cd,n.
$$
\pause
特别地,有
$$
\abd_j^T \abd_j = 0, \quad j=1,2,\cd,n,
$$
即
$$
a_{1j}^2+a_{2j}^2+\cd+a_{mj}^2=0  ~\Rightarrow~ a_{1j}=a_{2j}=\cd=a_{mj}=0 ~\Rightarrow~ \A = \zero.
$$
\end{proof}
\end{frame}

\begin{frame}\ft{\secname}

\begin{li}
  若$n$阶矩阵$\C,\D$可以分块成同型对角块矩阵,即
  $$
  \C = \left(
    \begin{array}{cccc}
      \C_1&&&\\
      &\C_2&&\\
      &&\cd&\\
      &&&\C_m
    \end{array}
  \right),\quad
  \D = \left(
    \begin{array}{cccc}
      \D_1&&&\\
      &\D_2&&\\
      &&\cd&\\
      &&&\D_m
    \end{array}
  \right)
  $$
  其中$\C_i$和$\D_i$为同阶方阵($i=1,2,\cd,m$),则
  $$
  \C\D = \left(
    \begin{array}{cccc}
      \C_1\D_1&&&\\
      &\C_2\D_2&&\\
      &&\cd&\\
      &&&\C_m\D_m
    \end{array}
  \right)
  $$
\end{li}

\end{frame}

\begin{frame}\ft{\secname}





\begin{li}
  证明:若方阵$\A$为可逆的上三角阵,则$\A^{-1}$也为上三角阵。
\end{li}
\end{frame}

\begin{frame}\ft{\secname}
\begin{proof}
对阶数$n$用数学归纳法。\pause
\begin{itemize}
\item[1] 当$n=1$时,$(a)^{-1}=(\frac1a)$,结论成立。 \pause
\item[2] 假设命题对$n-1$阶可逆上三角矩阵成立,考虑$n$阶情况,设
  \begin{center}
    \begin{tikzpicture} [column 1/.style={anchor=base east}]
      \matrix (M) [matrix of math nodes]  { 
      \A = \\
    };
      \matrix(MM) [right=.1in of M, matrix of math nodes,nodes in empty cells,
      column sep=3ex,row sep=1ex,ampersand replacement=\&,left delimiter=(,right delimiter=)] {
      a_{11} \& a_{12} \& a_{13}  \& a_{14} \\
      0    \& a_{22} \& a_{23}  \& a_{24} \\
      \vd  \& \vd   \& \dd  \& \vd \\
      0    \& 0     \& \cd  \& a_{nn} \\
    };

      \draw[red, dashed, very thick]
      (MM-1-1.north east) -- (MM-4-1.south east)
      (MM-1-1.south west) -- (MM-1-4.south east);

      \matrix (MMM) [right=.1in of MM,matrix of math nodes]  { 
      = \\
    };
      \matrix(MMMM) [right=.1in of MMM, matrix of math nodes,nodes in empty cells,
      column sep=3ex,row sep=1ex,ampersand replacement=\&,left delimiter=(,right delimiter=)] {
      a_{11} \& \alphabd\\
      \zero \& \A_1\\
    };
    \end{tikzpicture}        
  \end{center}
  其中$\A_1$为$n-1$阶可逆上三角阵。
\end{itemize}
\end{proof}
\end{frame}

\begin{frame}\ft{\secname}
\begin{proof}[续]
设$\A$的逆阵为
$$
\begin{aligned}
\B = \left(
  \begin{array}{cc}
    b_{11} & \betabd\\
    \gammabd & \B_1 
  \end{array}    
\right),  
\end{aligned}
$$
其中
$$
\begin{aligned}
 \betabd = \left(
  \begin{array}{c}
    b_{12}\\
    \vd \\
    b_{1n}
  \end{array}
\right)^T,  
\quad \gammabd = \left(
  \begin{array}{c}
    b_{21}\\
    \vd \\
    b_{n1}
  \end{array}
\right), \quad
\B_1 = \left(
  \begin{array}{ccc}
    b_{22} & \cd & b_{2n}\\
    \vd   & \dd & \vd \\
    b_{n2} & \cd & b_{nn}
  \end{array}
\right),
\end{aligned}
$$\pause
则
$$
\begin{aligned}
\A\B &= \left(
  \begin{array}{cc}
    a_{11} & \alphabd \\
    \zero & \A_1
  \end{array}
\right)\left(
  \begin{array}{cc}
    b_{11} & \betabd \\
    \gammabd & \B_1
  \end{array}
\right) \\
& = \left(
  \begin{array}{cc}
    a_{11}b_{11}+\alphabd\gammabd & a_{11}\betabd+\alphabd \B_1\\
    \A_1\gammabd & \A_1\B_1
  \end{array}
\right)  \textcolor{acolor3}{
= \left(
  \begin{array}{cc}
    1 & \zero \\
    \zero & \II_{n-1}
  \end{array}
\right)
}
\end{aligned}
  $$
\end{proof}
\end{frame}

\begin{frame}\ft{\secname}
\begin{proof}[续]
          于是
  $$
  \begin{array}{l}
    \A_1 \gammabd = \zero ~ \Rightarrow ~ \gammabd=\zero, \\[0.2cm]
    \A_1\B_1 = \II_1 ~\Rightarrow~ \B_1=\A_1^{-1}.
  \end{array}
  $$\pause
  由归纳假设,$\B_1$为$n-1$阶上三角矩阵,因此
  $$
  \A^{-1} = \B = \left(
    \begin{array}{cc}
      b_{11} & \betabd\\
      \zero & \B_1 
    \end{array}    
  \right)
  $$
  为上三角矩阵。
\end{proof}
\end{frame}

\begin{frame}\ft{\secname}



\begin{dingyi}[分块矩阵的转置]
  分块矩阵$\A=(\A_{kl})_{s\times t}$的转置矩阵为
  $$
  \A^T = (\B_{lk})_{t\times s},
  $$
  其中$\B_{lk}=\A_{kl}$.
\end{dingyi}
\pause
\begin{li}
  $$
  \A = \left(
    \begin{array}{ccc}
      \A_{11} & \A_{12} & \A_{13}\\
      \A_{21} & \A_{22} & \A_{23}
    \end{array}
  \right) ~\Rightarrow~
  \A = \left(
    \begin{array}{cc}
      \A_{11}^T & \A_{21}^T \\[0.2cm]
      \A_{12}^T & \A_{22}^T \\[0.2cm]
      \A_{13}^T & \A_{23}^T
    \end{array}
  \right)
  $$
\pause
  $$
  \B \xlongequal[]{\mbox{按行分块}} \left(
    \begin{array}{c}
      \bb_1\\
      \bb_2\\
      \vd\\
      \bb_m
    \end{array}
  \right) ~\Rightarrow~
  \B^T = \left(
    \begin{array}{cccc}
      \bb_1^T & \bb_2^T & \cd & \bb_m^T
    \end{array}
  \right)
  $$
\end{li}



\end{frame}

\begin{frame}\ft{\secname}


\begin{dingyi}[可逆分块矩阵的逆矩阵]
  对角块矩阵(准对角矩阵)
  $$
  \A = \left(
    \begin{array}{cccc}
      \A_1&&&\\
          &\A_2&&\\
          &&\dd&\\
          &&&\A_m
    \end{array}
  \right)
  $$
  的行列式为$|\A|=|\A_1||\A_2|\cd|\A_m|$,因此,$\A$可逆的充分必要条件为
  $$
  |\A_i|\ne 0, \quad i=1,2,\cd, m.
  $$

  其逆矩阵为
  $$
  \A^{-1} = \left(
    \begin{array}{cccc}
      \A_1^{-1}&&&\\
               &\A_2^{-1}&&\\
               &&\dd&\\
               &&&\A_m^{-1}
    \end{array}
  \right)
  $$
\end{dingyi}
\end{frame}

\begin{frame}\ft{\secname}
分块矩阵的作用:
\begin{itemize}
\item   用分块矩阵求逆矩阵,可将高阶矩阵的求逆转化为低阶矩阵的求逆。
\item   一个$2\times 2$的分块矩阵求逆,可以根据逆矩阵的定义,用解矩阵方程的方法解得。
\end{itemize}
\end{frame}

\begin{frame}\ft{\secname}
\begin{li}
  设$\A=\left(
    \begin{array}{cc}
      \B&\zero\\
      \C&\D
    \end{array}
  \right)$,其中$\B,\D$皆为可逆矩阵,证明$\A$可逆并求$\A^{-1}$.
\end{li}
\end{frame}

\begin{frame}\ft{\secname}
\begin{jie}
  因$|\A|=|\B||\D|\ne 0$,故$\A$可逆。\pause 设$\A^{-1}=\left(
    \begin{array}{cc}
      \X&\Y\\
      \Z&\T
    \end{array}
  \right)$,则
  $$
  \left(
    \begin{array}{cc}
      \B&\zero\\
      \C&\D
    \end{array}
  \right) \left(
    \begin{array}{cc}
      \X&\Y\\
      \Z&\T
    \end{array}
  \right)=\left(
    \begin{array}{cc}
      \B\X&\B\Y\\
      \C\X+\D\Z&\C\Y+\D\T
    \end{array}
  \right) = \left(
    \begin{array}{cc}
      \II & \zero\\
      \zero & \II
    \end{array}
  \right)
  $$
\pause
  由此可知
  $$
  \begin{array}{ll}
    \B\X = \II   & \Rightarrow ~ \X = \B^{-1}\\[0.2cm]
    \B\Y = \zero & \Rightarrow ~ \Y = \zero\\[0.2cm]
    \C\X+\D\Z = \zero & \Rightarrow ~ \Z = -\D^{-1}\C\B^{-1}\\[0.2cm]
    \C\Y+\D\T = \II & \Rightarrow ~ \T = \D^{-1}
  \end{array}
  $$
\pause
  故
  $$
  \A^{-1} = \left(
    \begin{array}{cc}
      \B^{-1} & \zero\\
      -\D^{-1}\C\B^{-1} & \D^{-1}
    \end{array}
  \right).
  $$
\end{jie}
\end{frame}

\begin{frame}\ft{\secname}




\begin{dingyi}[分块矩阵的初等变换与分块初等矩阵]
  对于分块矩阵
  $$
  \A = \left(
    \begin{array}{cc}
      \A_{11} & \A_{12}\\
      \A_{21} & \A_{22}
    \end{array}
  \right)
  $$
  同样可以定义它的3类初等行变换与列变换,并相应地定义3类分块矩阵:
  \begin{itemize}
  \item[(i)] 分块倍乘矩阵($\C_1,\C_2$为可逆阵)
    $$
    \left(
      \begin{array}{cc}
        \C_1 & \zero\\
        \zero & \II_n
      \end{array}
    \right) ~~\mbox{或}~~
    \left(
      \begin{array}{cc}
        \II_m & \zero\\
        \zero & \C_2
      \end{array}
    \right)
    $$
  \item[(ii)] 分块倍加矩阵
    $$
    \left(
      \begin{array}{cc}
        \II_m & \zero\\
        \C_3 & \II_n
      \end{array}
    \right) ~~\mbox{或}~~
    \left(
      \begin{array}{cc}
        \II_m & \C_4\\
        \zero & \II_n
      \end{array}
    \right)
    $$
  \item[(iii)] 分块对换矩阵
    $$
    \left(
      \begin{array}{cc}
        \zero & \II_n\\
        \II_m & \zero
      \end{array}
    \right)
    $$
  \end{itemize}
\end{dingyi}
\end{frame}

\begin{frame}\ft{\secname}

\begin{li}
  设$n$阶矩阵$\A$分块表示为
  $$
  \A = \left(
    \begin{array}{cc}
      \A_{11} & \A_{12}\\
      \A_{21} & \A_{22}
    \end{array}
  \right)
  $$
  其中$\A_{11},\A_{22}$为方阵,且$\A$与$\A_{11}$可逆。证明:$\A_{22}-\A_{21}\A_{11}^{-1}\A_{12}$可逆,并求$\A^{-1}$。
\end{li}
\end{frame}

\begin{frame}\ft{\secname}
\begin{jie}
  构造分块倍加矩阵
  $$
  \PP_1 = \left(
    \begin{array}{cc}
      \II_1 & \zero\\
      -\A_{21}\A_{11}^{-1} & \II_2
    \end{array}
  \right)
  $$
  则$$
  \PP_1\A = \left(
    \begin{array}{cc}
      \A_{11} & \A_{12} \\
      \zero & \A_{22}-\A_{21}\A_{11}^{-1}\A_{12}
    \end{array}
  \right)
  $$
  两边同时取行列式可知
  $$
  |\A| = |\PP_1\A| = |\A_{11}|\cdot |\A_{22}-\A_{21}\A_{11}^{-1}\A_{12}|
  $$
  故$\A_{22}-\A_{21}\A_{11}^{-1}\A_{12}$可逆。
\end{jie}
\end{frame}

\begin{frame}\ft{\secname}
\begin{jie}[续]
  $$
  \PP_1\A = \left(
    \begin{array}{cc}
      \A_{11} & \A_{12} \\
      \zero & \A_{22}-\A_{21}\A_{11}^{-1}\A_{12}
    \end{array}
  \right)\xlongequal[]{\textcolor{acolor3}{\ds \QQ=\A_{22}-\A_{21}\A_{11}^{-1}\A_{12}}}
  \left(
    \begin{array}{cc}
      \A_{11} & \A_{12} \\
      \zero & \QQ
    \end{array}
  \right)
  $$ \pause
  构造分块倍加矩阵
  $$
  \PP_2 = \left(
    \begin{array}{cc}
      \II_1 & -\A_{12}\QQ^{-1}\\
      \zero & \II_2
    \end{array}
  \right)
  $$ \pause
  则
  $$
  \PP_2\PP_1\A = \left(
    \begin{array}{cc}
      \A_{11} & \zero\\
      \zero & \QQ
    \end{array}
  \right)
  $$ \pause
  于是
  $$
  \begin{array}{rl}
    \A^{-1} & = \left(
      \begin{array}{cc}
        \A_{11}^{-1} & \zero\\
        \zero & \QQ^{-1}
      \end{array}
    \right)\left(
      \begin{array}{cc}
        \II_1 & -\A_{12}\QQ^{-1}\\
        \zero & \II_2
      \end{array}
    \right)\left(
      \begin{array}{cc}
        \II_1 & \zero\\[0.2cm]
        -\A_{21}\A_{11}^{-1} & \II_2
      \end{array}
    \right) \\[0.3in]
    & = \left(
      \begin{array}{cc}
        \A_{11}^{-1} & \zero\\[0.2cm]
        \zero & \QQ^{-1}
      \end{array}
    \right)\left(
      \begin{array}{cc}
        \II_1+ \A_{12}\QQ^{-1}\A_{21}\A_{11}^{-1}& -\A_{12}\QQ^{-1}\\[0.2cm]
        -\A_{21}\A_{11}^{-1} & \II_2
      \end{array}
    \right)\\[0.3in]
    & = \left(
      \begin{array}{cc}
        \A_{11}^{-1}+ \A_{11}^{-1}\A_{12}\QQ^{-1}\A_{21}\A_{11}^{-1}& -\A_{11}^{-1}\A_{12}\QQ^{-1}\\[0.2cm]
        -\QQ^{-1}\A_{21}\A_{11}^{-1} & \QQ^{-1}
      \end{array}
    \right)
  \end{array}
  $$
\end{jie}
\end{frame}

\begin{frame}\ft{\secname}

  \begin{li}
    设$\QQ=\left(
      \begin{array}{cc}
        \A&\B\\
        \C&\D
      \end{array}
    \right)$,且$\A$可逆,证明:
    $$
    |\QQ| = |\A| \cdot |\D-\C\A^{-1}\B|
    $$
  \end{li}\pause
\begin{proof}
  构造分块倍加矩阵
  $$
  \PP_1 = \left(
    \begin{array}{cc}
      \II_1 & \zero\\
      -\C\A^{-1} & \II_2
    \end{array}
  \right)
  $$ \pause
  则
  $$
  \PP_1 \QQ = \left(
    \begin{array}{cc}
      \A & \B\\
      \zero & \D-\C\A^{-1}\B
    \end{array}
  \right)
  $$
\pause
  两边同时取行列式得
  $$
  |\QQ| = |\PP_1\QQ| = |\A|\cdot |\D-\C\A^{-1}\B|.
  $$
\end{proof}
\end{frame}

\begin{frame}\ft{\secname}
  \begin{li}
    设$\A$与$\B$均为$n$阶分块矩阵,证明
    $$
    \left|
      \begin{array}{cc}
        \A&\B\\
        \B&\A
      \end{array}
    \right| = |\A+\B|~|\A-\B|
    $$
  \end{li}
\end{frame}

\begin{frame}\ft{\secname}
\begin{proof}

  将分块矩阵$
  \PP = 
  \left(
    \begin{array}{cc}
      \A&\B\\
      \B&\A
    \end{array}
  \right)$的第一行加到第二行,得\pause
  $$
  \left(
    \begin{array}{cc}
      \II & \zero\\
      \II & \II
    \end{array}
  \right) \left(
    \begin{array}{cc}
      \A&\B\\
      \B&\A
    \end{array}
  \right) = \left(
    \begin{array}{cc}
      \A&\B\\
      \A+\B&\A+\B
    \end{array}
  \right)
  $$\pause
  再将第一列减去第二列,得\pause
  $$
  \left(
    \begin{array}{cc}
      \A&\B\\
      \A+\B&\A+\B
    \end{array}
  \right) \left(
    \begin{array}{cc}
      \II&\zero\\
      -\II&\II
    \end{array}
  \right) = \left(
    \begin{array}{cc}
      \A-\B & \B\\
      \zero & \A+\B
    \end{array}
  \right)
  $$\pause
  总之有
  $$
  \left(
    \begin{array}{cc}
      \II & \zero\\
      \II & \II
    \end{array}
  \right) \left(
    \begin{array}{cc}
      \A&\B\\
      \B&\A
    \end{array}
  \right) 
  \left(
    \begin{array}{cc}
      \II&\zero\\
      -\II&\II
    \end{array}
  \right) = \left(
    \begin{array}{cc}
      \A-\B & \B\\
      \zero & \A+\B
    \end{array}
  \right)
  $$
  两边同时取行列式即得结论。
\end{proof}
\end{frame}
