\section{向量空间}

\begin{frame}\ft{\secname}
\begin{dingyi}[向量空间]
  设$V$是非空的$n$维向量集合,如果集合$V$对于向量的加法和数乘运算满足以下条件:
  \begin{itemize}
  \item[(1)] 对任意的$\alphabd,\betabd\in V$,有$\alphabd+\betabd\in V$;
  \item[(2)] 对任意的$\alphabd\in V,\lambda \in \mathbb R$,有$\lambda\alphabd\in V$,
  \end{itemize}
  则称$V$为向量空间。
\end{dingyi} \pause 
定义中的(1)表示该集合中向量对加法运算封闭;(2)表示该集合中向量对数乘运算封闭。因此,向量空间也可表述为:\blue{对加法和数乘运算封闭的非空集合。}
\end{frame}

\begin{frame}\ft{\secname}
\begin{li}
  设三维向量的全体
  $$
  \R^3=\{\alphabd=(x_1,x_2,x_3)^T|x_1,x_2,x_3\in\R\}
  $$
  以及三维向量集合
  $$
  V_1=\{\alphabd=(0,x_2,x_3)^T|x_2,x_3\in\R\}
  $$
  和
  $$
  V_2=\{\alphabd=(1,x_2,x_3)^T|x_2,x_3\in\R\}
  $$
  验证$\R^3, V_1$是向量空间,而$V_2$不是向量空间。
\end{li}
\pause 
类似地,$n$维向量的全体$\R^n$也是一个向量空间。

\end{frame}

\begin{frame}\ft{\secname}

  \begin{li}
  验证齐次线性方程组的全体
  $$
  S=\{x|Ax=0\}
  $$
  是向量空间。
\end{li} \pause 
\begin{jie}
  因$0\in S$,$S$为非空集合,由齐次线性方程组解的性质知,对任意的$\xibd_1,\xibd_2\in S,\lambda \in R$,有$\xibd_1+\xibd_2\in S$且$\lambda \xibd_1\in S$,故$S$中的向量关于加法与数乘运算封闭,称这个向量空间为齐次线性方程组的解空间。 
\end{jie}

\end{frame}

\begin{frame}\ft{\secname}

\begin{li}
  验证非齐次线性方程组解的全体
  $$
  S=\{x|Ax=b, b\ne 0\}
  $$
  不是向量空间。
\end{li} \pause 
\begin{jie}
  因对任意的$\etabd_1,\etabd_2\in S, \lambda\in \R$,有
  $$
  A(\etabd_1+\etabd_2)=A\etabd_1+A\etabd_2=b+b=2b,
  $$
  故$\etabd_1+\etabd_2\notin S$,即$S$关于向量的加法与数乘运算不封闭。
\end{jie}
\end{frame}

\begin{frame}\ft{\secname}

\begin{li}
  设$\alphabd,\betabd$是$n$维向量,验证集合
  $$
  V(\alphabd,\betabd)=\{\etabd=\lambda\alphabd+\mu\betabd|\lambda,\mu\in\R\}
  $$
  是向量空间。
\end{li} \pause 
\begin{jie}
  因$0=0\alphabd+0\betabd\in V$,故$V$为非空集合。另外,对任意的$\etabd_1,\etabd_2\in V, k\in \R$,有
  $$
  \etabd_1=\lambda_1\alphabd+\mu_1\betabd, \quad
  \etabd_2=\lambda_2\alphabd+\mu_2\betabd
  $$
  从而
  $$
  \begin{aligned}
    \etabd_1+\etabd_2=(\lambda_1+\lambda_2)\alphabd+(\mu_1+\mu_2)\betabd\in V,\\
    k\etabd_1=(k\lambda_1)\alphabd+(k\mu_1)\betabd\in V,
  \end{aligned}
  $$
  称该向量空间为由向量$\alphabd,\betabd$生成的向量空间。
\end{jie}
\end{frame}

\begin{frame}\ft{\secname}

\begin{dingyi}
  由\blue{向量组$\alphabd_1,\alphabd_2,\cd,\alphabd_m$}生成的向量空间可表示为
  $$
  V(\alphabd_1,\alphabd_2,\cd,\alphabd_m)=\{\lambda_1\alphabd_1+\lambda_2\alphabd_2+\cd+\lambda_m\alphabd_m|\lambda_1,\lambda_2,\cd,\lambda_m\in \R\}.
  $$
\end{dingyi}
\end{frame}

\begin{frame}\ft{\secname}

\begin{li}
证明:等价向量组生成的向量空间等价。
\end{li} \pause 
\begin{proof}
  设向量组$\alphabd_1,\alphabd_2,\cd,\alphabd_m$与向量组$\betabd_1,\betabd_2,\cd,\betabd_m$等价,记
  $$
  \begin{aligned}
    V_1=\{x=\lambda_1\alphabd_1+\lambda_2\alphabd_2+\cd+\lambda_m\alphabd_m|\lambda_1,\lambda_2,\cd,\lambda_m\in \R\},\\
    V_2=\{x=\mu_1\betabd_1+\mu_2\betabd_2+\cd+\mu_m\betabd_m|\mu_1,\mu_2,\cd,\mu_m\in \R\},
  \end{aligned}
  $$
  需证明$V_1=V_2$。 \pause 

  设$x\in V_1$,则$x$可由$\alphabd_1,\alphabd_2,\cd,\alphabd_m$线性表示。因$\alphabd_1,\alphabd_2,\cd,\alphabd_m$可由$\betabd_1,\betabd_2,\cd,\betabd_m$线性表示,故$x$可由$\betabd_1,\betabd_2,\cd,\betabd_m$线性表示,即$x\in V_2$,于是$V_1\subset V_2$。同理可证$V_2\subset V_1$。
\end{proof}

\end{frame}