\section{正定二次型和正定矩阵}

\begin{frame}
  二次型的标准形是不唯一的,但标准形中所含项数(即二次型的秩)是确定的。不仅如此,在限定变换为实变换时,标准形中正系数的个数是不变的,从而负系数的个数也是不变的。
\end{frame}

\begin{frame}
  \begin{dingli}[惯性定理]
    设有二次型$f=\xx^T\A\xx$,它的秩为$r$,有两个可逆变换
    $$
    \xx = \C\yy, \quad \xx = \QQ \zz,
    $$
    使得
    $$
    \begin{aligned}
      f & = k_1y_1^2+k_2y_2^2+\cd+k_ry_r^2, ~~ k_i\ne 0, \\[0.1in]
      f & = \lambda_1y_1^2+\lambda_2y_2^2+\cd+\lambda_ry_r^2, ~~ \lambda_i\ne 0,
    \end{aligned}
    $$
    则$k_1,k_2,\cd,k_r$中正数的个数与$\lambda_1,\lambda_2,\cd,\lambda_r$中正数的个数相等。
  \end{dingli}
\end{frame}

\begin{frame}
  二次型的标准形中,正系数的个数称为二次型的\blue{正惯性指数},负系数的个数称为\blue{负惯性指数}。若二次型$f$的正惯性指数为$p$,秩为$r$,则$f$的规范形便可确定为
  $$
  f=y_1^2+\cd+y_p^2-y_{p+1}^2-\cd-y_r^2.
  $$
\end{frame}

\begin{frame}
  
    \begin{dingyi}
      如果对于任意的非零向量$\xx=(x_1,x_2,\cd,x_n)^T$,恒有
      $$
      \xx^T\A\xx=\sum_{i=1}^n\sum_{j=1}^na_{ij}x_ix_j>0,
      $$
      就称$\xx^T\A\xx$为正定二次型,称$\A$为正定矩阵。
    \end{dingyi}
    \pause\vspace{0.1in}

    
    注:正定矩阵是针对对称矩阵而言的。
    
  
\end{frame}

\begin{frame}
  
    \begin{jielun}
      二次型$f(y_1,y_2,\cd,y_n)=d_1y_1^2+d_2y_2^2+\cd+d_ny_n^2$正定
      $~~~\Longleftrightarrow~~~d_i>0~~(i=1,2,\cd,n)$
    \end{jielun}\pause

    \begin{proof}
    \begin{itemize}
    \item[$\Leftarrow$] 显然 \pause
    \item[$\Rightarrow$] 设$d_i\le 0$,取$y_i=1, y_j=0(j\ne i)$,代入二次型,得
      $$
      f(0,\cd,0,1,0,\cd,0)=d_i\le 0
      $$
      这与二次型$f(y_1,y_2,\cd,y_n)$正定矛盾。
    \end{itemize}
    \end{proof}
  
\end{frame}


\begin{frame}
  
  \begin{jielun}
      一个二次型$\xx^T\A\xx$,经过非退化线性变换$\xx=\C\yy$,化为$\yy^T(\C^T\A\C)\yy$,其正定性保持不变。即当
      $$\xx^T\A\xx~~~\xLongleftrightarrow[]{\ds \xx=\C\yy}~~~\yy^T(\C^T\A\C)\yy\quad (\C\mbox{可逆})$$
      时,等式两端的二次型有相同的正定性。
    \end{jielun}\pause

    \begin{proof}
    $\forall \yy=(y_1,y_2,\cd,y_n)\ne\zero$,由于$\xx=\C\yy(\C\mbox{可逆})$,则$\xx\ne \zero$。若$\xx^T\A\xx$正定,则$\xx^T\A\xx>0$。
    从而有:$\forall \yy\ne\zero$,
    $$
    \yy^T(\C^T\A\C)\yy=\xx^T\A\xx>0
    $$
    故$\yy^T(\C^T\A\C)\yy$是正定二次型。\pause 反之亦然。
    \end{proof}
  
\end{frame}


\begin{frame}
  
    \begin{dingli}
      若$\A$是$n$阶实对称矩阵,则以下命题等价:
      \begin{itemize}
      \item[(1)]$\xx^T\A\xx$是正定二次型($\A$是正定矩阵);
      \item[(2)]$\A$的正惯性指数为$n$,即$\A\simeq\II$;
      \item[(3)]存在可逆矩阵$\PP$使得$\A=\PP^T\PP$;
      \item[(4)]$\A$的$n$个特征值$\lambda_1,\lambda_2,\cd,\lambda_n$全大于零。
      \end{itemize}
    \end{dingli}
  
\end{frame}

\begin{frame}
  \begin{proof}[(1)$\Rightarrow$(2)]
    对$\A$,存在可逆阵$\C$,使得
    $$
    \C^T\A\C=\diag(d_1,d_2,\cd,d_n).
    $$
    设$\A$的正惯性指数$<n$,则至少存在一个$d_i\le 0$。做变换$\xx=\C\yy$,则
    $$
    \xx^T\A\xx=\yy^T(\C^T\A\C)\yy=d_1y_1^2+d_2y_2^2+\cd+d_ny_n^2
    $$
    不恒大于零,与命题(1)矛盾。故$\A$的正惯性指数为$n$。
  \end{proof}
\end{frame}

\begin{frame}
  \begin{proof}[(2)$\Rightarrow$(3)]
    由$\C^T\A\C=\II$得$\A=(\C^T)^{-1}\C^{-1}=(\C^{-1})^T\C^{-1}$,取$\PP=\C^{-1}$,则有$\A=\PP^{T}\PP$。
  \end{proof}
\end{frame}

\begin{frame}
  \begin{proof}[(3)$\Rightarrow$(4)]
    设$\A\xx=\lambda\xx$,即$(\PP^{T}\PP)\xx=\lambda\xx$,于是
    $$
    \xx^T\PP^{T}\PP\xx=\lambda\xx^T\xx
    $$
    即
    $$
    (\PP\xx,\PP\xx)=\lambda(\xx,\xx)
    $$
    因特征向量$\xx\ne 0$,从而$\PP\xx\ne 0$,故
    $$
    \lambda = \frac{(\PP\xx,\PP\xx)}{(\xx,\xx)}>0.
    $$
  \end{proof}
\end{frame}

\begin{frame}
  \begin{proof}[(4)$\Rightarrow$(1)]
    对于实对称矩阵$\A$,存在正交矩阵$\QQ$,使得
    $$
    \QQ^T\A\QQ=\diag(\lambda_1,\lambda_2,\cd,\lambda_n),
    $$
    做正交变换$\xx=\QQ\yy$得
    $$
    \xx^T\A\xx=\lambda_1y_1^2+\lambda_2y_2^2+\cd+\lambda_ny_n^2.
    $$
    由于特征值$\lambda_1,\lambda_2,\cd,\lambda_n$都大于零,故$\xx^T\A\xx$正定。
  \end{proof}
\end{frame}


\begin{frame}
  
    \begin{li}
      $\A\mbox{正定} ~~\Longrightarrow~~ \A^{-1}\mbox{正定}$
    \end{li}
    \pause 
    \begin{proof}
      设$\lambda$为$\A$的任一特征值,因$\A$正定,故$\lambda>0$。而$\lambda^{-1}$为$\A^{-1}$的特征值,显然大于零,故$\A^{-1}$正定。
    \end{proof}
\end{frame}


\begin{frame}
  
    \begin{li}
      判断二次型
      $$
      f(x_1,x_2,x_3)=x_1^2+2x_2^2+3x_3^2+2x_1x_2-2x_2x_3
      $$
      是否为正定二次型。
    \end{li}
    \begin{jie}
      用配方法得
      $$
      \begin{aligned}
        f&=(x_1^2+2x_1x_2+x_2^2)+(x_2^2-2x_2x_3+x_3^2)+2x_3^2\\
        &=(x_1+x_2)^2+(x_2-x_3)^2+2x_3^2\ge 0.
      \end{aligned}
      $$
      等号成立的充要条件是
      $$
      \left\{
        \begin{aligned}
          x_1+x_2=0,\\
          x_2-x_3=0,\\
          x_3=0,
        \end{aligned}
      \right.
      $$
      即$x_1=x_2=x_3=0$,故$f$正定。
    \end{jie}
\end{frame}

\begin{frame}
  
    \begin{li}
      判断二次型
      $$
      f(x_1,x_2,x_3)=3x_1^2+x_2^2+3x_3^2-4x_1x_2-4x_1x_3+4x_2x_3
      $$
      是否为正定二次型。
    \end{li}
    \begin{jie}
      任何二次型都可用配方法判断正定性,但此题配方时系数较为复杂,可考虑用特征值判定。
    \end{jie}
\end{frame}

\begin{frame}
  
    \begin{dingli}
      $$
      \A\mbox{正定}~~\Longrightarrow~~
      a_{ii}>0(i=1,2,\cd,n) \mbox{~~且~~}
      |\A|>0
      $$
    \end{dingli}
    \pause
    \begin{jie}
      因
      $$
      \xx^T\A\xx=\sum_{i=1}^n\sum_{j=1}^na_{ij}x_ix_j
      $$
      正定,取$\xx=(0,\cd,0,1,0,\cd,0)^T\ne 0$(其中第$i$个分量$x_i=1$),则必有
      $$
      \xx^T\A\xx=a_{ii}x_i^2=a_{ii}>0, \quad i=1,2,\cd,n.
      $$ \vspace{0.1in}

      因$\A$正定,故$\A$的所有特征值均大于零,即得$|\A|=\lambda_1\lambda_2\cd\lambda_n>0$。
    \end{jie}
\end{frame}

\begin{frame}
  \begin{dingyi}
    设$A=(a_{ij})$,则
    $$
    \det~ \A_k = \left|
      \begin{array}{cccc}
        a_{11}&a_{12}&\cd&a_{1k}\\
        a_{21}&a_{22}&\cd&a_{2k}\\
        \vd&\vd&&\vd\\
        a_{k1}&a_{k2}&\cd&a_{kk}\\
      \end{array}   
    \right|
    $$
    称为$n$阶矩阵$\A$的$k$阶顺序主子式。当$k$取$1,2,\cd,n$时,就得$\A$的$n$个顺序主子式。
  \end{dingyi}
  \end{frame}

\begin{frame}

    \begin{dingli}
      $$
     \begin{aligned}
       \A\mbox{正定} ~~\Longleftrightarrow~~ \A\mbox{的各阶顺序主子式全为正。}\\        \A\mbox{负定} ~~\Longleftrightarrow~~ \A\mbox{的奇数阶顺序主子式全为负,偶数阶顺序主子式全为正。}\\       
     \end{aligned}
     $$
    \end{dingli}

\end{frame}

\begin{frame}
\begin{li}
  判断二次型$f=-5x^2-6y^2-4z^2+4xy+4xz$的正定性。
\end{li}
\pause
\begin{jie}
  $f$的矩阵为
  $$
  \A=\left(
    \begin{array}{rrr}
      -5&2&2\\
      2&-6&0\\
      2&0&-4\\
    \end{array}
  \right)
  $$
  因
  $$
  a_{11}=-5<0, ~~\left|
    \begin{array}{rr}
      -5&2\\
      2&-6
    \end{array}
  \right|=26>0, ~~\left|
    \begin{array}{rrr}
      -5&2&2\\
      2&-6&0\\
      2&0&-4\\
    \end{array}
  \right|=-80<0
  $$
  故$f$负定。
\end{jie}
\end{frame}
