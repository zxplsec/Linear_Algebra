\section{二次型的定义与矩阵表示\quad 合同矩阵}

\begin{frame}
  
  \begin{dingyi}[二次型]
    $n$元变量$x_1,x_2,\cd,x_n$的二次齐次多项式
    \begin{equation}\label{qf}
    \begin{array}{rcccccccc}
      f(x_1,x_2,\cd,x_n) &=& \\[0.1in]
      a_{11}x_1^2&+&2a_{12}x_1x_2&+&2a_{13}x_1x_3&+&\cd&+&2a_{1n}x_1x_n\\[0.1in]
                         &+&a_{22}x_2^2&+&2a_{23}x_2x_3&+&\cd&+&2a_{2n}x_2x_n\\[0.1in]
                         &&&&\cd&&\cd\\[0.1in]
                         &&&&&&&+&a_{nn}x_n^2
    \end{array}
    \end{equation}
    当系数属于数域$F$时,称为数域$F$上的一个\blue{\underline{$n$元二次型}}。
  \end{dingyi}
  
\end{frame}


\begin{frame}[allowframebreaks]
  设$a_{ij}=a_{ji}$,则
  $$
  \begin{array}{l}
    f(x_1,x_2,\cd,x_n) =  \\[0.1in]
    \begin{array}{rcccccccccc}
      &a_{11}x_1^2&+&a_{12}x_1x_2&+&a_{13}x_1x_3&+&\cd&+&a_{1n}x_1x_n\\[0.1in]
      +&a_{21}x_2x_1&+&a_{22}x_2^2&+&a_{23}x_2x_3&+&\cd&+&a_{2n}x_2x_n\\[0.1in]
      &\cd&&\cd&&\cd&&\cd\\[0.1in]
      +&a_{n1}x_nx_1&+&a_{n2}x_nx_2&+&a_{n3}x_2x_3&+&\cd&+&a_{nn}x_n^2
    \end{array} \\[0.5in] 
    \ds
    \blue{= \sum_{i=1}^nx_i(a_{i1}x_1+a_{i2}x_2+\cd+a_{in}x_n)
    = \sum_{i=1}^nx_i\sum_{j=1}^n a_{ij}x_{ij}
    = \sum_{i=1}^n\sum_{j=1}^na_{ij}x_ix_j}\\[0.2in] 
    % \ds = (x_1,x_2,\cd,x_n)\left(
    % \begin{array}{c}
    %   a_{11}x_1+a_{12}x_2+\cd+a_{1n}x_n\\
    %   a_{21}x_1+a_{22}x_2+\cd+a_{2n}x_n\\
    %   \vd\\
    %   a_{n1}x_1+a_{n2}x_2+\cd+a_{nn}x_n
    % \end{array}
    % \right)\\[0.3in]
    \ds = (x_1,x_2,\cd,x_n)\left(
    \begin{array}{cccc}
      a_{11}&a_{12}&\cd&a_{1n}\\
      a_{21}&a_{22}&\cd&a_{2n}\\
      \vd&\vd&&\vd\\
      a_{n1}&a_{n2}&\cd&a_{nn}\\
    \end{array}
    \right)\left(
    \begin{array}{c}
      x_1\\
      x_2\\
      \vd\\
      x_n
    \end{array}\right)  = \xx^T\A\xx
  \end{array}
  $$
  
\end{frame}


\begin{frame}
  
  \begin{itemize}
  \item
    对于任意一个二次型,总可以写成对称形式
    $$
    f(x_1,x_2,\cd,x_n)=\xx^T\A\xx
    $$
    其中$\A$为对称矩阵。\\[0.1in]\pause
  \item
    若$\A,\B$为对称矩阵,且
    $$
    f(x_1,x_2,\cd,x_n)=\xx^T\A\xx=\xx^T\B\xx
    $$
    则必有$\A=\B$。\\[0.1in]\pause
  \item 二次型和它的矩阵是相互唯一确定的,因此研究二次型的性质就转化为研究对称矩阵$\A$所具有的性质。\\[0.1in]\pause
  \item \blue{对于二次型$f=\xx^T\A\xx$,对称阵$\A$称为二次型$f$的矩阵,$f$称为对称阵$\A$的二次型,而$\A$的秩称为二次型$f$的秩。}
  \end{itemize}

  
\end{frame}

\begin{frame}
  
  \begin{li}
    设$f(x_1,x_2,x_3,x_4)=2x_1^2+x_1x_2+2x_1x_3+4x_2x_4+x_3^2+5x_4^2$,则它的矩阵为
    $$
    \A=\left(
      \begin{array}{cccc}
        2&\frac12&1&0\\[0.2cm]
        \frac12&0&0&2\\[0.2cm]
        1&0&1&0\\[0.2cm]
        0&2&0&5
      \end{array}
    \right)
    $$
  \end{li}
  
\end{frame}

\begin{frame}
  
  对于二次型,我们讨论的主要问题是:寻求可逆的线性变换
  \begin{equation}\label{lt}
  \left\{
    \begin{aligned}
      x_1=c_{11}y_1+c_{12}y_2+\cd+c_{1n}y_n,\\
      x_2=c_{12}y_1+c_{22}y_2+\cd+c_{2n}y_n,\\
      \cd\cd\cd\\
      x_n=c_{n1}y_1+c_{n2}y_2+\cd+c_{nn}y_n
    \end{aligned}
  \right.
  \end{equation}
  使得二次型只含平方项。也就是用\eqref{lt}代入\eqref{qf},能使
  $$
  f=\lambda_1y_1^2+\lambda_2y_2^2+\cd+\lambda_ny_n^2.
  $$
  这种只含平方项的二次型,称为\blue{二次型的标准形}。
\end{frame}

\begin{frame}
  若标准形的系数$\lambda_1,\lambda_2,\cd,\lambda_n$只在$1,-1,0$中取值,也就是用\eqref{lt}代入\eqref{qf},能使
  $$
  f=y_1^2+\cd+y_p^2-y_{p+1}^2-\cd-y_r^2.
  $$
  则称上式为\blue{二次型的规范形}。
\end{frame}

\begin{frame}
  
  设$\C=(c_{ij})$,则可逆变换\eqref{lt}可记为
  $$
  \xx=\C\yy
  $$
  从而
  $$
  \blue{f=\xx^T\A\xx=\yy^T(\C^T\A\C)\yy.}
  $$ 
\end{frame}

\begin{frame}
  
  把一般的二次型$f(x_1,x_2,\cd,x_n)$化为$y_1,y_2,\cd,y_n$的纯平方项之代数和的基本方法是做坐标变换
  $$
  \xx=\C\yy \mbox{~~~~~($\C$为可逆矩阵)}
  $$
  使得
  $$
  \xx^T\A\xx=\yy^T\C^T\A\C\yy=d_1y_1^2+\cd+d_ny_n^2.
  $$
  \pause \vspace{0.1in}

  \blue{从矩阵的角度来说,就是对于一个实对称矩阵$\A$,寻找一个可逆矩阵,使得$\C^T\A\C$称为对角形。}
  
\end{frame}



% \begin{frame}
  
%   设$\alphabd$在两组基$\{\epsilonbd_1,\epsilonbd_2,\cd,\epsilonbd_n\}$和$\{\etabd_1,\etabd_2,\cd,\etabd_n\}$下的坐标向量分别为
%   $$
%   \xx=(x_1,x_2,\cd,x_n)^T\mbox{~~和~~}\yy=(y_1,y_2,\cd,y_n)^T
%   $$
%   又
%   $$
%   (\etabd_1,\etabd_2,\cd,\etabd_n)=(\epsilonbd_1,\epsilonbd_2,\cd,\epsilonbd_n)\C
%   $$
%   故
%   $$
%   \xx=\C\yy
%   $$
%   从而
%   $$
%   f(\alphabd)=\xx^T\A\xx=\yy^T(\C^T\A\C)\yy
%   $$ \pause


%   \blue{
%     二次型$f(\alphabd)$在两组基$\{\epsilonbd_1,\epsilonbd_2,\cd,\epsilonbd_n\}$和$\{\etabd_1,\etabd_2,\cd,\etabd_n\}$下所对应的矩阵分别为
%     $$
%     \A \mbox{~~和~~} \C^T\A\C     
%     $$
%   }
  
% \end{frame}


% \begin{frame}
  
%   \begin{li}
%     设$\alphabd$在自然基$\{\epsilonbd_1,\epsilonbd_2\}$下的坐标$(x_1,x_2)^T$满足方程
%     \begin{equation}\label{li2-1}
%       5x_1^2+5x_2^2-6x_1x_2=4.
%     \end{equation}      
%     将$\epsilonbd_1,\epsilonbd_2$逆时针旋转$\pi/4$变为$\etabd_1,\etabd_2$      
%   \end{li}
%   \pause 
%   $$
%   (\etabd_1,\etabd_2)=(\epsilonbd_1,\epsilonbd_2)\left(
%     \begin{array}{rr}
%       \cos \pi/4&-\sin \pi/4\\
%       \sin\pi/4&\cos\pi/4
%     \end{array}
%   \right)
%   $$
%   则$\alphabd$在基$\{\etabd_1,\etabd_2\}$下的坐标$(y_1,y_2)^T$满足
%   $$
%   \xx=\left(
%     \begin{array}{c}
%       x_1\\
%       x_2
%     \end{array}
%   \right)=\left(
%     \begin{array}{rr}
%       \cos \pi/4&-\sin \pi/4\\
%       \sin\pi/4&\cos\pi/4
%     \end{array}
%   \right)\left(
%     \begin{array}{c}
%       y_1\\
%       y_2
%     \end{array}
%   \right)=\C\yy
%   $$\pause 
%   (\ref{li2-1})的矩阵形式为
%   $$
%   \xx^T\A\xx=(x_1,x_2)\left(
%     \begin{array}{rr}
%       5&-3\\
%       -3&5
%     \end{array}
%   \right)\left(
%     \begin{array}{c}
%       x_1\\
%       x_2
%     \end{array}
%   \right)=4
%   $$

  
% \end{frame}


% \begin{frame}
  
%   $$
%   \begin{array}{rl}
%     \xx^T\A\xx& =\yy^T\C^T\A\C\yy\\[0.1in]
%               & =(y_1,y_2)\left(
%                 \begin{array}{rr}
%                   \frac{\sqrt{2}}2&\frac{\sqrt{2}}2\\[0.2cm]
%                   -\frac{\sqrt{2}}2&\frac{\sqrt{2}}2
%                 \end{array}
%                                      \right)\left(
%                                      \begin{array}{rr}
%                                        5&-3\\
%                                        -3&5
%                                      \end{array}
%                                            \right)\left(
%                                            \begin{array}{rr}
%                                              \frac{\sqrt{2}}2&-\frac{\sqrt{2}}2\\[0.2cm]
%                                              \frac{\sqrt{2}}2&\frac{\sqrt{2}}2
%                                            \end{array}
%                                                                \right)\left(
%                                                                \begin{array}{c}
%                                                                  y_1\\
%                                                                  y_2
%                                                                \end{array}
%     \right)\\[0.2in]
%               & =(y_1,y_2)\left(
%                 \begin{array}{rr}
%                   2&0\\
%                   0&8
%                 \end{array}
%                      \right)\left(
%                      \begin{array}{c}
%                        y_1\\
%                        y_2
%                      \end{array}
%     \right)  \\[0.2in]
%               &= 2y_1^2+8y_2^2=4.      
%   \end{array}
%   $$ \pause
%   此时,方程(\ref{li2-1})化成了在基$\{\etabd_1,\etabd_2\}$的坐标系下的标准方程,其图形是一个椭圆。

  
% \end{frame}



\begin{frame}
  
  \begin{dingyi}[矩阵的合同]
    对于两个矩阵$\A$和$\B$,若存在可逆矩阵$\C$,使得
    $$
    \C^T\A\C=\B,
    $$
    就称$\A$合同于$\B$,记作$\A\simeq\B$。
  \end{dingyi}
  
\end{frame}

\begin{frame}
  显然,\blue{若$\A$为对称阵,则$\B=\C^T\A\C$也是对称阵,且$\rank(\B)=\rank(\A)$。}
  \pause \vspace{.1in}

  事实上,
  $$
  \B^T = (\C^T\A\C)^T = \C^T\A^T\C = \C^T\A\C = \B,
  $$
  即$\B$为对称阵。又因为$\B=\C^T\A\C$,而$\C$可逆,从而$\C^T$也可逆,由矩阵秩的性质可知$\rank(\B)=\rank(\A)$。
\end{frame}

\begin{frame}
  \begin{jielun}

    经过可逆变换$\xx=\C\yy$后,二次型$f$的矩阵由$\A$变为与$\A$合同的矩阵$\C^T\A\C$,且二次型的秩不变。
  \end{jielun}
\end{frame}






