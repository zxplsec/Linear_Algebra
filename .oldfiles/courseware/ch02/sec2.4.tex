\section{逆矩阵}

\begin{frame}
  \begin{footnotesize}
    给定一个从$\xx$到$\yy$的线性变换
    \begin{equation}\label{yax}
      \yy = \A \xx  
    \end{equation}       
    即
    $$
    \left\{
    \begin{array}{c}
      y_1 = a_{11}x_1 + a_{12}x_2 + \cd + a_{1n}x_n\\[0.1cm]
      y_2 = a_{21}x_1 + a_{22}x_2 + \cd + a_{2n}x_n\\[0.1cm]
      \cd\cd \\[0.1cm]
      y_n = a_{n1}x_1 + a_{n2}x_2 + \cd + a_{nn}x_n
    \end{array}
    \right.
    $$
    \pause
    
    用$\A$的伴随阵$\A^*$左乘(\ref{yax}),得
    $$
    \A^* \yy = \A^* \A \xx \pause = |\A|\xx
    $$
    \pause
    当$|\A|\ne 0$时,有
    $$
    \xx = \frac{1}{|\A|}\A^* \yy
    $$
    \pause
    记
    $$
    \red{\B = \frac{1}{|\A|}\A^*,}
    $$
    则上式可记为
    \begin{equation}\label{xby}
      \xx = \B \yy,
    \end{equation}
    它表示一个从$\yy$到$\xx$的线性变换,称为线性变换(\ref{yax})的逆变换。
  \end{footnotesize}
\end{frame}



\begin{frame}
  \begin{footnotesize}
    \begin{block}{$\A$与$\B$的关系}
      \pause 
      \begin{itemize}
      \item[1] 将(\ref{xby})代入(\ref{yax})
        $$
        \yy = \A(\B\yy) = (\A\B)\yy
        $$
        可见$\A\B$为恒等变换对应的矩阵,故
        $$\A\B=\II.$$
        \pause
      \item[2] 将(\ref{yax})代入(\ref{xby})
        $$
        \xx = \B(\A\xx) = (\B\A)\xx
        $$
        可见$\B\A$为恒等变换对应的矩阵,故
        $$\B\A=\II.$$
      \end{itemize}
    \end{block}
    \pause 
    $$
    \red{
      \A\B = \B\A = \II.
    }
    $$

  \end{footnotesize}
\end{frame}


\begin{frame}
  \begin{footnotesize}
    \begin{block}{逆矩阵}
      对于$n$阶矩阵$\A$,如果有一个$n$阶矩阵$\B$,使
      $$
      \red{
        \A\B = \B\A = \II.
      }
      $$
      则称$\A$是\red{可逆}的,并把$\B$称为$\A$的\red{逆矩阵}。
    \end{block}
    \pause
    
    \begin{block}{注}
      \begin{itemize}
      \item[1] 可逆矩阵与其逆矩阵为同阶方阵。
      \item[2] $\A$与$\B$地位相等,也可称$\A$为$\B$的逆矩阵。      
      \end{itemize}
    \end{block}
  \end{footnotesize}
\end{frame}


\begin{frame}
  \begin{footnotesize}
    \begin{block}{定理1}
      若$\A$可逆,则$\A$的逆阵惟一。
    \end{block}
    \pause
    \proofname


    \pause 
    \vspace{2.cm}
    \red{
      $\A$的矩阵记作$\A^{-1}$,即
      $$
      \A\B = \B\A = \II ~ \Rightarrow ~ \B = \A^{-1}.
      $$
    }
  \end{footnotesize}
\end{frame}

\begin{frame}
  \begin{footnotesize}
    \begin{block}{定理2}
      若$\A$可逆,则$|\A|\ne 0$.
    \end{block}
    \pause
    \proofname


  \end{footnotesize}
\end{frame}

\begin{frame}
  \begin{footnotesize}
    \begin{block}{代数余子式矩阵,伴随矩阵}
      设$\A=(a_{ij})_{n\times n}$,$A_{ij}$为行列式$|\A|$中元素$a_{ij}$的代数余子式,称
      $$
      \mathrm{coef} \A = (A_{ij})_{n\times n}
      $$
      为$\A$的\red{代数余子式矩阵},并称$\mathrm{coef} \A$的转置矩阵为$\A$的\red{伴随矩阵},记为$\A^*$,
      即
      $$\red{
        \A^* = (\mathrm{coef}\A)^T = \left(
        \begin{array}{cccc}
          A_{11} & A_{21} & \cd & A_{n1} \\
          A_{12} & A_{22} & \cd & A_{n2} \\
          \vd   & \vd   &     & \vd   \\
          A_{1n} & A_{2n} & \cd & A_{nn} \\
        \end{array}
        \right)
      }
      $$
    \end{block}
    \pause
    之前已证
    $$ \red{
      \A\A^* = |\A|\II
    }
    $$
    \pause
    同理可证
    $$ \red{
      \A^*\A = |\A|\II
    }
    $$

  \end{footnotesize}
\end{frame}




\begin{frame}
  \begin{footnotesize}
    \begin{block}{定理3}
      若$|\A|\ne 0$,则$\A$可逆,且
      $$
      \A^{-1} = \frac1{|\A|} \A^*
      $$
    \end{block}
    \pause
    \proofname



    \pause 
    \vspace{2.cm}
    \red{
      该定理提供了求$\A^{-1}$的一种方法。
    }

  \end{footnotesize}
\end{frame}


\begin{frame}
  \begin{footnotesize}
    \begin{block}{推论}
      若$\A\B = \II$(或$\B\A=\II$),则
      $$
      \B = \A^{-1}.
      $$
    \end{block}
    \pause
    \proofname



    \pause 
    \vspace{2.cm}
    \red{
      该推论告诉我们,判断$\B$是否为$\A$的逆,只需验证$\A\B=\II$或$\B\A=\II$的一个等式成立即可。
    }

  \end{footnotesize}
\end{frame}


\begin{frame}
  \begin{footnotesize}
    \begin{block}{奇异阵与非奇异阵}
      当$|\A|=0$时,$\A$称为\red{奇异矩阵},否则称为\red{非奇异矩阵}。
    \end{block}

    \pause
    \begin{block}{注}
      \red{可逆矩阵就是非奇异矩阵。}
    \end{block}
  \end{footnotesize}
\end{frame}



\begin{frame}
  \begin{footnotesize}
    \begin{block}{可逆矩阵的运算规律}
      \begin{itemize}
      \item[1] 若$\A$可逆,则$\A^{-1}$亦可逆,且
        $$(\A^{-1})^{-1}=\A.$$\\
      \item[2] 若$\A$可逆,$k\ne 0$,则$k\A$可逆,且
        $$(k\A)^{-1}= k^{-1}A^{-1}.$$ \\
      \item[3] 若$\A, ~\B$为同阶矩阵且均可逆,则$\A\B$可逆,且
        $$(\A\B)^{-1} = \B^{-1}\A^{-1}.$$ 
      \item[] 若$\A_1,\A_2,\cd,\A_m$皆可逆,则
        $$
        (\A_1\A_2\cd\A_m)^{-1}=\A_m^{-1}\cd\A_2^{-1}\A_1^{-1}
        $$\\[0.3cm]
      \item[4] 若$\A$可逆,则$\A^T$亦可逆,且
        $$(\A^T)^{-1}=(\A^{-1})^T.$$ \\
      \item[5] 若$\A$可逆,则
        $$|\A^{-1}|=|\A|^{-1}$$
      \end{itemize}
    \end{block}
  \end{footnotesize}
\end{frame}


\begin{frame}
  \begin{footnotesize}
    \begin{exampleblock}{例1}
      已知$\A = \left(
      \begin{array}{cc}
        a & b \\
        c & d
      \end{array}
      \right)$,求$\A^{-1}$。
    \end{exampleblock}
    \pause 
    \jiename
    $$
    |\A| = ad-bc, \quad
    |\A^*| = \left(
    \begin{array}{rr}
      d & -b \\
      -c & a
    \end{array}
    \right)
    $$
    \pause 
    \begin{itemize}
    \item[1] 当$|\A|=ad-bc=0$时,逆阵不存在;\pause 
    \item[2] 当$|\A|=ad-bc\ne0$时,
      $$
      \A^{-1} = \frac1{|\A|} \A^* = \frac1{ad-bc}\left(
      \begin{array}{rr}
        d & -b \\
        -c & a
      \end{array}
      \right)
      $$
    \end{itemize}
    
  \end{footnotesize}
\end{frame}
 

\begin{frame}
  \begin{footnotesize}
    \begin{exampleblock}{例2}
      求方阵
      $
      \A = \left(
      \begin{array}{ccc}
        1 & 2 & 3\\
        2 & 2 & 1\\
        3 & 4 & 3
      \end{array}
      \right)
      $
      的逆阵。
    \end{exampleblock}
    \jiename
    $|\A| = 2$,故$\A$可逆。\pause 计算$\A$的余子式
    $$
    \begin{array}{lll}
      M_{11}=2 & M_{12}=3 & M_{13}=2\\
      M_{21}=-6 & M_{22}=-6 & M_{23}=-2\\
      M_{31}=-4 & M_{32}=-5 & M_{33}=-2
    \end{array}
    $$
    \pause 
    $$
    \mathrm{coef} \A = \left(
    \begin{array}{rrr}
      M_{11} & -M_{12} &  M_{13}\\
     -M_{21} &  M_{22} & -M_{23}\\
      M_{31} & -M_{32} &  M_{33}
    \end{array}
    \right) \pause = \left(
    \begin{array}{rrr}
      2 & -3 &  2\\
      6 & -6 &  2 \\
      -4 & 5 & -2
    \end{array}
    \right)
    $$
    \pause 
    $$
    \A^* =  \left(
    \begin{array}{rrr}
      2 & 6 &  -4\\
      -3 & -6 & 5 \\
      2 & 2 & -2
    \end{array}
    \right)
    $$
    \pause 
    故
    $$
    \A^{-1} = \frac1{|A|}\A^* = \left(
    \begin{array}{rrr}
      1 & 3 &  -2\\
      -\frac32 & -3 & \frac52 \\
      1 & 1 & -1
    \end{array}
    \right)
    $$
  \end{footnotesize}
\end{frame}


\begin{frame}
  \begin{footnotesize}
    \begin{exampleblock}{例3}
      设方阵$\A$满足方程
      $$
      \A^2 - 3\A - 10 \II = \zero,
      $$
      证明:$\A, \A-4\II$都可逆,并求它们的逆矩阵。      
    \end{exampleblock}
    \pause
    \proofname
    $$
    \A^2-3\A-10\II=\zero ~\Rightarrow~ \A(\A-3\II) = 10\II \pause
    ~\Rightarrow~ \A\left[\frac1{10}(\A-3\II)\right] = \II
    $$ \pause 
    故$\A$可逆,且\red{$\disp \A^{-1} = \frac1{10}(\A-3\II)$}.
    \pause
    $$
    \A^2-3\A-10\II=\zero ~\Rightarrow~ (\A+\II)(\A-4\II) = 6\II \pause
    ~\Rightarrow~ \frac1{6}(\A+\II)(\A-4\II) = \II
    $$ \pause    
    故$\A-4\II$可逆,且\red{$\disp (\A-4\II)^{-1} = \frac1{6}(\A+\II)$}.
  \end{footnotesize}
\end{frame}


\begin{frame}
  \begin{footnotesize}
    \begin{exampleblock}{例4}
      证明:可逆对称矩阵的逆矩阵仍为对称矩阵;可逆反对称矩阵的逆矩阵仍为反对称矩阵。
    \end{exampleblock}
  \end{footnotesize}
\end{frame}


\begin{frame}
  \begin{footnotesize}
    \begin{exampleblock}{例5}
      设$\A=(a_{ij})_{n\times n}$为非零实矩阵,证明:若$\A^*=\A^T$,则$\A$可逆。
    \end{exampleblock}
    \proofname
    欲证$\A$可逆,只需证$|\A|\ne 0$。
    \vspace{0.2cm}
    \pause
    
    由$\A^* = \A^T$及$\A^*$的定义可知,$\A$的元素$a_{ij}$等于自身的代数余子式$A_{ij}$。
    \pause
    再根据行列式的按行展开定义,有
    $$
    |\A| = \sum_{j=1}^n a_{ij} A_{ij} = \sum_{j=1}^n a_{ij}^2.
    $$
    \pause 
    由于$\A$为非零实矩阵,故$|\A|\ne 0$,即$\A$可逆。
  \end{footnotesize}
\end{frame}


\begin{frame}
  \begin{footnotesize}
    \begin{exampleblock}{例6}
      设$\A$可逆,且$\A^*\B = \A^{-1}+\B$,证明$\B$可逆,当$\A=\left(
      \begin{array}{ccc}
        2 & 6 & 0 \\
        0 & 2 & 6\\
        0 & 0 & 2
      \end{array}
      \right)$时,求$\B$.
    \end{exampleblock}
    \pause 
    \jiename
    $$
    \A^*\B = \A^{-1}+\B  \Rightarrow (\A^*-\II)\B = \A^{-1}
    \Rightarrow |\A^*-\II|\cdot |\B| = |\A^{-1}|\ne 0 
    $$
    故$\B$与$\A^*-\II$可逆。
    \pause 
    $$
    \B = (\A^*-\II)^{-1} \A^{-1} = [\A(\A^*-\II)]^{-1} = (\A\A^*-\A)^{-1} = (|\A|\II-\A)^{-1}.
    $$
    其中
    $$
    |\A|\II-\A = \left(
    \begin{array}{ccc}
      8 & &\\
      & 8 &\\
      & & 8
    \end{array}
    \right) - \left(
    \begin{array}{ccc}
      2 & 6 & 0 \\
      0 & 2 & 6\\
      0 & 0 & 2
    \end{array}
    \right) = 6 \left(
    \begin{array}{rrr}
      1 & -1 & 0 \\
      0 &  1 & -1\\
      0 & 0 & 1
    \end{array}
    \right)
    $$
    \pause 
    易算得
    $$
    \B = \frac16 \left(
    \begin{array}{rrr}
      1 &  1 & 1 \\
      0 &  1 & 1\\
      0 & 0 & 1
    \end{array}
    \right)
    $$
  \end{footnotesize}
\end{frame}


\begin{frame}
  \begin{footnotesize}
    \begin{exampleblock}{例7}
      设$\A,~\B$均为$n$阶可逆矩阵,证明:
      \begin{itemize}
      \item[(1)] $(\A\B)^*=\B^*\A^*$\\[0.2cm]
      \item[(2)] $(\A^*)^*=|\A|^{n-2}\A$ 
      \end{itemize}
    \end{exampleblock}
    \pause 
    \begin{block}{知识点}
      $$
      \red{\A^{-1}=\frac1{|\A|}\A^* ~~\Rightarrow~~ \A^*=|\A|\A^{-1}}
      $$
    \end{block}
    \pause 
    \proofname
    (1) 由$|\A\B| = |\A||\B| \ne 0$可知$\A\B$可逆,且有$(\A\B)(\A\B)^*=|\A\B|\II$。故
    $$
    \begin{array}{cl}
      (\A\B)^* &  \pause=|\A\B| (\A\B)^{-1}
      \pause= |\A||\B| \B^{-1}\A^{-1} \\[0.2cm]
      & \disp \pause= |\B| \B^{-1} |\A| \A^{-1}   \pause= \B^*\A^*.      
    \end{array}
    $$
    \pause

    (2) 由$(\A^*)^* \A^* = |\A^*|\II$,得\pause 
    $$(\A^*)^* \red{|\A|\A^{-1}} = |\A|^{n-1}\II$$ \pause 
    两边同时右乘$\A$得
    $$
    (\A^*)^*=|\A|^{n-2}\A.
    $$
  \end{footnotesize}
\end{frame}


\begin{frame}
  \begin{footnotesize}
    \begin{exampleblock}{例8}
      设$\PP = \left(
      \begin{array}{cc}
        1 & 2\\
        1 & 4
      \end{array}
      \right), ~~ \LLambda=\left(
      \begin{array}{cc}
        1 & \\
         & 2
      \end{array}
      \right), ~~ \A\PP=\PP\LLambda$,求$\A^n$.
    \end{exampleblock}
    \pause
    \jiename
    $$
    |\PP|=2, \quad \PP^{-1} = \frac12 \left(
    \begin{array}{rr}
      4 & -2\\
      -1 & 1
    \end{array}
    \right).
    $$
    \pause
    $$
    \A = \PP\LLambda\PP^{-1}, \quad \pause
    \A^2 = \PP\LLambda\PP^{-1}\PP\LLambda\PP^{-1} = \PP\LLambda^2\PP^{-1}, \quad \pause
    \cd, \quad \pause
    \A^n = \PP\LLambda^n\PP^{-1}.
    $$
    \pause 
    $$
    \LLambda^n = \left(
    \begin{array}{cc}
      1 & \\
      & 2^n
    \end{array}
    \right).
    $$
    \pause 
    $$
    \A^n =  \left(
    \begin{array}{cc}
      1 & 2\\
      1 & 4
    \end{array}
    \right) \cdot \left(
    \begin{array}{cc}
      1 & \\
      & 2^n
    \end{array}
    \right) \cdot \frac12 \left(
    \begin{array}{rr}
      4 & -2\\
      -1 & 1
    \end{array}
    \right) = \left(
    \begin{array}{cc}
      2-2^n & 2^n-1\\
      2-2^{n+1} & 2^{n+1}-1
    \end{array}
    \right).
    $$
  \end{footnotesize}
\end{frame}


\begin{frame}
  \begin{footnotesize}
    \begin{block}{结论}
      令
      $$
      \varphi(\A) = a_0 \II + a_1 \A + \cd + a_m \A^m.
      $$
      \begin{itemize}
      \item[(i)]
        若$\A = \PP \LLambda \PP^{-1}$,则$\A^k = \PP \LLambda^k \PP^{-1}$,从而
        $$
        \begin{array}{rcl}
          \varphi(\A) &=& a_0 \II + a_1 \A + \cd + a_m \A^m \\[0.2cm]
          &=& \PP a_0 \II \PP^{-1} + \PP a_1 \LLambda\PP^{-1} + \cd + \PP a_m \LLambda^m \PP^{-1} \\[0.2cm]
          &=& \PP \varphi(\LLambda) \PP^{-1}.
        \end{array}
        $$
      \end{itemize}
    \end{block}
  \end{footnotesize}
\end{frame}


\begin{frame}
  \begin{scriptsize}
    \begin{block}{结论}
      令
      $$
      \varphi(\A) = a_0 \II + a_1 \A + \cd + a_m \A^m.
      $$
      \begin{itemize}
      \item[(ii)] 若$\LLambda=\mathrm{diag}(\lambda_1,\lambda_2,\cd,\lambda_n)$为对角阵,则$\LLambda^k=\mathrm{diag}(\lambda_1^k,\lambda_2^k,\cd,\lambda_n^k)$,从而
        $$
        \begin{array}{l}
          \varphi(\LLambda) = a_0 \II + a_1 \LLambda + \cd + a_m \LLambda^m \\[0.2cm]
          =  a_0 \left(
          \begin{array}{cccc}
            1 & & &\\
            & 1 & & \\
            & & \dd & \\
            & & & 1
          \end{array}
          \right)
          + a_1 \left(
          \begin{array}{cccc}
            \lambda_1 & & &\\
            & \lambda_2 & & \\
            & & \dd & \\
            & & & \lambda_n
          \end{array}
          \right) + \cd +  a_m \left(
          \begin{array}{cccc}
            \lambda_1^m & & &\\
            & \lambda_2^m & & \\
            & & \dd & \\
            & & & \lambda_n^m
          \end{array}
          \right)  \\[0.6cm]
          =\left(
          \begin{array}{cccc}
            \varphi(\lambda_1) & & &\\
            & \varphi(\lambda_2) & & \\
            & & \dd & \\
            & & & \varphi(\lambda_n)
          \end{array}
          \right)
        \end{array}
        $$
      \end{itemize}
    \end{block}
  \end{scriptsize}
\end{frame}
