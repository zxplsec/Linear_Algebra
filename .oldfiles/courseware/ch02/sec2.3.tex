\section{矩阵的转置、对称矩阵}

\begin{frame}
  \begin{footnotesize}
    \begin{block}{转置矩阵}
      把一个$m\times n$矩阵
      $$
      \A = \left(
      \begin{array}{cccc}
        a_{11} & a_{12} & \cd & a_{1n} \\
        a_{21} & a_{22} & \cd & a_{2n} \\
        \vd   & \vd &  & \vd \\
        a_{m1} & a_{m2} & \cd & a_{mn} 
      \end{array}
      \right)
      $$
      的行列互换得到的一个$n\times m$矩阵,称之为$\A$的\red{转置矩阵},记为$\A^T$或$\A^\prime$,即
      $$
      \A^\prime = \left(
      \begin{array}{cccc}
        a_{11} & a_{21} & \cd & a_{m1} \\
        a_{12} & a_{22} & \cd & a_{m2} \\
        \vd   & \vd &  & \vd \\
        a_{1n} & a_{2n} & \cd & a_{mn} 
      \end{array}
      \right).
      $$
      
    \end{block}
    
  \end{footnotesize}
\end{frame}


\begin{frame}
  \begin{footnotesize}
    \begin{block}{矩阵转置的运算律}
      \begin{itemize}
      \item[(i)] $(\A^T)^T=\A$\\[0.2cm]
      \item[(ii)] $(\A+\B)^T=\A^T+\B^T$\\[0.2cm]
      \item[(iii)] $(k\A)^T= k\A^T$\\[0.2cm]
      \item[(iv)] $(\A\B)^T=\B^T\A^T$
      \end{itemize}
    \end{block}
    \pause
    \proofname
    只证(iv)。 设$\A=(a_{ij})_{m\times n}, \B=(b_{ij})_{n\times s}, \A^T=(a_{ij}^T)_{n\times m}, \B^T=(b_{ij}^T)_{s\times n}$,
    注意到
    $$a_{ij} = a_{ji}^T, b_{ij} = b_{ji}^T,$$ \pause 
    有
    $$
    (\B^T\A^T)_{ji} = \sum_{k=1}^n b_{jk}^Ta_{ki}^T \pause = \sum_{k=1}^n a_{ik}b_{kj} \pause = (\A\B)_{ij} \pause = (\A\B)_{ji}^T,
    $$\pause 
    于是$(\A\B)^T=\B^T\A^T$.
  \end{footnotesize}
\end{frame}


\begin{frame}
  \begin{footnotesize}
    \begin{block}{对称矩阵、反对称矩阵}
      设
      $$
      \A = \left(
      \begin{array}{cccc}
        a_{11} & a_{12} & \cd & a_{1n} \\
        a_{21} & a_{22} & \cd & a_{2n} \\
        \vd   & \vd &  & \vd \\
        a_{n1} & a_{n2} & \cd & a_{nn} 
      \end{array}
      \right)
      $$
      是一个$n$阶矩阵。
      \begin{itemize}
      \item[1]
        如果
        $$
        a_{ij} = a_{ji},
        $$
        则称$\A$为\red{对称矩阵};
      \item[2]
        如果
        $$
        a_{ij} = -a_{ji},
        $$
        则称$\A$为\red{反对称矩阵}。
      \end{itemize}      
    \end{block}
  \end{footnotesize}
\end{frame}


\begin{frame}
  \begin{footnotesize}
    \begin{block}{注}
      \begin{itemize}
      \item[1] $\A$为对称矩阵的充分必要条件是$\A^T=\A$;\\[0.2cm] \pause 
      \item[2] $\A$为反对称矩阵的充分必要条件是$\A^T=-\A$;\\[0.2cm] \pause
      \item[3] 反对称矩阵的主对角元全为零。\\[0.2cm] \pause 
      \item[4] 奇数阶反对称矩阵的行列式为零。\\[0.2cm] \pause
      \item[5] 任何一个方阵都可表示成一个对称矩阵与一个反对称矩阵的和。\\[0.2cm] \pause
      \item[]  设$\A$为一$n$阶方阵,则
        $$
        \A = \frac{\A+\A^T}2 + \frac{\A-\A^T}2
        $$
        容易验证$\frac{\A+\A^T}2$为对称阵,$\frac{\A-\A^T}2$为反对称阵。 \\[0.2cm] \pause
      \item[6] 对称矩阵的乘积不一定为对称矩阵。\\[0.2cm] \pause
      \item[]  \red{若$\A$与$\B$均为对称矩阵,则$\A\B$对称的充分必要条件是$\A\B$可交换。}
      \end{itemize}
    \end{block}

  \end{footnotesize}
\end{frame}


\begin{frame}
  \begin{footnotesize}
    \begin{block}{例1}
      设$\A$是一个$m\times n$矩阵,则$\A^T\A$和$\A\A^T$都是对称矩阵。      
    \end{block}
    \pause 
    \proofname
    $$
    \begin{array}{l}
      (\A^T\A)^T \pause = \A^T(\A^T)^T \pause = \A^T \A\\[0.3cm] \pause 
      (\A\A^T)^T \pause = (\A^T)^T\A^T \pause = \A \A^T
    \end{array}
    $$
  \end{footnotesize}
\end{frame}


\begin{frame}
  \begin{footnotesize}
    \begin{block}{例2}
      设$\A$为$n$阶反对称矩阵,$\B$为$n$阶对称矩阵,则$\A\B+\B\A$为$n$阶反对称矩阵。
    \end{block}
    \pause
    \proofname
    $$
    \begin{array}{rl}
      (\A\B+\B\A)^T & \pause =(\A\B)^T+(\B\A)^T \pause = \B^T\A^T+\A^T\B^T \\[0.3cm]
      & \pause
      = \B(-\A) + (-\A^T)\B \pause = - (\A\B+\B\A).      
    \end{array}
    $$
  \end{footnotesize}
\end{frame}
