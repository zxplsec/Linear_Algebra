\section{习题}

\begin{frame}
  \begin{footnotesize}
    \begin{exampleblock}{28}
      求与$\A=\left(
      \begin{array}{rrr}
        1&0&0\\
        0&1&2\\
        0&1&-2
      \end{array}
      \right)$可交换的全体三阶矩阵。
    \end{exampleblock}
  \end{footnotesize}
\end{frame}



\begin{frame}
  \begin{footnotesize}
    \begin{exampleblock}{29}
      已知$\A$是对角元互不相等的$n$阶对角矩阵,即
      $$
      \A=\left(
      \begin{array}{cccc}
        a_1&&&\\
        &a_2&&\\
        &&\dd&\\
        &&&a_n
      \end{array}
      \right)
      $$
      当$i\ne j$时,$a_i\ne a_j$。证明:与$\A$可交换的矩阵必是对角矩阵。
    \end{exampleblock}
  \end{footnotesize}
\end{frame}



\begin{frame}
  \begin{footnotesize}
    \begin{exampleblock}{30}
      证明:两个$n$阶下三角矩阵的乘积仍是下三角矩阵。
    \end{exampleblock}
  \end{footnotesize}
\end{frame}



\begin{frame}
  \begin{footnotesize}
    \begin{exampleblock}{31}
      证明:若$\A$是对角元全为零的上三角矩阵,则$\A^2$也是主对角元全为零的上三角矩阵。
    \end{exampleblock}
  \end{footnotesize}
\end{frame}



\begin{frame}
  \begin{footnotesize}
    \begin{exampleblock}{32}
      证明:对角元全为1的上三角矩阵的乘积,仍是主对角元全为1的上三角矩阵。
    \end{exampleblock}
  \end{footnotesize}
\end{frame}


\begin{frame}
  \begin{footnotesize}
    \begin{exampleblock}{33}
      设$\A=\left(
      \begin{array}{rrr}
        5&-2&1\\
        3&4&-1
      \end{array}
      \right),~~\B=\left(
      \begin{array}{rrr}
        -3&2&0\\
        -2&0&1
      \end{array}
      \right)$,计算$\A\B^T,~~\B^T\A,~~\A^T\A,~~\B\B^T+\A\B^T$。
    \end{exampleblock}
  \end{footnotesize}
\end{frame}



\begin{frame}
  \begin{footnotesize}
    \begin{exampleblock}{34}
      证明:$(\A_1\A_2\cd\A_k)^T=\A_k^T\cd\A_2^T\A_1^T$。
    \end{exampleblock}
  \end{footnotesize}
\end{frame}



\begin{frame}
  \begin{footnotesize}
    \begin{exampleblock}{35}
      证明:若$\A$和$\B$都是$n$阶对称矩阵,则$\A+\B,~~\A-2\B$也是对称矩阵。
    \end{exampleblock}
  \end{footnotesize}
\end{frame}



\begin{frame}
  \begin{footnotesize}
    \begin{exampleblock}{36}
      对于任意的$n$阶矩阵$\A$,证明:
      \begin{itemize}
      \item[(1)]$\A+\A^T$是对称矩阵,$\A-\A^T$是反对称矩阵。
      \item[(2)]$\A$可表示对称矩阵和反对称矩阵之和。
      \end{itemize}
    \end{exampleblock}
  \end{footnotesize}
\end{frame}



\begin{frame}
  \begin{footnotesize}
    \begin{exampleblock}{37}
      证明:若$\A$和$\B$都是$n$阶对称矩阵,则$\A\B$是对称矩阵的充要条件是$\A$与$\B$可交换。
    \end{exampleblock}
  \end{footnotesize}
\end{frame}


\begin{frame}
  \begin{footnotesize}
    \begin{exampleblock}{38}
      设$\A$是实对称矩阵,且$\A^2=\zero$,证明$\A=\zero$.
    \end{exampleblock}
  \end{footnotesize}
\end{frame}



\begin{frame}
  \begin{footnotesize}
    \begin{exampleblock}{39}
      已知$\A$是一个$n$阶对称矩阵,$\B$是一个$n$阶反对称矩阵。
      \begin{itemize}
      \item[(1)]问$\A^k,~~\B^k$是否为对称或反对称矩阵?
      \item[(2)]证明:$\A\B+\B\A$是一个反对称矩阵。
      \end{itemize}
    \end{exampleblock}
  \end{footnotesize}
\end{frame}



\begin{frame}
  \begin{footnotesize}
    \begin{exampleblock}{40(求逆矩阵)}
      \begin{itemize}
      \item[(1)]
        $$
        \left(
        \begin{array}{rr}
          8&-4\\
          -5&3
        \end{array}
        \right)
        $$
      \end{itemize}
    \end{exampleblock}
  \end{footnotesize}
\end{frame}

\begin{frame}
  \begin{footnotesize}
    \begin{exampleblock}{40(求逆矩阵)}
      \begin{itemize}
      \item[(2)]
        $$
        \left(
        \begin{array}{rr}
          \cos\theta&\sin\theta\\
          -\sin\theta&\cos\theta
        \end{array}
        \right)
        $$
      \end{itemize}
    \end{exampleblock}
  \end{footnotesize}
\end{frame}


\begin{frame}
  \begin{footnotesize}
    \begin{exampleblock}{40(求逆矩阵)}
      \begin{itemize}
      \item[(3)]
        $$
        \left(
        \begin{array}{rrr}
          1&2&-2\\
          2&1&-2\\
          2&-2&1
        \end{array}
        \right)
        $$
      \end{itemize}
    \end{exampleblock}
  \end{footnotesize}
\end{frame}

\begin{frame}
  \begin{footnotesize}
    \begin{exampleblock}{40(求逆矩阵)}
      \begin{itemize}
      \item[(4)]
        $$
        \left(
        \begin{array}{rrr}
          2&3&-1\\
          1&2&0\\
          -1&2&-2
        \end{array}
        \right)
        $$
      \end{itemize}
    \end{exampleblock}
  \end{footnotesize}
\end{frame}

\begin{frame}
  \begin{footnotesize}
    \begin{exampleblock}{40(求逆矩阵)}
      \begin{itemize}
      \item[(5)]
        $$
        \left(
        \begin{array}{rrr}
          1&0&0\\
          1&1&0\\
          1&1&1
        \end{array}
        \right)
        $$
      \end{itemize}
    \end{exampleblock}
  \end{footnotesize}
\end{frame}

\begin{frame}
  \begin{footnotesize}
    \begin{exampleblock}{40(求逆矩阵)}
      \begin{itemize}
      \item[(6)]
        $$
        \left(
        \begin{array}{rrrr}
          1&1&0&0\\
          0&1&1&0\\
          0&0&1&1\\
          0&0&0&1
        \end{array}
        \right)
        $$
      \end{itemize}
    \end{exampleblock}
  \end{footnotesize}
\end{frame}


\begin{frame}
  \begin{footnotesize}
    \begin{exampleblock}{41(解矩阵方程)}
      \begin{itemize}
      \item[(2)]
        $$
        \left(
        \begin{array}{rrr}
          2&3&-1\\
          1&2&0\\
          -1&2&-2
        \end{array}
        \right)\X=
        \left(
        \begin{array}{rr}
          1&1\\
          -1&0
        \end{array}
        \right)        
        $$
      \end{itemize}
    \end{exampleblock}
  \end{footnotesize}
\end{frame}

\begin{frame}
  \begin{footnotesize}
    \begin{exampleblock}{41(解矩阵方程)}
      \begin{itemize}
      \item[(1)]
        $$
        \left(
        \begin{array}{rr}
          2&5\\
          1&3
        \end{array}
        \right)\B=
        \left(
        \begin{array}{rr}
          1&1\\
          -1&0
        \end{array}
        \right)        
        $$
      \end{itemize}
    \end{exampleblock}
  \end{footnotesize}
\end{frame}


\begin{frame}
  \begin{footnotesize}
    \begin{exampleblock}{41(解矩阵方程)}
      \begin{itemize}
      \item[(1)]
        $$
        \left(
        \begin{array}{rr}
          2&5\\
          1&3
        \end{array}
        \right)\B=
        \left(
        \begin{array}{rr}
          1&1\\
          -1&0
        \end{array}
        \right)        
        $$
      \end{itemize}
    \end{exampleblock}
  \end{footnotesize}
\end{frame}



\begin{frame}
  \begin{footnotesize}
    \begin{exampleblock}{28}
      
    \end{exampleblock}
  \end{footnotesize}
\end{frame}


\begin{frame}
  \begin{footnotesize}
    \begin{exampleblock}{28}
      
    \end{exampleblock}
  \end{footnotesize}
\end{frame}



\begin{frame}
  \begin{footnotesize}
    \begin{exampleblock}{28}
      
    \end{exampleblock}
  \end{footnotesize}
\end{frame}



\begin{frame}
  \begin{footnotesize}
    \begin{exampleblock}{28}
      
    \end{exampleblock}
  \end{footnotesize}
\end{frame}



\begin{frame}
  \begin{footnotesize}
    \begin{exampleblock}{28}
      
    \end{exampleblock}
  \end{footnotesize}
\end{frame}



\begin{frame}
  \begin{footnotesize}
    \begin{exampleblock}{28}
      
    \end{exampleblock}
  \end{footnotesize}
\end{frame}


\begin{frame}
  \begin{footnotesize}
    \begin{exampleblock}{28}
      
    \end{exampleblock}
  \end{footnotesize}
\end{frame}



\begin{frame}
  \begin{footnotesize}
    \begin{exampleblock}{28}
      
    \end{exampleblock}
  \end{footnotesize}
\end{frame}



\begin{frame}
  \begin{footnotesize}
    \begin{exampleblock}{28}
      
    \end{exampleblock}
  \end{footnotesize}
\end{frame}



\begin{frame}
  \begin{footnotesize}
    \begin{exampleblock}{28}
      
    \end{exampleblock}
  \end{footnotesize}
\end{frame}



\begin{frame}
  \begin{footnotesize}
    \begin{exampleblock}{28}
      
    \end{exampleblock}
  \end{footnotesize}
\end{frame}


\begin{frame}
  \begin{footnotesize}
    \begin{exampleblock}{28}
      
    \end{exampleblock}
  \end{footnotesize}
\end{frame}



\begin{frame}
  \begin{footnotesize}
    \begin{exampleblock}{28}
      
    \end{exampleblock}
  \end{footnotesize}
\end{frame}



\begin{frame}
  \begin{footnotesize}
    \begin{exampleblock}{28}
      
    \end{exampleblock}
  \end{footnotesize}
\end{frame}



\begin{frame}
  \begin{footnotesize}
    \begin{exampleblock}{28}
      
    \end{exampleblock}
  \end{footnotesize}
\end{frame}



\begin{frame}
  \begin{footnotesize}
    \begin{exampleblock}{28}
      
    \end{exampleblock}
  \end{footnotesize}
\end{frame}
