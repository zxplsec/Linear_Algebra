\section{矩阵分块}

\begin{frame}
  \begin{center}
  \begin{tikzpicture}
    \matrix(A) [matrix of math nodes,nodes in empty cells,ampersand replacement=\&,left delimiter=(,right delimiter=)] {
      a_{11} \& a_{12} \& a_{13}  \& a_{14} \\
      a_{21} \& a_{22} \& a_{23}  \& a_{24} \\
      a_{31} \& a_{32} \& a_{33}  \& a_{34} \\
    };
    \draw[red, dashed, very thick] (A-1-2.north east) -- (A-3-2.south east) (A-2-1.south west) -- (A-2-4.south east);;
  \end{tikzpicture}
    
  \end{center}
  记为
  $$
  \left(
  \begin{array}{cc}
    A_{11} &  A_{12}\\
    A_{21} &  A_{22}
  \end{array}
  \right)
  $$
  其中
  $$
  \begin{array}{ll}
  A_{11} = 
  \left(
  \begin{array}{cc}
    a_{11} &  a_{12}\\
    a_{21} &  a_{22}
  \end{array}
  \right),  &
  A_{12} = 
  \left(
  \begin{array}{cc}
    a_{13} &  a_{14}\\
    a_{23} &  a_{24}
  \end{array}
  \right)\\ [0.3cm]
  A_{21} = 
  \left(
  \begin{array}{cc}
    a_{31} &  a_{32}
  \end{array}
  \right) , &
  A_{22} = 
  \left(
  \begin{array}{cc}
    a_{33} &  a_{34}
  \end{array}
  \right)    
  \end{array}
  $$
  
\end{frame}


\begin{frame}
  \begin{overprint}
    \onslide<1->
    \begin{center}
      \blue{分块矩阵的运算}
    \end{center}
  \end{overprint}

  \begin{overprint}
    \onslide<1>
    \begin{itemize}
    \item[1] 
      加法
    \item[]
      设A, B为同型矩阵,采用相同的分块法,有
      $$
      A = \left(
      \begin{array}{ccc}
        A_{11} & \cd & A_{1r} \\
        \vd   &     & \vd   \\
        A_{s1} & \cd & A_{sr}
      \end{array}
      \right), \ \ 
      B = \left(
      \begin{array}{ccc}
        B_{11} & \cd & B_{1r} \\
        \vd   &     & \vd   \\
        B_{s1} & \cd & B_{sr}
      \end{array}
      \right),
      $$
      其中$A_{ij}$与$B_{ij}$为同型矩阵,则
      $$
      A = \left(
      \begin{array}{ccc}
        A_{11} + B_{11}  & \cd & A_{1r} + B_{1r} \\
        \vd   &     & \vd   \\
        A_{s1} + B_{s1}  & \cd & A_{sr} + B_{sr}
      \end{array}
      \right).
      $$
    \end{itemize}

    \onslide<2>
    \begin{itemize}
    \item[2] 
      数乘
    \item[]
      $$
      \lambda A = \left(
      \begin{array}{ccc}
        \lambda A_{11} & \cd & \lambda A_{1r} \\
        \vd   &     & \vd   \\
        \lambda A_{s1} & \cd & \lambda A_{sr}
      \end{array}
      \right)
      $$    
    \end{itemize}


    \onslide<3>
    \begin{itemize}
    \item[3] 
      分块矩阵的乘法
    \item[]
      设A为$m\times l$矩阵, B为$l \times n$矩阵,
      $$
      A = \left(
      \begin{array}{ccc}
        A_{11} & \cd & A_{1t} \\
        \vd   &     & \vd   \\
        A_{s1} & \cd & A_{st}
      \end{array}
      \right), \ \ 
      B = \left(
      \begin{array}{ccc}
        B_{11} & \cd & B_{1r} \\
        \vd   &     & \vd   \\
        B_{t1} & \cd & B_{tr}
      \end{array}
      \right),
      $$
      其中$A_{i1}, A_{i2}, \cd, A_{it}$的列数分别等于$B_{1j}, B_{2j}, \cd, B_{tj}$的行数,则
      $$
      A B = \left(
      \begin{array}{ccc}
        C_{11}   & \cd & C_{1r}  \\
        \vd   &     & \vd   \\
        C_{s1}   & \cd & C_{sr}
      \end{array}
      \right),
      $$
      其中
      $$
      C_{ij} = \sum_{k=1}^t A_{ik} B_{kj}.
      $$
    \end{itemize}


  \end{overprint}

\end{frame}
