\section{矩阵可对角化的条件}
\begin{frame}
  \begin{footnotesize}
    矩阵可对角化,即矩阵与对角阵相似。    
  \end{footnotesize}
\end{frame}

\begin{frame}
  \begin{footnotesize}
    \begin{block}{定理}
      $\mbox{矩阵可对角化} ~~\Longleftrightarrow~~
      \mbox{$n$阶矩阵有$n$个线性无关的特征向量}$ 
    \end{block}
    \pause\proofname
    \begin{itemize}
\item[$\Rightarrow$] 
$$
\PP^{-1}\A\PP=\Lambdabd  ~~\Longrightarrow~~
\A\PP=\PP\Lambdabd
$$\pause
将$\PP$按列分块,即
$$
\PP=(\xx_1,~\xx_2,~\cd,~\xx_n),
$$
则
$$
\A(\xx_1,~\xx_2,~\cd,~\xx_n)=(\xx_1,~\xx_2,~\cd,~\xx_n)\left(
\begin{array}{cccc}
\lambda_1&&&\\
&\lambda_2&&\\
&&\dd&\\
&&&\lambda_n
\end{array}
\right)
$$\pause
于是
$$
\A\xx_i=\lambda_i\xx_i\quad(i=1,2,\cd,n).
$$\pause
故$\xx_1,~\xx_2,~\cd,~\xx_n$是$\A$分别对应于$\lambda_1,~\lambda_2,~\cd,~\lambda_n$的特征向量。\pause由于$\PP$可逆,所以它们是线性无关的。
\end{itemize}•
  \end{footnotesize}
\end{frame}


\begin{frame}
  \begin{footnotesize}
    若$\A$与$\Lambdabd$相似,则$\Lambdabd$的主对角元都是$\A$的特征值。
    若不计$\lambda_k$的排列次序,则$\Lambdabd$是唯一的,称$\Lambdabd$为$\A$的相似标准型。
  \end{footnotesize}
\end{frame}


\begin{frame}
  \begin{footnotesize}
    \begin{block}{定理}
      $\A$的属于不同特征值的特征向量是线性无关的。
    \end{block}
    \pause\proofname
    设$\A$的$m$个互不相同的特征值为$\lambda_1,\lambda_2,\cd,\lambda_m$,其相应的特征向量为$\xx_1,~\xx_2,~\cd,~\xx_m$.
    对$m$做数学归纳法。
    \begin{itemize}
\item[$1^o$] 当$m=1$时,结论显然成立。
\item[$2^o$] 设$k$个不同特征值$\lambda_1,\lambda_2,\cd,\lambda_k$的特征向量$\xx_1,~\xx_2,~\cd,~\xx_k$。下面考虑$k+1$个不同特征值的特征向量。
\pause
设
$$
\begin{array}{rc}
& a_1\xx_1+a_2\xx_2+\cd+a_k\xx_k+a_{k+1}\xx_{k+1}=\zero\qquad(1)\\[0.1cm]\pause
\Longrightarrow&
\A(a_1\xx_1+a_2\xx_2+\cd+a_k\xx_k+a_{k+1}\xx_{k+1})=\zero\\[0.1cm]\pause
\Longrightarrow& 
a_1\lambda_1\xx_1+a_2\lambda_2\xx_2+\cd+a_k\lambda_k\xx_k+a_{k+1}\lambda_{k+1}\xx_{k+1}=\zero\quad(2)\\[0.1cm]\pause
\xLongrightarrow[]{(2)-\lambda_{k+1}(1)}&
a_1(\lambda_{k+1}-\lambda_1)\xx_1+a_2(\lambda_{k+1}-\lambda_2)\xx_2+\cd+a_k(\lambda_{k+1}-\lambda_k)\xx_k=\zero\\[0.1cm]\pause
\Longrightarrow&
a_i(\lambda_{k+1}-\lambda_i)=0, ~~i=1,2,\cd,k\\[0.1cm]\pause
\Longrightarrow&
a_i=0, ~~i=1,2,\cd,k\\[0.1cm]\pause
\Longrightarrow&
a_{k+1}\xx_{k+1}=0\\[0.1cm]\pause
\Longrightarrow&
a_{k+1}=0\\[0.1cm]\pause
\Longrightarrow&
 \xx_1,~\xx_2,~\cd,~\xx_k,~~\xx_{k+1}\mbox{线性无关}
\end{array}
$$

\end{itemize}•
  \end{footnotesize}
\end{frame}

\begin{frame}
  \begin{footnotesize}
    \begin{block}{推论}
      若$\A$有$n$个互不相同的特征值,则$\A$与对角阵相似。
    \end{block}
  \end{footnotesize}
\end{frame}

\begin{frame}
  \begin{footnotesize}
    \begin{exampleblock}{例1}
      设实对称矩阵
      $$
      \A=\left(
      \begin{array}{rrrr}
        1&-1&-1&-1\\
        -1&1&-1&-1\\
        -1&-1&1&-1\\
        -1&-1&-1&1
      \end{array}
      \right)
      $$
      问$\A$是否可对角化?若可对角化,求对角阵$\Lambdabd$及可逆矩阵$\PP$使得$\PP^{-1}\A\PP=\Lambdabd$,再求$\A^k$。
    \end{exampleblock}
  \end{footnotesize}
\end{frame}

\begin{frame}
  \begin{footnotesize}
    \begin{exampleblock}{例2}
      设$\A=(a_{ij})_{n\times n}$是主对角元全为$2$的上三角矩阵,且存在$a_{ij}\ne 0(i<j)$,问$\A$是否可对角化?
    \end{exampleblock}
  \end{footnotesize}
\end{frame}
