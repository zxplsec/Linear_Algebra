\subsection{特征值与特征向量}
\begin{frame}
  \begin{footnotesize}
    \begin{block}{定义(特征值与特征向量)}
      设$\A$为复数域$\CCC$上的$n$阶矩阵,如果存在数$\lambda\in\CCC$和非零的$n$维向量$\xx$使得
      $$
      \A\xx=\lambda\xx
      $$
      则称$\lambda$为矩阵$\A$的\blue{\underline{特征值}},$\xx$为$\A$的对应于特征值$\lambda$的\blue{\underline{特征向量}}。
    \end{block} \pause

    \begin{itemize}
    \item[(1)] 特征向量$\xx\ne\zero$;
    \item[(2)] 特征值问题是对方针而言的。 
    \end{itemize}
  \end{footnotesize}
\end{frame}


\begin{frame}
  \begin{footnotesize}
    由定义,$n$阶矩阵$\A$的特征值,就是使齐次线性方程组
    $$
    (\lambda\II-\A)\xx=\zero
    $$
    有非零解的$\lambda$值,即满足方程
    $$
    \det(\lambda\II-\A)=0
    $$
    的$\lambda$都是矩阵$\A$的特征值。
    \pause

    \begin{block}{重要结论}
      特征值$\lambda$是关于$\lambda$的多项式$\det(\lambda\II-\A)$的根。
    \end{block}
    
  \end{footnotesize}
\end{frame}

\begin{frame}
  \begin{footnotesize}
    \begin{block}{定义(特征多项式、特征矩阵、特征方程)}
      设$n$阶矩阵$\A=(a_{ij})$,则
      $$
      f(\lambda)=\det(\lambda\II-\A)
      =\left|
      \begin{array}{cccc}
        \lambda-a_{11}&-a_{12}&\cd&-a_{1n}\\[0.2cm]
        -a_{21}&\lambda-a_{22}&\cd&-a_{2n}\\[0.2cm]
        \vd&\vd&&\vd\\[0.2cm]
        -a_{n1}&-a_{n2}&\cd&\lambda-a_{nn}
      \end{array}
      \right|
      $$
      称为矩阵$\A$的特征多项式,$\lambda\II-\A$称为$\A$的特征矩阵,$\det(\lambda\II-\A)=0$称为$\A$的特征方程。
    \end{block}
    \pause
    \begin{itemize}
    \item[(1)]  $n$阶矩阵$\A$的特征多项式是$\lambda$的$n$次多项式。\pause
    \item[(2)]  特征多项式的$k$重根称为$k$重特征值。
    \end{itemize}
  \end{footnotesize}
\end{frame}

\begin{frame}
  \begin{footnotesize}
    \begin{exampleblock}{例1}
      求矩阵
      $$
      \A=\left(
      \begin{array}{rrr}
        5&-1&-1\\
        3&1&-1\\
        4&-2&1
      \end{array}
      \right)
      $$
      的特征值与特征向量。
    \end{exampleblock}
    \pause\jiename
    $$
    \begin{array}{rl}
      \det(\lambda\II-\A)&=\left|\begin{array}{rrr}
      \lambda-5&1&1\\
      -3&\lambda-1&1\\
      -4&2&\lambda-1
      \end{array}\right| = (\lambda-3)
      \left|\begin{array}{rrr}
      1&1&1\\
      0&\lambda-2&0\\
      0&1&\lambda-2
      \end{array}\right|\\[0.3in]
      &=(\lambda-3)(\lambda-2)^2=0
    \end{array}
    $$
    故$\A$的特征值为$\lambda_1=3,~\lambda_2=2\mbox{(二重特征值)}$。
  \end{footnotesize}
\end{frame}

\begin{frame}
  \begin{footnotesize}
    当$\lambda_1=3$时,由$(\lambda\II-\A)\xx=\zero$,即
    $$
    \left(\begin{array}{rrr}
      -2&1&1\\
      -3&2&1\\
      -4&2&2
      \end{array}\right)\left(
    \begin{array}{c}
      x_1\\
      x_2\\
      x_3
    \end{array}
    \right)=\left(
    \begin{array}{c}
      0\\
      0\\
      0
    \end{array}
    \right)
    $$
    得其基础解系为$\xx_1=(1,1,1)^T$,因此$k_1\xx_1$($k_1$为非零任意常数)是$\A$对应于$\lambda_1=3$的全部特征向量。
    \pause\vspace{0.1in}

    当$\lambda_1=2$时,由$(\lambda\II-\A)\xx=\zero$,即
    $$
    \left(\begin{array}{rrr}
      -3&1&1\\
      -3&1&1\\
      -4&2&1
      \end{array}\right)\left(
    \begin{array}{c}
      x_1\\
      x_2\\
      x_3
    \end{array}
    \right)=\left(
    \begin{array}{c}
      0\\
      0\\
      0
    \end{array}
    \right)
    $$
    得其基础解系为$\xx_2=(1,1,2)^T$,因此$k_2\xx_2$($k_1$为非零任意常数)是$\A$对应于$\lambda_1=2$的全部特征向量。

  \end{footnotesize}
\end{frame}

\begin{frame}
  \begin{footnotesize}
    \begin{exampleblock}{例2}
      $$
      \left(
      \begin{array}{cccc}
        a_{11}&0&\cd&0\\
        0&a_{22}&\cd&0\\
        \vd&\vd&&\vd\\
        0&0&\cd&a_{nn}
      \end{array}
      \right), \left(
      \begin{array}{cccc}
        a_{11}&a_{12}&\cd&a_{1n}\\
        0&a_{22}&\cd&a_{2n}\\
        \vd&\vd&&\vd\\
        0&0&\cd&a_{nn}
      \end{array}
      \right),\left(
      \begin{array}{cccc}
        a_{11}&0&\cd&0\\
        a_{21}&a_{22}&\cd&0\\
        \vd&\vd&&\vd\\
        a_{n1}&a_{n2}&\cd&a_{nn}
      \end{array}
      \right)
      $$
      的特征多项式为
      $$
      (\lambda-a_{11})(\lambda-a_{22})\cd(\lambda-a_{nn})
      $$
      故其$n$个特征值为$n$个对角元。
    \end{exampleblock}
  \end{footnotesize}
\end{frame}

