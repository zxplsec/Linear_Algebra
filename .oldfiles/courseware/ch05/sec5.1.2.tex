\subsection{特征值与特征值的性质}

\begin{frame}
  \begin{footnotesize}
    \begin{block}{定理1}
      若$\xx_1$和$\xx_2$都是$\A$的对应于特征值$\lambda_0$的特征向量,则$k_1\xx_1+k_2\xx_2$也是$\A$的对应于特征值$\lambda_0$的特征向量(其中$k_1,k_2$为任意常数,但$k_1\xx_1+k_2\xx_2\ne 0$)。
    \end{block}
  \end{footnotesize}
\end{frame}

\begin{frame}
  \begin{footnotesize}
    \begin{block}{定理2}
      设$n$阶矩阵$\A=(a_{ij})$的$n$个特征值为$\lambda_1,\lambda_2,\cd,\lambda_n$,则
      \begin{itemize}
      \item[(1)] $\ds \sum_{i=1}^n\lambda_i=\sum_{i=1}^na_{ii}$
      \item[(2)] $\ds \prod_{i=1}^n\lambda_i=\det(\A)$         
      \end{itemize}
    \end{block}
    \pause

    \begin{itemize}
    \item 当$\det(\A)\ne 0$,即$\A$为可逆矩阵时,其特征值全为非零数;
    \item 奇异矩阵$\A$至少有一个零特征值。      
    \end{itemize}
  \end{footnotesize}
\end{frame}

\begin{frame}
  \begin{footnotesize}
    \begin{block}{定理3}
      一个特征向量不能属于不同的特征值。
    \end{block}
    \pause \proofname
    若$\xx$是$\A$的属于特征值$\lambda_1,\lambda_2(\lambda_1\ne\lambda_2)$的特征向量,
    即有
    $$
    \A\xx=\lambda_1\xx, ~~ \A\xx=\lambda_2\xx
    ~~\Rightarrow~~ (\lambda_1-\lambda_2)\xx=\zero
    ~~\Rightarrow~~ \xx=\zero
    $$
    这与$\xx\ne\zero$矛盾。
  \end{footnotesize}
\end{frame}

\begin{frame}
  \begin{footnotesize}
    \begin{block}{性质1}
      \begin{table}
        \caption{特征值与特征向量}
        
        \begin{tabular}{|c|c|c|}\hline
          &特征值&特征向量\\\hline
          \red{$\A$}&\red{$\lambda$}&\red{$\xx$}\\ \hline 
          \hline 
          $k\A$&$k\lambda$&$\xx$\\\hline
          $\A^m$&$\lambda^m$&$\xx$\\\hline
          $\A^{-1}$&$\lambda^{-1}$&$\xx$\\\hline
        \end{tabular}
      \end{table}
    \end{block}
  \end{footnotesize}
\end{frame}

\begin{frame}
  \begin{footnotesize}
    \begin{block}{性质2}
      矩阵$\A$与$\A^T$的特征值相同。
    \end{block}
  \end{footnotesize}
\end{frame}

\begin{frame}
  \begin{footnotesize}
    \begin{block}{定理4}
      设$\A=(a_{ij})$是$n$阶矩阵,若
      \begin{itemize}
      \item[(1)] $\ds \sum_{j=1}^n|a_{ij}|<1, ~~(i=1,2,\cd,n)$
      \item[(2)] $\ds \sum_{i=1}^n|a_{ij}|<1, ~~(j=1,2,\cd,n)$
      \end{itemize}
      有一个成立,则$\A$的所有特征值的模都小于$1$。
    \end{block}
  \end{footnotesize}
\end{frame}

\begin{frame}
  \begin{footnotesize}
    \begin{exampleblock}{例1}
      设$\A=\left(
      \begin{array}{rrr}
        1&-1&1\\
        2&-2&2\\
        -1&1&-1
      \end{array}
      \right)$
      \begin{itemize}
      \item[(i)]求$\A$的特征值与特征向量
      \item[(ii)] 求可逆矩阵$\PP$,使得$\PP^{-1}\A\PP$为对角阵。 
      \end{itemize}
    \end{exampleblock}
  \end{footnotesize}
\end{frame}

