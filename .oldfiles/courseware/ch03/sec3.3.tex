\section{矩阵的秩 \quad 相抵标准形}


\begin{frame}
  \begin{footnotesize}
    \begin{block}{定义(行秩 \& 列秩)}
      \begin{itemize}
      \item
        对于矩阵$\A$,把它的每一行称为$\A$的一个\red{行向量}。
        把$\A$的行向量组的秩,称为矩阵$\A$的\red{行秩}。
      \item
        对于矩阵$\A$,把它的每一列称为$\A$的一个\red{列向量}。
        把$\A$的列向量组的秩,称为矩阵$\A$的\red{列秩}。
      \end{itemize}      
    \end{block}
    \pause 
    \vspace{0.2in}

    对于$m\times n$阶矩阵$\A$,
    \begin{itemize}
    \item $\A$的行秩$~\le~ m$;
    \item $\A$的列秩$~\le~ n$。
    \end{itemize}
  \end{footnotesize}
\end{frame}

\begin{frame}
  \begin{footnotesize}
    阶梯形矩阵
    \begin{figure}
      \centering
      \begin{tikzpicture}
        \matrix (M) [matrix of math nodes]  { 
          \A = \\
        };
        \matrix(MM) [right=.1in of M, matrix of math nodes,nodes in empty cells,
          column sep=3ex,row sep=1ex,ampersand replacement=\&,left delimiter=(,right delimiter=)] {
          a_{11} \& a_{12} \& a_{13}  \& a_{14} \& a_{15}\\
          0 \& 0 \& a_{23}  \& a_{24} \& a_{25}\\
          0 \& 0 \& 0  \& a_{34} \& a_{35}\\
          0 \& 0 \& 0  \& 0 \& 0\\
        };
        \draw[thick,red,dashed] (MM-2-1.north west)--(MM-2-2.north east)
        --(MM-3-2.north east)--(MM-3-3.north east)
        --(MM-4-3.north east)--(MM-4-5.north east);
      \end{tikzpicture}
    \end{figure}
    其中$a_{11}\ne0, a_{23}\ne 0, a_{34}\ne 0$。
    \red{验证$\A$的行秩$=3$,列秩$=3$}。
  \end{footnotesize}
\end{frame}

\begin{frame}
  \begin{footnotesize}
    把$\A$按行和列分块为
    $$
    \A = \left(
    \begin{array}{c}
      \alphabd_1\\
      \alphabd_2\\
      \alphabd_3\\
      \alphabd_4
    \end{array}
    \right), \quad \B = (\betabd_1,\betabd_2,\betabd_3,\betabd_4,\betabd_5)
    $$
    下证$\alphabd_1,\alphabd_2,\alphabd_3$线性无关,$\betabd_1,\betabd_3,\betabd_4$线性无关。
          
  \end{footnotesize}
\end{frame}

\begin{frame}
  \begin{footnotesize}
    \begin{itemize}
    \item[(1)] 设
      $$
      x_1\alphabd_1+x_2\alphabd_2+x_3\alphabd_3=\zero,
      $$
      即
      $$
      x_1(a_{11},a_{12},a_{13},a_{14},a_{15})+
      x_2(0,0,a_{23},a_{24},a_{25})+
      x_3(0,0,0,a_{34},a_{35})=(0,0,0,0,0)
      $$ \pause 
      比较第一个分量
      $$
      x_1a_{11} = 0 \Rightarrow x_1=0.
      $$ \pause 从而
      $$
      x_2(0,0,a_{23},a_{24},a_{25})+
      x_3(0,0,0,a_{34},a_{35})=(0,0,0,0,0)
      $$ \pause 
      比较第3个分量
      $$
      x_2a_{23} = 0 \Rightarrow x_2=0.
      $$ \pause  从而
      $$
      x_3(0,0,0,a_{34},a_{35})=(0,0,0,0,0)
      $$\pause 
      同理得$x_3=0$。  于是$\alphabd_1,\alphabd_2,\alphabd_3$线性无关。 
      \pause \vspace{0.1in}

      又$\alphabd_4=\zero$,而零向量可由任何向量线性表示,这里
      $$
      \zero = 0\alphabd_1+0\alphabd_2+0\alphabd_3.
      $$
      故$\alphabd_1,\alphabd_2,\alphabd_3$是向量组$\alphabd_1,\alphabd_2,\alphabd_3,\alphabd_4$的极大无关组。所以矩阵$\A$的行秩为3。
    \end{itemize}
  \end{footnotesize}
\end{frame}


\begin{frame}
  \begin{footnotesize}
    \begin{itemize}
    \item[(2)] 设
      $$
      y_1\betabd_1 + y_3\betabd_3 + y_4\betabd_4=\zero
      $$
      即
      $$
      y_1\left(
      \begin{array}{c}
        a_{11}\\
        0\\
        0\\
        0
      \end{array}
      \right) + y_3\left(
      \begin{array}{c}
        a_{13}\\
        a_{23}\\
        0\\
        0
      \end{array}
      \right) + y_4\left(
      \begin{array}{c}
        a_{14}\\
        a_{24}\\
        a_{34}\\
        0
      \end{array}
      \right) = \left(
      \begin{array}{c}
        0\\
        0\\
        0\\
        0
      \end{array}
      \right)
      $$
      比较第三个分量得$y_4=0$。从而
      $$
      y_1\left(
      \begin{array}{c}
        a_{11}\\
        0\\
        0\\
        0
      \end{array}
      \right) + y_3\left(
      \begin{array}{c}
        a_{13}\\
        a_{23}\\
        0\\
        0
      \end{array}
      \right) = \left(
      \begin{array}{c}
        0\\
        0\\
        0\\
        0
      \end{array}
      \right)
      $$比较第二个分量得$y_3=0$。从而
      $$
      y_1\left(
      \begin{array}{c}
        a_{11}\\
        0\\
        0\\
        0
      \end{array}
      \right) = \left(
      \begin{array}{c}
        0\\
        0\\
        0\\
        0
      \end{array}
      \right)
      $$ 比较第一个分量得$y_1=0$。故$\betabd_1,\betabd_3,\betabd_4$线性无关。
    \end{itemize}
  \end{footnotesize}
\end{frame}

\begin{frame}
  \begin{footnotesize}
\begin{figure}
      \centering
      \begin{tikzpicture}
        \matrix(MM) [ matrix of math nodes,nodes in empty cells,
          column sep=3ex,row sep=1ex,ampersand replacement=\&,left delimiter=(,right delimiter=)] {
          \red{a_{11}} \& a_{12} \& \red{a_{13}}  \& \red{a_{14}} \& a_{15}\\
          \red{0} \& 0 \& \red{a_{23}}  \& \red{a_{24}} \& a_{25}\\
          \red{0} \& 0 \& \red{0}  \& \red{a_{34}} \& a_{35}\\
          \red{0} \& 0 \& \red{0}  \& \red{0} \& 0\\
        };
      \end{tikzpicture}
    \end{figure}
    
\pause 
    去掉向量组
    $$B:\betabd_1,\betabd_2,\betabd_3,\betabd_4,\betabd_5$$
    的最后一个分量,
    所得的新向量记为
    $$B^*:\betabd_1^*,\betabd_2^*,\betabd_3^*,\betabd_4^*,\betabd_5^*.$$
    注意去掉的分量全为$0$,故这两个向量组的相关性是一致的。
    \pause 
    \vspace{0.1in}
    
    由$\betabd_1,\betabd_3,\betabd_4$线性无关,
    则$\betabd_1^*,\betabd_3^*,\betabd_4^*$也线性无关。
    \pause 
    \vspace{0.1in}
    

    因任意$(3+1)=4$个$3$维向量必线性相关,
    故$\betabd_1^*,\betabd_3^*,\betabd_4^*$为向量组$B^*$的极大无关组,
    \pause 即向量组$B^*$中任何一个向量都可由$\betabd_1^*,\betabd_3^*,\betabd_4^*$线性表示,
    从而向量组$B$的任何一个向量都可以由$\betabd_1,\betabd_3,\betabd_4$线性表示。
    \pause 
    \vspace{0.1in}
    
    得证$\betabd_1,\betabd_3,\betabd_4$是向量组$B$的极大无关组,即矩阵$\A$的列秩为$3$。
  \end{footnotesize}
\end{frame}

\begin{frame}
  \begin{footnotesize}
    \begin{block}{结论}
      阶梯形矩阵的行秩等于列秩,其值等于阶梯形矩阵的非零行的行数。
    \end{block}
  \end{footnotesize}
\end{frame}


\begin{frame}
  \begin{footnotesize}
    \begin{block}{定理3.3.1}
      初等行变换不改变矩阵的行秩。
    \end{block}
    \pause
    \proofname
    只需证明每做一次对换、倍乘和倍加变换,矩阵的行秩不改变。    
    \pause\vspace{0.1in}

    设$\A$是$m\times n$矩阵,进行一次初等变换所得矩阵为$\B$。记$\A$的行向量为
    $$\red{A:~\alphabd_1,\alphabd_2,\cd,\alphabd_m.}$$ \pause     
    \begin{itemize}
    \item[(1)] 证明对换变换不改变矩阵的行秩。 \pause
      $$
      \A \xlongrightarrow[]{r_i\leftrightarrow r_j}\B
      $$
      因$\B$的行向量组
      $$\red{B:~\alphabd_1,\alphabd_2,\cd,\blue{c\alphabd_j},,\cd,\blue{c\alphabd_i},\cd,\alphabd_m}$$
      与$\A$的行向量组$$\red{B:~\alphabd_1,\alphabd_2,\cd,\blue{c\alphabd_i},,\cd,\blue{c\alphabd_j},\cd,\alphabd_m}$$
      一致,故$\B$的行秩等于$\A$的行秩。 \\[0.1in] \pause
    \end{itemize}
  \end{footnotesize}
\end{frame}


\begin{frame}
  \begin{footnotesize}
    \begin{itemize}
    \item[(2)] 证明倍乘变换不改变矩阵的行秩。 \pause
      $$
      \A \xlongrightarrow[]{r_i\times c }\B,
      $$
      其中$c\ne 0$。因$\B$的行向量组
      $$\red{B:~\alphabd_1,\alphabd_2,\cd,\blue{c\alphabd_i},\cd,\alphabd_m}$$
      与$\A$的行向量组
      $$\red{A:~\alphabd_1,\alphabd_2,\cd,\blue{\alphabd_i},\cd,\alphabd_m}$$
      等价,故$\B$的行秩等于$\A$的行秩。
    \end{itemize}
  \end{footnotesize}
\end{frame}

\begin{frame}
  \begin{footnotesize}
    \begin{itemize}
    \item[(3)] 证明倍乘变换不改变矩阵的行秩。 \pause
      $$
      \A \xlongrightarrow[]{r_i+ r_j \times c  }\B,
      $$
      因$\B$的行向量组
      $$B:~\alphabd_1,\alphabd_2,\cd,\red{\alphabd_i+c\alphabd_j},\cd,\alphabd_m$$
      与$\A$的行向量组
      $$\red{A:~\alphabd_1,\alphabd_2,\cd,\blue{\alphabd_i},\cd,\alphabd_m}$$
      等价,故$\B$的行秩等于$\A$的行秩。
    \end{itemize}
  \end{footnotesize}
\end{frame}

\begin{frame}
  \begin{footnotesize}
    \begin{block}{定理3.3.2}
      初等行变换不改变矩阵的列秩。
    \end{block}
    \pause
    \proofname
    设
    $$
    \A = (\alphabd_1,\alphabd_2,\cd,\alphabd_m) \xlongrightarrow[]{\mbox{初等行变换}}
    (\betabd_1,\betabd_2,\cd,\betabd_m) = \B
    $$ \pause
    在$\A,\B$中相同位置任取某$s$个列向量:
    $$
    \alphabd_{i_1},\alphabd_{i_2},\cd,\alphabd_{i_s} \mbox{~~和~~}
    \betabd_{i_1},\betabd_{i_2},\cd,\betabd_{i_s},
    $$
    分别记为向量组$A^*$和$B^*$。\pause设
    \begin{eqnarray}
      x_1\alphabd_{i_1}+x_2\alphabd_{i_2}+\cd+x_s\alphabd_{i_s} =\zero, \label{thm3.3.2-1}\\[0.1in]
      x_1\betabd_{i_1}+x_2\betabd_{i_2}+\cd+x_s\betabd_{i_s} =\zero, \label{thm3.3.2-2}
    \end{eqnarray} \pause
    注意到方程组(\ref{thm3.3.2-2})是方程组(\ref{thm3.3.2-1})经过高斯消元法得到,故两方程组同解。\pause 即向量组$A^*$和$B^*$有完全相同的线性关系。得证$\A,\B$列秩相等。
    \pause
    \begin{block}{注}
      定理3.3.2提供了求向量组的秩与极大无关组的一种简便而有效的方法。
    \end{block}
  \end{footnotesize}
\end{frame}

\begin{frame}
  \begin{footnotesize}
    \begin{exampleblock}{例1}
      设向量组
      $$
      \alphabd_1=\left(
      \begin{array}{r}
        -1\\-1\\0\\0
      \end{array}
      \right),~~ \alphabd_2=\left(
      \begin{array}{r}
        1\\2\\1\\-1
      \end{array}
      \right),~~ \alphabd_3=\left(
      \begin{array}{r}
        0\\1\\1\\-1
      \end{array}
      \right),~~ \alphabd_4=\left(
      \begin{array}{r}
        1\\3\\2\\1
      \end{array}
      \right),~~ \alphabd_5=\left(
      \begin{array}{r}
        2\\6\\4\\-1
      \end{array}
      \right)
      $$
      求向量组的秩及其一个极大无关组,并将其余向量用该极大无关组线性表示。
    \end{exampleblock}
    \pause\jiename
    作矩阵$\A=(\alphabd_1,\alphabd_2,\alphabd_3,\alphabd_4,\alphabd_5)$,由
    $$
    \begin{array}{rl}
    \A &= \left(
    \begin{array}{rrrrr}
      -1&1&0&1&2\\
      -1&2&1&3&6\\
      0&1&1&2&4\\
      0&-1&-1&1&-1
    \end{array}
    \right) \xlongrightarrow[r_2+r_1]{ r_1\times(-1)}
    \left(
    \begin{array}{rrrrr}
      1&-1&0&-1&-2\\
      0&1&1&2&4\\
      0&1&1&2&4\\
      0&-1&-1&1&-1
    \end{array}
    \right)\\[0.4in]
    &\xlongrightarrow[r_4+r_2]{r_3- r_2}
    \left(
    \begin{array}{rrrrr}
      1&-1&0&-1&-2\\
      0&1&1&2&4\\
      0&0&0&0&0\\
      0&0&0&3&3
    \end{array}
    \right) \xlongrightarrow[r_3\leftrightarrow r_4]{r_4\div 3}
    \left(
    \begin{array}{rrrrr}
      1&-1&0&-1&-2\\
      0&1&1&2&4\\
      0&0&0&1&1\\
      0&0&0&0&0
    \end{array}
    \right)
    \end{array}
    $$
  \end{footnotesize}
\end{frame}

\begin{frame}
  \begin{footnotesize}
    $$
    \begin{array}{rl}
      & \xlongrightarrow[r_2 -2 r_3]{r_1+r_3}
    \left(
    \begin{array}{rrrrr}
      1&-1&0&0&-1\\
      0&1&1&0&2\\
      0&0&0&1&1\\
      0&0&0&0&0
    \end{array}
    \right) \xlongrightarrow[]{r_1+r_2}
    \left(
    \begin{array}{rrrrr}
      1&0&1&0&1\\
      0&1&1&0&2\\
      0&0&0&1&1\\
      0&0&0&0&0
    \end{array}
    \right) = \B
    \end{array}
    $$
    将最后一个阶梯矩阵$\B$记为$(\betabd_1,\betabd_2,\betabd_3,\betabd_4,\betabd_5)$
    \pause 
    \vspace{0.1in}

    易知$\betabd_1,\betabd_2,\betabd_4$为$\B$的列向量组的一个极大无关组,故$\alphabd_1,\alphabd_2,\alphabd_4$也为$\A$的列向量组的一个极大无关组,故
    $$
    \rr(\alphabd_1,\alphabd_2,\alphabd_3,\alphabd_4,\alphabd_5)=3,
    $$
    且
    $$
    \begin{array}{l}
      \alphabd_3=\alphabd_1+\alphabd_2,\\
      \alphabd_5=\alphabd_1+2\alphabd_2+\alphabd_4,\\
    \end{array}
    $$
    
  \end{footnotesize}
\end{frame}

\begin{frame}
  \begin{footnotesize}
    由定理3.3.1与定理3.3.2可以推出:
    \purple{初等列变换也不改变矩阵的列秩与行秩。}
    \pause
    \begin{block}{定理3.3.3}
      初等变换不改变矩阵的行秩与列秩。
    \end{block}

    \pause
    \begin{block}{定理3.3.4}
      矩阵的行秩等于其列秩。
    \end{block}
    \pause \proofname
    对$\A$做初等行变换得到阶梯矩阵$\U$,则有
    $$
    \begin{array}{rl}
      \A\mbox{的行秩}&=\U\mbox{的行秩}\\[0.1in]
      &=\U\mbox{的列秩}=\A\mbox{的列秩}
    \end{array}
    $$

  \end{footnotesize}
\end{frame}

\begin{frame}
  \begin{footnotesize}
    \begin{block}{定义(矩阵的秩)}
      矩阵的行秩或列秩的数值,称为\red{矩阵的秩}。记作
      $$
      \rr(\A)  \quad \mbox{或} \quad 
      \mathrm{R} (\A)  \quad \mbox{或} \quad
      \mathrm{rank} (\A)
      $$
    \end{block}
    \pause 
    \begin{block}{定义(满秩矩阵)}
      对于$n$阶方阵,若
      $$
      \rr(\A) = n,
      $$
      则称$\A$为\red{满秩矩阵}。
    \end{block}
  \end{footnotesize}
\end{frame}

\begin{frame}
  \begin{footnotesize}
    \begin{block}{定理3.3.5}
      对于$n$阶方阵,下列表述等价:
      \begin{itemize}
      \item[(1)] $\A$为满秩矩阵。
      \item[(2)] $\A$为可逆矩阵。
      \item[(3)] $\A$为非奇异矩阵。
      \item[(4)] $\A\ne 0$。
      \end{itemize}
    \end{block}
    \pause\proofname
    只需证明前两个表述等价。 \pause

    \begin{itemize}
    \item [\red{(1)$\Rightarrow$(2)}]    
      设$\rr(\A)=n$,记$\A$的行简化阶梯形矩阵为$\B$,则$\B$有$n$个非零行,\pause 由行简化阶梯形矩阵的结构知
    $
    \B=\II,
    $ \pause
    即存在可逆矩阵$\PP$使得
    $$
    \PP\A=\II,
    $$
    故$\A^{-1}=\PP$,即$\A$可逆。\pause
  \item [\red{(2)$\Rightarrow$(1)}]   
    若$\A$可逆,记$\A^{-1}=\PP_0$,则
    $$
    \PP_0\A=\II,
    $$ \pause
    即$\A$经过初等行变换可以得到$\II$,故$\rr(\A)=\rr(\II)=n$。
    \end{itemize}
  \end{footnotesize}
\end{frame}

\begin{frame}
  \begin{footnotesize}
    \begin{block}{子式与主子式}
      对矩阵$\A=(a_{ij})_{m\times n}$,任意挑选$k$行($i_1,i_2,\cd,i_k$行)与$k$列($j_1,j_2,\cd,j_k$列),
      其交点上的$k^2$个元素按原顺序排成的$k$阶行列式
      \begin{equation}\label{subdet}
        \left|
        \begin{array}{cccc}
          a_{i_1j_1} & a_{i_1j_2} & \cd & a_{i_1j_k}\\
          a_{i_2j_1} & a_{i_2j_2} & \cd & a_{i_2j_k}\\
          \vd & \vd && \vd\\
          a_{i_kj_1} & a_{i_kj_2} & \cd & a_{i_kj_k}\\
        \end{array}
        \right|
      \end{equation}
      称为$\A$的\red{$k$阶子行列式},简称$\A$的\red{$k$阶子式}。\pause 
      \begin{itemize}
      \item 当(\ref{subdet})等于零时,称为\red{$k$阶零子式};
      \item 当(\ref{subdet})不等于零时,称为\red{$k$阶非零子式};
      \item 当(\ref{subdet})的$j_1=i_1,~j_2=i_2,~\cd,~j_k=i_k$,称为$\A$的\red{$k$阶主子式}。
      \end{itemize}
    \end{block}
  \end{footnotesize}
\end{frame}

\begin{frame}
  \begin{footnotesize}
    \begin{block}{注}
      若$\A$存在$r$阶非零子式,而所有$r+1$阶子式(如果有)都等于零,则矩阵$\A$的非零子式的最高阶数为$r$。
    \end{block}
    \pause
    事实上,由行列式的按行展开可知,若所有$r+1$阶子式都等于零,可得到所有更高阶的子式都等于零。
  \end{footnotesize}
\end{frame}

\begin{frame}
  \begin{footnotesize}
    \begin{block}{定理3.3.6}
      $\rr(\A)=r$的充分必要条件是$\A$的非零子式的最高阶数为$r$。
    \end{block}
    \pause
    \proofname
    \begin{itemize}
    \item[$(\Rightarrow)$] 设$\rr(\A)=r$,即$\A$的行秩为$r$,不妨设$\A$的前$r$行构成的矩阵$\A_1$的行秩为$r$,
      其列秩也为$r$;不妨设$\A_1$的前$r$个列向量线性无关。\vspace{0.05in} \pause 

      由定理3.3.5可知,$\A$的左上角$r$阶子式为非零子式。\vspace{0.05in} \pause 

      又因为$\A$的任意$r+1$个行向量线性相关,所以$\A$的任意$r+1$阶子式都是零子式(\purple{因为其中有一行可由其余$r$行线性表示}),
      因此$\A$的非零子式的最高阶数为$r$。 \vspace{0.05in} \pause 

    \item[$(\Leftarrow)$] 
      不妨设$\A$的左上角$r$阶子式$|\A_r|\ne 0$,于是$\A_r$可逆,其$r$个行向量线性无关。\pause
      将它们添加分量称为$\A$的前$r$个行向量,它们也线性无关。\vspace{0.05in} \pause 

      而$\A$的任何$r+1$个行向量必线性相关(\purple{否则,$\A$中存在$r+1$阶非零子式,这与题设矛盾}),故$\A\mbox{的行秩}=\rr(\A)=r$.
    \end{itemize}
  \end{footnotesize}
\end{frame}


\begin{frame}
  \begin{footnotesize}
    \begin{block}{关于矩阵的秩的基本结论}
      \begin{itemize}
      \item[(1)]  $\red{\rr(\A)=\A\mbox{的行秩}=\A\mbox{的列秩}=\A\mbox{的非零子式的最高阶数}}$
      \item[(2)]  \red{初等变换不改变矩阵的秩}
      \end{itemize}
    \end{block}
  \end{footnotesize}
\end{frame}


\begin{frame}
  \begin{footnotesize}
    \begin{block}{性质1}
      $$
      \red{\max\{\rr(\A),~\rr(\B)\}~~\le~~ \rr(\A,~\B) ~~\le~~ \rr(\A) + \rr(\B).}
      $$
      特别地,当$\B=\bb$为非零向量时,有
      $$
      \red{\rr(\A)~~\le~~\rr(\A,~\bb)~~\le~~\rr(\A)+1.}
      $$
    \end{block}
    \pause
    $$
    \rr(\A,\bb) = \left\{
    \begin{array}{ll}
      \rr(\A) & \Longleftrightarrow~~ \bb\mbox{可以被}\A\mbox{的列向量线性表示}\\[0.1in]
      \rr(\A)+1 & \Longleftrightarrow~~ \bb\mbox{不能被}\A\mbox{的列向量线性表示}
    \end{array}
    \right.
    $$
  \end{footnotesize}
\end{frame}

\begin{frame}
  \begin{footnotesize}
    设$$\A=\left(
    \begin{array}{cc}
      1&0\\
      0&1\\
      0&0
    \end{array}
    \right),$$
    \begin{itemize}
    \item[(1)] 取$\bb=\left(
    \begin{array}{cc}
      1\\
      2\\
      0
    \end{array}
    \right)$,则
    $$
    (\A,~\bb) = \left(
    \begin{array}{ccc}
      1&0&\red{1}\\
      0&1&\red{2}\\
      0&0&\red{0}
    \end{array}
    \right) \xlongrightarrow[]{c_3-(c_1+2c_2)}
    \left(
    \begin{array}{ccc}
      1&0&\red{0}\\
      0&1&\red{0}\\
      0&0&\red{0}
    \end{array}
    \right) = (\A, \zero),
    $$
    故$\rr(\A,\bb)=\rr(\A,\zero)=\rr(\A)$。\\[0.1in] \pause 
  \item[(2)] 取$\bb=\left(
    \begin{array}{cc}
      0\\
      0\\
      1
    \end{array}
    \right)$,则
    $$
    (\A,~\bb) = \left(
    \begin{array}{ccc}
      1&0&\red{0}\\
      0&1&\red{0}\\
      0&0&\red{1}
    \end{array}
    \right),
    $$
    $\bb$不能由$\A$的列向量线性表示,故
    $\rr(\A,\bb)=\rr(\A)+1.$
    \end{itemize}
    
  \end{footnotesize}
\end{frame}


\begin{frame}
  \begin{footnotesize}
    \proofname
    \begin{itemize}
    \item
      因为$\A$的列均可由$(\A,\B)$的列线性表示,故
      $$
      \rr(\A) \le \rr(\A,\B),
      $$ \pause 
      同理
      $$
      \rr(\B) \le \rr(\A,\B),
      $$\pause 
      所以
      $$
      \max\{\rr(\A),~\rr(\B)\} \le \rr(\A,\B),
      $$ \\[0.1in]
    \item \pause 
      设$\rr(\A)=p, ~\rr(\B)=q$,%将$\A,\B$按列分块为
    %% $$
    %% \A=(\alphabd_1,~\cd,~\alphabd_n), ~~
    %% \B=(\betabd_1,~\cd,~\betabd_n).
    %% $$ \pause 
    $\A$和$\B$的列向量组的极大无关组分别为
    $$
    \alphabd_1,~\cd,~\alphabd_p \mbox{~~和~~}
    \betabd_1,~\cd,~\betabd_q. \pause 
    $$ \pause 
    显然$(\A,~\B)$的列向量组可由向量组$\alphabd_1,~\cd,~\alphabd_p,~
    \betabd_1,~\cd,~\betabd_q$线性表示,故
    $$
    \rr(\A,~\B) = (\A,~\B)\mbox{的列秩} \le \rr(\alphabd_1,~\cd,~\alphabd_p,~
    \betabd_1,~\cd,~\betabd_q) \le p+q.
    $$
    \end{itemize}
  \end{footnotesize}
\end{frame}


\begin{frame}
  \begin{footnotesize}
    \begin{block}{注}
      \begin{itemize}
      \item 不等式
        $$
        \min\{\rr(\A),~\rr(\B)\} ~~\le~~ \rr(\A,~\B)
        $$
        意味着:在$\A$的右侧添加新的列,只有可能使得秩在原来的基础上得到增加;当$\B$的列向量能被$\A$的列向量线性表示时,等号成立。\\[0.1in]\pause 
      \item 不等式
        $$
        \rr(\A,~\B) ~~\le~~ \rr(\A)+\rr(\B)
        $$
        意味着:对$(\A,~\B)$,有可能$\A$的列向量与$\B$的列向量出现线性相关,合并为$(\A,~\B)$的秩一般会比$\rr(\A)+\rr(\B)$要小。
      \end{itemize}
    \end{block}
  \end{footnotesize}
\end{frame}




\begin{frame}
  \begin{footnotesize}
     \begin{block}{性质2}
      $$
      \red{\rr(\A+\B) \le \rr(\A)+\rr(\B).}
      $$
    \end{block}
    \pause\proofname
    设$\rr(\A)=p, ~\rr(\B)=q$,
    $\A$和$\B$的列向量组的极大无关组分别为
    $$
    \alphabd_1,~\cd,~\alphabd_p \mbox{~~和~~}
    \betabd_1,~\cd,~\betabd_q. \pause 
    $$ \pause 
    显然$\A+\B$的列向量组可由向量组$\alphabd_1,~\cd,~\alphabd_p,~
    \betabd_1,~\cd,~\betabd_q$线性表示,故
    $$
    \rr(\A+\B) = \A+\B\mbox{的列秩} \le \rr(\alphabd_1,~\cd,~\alphabd_p,~
    \betabd_1,~\cd,~\betabd_q) \le p+q.
    $$
    \pause 
    \begin{block}{注}
      将矩阵$\A$和$\B$合并、相加,只可能使得秩减小。
    \end{block}
  \end{footnotesize}
\end{frame}




\begin{frame}
  \begin{footnotesize}
    \begin{block}{性质3}
      $$
      \red{\rr(\A\B) \le \min(\rr(\A),~\rr(\B)).}
      $$
    \end{block}
    \pause
    \proofname
    设$\A,\B$分别为$m\times n, n\times s$矩阵,将$\A$按列分块,则
    $$
    \A\B = (\alphabd_1,~\cd,~\alphabd_n) \left(
    \begin{array}{cccc}
      b_{11}&b_{12}&\cd&b_{1s}\\
      b_{21}&b_{22}&\cd&b_{2s}\\
      \vd&\vd&&\vd\\
      b_{n1}&b_{n2}&\cd&b_{ns}
    \end{array}
    \right).
    $$ \pause 
    由此可知,$\A\B$的列向量组可由$\alphabd_1,~\alphabd_2,~\cd,~\alphabd_n$线性表示,故
    $$
    \rr(\A\B) = \A\B\mbox{的列秩} \le \A\mbox{的列秩} = \rr(\A).
    $$
    \pause
    类似地,将$\B$按行分块,可得$$\rr(\A\B)\le \rr(\B).$$
      \end{footnotesize}
\end{frame}


\begin{frame}
  \begin{footnotesize}
    该定理告诉我们,
    \begin{block}{}
      对一个向量组进行线性组合,可能会使向量组的秩减小。
    \end{block}
  \end{footnotesize}
\end{frame}


\begin{frame}
  \begin{footnotesize}
    \begin{block}{性质4}
      设$\A$为$m\times n$矩阵,$\PP,\QQ$分别为$m$阶、$n$阶可逆矩阵,则
      $$
      \rr(\A) = \rr(\PP\A) = \rr(\A\QQ)  = \rr(\PP\A\QQ).
      $$
    \end{block}
    \pause\proofname
    \begin{itemize}
    \item[方法一] 
      可逆矩阵$\PP,~\QQ$可表示为若干个初等矩阵的乘积,而初等变换不改变矩阵的秩,故结论成立。 \pause
    \item[方法二]
      因
      $$
      \rr(\A) = \rr(\PP^{-1}(\PP\A)) \le \rr(\PP\A) \le \rr(\A)
      $$
      故
      $$
      \rr(\A) = \rr(\PP\A).
      $$
      \pause 
      同理可证其他等式。
    \end{itemize}
  \end{footnotesize}
\end{frame}


\begin{frame}
  \begin{footnotesize}
    \begin{exampleblock}{例3.3.2}
      设$\A$是$m\times n$矩阵,且$m<n$,证明:$|\A^T\A|=0$.
    \end{exampleblock}
    \pause
    \jiename
    由于$\rr(\A)=\rr(\A^T)\le \min\{m,n\}<n$,根据性质2,有
    $$
    \rr(\A^T\A) \le \min\{\rr(\A^T),~\rr(\A)\} < n,
    $$
    而$\A^T\A$为$n$阶矩阵,故$|\A^T\A|=0$。
  \end{footnotesize}
\end{frame}


\begin{frame}
  \begin{footnotesize}
    \begin{block}{矩阵的相抵}
      若矩阵$\A$经过初等变换化为$\B$(\purple{亦即存在可逆矩阵$\PP$和$\QQ$使得$\PP\A\QQ=\B$}),就称$\A$\red{相抵于}$\B$,记作$\A\cong\B$
    \end{block}
    \pause
    \begin{block}{相抵关系的性质}
      \begin{itemize}
      \item 反身性
        $$
        \A\cong\A
        $$
      \item 对称性
        $$
        \A\cong\B ~~\Rightarrow~~ \B\cong\A
        $$
      \item 传递性
        $$
        \A\cong\B,~~\B\cong\C ~~\Rightarrow~~ \A\cong\C
        $$
      \end{itemize}
    \end{block}
    
  \end{footnotesize}
\end{frame}


\begin{frame}
  \begin{footnotesize}
    \begin{block}{阶梯形矩阵}
      若矩阵$\A$满足
      \begin{itemize}
      \item[(1)] 零行在最下方;
      \item[(2)] 非零行首元的列标号随行标号的增加而严格递增,
      \end{itemize}
      则称$\A$为\red{阶梯形矩阵}。
    \end{block}
    \pause
    \begin{exampleblock}{例}
      $$
      \left(
      \begin{array}{rrrr}
        2&0&2&1\\
        0&5&2&-2\\
        0&0&3&2\\
        0&0&0&0
      \end{array}
      \right)
      $$
    \end{exampleblock}
  \end{footnotesize}
\end{frame}


\begin{frame}
  \begin{footnotesize}
    \begin{block}{阶梯形矩阵}
      若矩阵$\A$满足
      \begin{itemize}
      \item[(1)] 它是阶梯形矩阵;
      \item[(2)] 非零行首元所在的列除了非零行首元外,其余元素全为零,
      \end{itemize}
      则称$\A$为\red{行简化阶梯形矩阵}。
    \end{block}
    \pause
    \begin{exampleblock}{例}
      $$
      \left(
      \begin{array}{rrrr}
        2&0&0&1\\
        0&5&0&-2\\
        0&0&3&2\\
        0&0&0&0
      \end{array}
      \right)
      $$
    \end{exampleblock}
  \end{footnotesize}
\end{frame}

\begin{frame}
  \begin{footnotesize}
    \begin{block}{定理3.3.7}
      若$\A$为$m\times n$矩阵,且$\rr(\A)=r$,则一定存在可逆的$m$阶矩阵$\PP$和$n$阶矩阵$\QQ$使得
      $$
      \PP\A\QQ=\left(
      \begin{array}{cc}
        \II_r&\zero\\
        \zero&\zero
      \end{array}
      \right)_{m\times n} = \U.
      $$
    \end{block}
    \pause \proofname
    对$\A$做初等行变换,可将$\A$化为有$r$个非零行的行简化阶梯形矩阵,即存在初等矩阵$\PP_1,\PP_2,\cd,\PP_s$使得
    $$
    \PP_s\cd\PP_2\PP_1\A=\U_1.
    $$
    \pause
    对$\U_1$做初等列变换可将$\U_1$化为$\U$,即存在初等矩阵$\QQ_1,\QQ_2,\cd,\QQ_t$使得
    $$
    \U_1\QQ_1\QQ_2\cd\QQ_t=\U
    $$
    \pause
    记
    $$
    \purple{\PP_s\cd\PP_2\PP_1=\PP, ~~\QQ_1\QQ_2\cd\QQ_t=\QQ,}
    $$
    则有
    $$
    \PP\A\QQ=\U.
    $$
  \end{footnotesize}
\end{frame}

\begin{frame}
  \begin{footnotesize}
    \begin{block}{定义(相抵标准形)}
      设$\rr(\A_{m\times n})=r$,则矩阵
      $$
      \left(
      \begin{array}{cc}
        \II_r&\zero\\
        \zero&\zero
      \end{array}
      \right)_{m\times n} 
      $$称为$\A$的\blue{相抵标准形},简称\blue{标准形}。
    \end{block}
    \pause 
    \begin{block}{注}
      \begin{itemize}
      \item 秩相等的同型矩阵,必有相同的标准形。\\[0.1in]
      \item 两个秩相等的同型矩阵是相抵的。
      \end{itemize}
    \end{block}
  \end{footnotesize}
\end{frame}

\begin{frame}
  \begin{footnotesize}
    \begin{exampleblock}{例3}
      设$\A$为$m\times n$矩阵($m>n$),$\rr(\A)=n$,证明:存在$n\times m$矩阵$\B$使得
      $$
      \B\A=\II_n.
      $$
    \end{exampleblock}
    \pause
    \proofname
    由定理3.3.7可知,存在$m$阶可逆矩阵$\PP$与$n$阶可逆矩阵$\QQ$使得
    $$
    \PP\A\QQ=\left(
    \begin{array}{c}
      \II_n\\
      \zero_1
    \end{array}
    \right) \pause    
    ~~~\Rightarrow~~~
    \PP\A=\left(
    \begin{array}{c}
      \II_n\\
      \zero_1
    \end{array}
    \right)\QQ^{-1}=\left(
    \begin{array}{c}
      \QQ^{-1}\\
      \zero_1
    \end{array}
    \right)
    $$
    其中$\zero_1$为$(m-n)\times n$零矩阵。 \pause
    
    取
    $$
    \C=(\QQ~~\zero_2),
    $$
    其中$\zero_2$为$n\times(m-n)$阶零矩阵,\pause 则
    $$
    \C\PP\A=(\QQ~~\zero_2)\left(
    \begin{array}{c}
      \QQ^{-1}\\
      \zero_1
    \end{array}
    \right)=\QQ\QQ^{-1}+\zero_2\zero_1=\II_n.
    $$\pause 
    故存在$\B=\C\PP$使得
    $$
    \B\A=\II_n.
    $$
  \end{footnotesize}
\end{frame}


\begin{frame}
  \begin{footnotesize}
    \begin{exampleblock}{例4}
      设$\alphabd_1=(1,3,1,2), ~\alphabd_2=(2,5,3,3), ~\alphabd_3=(0,1,-1,a), ~\alphabd_4=(3,10,k,4)$,
      试求向量组$\alphabd_1,~\alphabd_2,~\alphabd_3,~\alphabd_4$的秩,并将$\alphabd_4$用$\alphabd_1,~\alphabd_2,~\alphabd_3$线性表示。
    \end{exampleblock}
    \pause \jiename
    将4个向量按列排成一个矩阵$\A$,对$\A$进行初等变换,将其化为阶梯形矩阵$\U$,即
    $$
    \A=\left(
    \begin{array}{rrrr}
    1&2&0&3\\
    3&5&1&10\\
    1&3&-1&k\\
    2&3&a&4
    \end{array}
    \right) \xlongrightarrow[]{\mbox{初等行变换}}
    \left(
    \begin{array}{rrcc}
    1&2&0&3\\
    0&-1&1&1\\
    0&0&a-1&-3\\
    0&0&0&k-2
    \end{array}
    \right)
    $$
    \pause 
    \begin{itemize}
    \item[(1)] 当$a=1$或$k=2$时,$\U$只有3个非零行,故
      $$\rr(\alphabd_1,\alphabd_2,\alphabd_3,\alphabd_4)=\rr(\A)=3. $$ 
    \item[(2)] \pause 当$a\ne1$且$k\ne2$时,
      $$\rr(\alphabd_1,\alphabd_2,\alphabd_3,\alphabd_4)=\rr(\A)=4.$$
    \end{itemize}
      \end{footnotesize}
\end{frame}


\begin{frame}
  \begin{footnotesize}
    $$
    \A=\left(
    \begin{array}{rrrr}
      1&2&0&3\\
      3&5&1&10\\
      1&3&-1&k\\
      2&3&a&4
    \end{array}
    \right) \xlongrightarrow[]{\mbox{初等行变换}}
    \left(
    \begin{array}{rrcc}
      1&2&0&3\\
      0&-1&1&1\\
      0&0&a-1&-3\\
      0&0&0&k-2
    \end{array}
    \right)
    $$
    \begin{itemize}
    \item 当$k=2$且$a\ne1$时,$\alphabd_4$可由$\alphabd_1,~\alphabd_2,~\alphabd_3$线性表示,
      且
      $$
      \alphabd_4=-\frac{1+5a}{1-a}\alphabd_1+\frac{2+a}{1-a}\alphabd_2+\frac{3}{1-a}\alphabd_3.
      $$
    \item \pause 当$k\ne2$或$a=1$时,$\alphabd_4$不能由$\alphabd_1,~\alphabd_2,~\alphabd_3$线性表示。
    \end{itemize}
  \end{footnotesize}
\end{frame}


\begin{frame}
  \begin{footnotesize}
    \begin{exampleblock}{例5}
      设
      $$
      \A=\left(
      \begin{array}{rrr}
        1&2&1\\
        2&2&-2\\
        -1&t&5\\
        1&0&-3
      \end{array}
      \right)
      $$
      已知$\rr(\A)=2$,求$t$。
    \end{exampleblock}
    \pause
    \jiename
    $$
    \A \xlongrightarrow[]{\mbox{初等行变换}} \left(
    \begin{array}{ccr}
      1&2&1\\
      0&-2&-4\\
      0&2+t&6\\
      0&0&0
    \end{array}
    \right)=\B
    $$ \pause
    由于$\rr(\B)=\rr(\A)$,故$\B$中第2、3行必须成比例,即
    $$
    \frac{-2}{2+t}=\frac{-4}6,
    $$
    即得$t=1$。
  \end{footnotesize}
\end{frame}

%% \begin{frame}
%%   \begin{footnotesize}
%%     \begin{exampleblock}{例6}
%%       已知$\rr(\B)=2$,
%%       $$
%%       \A=\left(
%%       \begin{array}{ccc}
%%         1&2&0\\
%%         0&a&1\\
%%         1&3&b
%%       \end{array}
%%       \right).
%%       $$
%%       问
%%       \begin{itemize}
%%       \item[(1)] $a,b$满足什么条件时,$\rr(\A\B)=2$;
%%       \item[(2)] $\A$与$\B$满足什么条件时,$\rr(\A\B)=1$。
%%       \end{itemize}
%%     \end{exampleblock}
%%     \jiename\pause
%%     \begin{itemize}
%%     \item[(1)] 当$\A$可逆时,$\rr(\A\B)=\rr(\B)=2$。此时,
%%       $$
%%       |\A| = ab-1\ne 0.
%%       $$
%%     \item[(2)]\pause 当$ab-1=0$时,$\A$不可逆,且$\rr(\A)=2$。\pause
%%       $\A$的列向量组线性相关,故$\A\xx=\zero$有非零解。
%%     \end{itemize}
%%   \end{footnotesize}
%% \end{frame}

%% \begin{frame}
%%   \begin{footnotesize}
    
%%   \end{footnotesize}
%% \end{frame}


%% \begin{frame}
%%   \begin{footnotesize}
    
%%   \end{footnotesize}
%% \end{frame}


%% \begin{frame}
%%   \begin{footnotesize}
    
%%   \end{footnotesize}
%% \end{frame}

%% \begin{frame}
%%   \begin{footnotesize}
    
%%   \end{footnotesize}
%% \end{frame}

%% \begin{frame}
%%   \begin{footnotesize}
    
%%   \end{footnotesize}
%% \end{frame}
