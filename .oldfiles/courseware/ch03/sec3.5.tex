\section{非齐次线性方程组有解的条件及解的结构}

\begin{frame}
  \begin{footnotesize}
    \begin{block}{定理3.5.1}
      对于非齐次线性方程组$\A\xx=\bb$,以下命题等价:
      \begin{itemize}
      \item[(i)] $\A\xx=\bb$有解;
      \item[(ii)] $\bb$可由$\A$的列向量组线性表示;
      \item[(iii)] $\rr(\A,\bb)=\rr(\A)$。
      \end{itemize}
    \end{block}
    \pause\proofname
    \begin{itemize}
    \item[(i)$\Leftrightarrow$(ii)] 记$\A=(\alphabd_1,\alphabd_2,\cd,\alphabd_n)$,则$\A\xx=\bb$等价于
      $$
      x_1\alphabd_1+x_2\alphabd_2+\cd+x_n\alphabd_n=\bb.
      $$
    \item[(ii)$\Leftrightarrow$(iii)] 若$\bb$可由$\A$的列向量组$\alphabd_1,\alphabd_2,\cd,\alphabd_n$线性表示,则$(\A,\bb)$的列向量组与$\A$的列向量组等价,故$\rr(\A,\bb)=\rr(\A)$。
      \pause \vspace{0.1in}
      
      反之,若$\rr(\A,\bb)=\rr(\A)$,则$\bb$能由向量组$\alphabd_1,\alphabd_2,\cd,\alphabd_n$线性表示,否则$\rr(\A,\bb)=\rr(\A)+1$,导致矛盾。
    \end{itemize}
  \end{footnotesize}
\end{frame}


\begin{frame}
  \begin{footnotesize}
    \begin{block}{注}
      $\rr(\A,\bb)=\rr(\A)+1$会导致矛盾方程的出现
    \end{block}
    记$\rr(\A)=r$,若$\rr(\A,\bb)=\rr(\A)+1$,则增广矩阵$(\A,\bb)$经过初等行变换所得的行阶梯形矩阵形如
    \begin{center}
    \begin{tikzpicture}
      \matrix(MM) [matrix of math nodes,nodes in empty cells,ampersand replacement=\&,left delimiter=(,right delimiter=)] {
        1\&\cd\&c_{1r}\&c_{1,r+1}\&\cd\&c_{1n}\&\&d_1\\        
        \vd\&\&\vd\&\vd\&\&\vd\&\&\vd\\
        0\&\cd\&1\&c_{r,r+1}\&\cd\&c_{rn}\&\&d_r\\
        0\&\cd\&0\&0\&\cd\&0\&\&\red{d_{r+1}}\\
        0\&\cd\&0\&0\&\cd\&0\&\&0\\        
        \vd\&\&\vd\&\vd\&\&\vd\&\&\vd\\
        0\&\cd\&0\&0\&\cd\&0\&\&0\\
      };  
      \draw[thick,dashed] (MM-1-7.north)--(MM-7-7.south);
    \end{tikzpicture}
    \end{center}
    其中$d_{r+1}\ne 0$(否则$\rr(\A,\bb)=r$)。这意味着出现了矛盾方程
    $$
    0 = \red{d_{r+1}}.
    $$    
  \end{footnotesize}
\end{frame}

\begin{frame}
  \begin{footnotesize}
    \begin{block}{推论}
      $$
      \A\xx=\bb\mbox{有唯一解} ~~\Longleftrightarrow~~
      \rr(\A,\bb)=\rr(\A)=\A\mbox{的列数}.
      $$
    \end{block}
\begin{center}
    \begin{tikzpicture}
      \matrix(MM) [matrix of math nodes,nodes in empty cells,ampersand replacement=\&,left delimiter=(,right delimiter=)] {
        1\&\cd\&0\&\&d_1\\        
        \vd\&\&\vd\&\&\vd\\
        0\&\cd\&1\&\&d_r\\
        0\&\cd\&0\&\&0\\
        0\&\cd\&0\&\&0\\
        \vd\&\&\vd\&\&\vd\\
        0\&\cd\&0\&\&0\\
      };  
      \draw[thick,dashed] (MM-1-4.north)--(MM-7-4.south);
    \end{tikzpicture}
    \end{center}
  \end{footnotesize}
\end{frame}

\begin{frame}
  \begin{footnotesize}
    \begin{block}{定理3.5.2}
      若$\xx_1,~\xx_2$是$\A\xx=\bb$的解,则$\xx_1-\xx_2$是$\A\xx=\zero$的解。
    \end{block}
    \pause
    \proofname
    $$
    \A(\xx_1-\xx_2)=\A\xx_1-\A\xx_2=\bb-\bb=\zero,
    $$
    故$\xx_1-\xx_2$是$\A\xx=\zero$的解。
  \end{footnotesize}
\end{frame}

\begin{frame}
  \begin{footnotesize}
    \begin{block}{定理3.5.3}
      若$\A\xx=\bb$有解,则其一般解(或称通解)为
      $$
      \xx=\xx_0+\bar\xx
      $$
      其中$\xx_0$是$\A\xx=\bb$的一个特解,而
      $$
      \bar\xx=k_1\xx_1+k_2\xx_2+\cd+k_p\xx_p
      $$
      为$\A\xx=\zero$的一般解。
    \end{block}
    \pause\proofname
    $$
    \A(\xx_0+\bar\xx)=\A\xx_0+\A\bar\xx=\bb ~~\Rightarrow~~
    \xx_0+\bar\xx\mbox{是}\A\xx=\bb\mbox{的解}
    $$
    设$\xx^*$是$\A\xx=\bb$的任意一个解,则$\xx^*-\xx_0$是$\A\xx=\zero$的解,而
    $$
    \xx^*=\xx_0+(\xx^*-\xx_0).
    $$
    故$\xx^*$可表示为$\xx_0+\bar\xx$的形式。
  \end{footnotesize}
\end{frame}


\begin{frame}
  \begin{footnotesize}
    非齐次线性方程组
    $$\A\xx=\bb$$
    的通解为
    $$
    k_1\xx_1+k_2\xx_2+\cd+k_p\xx_p + \red{\xx_0}
    $$
    其中$\xx_1,\xx_2,\cd,\xx_p$为$\A\xx=\zero$的基础解系,$\xx_0$为$\A\xx=\bb$的一个特解。

    \begin{block}{注}
      “$\A\xx=\bb$的通解” =  “$\A\xx=\zero$的通解” + “$\A\xx=\bb$的特解”
    \end{block}
  \end{footnotesize}
\end{frame}

\begin{frame}
  \begin{footnotesize}
    \begin{exampleblock}{例1}
      求非齐次线性方程组$\A\xx=\bb$的一般解,其中增广矩阵为
      $$
      (\A,\bb) = \left(
      \begin{array}{rrrrr}
        1&-1&-1& 1&\red{0}\\
        1&-1& 1&-3&\red{1}\\
        1&-1&-2& 3&\red{-\frac12}
      \end{array}
      \right)
      $$
    \end{exampleblock}
    \pause\jiename
    $$
    \begin{array}{rl}
        \left(
    \begin{array}{rrrrr}
      1&-1&-1& 1&\red{0}\\
      1&-1& 1&-3&\red{1}\\
      1&-1&-2& 3&\red{-\frac12}
    \end{array}
    \right)  \xlongrightarrow[r_3-r_1]{r_2-r_1} &
    \left(
    \begin{array}{rrrrr}
      1&-1&-1& 1&\red{0}\\
      0& 0& 2&-4&\red{1}\\
      0& 0&-1& 2&\red{-\frac12}
    \end{array}
    \right) \\[0.4in]
     \xlongrightarrow[r_2\div2]{r_1-r_3,r_3+\frac12r_2} &
    \left(
    \begin{array}{rrrrr}
      1&-1&-1& 1&\red{0}\\
      0& 0& 1&-2&\red{\frac12}\\
      0& 0& 0& 0&\red{0}
    \end{array}
    \right)
    \end{array}
    $$
  \end{footnotesize}
\end{frame}

\begin{frame}
  \begin{footnotesize}
    同解方程为
    $$
    \left\{
    \begin{array}{rcrcrcr}
      x_1&=&x_2&+&x_4&+&\frac12\\[0.1in]
      x_3&=&&&2x_4&+&\frac12
    \end{array}
    \right.
    $$
    亦即
    $$
    \left\{
    \begin{array}{rcrcrcr}
      x_1&=&x_2&+&x_4&+&\frac12\\[0.1in]
      x_2&=&x_2&&&&\\[0.1in]
      x_3&=&&&2x_4&+&\frac12\\[0.1in]
      x_4&=&&&x_4&&
    \end{array}
    \right.
    $$
    故通解为
    $$
    \left(
    \begin{array}{c}
      x_1\\x_2\\x_3\\x_4
    \end{array}
    \right) = c_1    \left(
    \begin{array}{c}
      1\\1\\0\\0
    \end{array}
    \right)+c_2    \left(
    \begin{array}{c}
      1\\0\\2\\1
    \end{array}
    \right)+    \left(
    \begin{array}{c}
      1/2\\0\\1/2\\0
    \end{array}
    \right) \quad c_1,c_2\in\mathbb R
    $$

  \end{footnotesize}
\end{frame}

\begin{frame}
  \begin{footnotesize}
    \begin{exampleblock}{例2(重要题型)}
      设有线性方程组
      $$
      \left\{
      \begin{array}{rrrcr}
        (1+\lambda)x_1&+x_2&+x_3&=&0\\[0.05in]
        x_1&+(1+\lambda)x_2&+x_3&=&3\\[0.05in]
        x_1&+x_2&+(1+\lambda)x_3&=&\lambda
      \end{array}
      \right.
      $$
      问$\lambda$取何值时,此方程组
      \begin{itemize}
      \item[(1)]有唯一解?
      \item[(2)]无解? 
      \item[(3)]有无穷多解? 并在有无穷多解时求其通解。
      \end{itemize}
    \end{exampleblock}
    \pause\jiename
    $$
    |\A|=\left|
    \begin{array}{ccc}
      1+\lambda&1&1\\
      1&1+\lambda&1\\
      1&1&1+\lambda
    \end{array}
    \right| = (3+\lambda)\lambda^2.
    $$
    故当$\lambda\ne0$且$\lambda\ne-3$时,有唯一解。
  \end{footnotesize}
\end{frame}


\begin{frame}
  \begin{footnotesize}
    当$\lambda=0$时,原方程组为
    $$
    \left\{
    \begin{array}{l}
      x_1+x_2+x_3=0,\\
      x_1+x_2+x_3=3,\\
      x_1+x_2+x_3=0      
    \end{array}
    \right.
    $$
    它为矛盾方程组,故无解。\pause \vspace{0.1in}

    当$\lambda=-3$时,增广矩阵为
    $$
    \left(
    \begin{array}{rrrr}
      -2&1&1&\red{0}\\
      1&-2&1&\red{3}\\
      1&1&-2&\red{-3}
    \end{array}
    \right) \xlongrightarrow[]{\mbox{初等行变换}}
    \left(
    \begin{array}{rrrr}
      1&0&-1&\red{-1}\\
      0&1&-1&\red{-2}\\
      0&0&0&\red{0}
    \end{array}
    \right)
    $$
    得同解方程组为
    $$
    \left\{
    \begin{array}{l}
      x_1=x_3-1\\[0.05in]
      x_2=x_3-2\\[0.05in]
      x_3=x_3
    \end{array}
    \right.
    $$
    通解为
    $$
    \left(
    \begin{array}{c}
      x_1\\x_2\\x_3
    \end{array}
    \right) = c\left(
    \begin{array}{c}
     1\\1\\1
    \end{array}
    \right)+\left(
    \begin{array}{r}
      -1\\-2\\0
    \end{array}
    \right) \quad c\in\mathbb R
    $$
  \end{footnotesize}
\end{frame}

\begin{frame}
  \begin{footnotesize}
    \begin{exampleblock}{例3}
      设$\etabd^*$为$\A\xx=\bb$的一个解,$\xibd_1,~\xibd_2,~\cd,~\xibd_{n-r}$为对应的齐次线性方程组的一个基础解系,证明:
      \begin{itemize}
      \item $\etabd^*,~\xibd_1,~\xibd_2,~\cd,~\xibd_{n-r}$线性无关
      \item $\etabd^*,~\etabd^*+\xibd_1,~\etabd^*+\xibd_2,~\cd,~\etabd^*+\xibd_{n-r}$线性无关
      \end{itemize}
    \end{exampleblock}
    \pause\proofname
    \begin{itemize}
    \item[(1)] 假设$\etabd^*,~\xibd_1,~\xibd_2,~\cd,~\xibd_{n-r}$线性相关,而$\xibd_1,~\xibd_2,~\cd,~\xibd_{n-r}$线性无关,故$\etabd^*$可由$\xibd_1,~\xibd_2,~\cd,~\xibd_{n-r}$线性表示,从而$\etabd^*$为$\A\xx=\zero$的解,这与$\etabd^*$为$\A\xx=\bb$的解矛盾。故假设不成立,即$\etabd^*,~\xibd_1,~\xibd_2,~\cd,~\xibd_{n-r}$线性无关。\pause 
    \item[(2)] 显然,
      $$\etabd^*,~\xibd_1,~\xibd_2,~\cd,~\xibd_{n-r}
      \mbox{~~等价于~~} 
      \etabd^*,~\etabd^*+\xibd_1,~\etabd^*+\xibd_2,~\cd,~\etabd^*+\xibd_{n-r},$$ \pause 
      由题(1)结论可知
      $$
      \rr(\etabd^*,~\etabd^*+\xibd_1,~\etabd^*+\xibd_2,~\cd,~\etabd^*+\xibd_{n-r}) = 
      \rr(\etabd^*,~\xibd_1,~\xibd_2,~\cd,~\xibd_{n-r}) = n-r+1
      $$
      从而结论成立。
    \end{itemize}
  \end{footnotesize}
\end{frame}

\begin{frame}
  \begin{footnotesize}
    \begin{exampleblock}{例3}
      设$\etabd_1,~\etabd_2,~\cd,~\etabd_s$为$\A\xx=\bb$的$s$个解,$k_1,~k_2,~\cd,~k_{s}$为实数,满足$k_1+k_2+\cd+k_s=1$。证明:
      $$
      \xx=k_1\etabd_1+k_2\etabd_2+\cd+k_s\etabd_s
      $$
      也是它的解。
    \end{exampleblock}
    \pause\proofname
    $$
    \begin{array}{rcl}
      \A(k_1\etabd_1+k_2\etabd_2+\cd+k_{s}\etabd_{s})&=&
      k_1\A\etabd_1+k_2\A\etabd_2+\cd+k_s\A\etabd_s\\[0.05in]
      &=&k_1\bb+k_2\bb+\cd+k_s\bb\\[0.05in]
      &=&\bb.
    \end{array}
    $$
  \end{footnotesize}
\end{frame}

\begin{frame}
  \begin{footnotesize}
    \begin{exampleblock}{例3}
      对于$\A\xx=\bb$,$\rr(\A)=r$,$\etabd_1,~\etabd_2,~\cd,~\etabd_{n-r+1}$为它的$n-r+1$个线性无关的解。证明它的任一解可表示为
      $$
      \xx=k_1\etabd_1+k_2\etabd_2+\cd+k_{n-r+1}\etabd_{n-r+1},
      $$
      其中$k_1+k_2+\cd+k_{n-r+1}=1$
    \end{exampleblock}
    \pause\proofname
    取向量组
    $$
    \etabd_2-\etabd_1,~~\etabd_2-\etabd_1,~~\cd,~\etabd_{n-r+1}-\etabd_1.
    $$
    下证该向量组为$\A\xx=\zero$的一个基础解系。\pause 
    $$
    (\etabd_1,~\etabd_2,~\cd,~\etabd_{n-r+1}) \xlongrightarrow[j=2,\cd,n-r+1]{c_j-c_1}
    (\etabd_1,~\etabd_2-\etabd_1,~\cd,~\etabd_{n-r+1}-\etabd_1)
    $$\pause 
    $$
    \begin{array}{rl}
      &\etabd_1,~\etabd_2,~\cd,~\etabd_{n-r+1}\mbox{线性无关}\\[0.1in] \pause 
      \Rightarrow&\etabd_1,~\etabd_2-\etabd_1,~\cd,~\etabd_{n-r+1}-\etabd_1\mbox{线性无关}\\[0.1in] \pause 
      \Rightarrow&\etabd_2-\etabd_1,~\cd,~\etabd_{n-r+1}-\etabd_1\mbox{线性无关}\\[0.1in] \pause 
      \Rightarrow& \etabd_2-\etabd_1,~\cd,~\etabd_{n-r+1}-\etabd_1\mbox{为}\A\xx=\zero
      \mbox{的基础解系}.      
    \end{array}
    $$
  \end{footnotesize}
\end{frame}

\begin{frame}
  \begin{footnotesize}
    于是$\A\xx=\bb$的任意一个解$\xx$可表示为
    $$
    \begin{array}{rl}
      & \xx = k_2(\etabd_2-\etabd_1)+\cd+k_{n-r+1}(\etabd_{n-r+1}-\etabd_1)+\red{\etabd_1}\\[0.1in] \pause
      \Rightarrow & 
      \xx = (1-k_2-\cd-k_{n+r-1})\etabd_1+k_2\etabd_2+\cd+k_{n-r+1}\etabd_{n-r+1}\\[0.1in]\pause
      \Rightarrow & 
      \xx =k_1\etabd_1+k_2\etabd_2+\cd+k_{n-r+1}\etabd_{n-r+1}
    \end{array}
    $$
  \end{footnotesize}
\end{frame}

\begin{frame}
  \begin{footnotesize}
    \begin{exampleblock}{例}
      设四元齐次线性方程组
      $$
      \mbox{I}:\left\{
      \begin{array}{l}
        x_1+x_2=0,\\
        x_2-x_4=0;
      \end{array}
      \right. \quad
      \mbox{II}:\left\{
      \begin{array}{l}
        x_1-x_2+x_3=0,\\
        x_2-x_3+x_4=0.
      \end{array}
      \right.
      $$
      求
      \begin{itemize}
      \item[(1)] 方程组I与II的基础解系
      \item[(1)] 方程组I与II的公共解        
      \end{itemize}
    \end{exampleblock}
    \pause\jiename
    \begin{itemize}
    \item[(1)]因为
      $
      \mbox{I} \Longleftrightarrow
      \left\{
      \begin{array}{rcr}
        x_1&=&-x_2\\[0.05in]
        x_4&=&x_2
      \end{array}
      \right. \Longleftrightarrow
      \left\{
      \begin{array}{rcrr}
        x_1&=&-x_2&\\[0.05in]
        x_2&=&x_2&\\[0.05in]
        x_3&=&&x_3\\[0.05in]
        x_4&=&x_2&
      \end{array}
      \right. 
      $\pause 
      
      故(I)的基础解系为
      $$
      \xibd_1=\left(
      \begin{array}{r}
        -1\\1\\0\\1
      \end{array}
      \right), \quad
      \xibd_1=\left(
      \begin{array}{r}
        0\\0\\1\\0
      \end{array}
      \right)
      $$
    \end{itemize}
  \end{footnotesize}
\end{frame}

\begin{frame}
  \begin{footnotesize}
    因为
      $
      \mbox{II} \Longleftrightarrow
      \left\{
      \begin{array}{rcrr}
        x_1&=&x_2&-x_3\\[0.05in]
        x_4&=&-x_2&+x_3
      \end{array}
      \right. \Longleftrightarrow
      \left\{
      \begin{array}{rcrr}
        x_1&=&x_2&-x_3\\[0.05in]
        x_2&=&x_2&\\[0.05in]
        x_3&=&&x_3\\[0.05in]
        x_4&=&-x_2&+x_3
      \end{array}
      \right. 
      $\pause 
      
      故(I)的基础解系为
      $$
      \xibd_1=\left(
      \begin{array}{r}
        1\\1\\0\\-1
      \end{array}
      \right), \quad
      \xibd_1=\left(
      \begin{array}{r}
        -1\\0\\1\\1
      \end{array}
      \right)
      $$
  \end{footnotesize}
\end{frame}

\begin{frame}
  \begin{footnotesize}
    \begin{itemize}
    \item[(2)] 方程I与II的公共解,即联立I和II所得新方程组的解:
      $$
      \left\{
      \begin{array}{l}
        x_1+x_2=0\\
        x_2-x_4=0\\
        x_1-x_2+x_3=0\\
        x_2-x_3+x_4=0
      \end{array}
      \right.
      $$\pause

      $$
      \begin{array}{rl}
        \left(
        \begin{array}{rrrr}
          1&1&0&0\\
          0&1&0&-1\\
          1&-1&1&0\\
          0&1&-1&1
        \end{array}
        \right) \xlongrightarrow[r_3-r_1]{r_4+r_2} & 
        \left(
        \begin{array}{rrrr}
          1&1&0&0\\
          0&1&0&-1\\
          0&-2&1&0\\
          0&2&-1&0
        \end{array}
        \right) \\[0.3in]
        \xlongrightarrow[r_3\times(-1)]{r_3+r_4}&
        \left(
        \begin{array}{rrrr}
          1&1&0&0\\
          0&1&0&-1\\
          0&2&-1&0\\
          0&0&0&0
        \end{array}
        \right)         
      \end{array}
      $$ \pause 
      即
      $$
      \left\{
      \begin{array}{rcr}
        x_1&=&-x_2\\
        x_2&=&x_2\\
        x_3&=&2x_2\\
        x_4&=&x_2
      \end{array}
      \right. \pause ~~ \Rightarrow ~~
      \xx = c\left(
      \begin{array}{r}
        -1\\1\\2\\1
      \end{array}
      \right) \quad c\in \mathbb R.
      $$
    \end{itemize}
  \end{footnotesize}
\end{frame}

