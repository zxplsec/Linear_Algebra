\section{向量组的秩及其极大线性无关组}

\begin{frame}
  \begin{footnotesize}
    \begin{block}{向量组的秩}
      向量组$\alphabd_1,\alphabd_2,\cd,\alphabd_s$中,若
      \begin{itemize}
      \item 存在$r$个\blue{\underline{线性无关}}的向量,
      \item 且其中\blue{\underline{任一向量}}可由这$r$个线性无关的向量线性表示, 
      \end{itemize}
      则数$r$称为\red{向量组的秩(rank)},记作
      $$
      \rr(\alphabd_1,\alphabd_2,\cd,\alphabd_s)=r
      $$
      或
      $$
      \mathrm{rank}(\alphabd_1,\alphabd_2,\cd,\alphabd_s)=r
      $$
      
    \end{block}
    \pause
    \begin{itemize}
    \item 若$\alphabd_1,\alphabd_2,\cd,\alphabd_s$线性无关,
      则$\rr(\alphabd_1,\alphabd_2,\cd,\alphabd_s)=s$;
    \item 只含零向量的向量组的秩为零。
    \item 只含一个非零向量的向量组的秩为1。
    \end{itemize}
    
  \end{footnotesize}
\end{frame}


\begin{frame}
  \begin{footnotesize}
    \begin{block}{定义}
      若向量组$B:~\betabd_1,\betabd_2,\cd,\betabd_t$中每个向量可由向量组$A:~\alphabd_1,\alphabd_2,\cd,\alphabd_s$线性表示,
      就称\red{向量组$B:~\betabd_1,\betabd_2,\cd,\betabd_t$可由向量组$A:~\alphabd_1,\alphabd_2,\cd,\alphabd_s$线性表示}。\pause 
      \vspace{0.1in}
      
      如果两个向量组可以互相线性表示,则称这两个向量组是\red{等价}的。
    \end{block}
    \pause 
    \vspace{0.1in}

    向量组的线性表示,具备
    \begin{itemize}
    \item \red{自反性}
    \item[]向量组自己可以由自己线性表示 \pause 
    \item \red{传递性}
    \item[] 设向量组$A$可以被向量组$B$线性表示,向量组$B$又可以被向量组$C$线性表示,
      则向量组$A$可以被向量组$C$线性表示 \pause 
    \item \red{不具备对称性}
    \item[] 向量组$A$可以被向量组$B$线性表示,不一定有向量组$B$又可以被向量组$A$线性表示。 \pause 
    \item[] \blue{如}:部分组总是可以由整体线性表示,但反之不成立
    \end{itemize} 
    
    
  \end{footnotesize}
\end{frame}

\begin{frame}
  \begin{footnotesize}
    向量组的等价,具备
    \begin{itemize}
    \item \red{自反性}
    \item[] 任一向量组和自身等价 \pause 
    \item \red{对称性}
    \item[] 向量组$A$与向量组$B$等价,当然向量组$B$与向量组$A$等价 \pause 
    \item \red{传递性}
    \item[] 设向量组$A$与向量组$B$等价,向量组$B$与向量组$C$等价,
      则向量组$A$与向量组$C$等价
    \end{itemize} 
  \end{footnotesize}
\end{frame}


\begin{frame}
  \begin{footnotesize}
    \begin{block}{定理3.2.1}
      若向量组$\blue{B:~\betabd_1,\betabd_2,\cd,\betabd_t}$可由向量组$\blue{A:~\alphabd_1,\alphabd_2,\cd,\alphabd_s}$线性表示,且$\blue{t>s}$,
      则$\blue{B:~\betabd_1,\betabd_2,\cd,\betabd_t}$线性相关。
    \end{block}
    \pause
    \proofname
    设
    $
    \ds \betabd_j=\sum_{i=1}^sk_{ij}\alphabd_i, \quad j=1,2,\cd,t.~~
    $ \pause 
    欲证$\betabd_1,\betabd_2,\cd,\betabd_t$线性相关,只需证:存在不全为零的数$x_1,x_2,\cd,x_t$使得
    \begin{equation}\label{thm4-1}
      x_1\betabd_1+x_2\betabd_2+\cd+x_t\betabd_t=\zero,
    \end{equation}   \pause  
    即
    $$
    \sum_{j=1}^t x_j \betabd_j = \sum_{j=1}^t x_j\left(\sum_{i=1}^sk_{ij}\alphabd_i\right)
    = \sum_{i=1}^s\left(\sum_{j=1}^t k_{ij} x_j \right)  \alphabd_i = \zero.
    $$ \pause 
    当其中$\alphabd_1,\alphabd_2,\cd,\alphabd_s$的系数
    \begin{equation}\label{thm4-2}
      \sum_{j=1}^tk_{ij}x_j = 0, \quad i=1,2,\cd,s
    \end{equation}
    时,(\ref{thm4-1})显然成立。\pause 
    注意到\blue{齐次线性方程组(\ref{thm4-2})含$t$个未知量,$s$个方程,而$t>s$},故(\ref{thm4-2})有非零解。\pause 即有不全为零的$x_1,x_2,\cd,x_t$使得(\ref{thm4-1})成立,故$\betabd_1,\betabd_2,\cd,\betabd_t$线性相关。
  \end{footnotesize}
\end{frame}


\begin{frame}
  \begin{footnotesize}
    \begin{block}{推论3.2.1(定理3.2.1的逆否命题)}
      若向量组$\blue{B:~\betabd_1,\betabd_2,\cd,\betabd_t}$可由向量组$\blue{A:~\alphabd_1,\alphabd_2,\cd,\alphabd_s}$线性表示,
      且$\blue{B:~\betabd_1,\betabd_2,\cd,\betabd_t}$线性无关,则
      $$\red{t\le s}$$。
    \end{block}
  \end{footnotesize}
\end{frame}


\begin{frame}
  \begin{footnotesize}
    \begin{block}{推论3.2.2}
      若$\blue{\rr(\alphabd_1,\alphabd_2,\cd,\alphabd_s)=r}$,
      则$\blue{\alphabd_1,\alphabd_2,\cd,\alphabd_s}$中任何$r+1$个向量都是线性相关的。
    \end{block}
    \pause 
    \proofname
    不妨设$\alphabd_1,\alphabd_2,\cd,\alphabd_r$是$\alphabd_1,\alphabd_2,\cd,\alphabd_s$中的$r$个线性无关的向量,由于该向量组中任一个向量可由$\alphabd_1,\alphabd_2,\cd,\alphabd_r$线性表示,由定理3.2.1可知,其中任意$r+1$个向量都线性相关。
  \end{footnotesize}
\end{frame}

\begin{frame}
  \begin{footnotesize}
    \begin{block}{定义(向量组的秩的等价定义 \& 极大线性无关组)}
      设有向量组$\alphabd_1,\alphabd_2,\cd,\alphabd_s$。
      如果能从其中选出$r$个向量$\alphabd_1,\alphabd_2,\cd,\alphabd_{\red{r}}$,满足
      \begin{itemize}
      \item 向量组$\alphabd_1,\alphabd_2,\cd,\alphabd_{\red{r}}$线性无关;
      \item 向量组$\alphabd_1,\alphabd_2,\cd,\alphabd_s$中任意$r+1$个向量都线性相关,
      \end{itemize}
      则称向量组$\alphabd_1,\alphabd_2,\cd,\alphabd_{\red{r}}$为原向量组的一个\red{极大线性无关组},简称\red{极大无关组}。\pause 
      \vspace{0.1in}

      \blue{\underline{极大线性无关组所含向量的个数$\red{r}$}},称为原向量组的\red{秩}。
      
    \end{block}

    \pause
    \begin{block}{注}
      \begin{itemize}
      \item   秩为$r$的向量组中,任一个线性无关的部分组最多含有$r$个向量;\\[0.1in]
      \item 一般情况下,极大无关组不惟一;\\[0.1in]
      \item 不同的极大无关组所含向量个数相同;\\[0.1in]
      \item 极大无关组与原向量组是等价的;\\[0.1in]
      \item 极大无关组是原向量组的\red{全权代表}。
      \end{itemize}

    \end{block}
  \end{footnotesize}
\end{frame}



\begin{frame}
  \begin{footnotesize}
    \begin{block}{推论3.2.3}
      设$\blue{\rr(\alphabd_1,\alphabd_2,\cd,\alphabd_s)=p,~~\rr(\betabd_1,\betabd_2,\cd,\betabd_t)=r}$,
      如果向量组$\blue{B:~\betabd_1,\betabd_2,\cd,\betabd_t}$可由$\blue{A:~\alphabd_1,\alphabd_2,\cd,\alphabd_s}$线性表示,则
      $$\red{r\le p.}$$
    \end{block}
    \pause 
    \proofname
    不妨设$\alphabd_1,\alphabd_2,\cd,\alphabd_p$与$\betabd_1,\betabd_2,\cd,\betabd_r$分别为两个向量组的极大线性无关组。 \pause 
    $$
    \begin{array}{rl}
      (1) & \betabd_1,\betabd_2,\cd,\betabd_r\mbox{等价于}\betabd_1,\betabd_2,\cd,\betabd_t\\[0.1in]
      \Rightarrow & 
      \blue{\betabd_1,\betabd_2,\cd,\betabd_r\mbox{可由}\betabd_1,\betabd_2,\cd,\betabd_t\mbox{线性表示}} \\[0.1in] \pause 
      (2) & \blue{\betabd_1,\betabd_2,\cd,\betabd_t\mbox{可由}\alphabd_1,\alphabd_2,\cd,\alphabd_s\mbox{线性表示}}\\[0.1in] \pause 
      (3)& \alphabd_1,\alphabd_2,\cd,\alphabd_s\mbox{等价于}\alphabd_1,\alphabd_2,\cd,\alphabd_p\\[0.1in]
      \Rightarrow & \blue{\alphabd_1,\alphabd_2,\cd,\alphabd_s\mbox{可由}\alphabd_1,\alphabd_2,\cd,\alphabd_p\mbox{线性表示}} \\[0.1in] \pause 
      \red{\Longrightarrow} &
      \red{\betabd_1,\betabd_2,\cd,\betabd_t\mbox{可由} \alphabd_1,\alphabd_2,\cd,\alphabd_p\mbox{线性表示}}
    \end{array}    
    $$ \pause 

    由推论3.2.1可知$r\le p$。
    \pause 

    \begin{block}{}
      \red{等价向量组的秩相等}。
    \end{block}
  \end{footnotesize}
\end{frame}
