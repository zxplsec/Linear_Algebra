%%%%%%%%%%%%%%%%%%%%%%%%%%%%%%%%%%%%%%%%%%%%%%%%%%%%%%%%%%%%%%%%%%%%

\subsection{典型例题1~~(线性相关性)}
\begin{frame}\ft{\subsecname}
  \begin{footnotesize}
    \begin{exampleblock}{例1~~2007-2008第一学期,2010-2011第一学期}
     若有不全为零的数$\lambda_1,\lambda_2,\cd,\lambda_m$使得$\lambda_1\alphabd_1+\lambda_2\alphabd_2+\cd+\lambda_m\alphabd_m+\lambda_1\betabd_1+\lambda_2\betabd_2+\cd+\lambda_m\betabd_m=\zero$ 成立,则$\alphabd_1,\alphabd_2,\cd,\alphabd_m$线性相关,$\betabd_1,\betabd_2,\cd,\betabd_m$也线性相关。试讨论该结论是否正确?
    \end{exampleblock}
    
    \pause
    该题可转换为:
    $$(\A+\B)\xx=\zero\mbox{有非零解}~~\xLongrightarrow[]{\ds \red{?}}~~ \A\xx=\zero\mbox{和}\B\xx=\zero\mbox{都有非零解}$$
   \end{footnotesize}
 \end{frame}
 
 \begin{frame}\ft{\subsecname}
  \begin{footnotesize}
    \begin{exampleblock}{例2~~2007-2008第二学期}
      设$\A$为$m\times n$矩阵,$\B$为$n\times m$矩阵,$\II$为单位矩阵,易知$\B\A=\II$,试判断$\A$的列向量组是否线性相关?为什么?
    \end{exampleblock}
    
    \pause \jiename
    一方面
    $$
    \rr(\A) \ge \rr(\B\A) =n,
    $$
    另一方面
    $$
    \rr(\A)\le n
    $$
    故$\rr(\A)=n$,于是$\A$的列向量组线性无关。
%      \end{footnotesize}
%\end{frame}
%
%
%\begin{frame}\ft{\subsecname}
%  \begin{footnotesize}
\pause 
    \begin{exampleblock}{例3~~2012-2013第二学期}
    设$\A$为$n\times m$矩阵,$\B$为$m\times n$矩阵,$n<m$且$\A\B=\II$,证明$\B$的列向量组线性无关。
    \end{exampleblock}

  \end{footnotesize}
\end{frame}

\begin{frame}\ft{\subsecname}
  \begin{footnotesize}
    \begin{exampleblock}{例4~~2008-2009第一学期}
      证明:与基础解系等价的线性无关的向量组也是基础解系。
    \end{exampleblock}
    \pause\proofname
    设$A:\alphabd_1,\alphabd_2,\cd,\alphabd_r$为基础解系,$B:\betabd_1,\betabd_2,\cd,\betabd_s$是$A$的等价组,且线性无关。
    由于$B$等价于$A$,故$A,B$可以互相线性表示。因$A$为基础解系,齐次线性方程组的全部解能由$A$线性表示,而$A$可由$B$线性表示,故齐次线性方程组的全部解能由$B$线性表示。注意到$r(A)=r$和$r(B=s)$,而$A$与$B$等价,故$r(A)=r(B)$,即$r=s$。综上所述,$B$也为基础解系。
  \end{footnotesize}
\end{frame}


\begin{frame}\ft{\subsecname}
  \begin{footnotesize}
    \begin{exampleblock}{例5~~2009-2010第一学期}
      已知向量组$\alphabd_1,\alphabd_2,\alphabd_3,\alphabd_4$线性无关,
      \begin{itemize}
      \item[1] 向量组$\alphabd_1,\alphabd_2,\alphabd_3$是否线性无关,并说明理由。
      \item[2] 常数$l,m$满足何种条件时,$l\alphabd_1+\alphabd_2,\alphabd_2+\alphabd_3,m\alphabd_3+\alphabd_1$线性无关,并说明理由。
      \end{itemize}
    \end{exampleblock}
    \pause \proofname
    \begin{itemize}
      \item[1] 整体无关,则部分无关。
      \item[2]   
      设$
      x_1(l\alphabd_1+\alphabd_2)+x_2(\alphabd_2+\alphabd_3)+x_3(m\alphabd_3+\alphabd_1)=\zero  
      $
      即
      $$
      (lx_1+x_3)\alphabd_1+(x_1+x_2)\alphabd_2 +(x_2+mx_3)\alphabd_3=\zero
      $$      
      由于$\alphabd_1,\alphabd_2,\alphabd_3$线性无关,故
      $$
      \left\{
      \begin{array}{rcl}
lx_1+x_3&=&0\\
x_1+x_2&=&0\\
x_2+mx_3&=&0
      \end{array}
      \right.
      $$
      只有零解。
      \end{itemize}
  \end{footnotesize}
\end{frame}




\subsection{典型例题2~~(极大无关组与向量组的秩)}

\begin{frame}\ft{\subsecname}
  \begin{scriptsize}
    \begin{exampleblock}{$\bigstar\bigstar\bigstar\bigstar\bigstar$}
      设向量组
      $$
      \alphabd_1=\left(
      \begin{array}{r}
        -1\\-1\\0\\0
      \end{array}
      \right),~~ \alphabd_2=\left(
      \begin{array}{r}
        1\\2\\1\\-1
      \end{array}
      \right),~~ \alphabd_3=\left(
      \begin{array}{r}
        0\\1\\1\\-1
      \end{array}
      \right),~~ \alphabd_4=\left(
      \begin{array}{r}
        1\\3\\2\\1
      \end{array}
      \right),~~ \alphabd_5=\left(
      \begin{array}{r}
        2\\6\\4\\-1
      \end{array}
      \right)
      $$
      求向量组的秩及其一个极大无关组,并将其余向量用该极大无关组线性表示。
    \end{exampleblock}
    \pause\jiename
    作矩阵$\A=(\alphabd_1,\alphabd_2,\alphabd_3,\alphabd_4,\alphabd_5)$,由
    $$
    \begin{array}{rl}
    \A &= \left(
    \begin{array}{rrrrr}
      -1&1&0&1&2\\
      -1&2&1&3&6\\
      0&1&1&2&4\\
      0&-1&-1&1&-1
    \end{array}
    \right) \xlongrightarrow[r_2+r_1]{ r_1\times(-1)}
    \left(
    \begin{array}{rrrrr}
      1&-1&0&-1&-2\\
      0&1&1&2&4\\
      0&1&1&2&4\\
      0&-1&-1&1&-1
    \end{array}
    \right)\\[0.28in]
    &\xlongrightarrow[r_4+r_2]{r_3- r_2}
    \left(
    \begin{array}{rrrrr}
      1&-1&0&-1&-2\\
      0&1&1&2&4\\
      0&0&0&0&0\\
      0&0&0&3&3
    \end{array}
    \right) \xlongrightarrow[r_3\leftrightarrow r_4]{r_4\div 3}
    \left(
    \begin{array}{rrrrr}
      1&-1&0&-1&-2\\
      0&1&1&2&4\\
      0&0&0&1&1\\
      0&0&0&0&0
    \end{array}
    \right)
    \end{array}
    $$
  \end{scriptsize}
\end{frame}

\begin{frame}\ft{\subsecname}
  \begin{scriptsize}
    $$
    \begin{array}{rl}
      & \xlongrightarrow[r_2 -2 r_3]{r_1+r_3}
    \left(
    \begin{array}{rrrrr}
      1&-1&0&0&-1\\
      0&1&1&0&2\\
      0&0&0&1&1\\
      0&0&0&0&0
    \end{array}
    \right) \xlongrightarrow[]{r_1+r_2}
    \left(
    \begin{array}{rrrrr}
      1&0&1&0&1\\
      0&1&1&0&2\\
      0&0&0&1&1\\
      0&0&0&0&0
    \end{array}
    \right) = \B
    \end{array}
    $$
    将最后一个阶梯矩阵$\B$记为$(\betabd_1,\betabd_2,\betabd_3,\betabd_4,\betabd_5)$
    \pause 
    \vspace{0.1in}

    易知$\betabd_1,\betabd_2,\betabd_4$为$\B$的列向量组的一个极大无关组,故$\alphabd_1,\alphabd_2,\alphabd_4$也为$\A$的列向量组的一个极大无关组,故
    $$
    \rr(\alphabd_1,\alphabd_2,\alphabd_3,\alphabd_4,\alphabd_5)=3,
    $$
    且
    $$
    \begin{array}{l}
      \alphabd_3=\alphabd_1+\alphabd_2,\\
      \alphabd_5=\alphabd_1+2\alphabd_2+\alphabd_4,\\
    \end{array}
    $$
    
  \end{scriptsize}
\end{frame}




\begin{frame}
  \begin{scriptsize}
    \begin{exampleblock}{$\bigstar\bigstar\bigstar\bigstar\bigstar$}
      设$\alphabd_1=(1,3,1,2), ~\alphabd_2=(2,5,3,3), ~\alphabd_3=(0,1,-1,a), ~\alphabd_4=(3,10,k,4)$,
      试求向量组$\alphabd_1,~\alphabd_2,~\alphabd_3,~\alphabd_4$的秩,并将$\alphabd_4$用$\alphabd_1,~\alphabd_2,~\alphabd_3$线性表示。
    \end{exampleblock}
    \pause \jiename
    将4个向量按列排成一个矩阵$\A$,对$\A$进行初等变换,将其化为阶梯形矩阵$\U$,即
    $$
    \A=\left(
    \begin{array}{rrrr}
    1&2&0&3\\
    3&5&1&10\\
    1&3&-1&k\\
    2&3&a&4
    \end{array}
    \right) \xlongrightarrow[]{\mbox{初等行变换}}
    \left(
    \begin{array}{rrcc}
    1&2&0&3\\
    0&-1&1&1\\
    0&0&a-1&-3\\
    0&0&0&k-2
    \end{array}
    \right)=\U
    $$
    \pause 
    \begin{itemize}
    \item[(1)] 当$a=1$或$k=2$时,$\U$只有3个非零行,故
      $$\rr(\alphabd_1,\alphabd_2,\alphabd_3,\alphabd_4)=\rr(\A)=3. $$ 
    \item[(2)] \pause 当$a\ne1$且$k\ne2$时,
      $$\rr(\alphabd_1,\alphabd_2,\alphabd_3,\alphabd_4)=\rr(\A)=4.$$
    \end{itemize}
      \end{scriptsize}
\end{frame}


\begin{frame}
  \begin{scriptsize}
    $$
    \A=\left(
    \begin{array}{rrrr}
      1&2&0&3\\
      3&5&1&10\\
      1&3&-1&k\\
      2&3&a&4
    \end{array}
    \right) \xlongrightarrow[]{\mbox{初等行变换}}
    \left(
    \begin{array}{rrcc}
      1&2&0&3\\
      0&-1&1&1\\
      0&0&a-1&-3\\
      0&0&0&k-2
    \end{array}
    \right)
    $$
    \begin{itemize}
    \item 当$k=2$且$a\ne1$时,$\alphabd_4$可由$\alphabd_1,~\alphabd_2,~\alphabd_3$线性表示,
      且
      $$
      \alphabd_4=-\frac{1+5a}{1-a}\alphabd_1+\frac{2+a}{1-a}\alphabd_2+\frac{3}{1-a}\alphabd_3.
      $$
    \item \pause 当$k\ne2$或$a=1$时,$\alphabd_4$不能由$\alphabd_1,~\alphabd_2,~\alphabd_3$线性表示。
    \end{itemize}
  \end{scriptsize}
\end{frame}


\begin{frame}
  \begin{scriptsize}
    \begin{exampleblock}{$\bigstar\bigstar\bigstar\bigstar\bigstar$}
      设
      $$
      \A=\left(
      \begin{array}{rrr}
        1&2&1\\
        2&2&-2\\
        -1&t&5\\
        1&0&-3
      \end{array}
      \right)
      $$
      已知$\rr(\A)=2$,求$t$。
    \end{exampleblock}
    \pause
    \jiename
    $$
    \A \xlongrightarrow[]{\mbox{初等行变换}} \left(
    \begin{array}{ccr}
      1&2&1\\
      0&-2&-4\\
      0&2+t&6\\
      0&0&0
    \end{array}
    \right)=\B
    $$ \pause
    由于$\rr(\B)=\rr(\A)$,故$\B$中第2、3行必须成比例,即
    $$
    \frac{-2}{2+t}=\frac{-4}6,
    $$
    即得$t=1$。
  \end{scriptsize}
\end{frame}


%% \begin{frame}\ft{\subsecname}
%%   \begin{scriptsize}
%%     \begin{exampleblock}{2005-2006第一学期}
%%       对于$\RR^3$中的向量组$A:\alphabd_1,\alphabd_2,\alphabd_3$和$B:\betabd_1,\betabd_2,\betabd_3$,讨论下面的问题:
%%       \begin{itemize}
%%       \item[(1)] 向量组$B$能否成为$\RR^3$中的基?能否用$A$线性表示$B$?如果可以,试求出由$\alphabd_1,\alphabd_2,\alphabd_3$到$\betabd_1,\betabd_2,\betabd_3$的过渡矩阵$P$,其中
%%         $$
%%         \begin{aligned}
%%           \alphabd_1=\left(
%%         \begin{array}{c}
%%           1\\0\\0
%%         \end{array}
%%         \right),
%%         \alphabd_2=\left(
%%         \begin{array}{c}
%%           1\\1\\0
%%         \end{array}
%%         \right),
%%         \alphabd_3=\left(
%%         \begin{array}{c}
%%           1\\1\\1
%%         \end{array}
%%         \right),
%%          \\
%%          \betabd_1=\left(
%%         \begin{array}{c}
%%           1\\1\\a
%%         \end{array}
%%         \right),
%%         \betabd_2=\left(
%%         \begin{array}{c}
%%           1\\1\\2-a
%%         \end{array}
%%         \right),
%%         \betabd_3=\left(
%%         \begin{array}{c}
%%           -1\\1\\0
%%         \end{array}
%%         \right), (a\in \RR)
%%         \end{aligned}
%%         $$
%%       \item[(2)]
%%         若$\betabd_1=k(2\alphabd_1+2\alphabd_2-\alphabd_3), \betabd_2=k(2\alphabd_1-\alphabd_2+2\alphabd_3),\betabd_3=k(\alphabd_1-2\alphabd_2-2\alphabd_3)$,$k$为非零实数,
%%         \begin{itemize}
%%         \item[(a)] 给出向量组$\betabd_1,\betabd_2,\betabd_3$线性无关的一个充要条件,并证明之;
%%         \item[(b)] 给出矩阵$(\betabd_1,\betabd_2,\betabd_3)$为正交阵的一个充要条件,并证明之.
%%         \end{itemize}
%%       \end{itemize}
%%     \end{exampleblock}
%%     \end{scriptsize}
%% \end{frame}



\begin{frame}\ft{\subsecname}
  \begin{footnotesize}
    \begin{exampleblock}{例1~~2005-2006第二学期}
      设$$\alphabd_1=(1,0,2,1),\alphabd_2=(2,0,1,-1),\alphabd_3=(1,1,0,1),\alphabd_4=(4,1,3,1),$$
      求向量组$\alphabd_1,\alphabd_2,\alphabd_3,\alphabd_4$的秩和一个极大无关组。
    \end{exampleblock}
%%   \end{footnotesize}
%% \end{frame}


%% \begin{frame}\ft{\subsecname}
%%   \begin{footnotesize}
    \begin{exampleblock}{例2~~2006-2007第二学期}
      计算向量组$$
      \begin{aligned}
      \alphabd_1=(1,-2,3,-1,2)^T,\alphabd_2=(2,1,2,-2,-3)^T, \\[1ex]
      \alphabd_3=(5,0,7,-5,-4)^T,\alphabd_4=(3,-1,5,-3,-1)^T
      \end{aligned}$$
      的秩和一个极大无关组,同时将其余向量表示成极大无关组的线性组合。
    \end{exampleblock}
  \end{footnotesize}
\end{frame}






 \begin{frame}\ft{\subsecname}
   \begin{footnotesize}
    \begin{exampleblock}{例3~~2007-2008第二学期}
      计算向量组
      $$\xibd_1=(1,2,3)^T,\xibd_2=(-8,4,8)^T,\xibd_3=(2,-1,-2)^T,
      \xibd_4=(10,5,6)^T
      $$的秩和一个极大无关组,同时将其余向量表示成极大无关组的线性组合。
    \end{exampleblock}
%  \end{footnotesize}
%\end{frame}
%
%
%
%\begin{frame}\ft{\subsecname}
%  \begin{footnotesize}
    \begin{exampleblock}{例4~~2008-2009第一学期}
      计算向量组
      $$\xibd_1=(1,0,2,1)^T,\xibd_2=(2,0,1,-1)^T,\xibd_3=(1,1,0,1)^T,
      \xibd_4=(4,1,3,1)^T
      $$的秩和一个极大无关组,同时将其余向量表示成极大无关组的线性组合。
    \end{exampleblock}
  \end{footnotesize}
\end{frame}




\begin{frame}\ft{\subsecname}
  \begin{footnotesize}
    \begin{exampleblock}{例5~~2008-2009第一学期}
      计算向量组$$\xibd_1=(1,1,0)^T,\xibd_2=(0,1,1)^T,\xibd_3=(1,1,1)^T,
      \xibd_4=(1,2,1)^T$$的秩和一个极大无关组,并给出向量组中不能由其余向量线性表示的向量。
    \end{exampleblock}
%  \end{footnotesize}
%\end{frame}
%
%
%\begin{frame}\ft{\subsecname}
%  \begin{footnotesize}
    \begin{exampleblock}{例6~~2009-2010第一学期}
    已知线性方程组$\A\xx=\bb$存在两个不同的解,其中$$\A=\left(
    \begin{array}{ccc}
        \lambda&1&1\\
        0&\lambda-1&0\\
        1&1&\lambda
      \end{array}
    \right),\bb=\left(
      \begin{array}{c}
        a\\1 \\1
      \end{array}
      \right).$$
      \begin{itemize}
      \item[1] 求$\lambda,a$.
      \item[2] 求其通解。
      \end{itemize}
    \end{exampleblock}

  \end{footnotesize}
\end{frame}


\begin{frame}\ft{\subsecname}
  \begin{footnotesize}
    \begin{exampleblock}{例7~~2009-2010第一学期}
    设有向量组$$\alphabd_1=(1,2,0)^T,\alphabd_2=(1,a+2,-3a)^T,\alphabd_3=(-1,-b-2,a+2b)^T,\betabd=(1,3,-3)^T,$$
    讨论当$a,b$为何值时,
    \begin{itemize}
      \item[1] $\betabd$不能由$\alphabd_1,\alphabd_2,\alphabd_3$线性表示;
      \item[2] $\betabd$可由$\alphabd_1,\alphabd_2,\alphabd_3$惟一地线性表示,并求出表示式;
      \item[3] $\betabd$可由$\alphabd_1,\alphabd_2,\alphabd_3$线性表示,但表示式不唯一,并求出表示式。
      \end{itemize}
    \end{exampleblock}
  \end{footnotesize}
\end{frame}



\begin{frame}\ft{\subsecname}
  \begin{footnotesize}
    \begin{exampleblock}{例8~~2012-2013第二学期}
    已知$\alphabd_1=\left(
      \begin{array}{c}
        1\\4\\0\\2
      \end{array}
      \right),\alphabd_2=\left(
      \begin{array}{c}
        2\\7\\1\\3
      \end{array}
      \right),\alphabd_3=\left(
      \begin{array}{r}
        0\\1\\-1\\a
      \end{array}
      \right),\betabd=\left(
      \begin{array}{c}
        3\\10\\b\\4
      \end{array}
      \right)$
      问$a,b$为何值时,
      \begin{itemize}
      \item[1] $\betabd$不能由$\alphabd_1,\alphabd_2,\alphabd_3$线性表示;
      \item[2] $\betabd$可由$\alphabd_1,\alphabd_2,\alphabd_3$惟一地线性表示,并求出表示式;
      \item[3] $\betabd$可由$\alphabd_1,\alphabd_2,\alphabd_3$线性表示,但表示式不唯一,并求出表示式;
      \item[4] $\alphabd_1,\alphabd_2,\alphabd_3$线性相关,并在此时求它的秩和一个最大无关组,且用该最大无关组表示其余向量。
      \end{itemize}
    \end{exampleblock}

  \end{footnotesize}
\end{frame}








