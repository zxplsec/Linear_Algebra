\section{第四章~~向量空间与线性变换}

\begin{frame}\ft{基与坐标}
  \begin{footnotesize}
    \begin{block}{定义($\mathbb R^n$的基与向量关于基的坐标)}
      设有序向量组$B=(\betabd_1,\betabd_2,\cd,\betabd_n)\subset\mathbb R^n$,如果$B$线性无关,
      则$\mathbb R^n$中任一向量$\alphabd$均可由$B$线性表示,即
      $$
      \alphabd=a_1\betabd_1+a_2\betabd_2+\cd+a_n\betabd_n,
      $$
      称$B$为$\mathbb R^n$的一组基,有序数组$(a_1,a_2,\cd,a_n)$是向量$\alphabd$在基$B$下的坐标,记作
      $$
      \alphabd_B=(a_1,a_2,\cd,a_n)\mbox{~~或~~}\alphabd_B=(a_1,a_2,\cd,a_n)^T
      $$
      并称之为$\alphabd$的坐标向量。
    \end{block}
  \end{footnotesize}
\end{frame}


\begin{frame}\ft{基与坐标}
  \begin{footnotesize}
    \begin{block}{注}
      \begin{itemize}
      \item $\RRR^n$的基不是唯一的
      \item 基本向量组
        $$
        \epsilonbd_i=(0,\cd,0,1,0,\cd,0), \quad i=1,2,\cd,n
        $$
        称为$\RRR^n$的自然基或标准基。
      \item 本书对于向量及其坐标,采用列向量的形式,即
        $$
        \alphabd=(\betabd_1,\betabd_2,\cd,\betabd_n)\left(
        \begin{array}{c}
          a_1\\
          a_2\\
          \vd\\
          a_n
        \end{array}
        \right)
        $$
      \end{itemize}
    \end{block}
  \end{footnotesize}
\end{frame}




\begin{frame}\ft{基与坐标}
  \begin{footnotesize}
    \begin{block}{定理}
      设$B=\{\alphabd_1,\alphabd_2,\cd,\alphabd_n\}$是$\RRR^n$的一组基,且
      $$
      \left\{
      \begin{array}{l}
        \etabd_1=a_{11}\alphabd_1+a_{21}\alphabd_2+\cd+a_{n1}\alphabd_n,\\[0.2cm]
        \etabd_2=a_{12}\alphabd_1+a_{22}\alphabd_2+\cd+a_{n2}\alphabd_n,\\[0.2cm]
        \cd\cd\\[0.2cm]
        \etabd_n=a_{1n}\alphabd_1+a_{2n}\alphabd_2+\cd+a_{nn}\alphabd_n.
      \end{array}
      \right.
      $$
      则$\etabd_1,\etabd_2,\cd,\etabd_n$线性无关的充要条件是
      $$
      \mathrm{det}\A=\left|
      \begin{array}{cccc}
        a_{11}&a_{12}&\cd&a_{1n}\\
        a_{21}&a_{22}&\cd&a_{2n}\\
        \vd&\vd&&\vd\\
        a_{n1}&a_{n2}&\cd&a_{nn}
      \end{array}
      \right|\ne 0.
      $$
    \end{block}
  \end{footnotesize}
\end{frame}


\begin{frame}\ft{基与坐标}
  \begin{footnotesize}
    设$\RRR^n$的两组基$B_1=\{\alphabd_1,\alphabd_2,\cd,\alphabd_n\}$和$B_2=\{\etabd_1,\etabd_2,\cd,\etabd_n\}$满足关系式
    $$\red{
    (\etabd_1,\etabd_2,\cd,\etabd_n)=(\alphabd_1,\alphabd_2,\cd,\alphabd_n)\left(
    \begin{array}{cccc}
      a_{11}&a_{12}&\cd&a_{1n}\\
      a_{21}&a_{22}&\cd&a_{2n}\\
      \vd&\vd&&\vd\\
      a_{n1}&a_{n2}&\cd&a_{nn}
    \end{array}
    \right)}
    $$
    则矩阵
    $$\blue{
    \A=\left(
    \begin{array}{cccc}
      a_{11}&a_{12}&\cd&a_{1n}\\
      a_{21}&a_{22}&\cd&a_{2n}\\
      \vd&\vd&&\vd\\
      a_{n1}&a_{n2}&\cd&a_{nn}
    \end{array}
    \right)}
    $$
    称为\purple{由旧基$B_1$到新基$B_2$的过渡矩阵}。
  \end{footnotesize}
\end{frame}

\begin{frame}\ft{基与坐标}
  \begin{footnotesize}
    \begin{block}{定理}
      设$\alphabd$在两组基$B_1=\{\alphabd_1,\alphabd_2,\cd,\alphabd_n\}$与$B_2=\{\etabd_1,\etabd_2,\cd,\etabd_n\}$的坐标分别为
      $$
      \xx=(x_1,x_2,\cd,x_n)^T\mbox{~~和~~}\yy=(y_1,y_2,\cd,y_n)^T
      $$
      基$B_1$到$B_2$的过渡矩阵为$\A$,则
      $$\red{
      \A\yy=\xx\mbox{~~或~~}\yy=\A^{-1}\xx
      }
      $$
    \end{block}
  \end{footnotesize}
\end{frame}


\begin{frame}\ft{基与坐标}
  \begin{footnotesize}
    \begin{exampleblock}{例1}
      已知$\RRR^3$的一组基为$B_2=\{\betabd_1,\betabd_2,\betabd_3\}$,其中
      $$\betabd_1=(1,2,1)^T,\betabd_2=(1,-1,0)^T,\betabd_3=(1,0,-1)^T,$$
      求自然基$B_1$到$B_2$的过渡矩阵。
    \end{exampleblock}
  \end{footnotesize}
\end{frame}


\begin{frame}\ft{基与坐标}
  \begin{footnotesize}
    \begin{exampleblock}{例2}
      已知$\RRR^3$的两组基为$B_1=\{\alphabd_1,\alphabd_2,\alphabd_3\}$和$B_2=\{\betabd_1,\betabd_2,\betabd_3\}$,
      其中
      $$
      \begin{array}{lll}
        \alphabd_1=(1,1,1)^T,&\alphabd_2=(0,1,1)^T,&\alphabd_3=(0,0,1)^T, \\[0.2cm]
        \betabd_1=(1,0,1)^T,&\betabd_2=(0,1,-1)^T,&\betabd_3=(1,2,0)^T.  
      \end{array}
      $$
      \begin{itemize}
      \item[(1)] 求基$B_1$到$B_2$的过渡矩阵。
      \item[(2)] 已知$\alpha$在基$B_1$的坐标为$(1,-2,-1)^T$,求$\alphabd$在基$B_2$下的坐标。
      \end{itemize}
      
    \end{exampleblock}
  \end{footnotesize}
\end{frame}



\begin{frame}\ft{内积}
  \begin{footnotesize}
    \begin{block}{定义(内积)}
      在$\RRR^n$中,对于$\alphabd=(a_1,a_2,\cd,a_n)^T$和$\betabd=(b_1,b_2,\cd,b_n)^T$,规定$\alphabd$和$\betabd$的内积为 
      $$
      (\alphabd,\betabd)=a_1b_1+a_2b_2+\cd+a_nb_n.
      $$
    \end{block}
    当$\alphabd$和$\betabd$为列向量时,
    $$
    (\alphabd,\betabd)=\alphabd^T\betabd=\betabd^T\alphabd.
    $$
  \end{footnotesize}
\end{frame}

\begin{frame}\ft{内积}
  \begin{footnotesize}
    \begin{block}{内积的运算性质}
      对于$\alphabd,\betabd,\gammabd\in\RRR^n$和$k\in\RRR$,
      \begin{itemize}
      \item[(i)]   $(\alphabd,\betabd)=(\betabd,\alphabd)$
      \item[(ii)]  $(\alphabd+\betabd,\gammabd)=(\alphabd,\gammabd)+(\betabd,\gammabd)$
      \item[(iii)] $(k\alphabd,\betabd)=k(\alphabd,\betabd)$
      \item[(iv)]  $(\alphabd,\alphabd)\ge0$, 等号成立当且仅当$\alphabd=\zero$.
      \end{itemize}
    \end{block}
    \pause
    \begin{block}{定义(向量长度)}
      向量$\alphabd$的长度定义为
      $$
      \|\alphabd\|=\sqrt{(\alphabd,\alphabd)}
      $$
    \end{block}
  \end{footnotesize}
\end{frame}


\begin{frame}\ft{内积}
  \begin{footnotesize}
    \begin{block}{定理(柯西-施瓦茨(Cauchy-Schwarz)不等式)}
      $$
      |(\alphabd,\betabd)|\le\|\alphabd\|\|\betabd\|
      $$
    \end{block}
  
%  \end{footnotesize}
%\end{frame}
%
%\begin{frame}
%  \begin{footnotesize}
    \begin{block}{定义(向量之间的夹角)}
    向量$\alphabd,\betabd$之间的夹角定义为
    $$
    <\alphabd,\betabd>=\arccos\frac{(\alphabd,\betabd)}{\|\alphabd\|\|\betabd\||}
    $$
    \end{block}
 
    \begin{block}{定理}
    $$\alphabd\perp\betabd ~~\Longleftrightarrow~~
   (\alphabd,\betabd)=0
    $$
    \end{block}


%  \end{footnotesize}
%\end{frame}
%
%
%
%\begin{frame}
%  \begin{footnotesize}
    \begin{block}{定理(三角不等式)}
      $$
      \|\alphabd+\betabd\|\le\|\alphabd\|+\|\betabd\|.
      $$
    \end{block}
   
  \end{footnotesize}
\end{frame}



\begin{frame}\ft{标准正交基}
  \begin{footnotesize}
    \begin{block}{定理}
      $\RRR^n$中两两正交且不含零向量的向量组$\alphabd_1,\alphabd_2,\cd,\alphabd_s$是线性无关的。
    \end{block}
    
%  \end{footnotesize}
%\end{frame}
%
%
%\begin{frame}\ft{标准正交基}
%  \begin{footnotesize}
    \begin{block}{定义(标准正交基)}
      设$\alphabd_1,\alphabd_2,\cd,\alphabd_n\in \RRR^n$,若
      $$
      (\alphabd_i,\alphabd_j)=\delta_{ij}=\left\{
      \begin{array}{ll}
        1,& i=j,\\
        0,& i\ne j.
      \end{array}
      \right. \quad i,j=1,2,\cd,n.
      $$
      则称$\{\alphabd_1,\alphabd_2,\cd,\alphabd_n\}$是$\RRR^n$中的一组标准正交基。
    \end{block}
  \end{footnotesize}
\end{frame}


\begin{frame}\ft{标准正交基}
  \begin{footnotesize}
    \begin{exampleblock}{例1}
      设$B=(\alphabd_1,\alphabd_2,\cd,\alphabd_n)$是$\RRR^n$中的一组标准正交基,求$\RRR^n$中向量$\betabd$在基$B$下的坐标。
    \end{exampleblock}
    \jiename
    $$
    \begin{array}{rl}
    & \betabd=x_1\alphabd_1+x_2\alphabd_2+\cd+x_n\alphabd_n\\[0.1in]
    \Longrightarrow&   (\betabd,\alphabd_j)=(x_1\alphabd_1+x_2\alphabd_2+\cd+x_n\alphabd_n,\alphabd_j)=x_j(\alphabd_j,\alphabd_j)\\[0.1in]
     \Longrightarrow& \ds x_j =  (\betabd,\alphabd_j) 
    \end{array}
     $$
    
  \end{footnotesize}
\end{frame}


\begin{frame}\ft{施密特正交化过程}
  \begin{footnotesize}
    \begin{block}{目标}
      从线性无关的向量组$\alphabd_1,\alphabd_2,\cd,\alphabd_m$出发,构造\red{标准正交向量组}。
    \end{block}
  \end{footnotesize}
\end{frame}


\begin{frame}\ft{施密特正交化过程}
  \begin{footnotesize}
    \begin{block}{施密特(Schmidt)正交化过程}
      给定$\RRR^n$中的线性无关组$\alphabd_1,\alphabd_2,\cd,\alphabd_m$, \pause
      \begin{itemize}
      \item[(1)] $$\betabd_1=\alphabd_1$$
      \item[(2)] 
        $$
        \betabd_2=\alphabd_2-\frac{(\alphabd_2,\betabd_1)}{(\betabd_1,\betabd_1)}\betabd_1
        $$
      \item[(3)] 
        $$
        \betabd_3=\alphabd_3-\frac{(\alphabd_3,\betabd_1)}{(\betabd_1,\betabd_1)}\betabd_1
        -\frac{(\alphabd_3,\betabd_2)}{(\betabd_2,\betabd_2)}\betabd_2
        $$
       \item[(4)]  $$\cd\cd$$
      \item[(5)] 
        $$
        \betabd_m=\alphabd_m-\frac{(\alphabd_m,\betabd_1)}{(\betabd_1,\betabd_1)}\betabd_1
        -\frac{(\alphabd_m,\betabd_2)}{(\betabd_2,\betabd_2)}\betabd_2
        -\cd
        -\frac{(\alphabd_m,\betabd_{m-1})}{(\betabd_{m-1},\betabd_{m-1})}\betabd_{m-1}.
        $$
      \end{itemize}
      则$\betabd_1,\betabd_2,\cd,\betabd_m$两两正交。
    \end{block}
  \end{footnotesize}
\end{frame}


\begin{frame}\ft{施密特正交化过程}
  \begin{footnotesize}
    \begin{block}{施密特(Schmidt)正交化过程}
      \begin{itemize}
      \item[(5)] 单位化
        $$
        \betabd_1, \betabd_2, \cd, \betabd_m \xlongrightarrow[]{\ds \eta_j=\frac{\betabd_j}{\|\betabd_j\|}}
        \etabd_1, \etabd_2, \cd, \etabd_m
        $$
      \end{itemize}
    \end{block}
  \end{footnotesize}
\end{frame}


\begin{frame}\ft{施密特正交化过程}
  \begin{footnotesize}
    \begin{exampleblock}{例}
      已知$B=\{\alphabd_1,\alphabd_2,\alphabd_3\}$是$\RRR^3$的一组基,其中
      $$
      \alphabd_1=(1,-1,0)^T,~~
      \alphabd_2=(1,0,1)^T,~~
      \alphabd_3=(1,-1,1)^T.
      $$
      试用施密特正交化方法,由$B$构造$\RRR^3$的一组标准正交基。
    \end{exampleblock}
    \pause\jiename
    $$
    \begin{array}{rl}
      \betabd_1&=\alphabd_1=(1,-1,0)^T, \\[0.2cm]\pause
      \betabd_2&\ds=\alphabd_2-\frac{(\alphabd_2,\betabd_1)}{(\betabd_1,\betabd_1)}\betabd_1\\[0.4cm]\pause
      &\ds=(1,0,1)^T-\frac12(1,-1,0)^T=\left(\frac12,\frac12,1\right),\\[0.4cm]\pause
      \betabd_3&\ds=\alphabd_3-\frac{(\alphabd_3,\betabd_1)}{(\betabd_1,\betabd_1)}\betabd_1
      -\frac{(\alphabd_3,\betabd_2)}{(\betabd_2,\betabd_2)}\betabd_2\\[0.4cm]\pause
      &\ds=(1,-1,1)^T-\frac23\left(\frac12,\frac12,1\right)^T-\frac22(1,-1,0)^T=\left(-\frac13,-\frac13,\frac13\right).
    \end{array}
    $$
  \end{footnotesize}
\end{frame}


\begin{frame}\ft{施密特正交化过程}
  \begin{footnotesize}
    $$
    \begin{array}{rl}
      \etabd_1&\ds =\frac{\betabd_1}{\|\betabd_1\|}=\left(\frac1{\sqrt{2}},-\frac1{\sqrt{2}},0\right),\\[0.4cm]
      \etabd_2&\ds =\frac{\betabd_2}{\|\betabd_2\|}=\left(\frac1{\sqrt{6}},\frac1{\sqrt{6}},\frac2{\sqrt{6}}\right),\\[0.4cm]
      \etabd_3&\ds =\frac{\betabd_3}{\|\betabd_3\|}=\left(-\frac1{\sqrt{3}},-\frac1{\sqrt{3}},\frac1{\sqrt{3}}\right).
    \end{array}
    $$
  \end{footnotesize}
\end{frame}



\begin{frame}\ft{正交矩阵}
  \begin{footnotesize}
    \begin{block}{定义}
      设$\A\in\RRR^{n\times n}$,如果
      $$
      \A^T\A=\II
      $$
      则称$\A$为正交矩阵。
    \end{block}
%  \end{footnotesize}
%\end{frame}
%
%
%\begin{frame}\ft{正交矩阵}
%  \begin{footnotesize}
    \begin{block}{定理}
      $$
      \A\mbox{为}\mbox{正交矩阵}
      ~~\Longleftrightarrow~~
      \A\mbox{的列向量组为一组标准正交基。}
      $$
    \end{block}
    
  \end{footnotesize}
\end{frame}


\begin{frame}\ft{正交矩阵}
  \begin{footnotesize}
    \begin{block}{定理}
      设$\A,\B$皆为$n$阶正交矩阵,则
      \begin{itemize}
      \item[(1)] $|\A|=1\mbox{~或~} -1$
      \item[(2)] $\A^{-1}=\A^T$
      \item[(3)] $\A^T$也是正交矩阵
      \item[(4)] $\A\B$也是正交矩阵
      \end{itemize}
    \end{block}
%  \end{footnotesize}
%\end{frame}
%
%\begin{frame}\ft{正交矩阵}
%  \begin{footnotesize}
    \begin{block}{定理}
      若列向量$\xx,\yy\in\RRR^n$在$n$阶正交矩阵$\A$的作用下变换为$\A\xx,\A\yy\in\RRR^n$,则向量的内积、长度与向量间的夹角都保持不变.
    \end{block}
  \end{footnotesize}
\end{frame}


%%%%%%%%%%%%%%%%%%%%%%%%%%%%%%
\subsection{往年试题}

\begin{frame}
   \begin{footnotesize}

    \begin{exampleblock}{13-14上}
      在$\RR^4$中,已知
$$
\begin{aligned}
\alphabd_1=\left(
\begin{array}{c}
1\\0\\0\\0
\end{array}
\right),
\alphabd_2=\left(
\begin{array}{c}
1\\2\\0\\0
\end{array}
\right),
\alphabd_3=\left(
\begin{array}{c}
1\\1\\1\\0
\end{array}
\right),
\alphabd_4=\left(
\begin{array}{c}
1\\1\\1\\1
\end{array}
\right);\\
\betabd_1=\left(
\begin{array}{c}
1\\-1\\a\\1
\end{array}
\right),
\betabd_2=\left(
\begin{array}{c}
-1\\1\\2-a\\1
\end{array}
\right),
\betabd_3=\left(
\begin{array}{c}
-1\\1\\0\\0
\end{array}
\right),
\betabd_4=\left(
\begin{array}{c}
1\\0\\0\\0
\end{array}
\right)
\end{aligned}
$$
\begin{itemize}
\item[1] 求$a$使得$\betabd_1,\betabd_2,\betabd_3,\betabd_4$为$\RR^4$的基;
\item[2] 求由基$\alphabd_1,\alphabd_2,\alphabd_3,\alphabd_4$到基$\betabd_1,\betabd_2,\betabd_3,\betabd_4$的过渡矩阵$\PP$.
\end{itemize}•
    \end{exampleblock}

  
  \end{footnotesize}
\end{frame}

