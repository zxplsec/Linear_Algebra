%%%%%%%%%%%%
\section{克莱姆法则}

\begin{frame}
  \begin{footnotesize}
    考察$n$元一次方程组
    \begin{equation}\label{ls}
      \left\{
      \begin{array}{l}
        a_{11}x_1 + a_{12}x_2 + \cdots + a_{1n}x_n = b_1, \\[0.3cm]
        a_{21}x_1 + a_{22}x_2 + \cdots + a_{2n}x_n = b_2, \\[0.3cm]
        \cd \\[0.2cm]
        a_{n1}x_1 + a_{n2}x_2 + \cdots + a_{nn}x_n = b_n.
      \end{array}
      \right.
    \end{equation}
    与二、三元线性方程组相类似,它的解可以用$n$阶行列式表示。
  \end{footnotesize}
\end{frame}


\begin{frame}
  \begin{footnotesize}
    \begin{block}{克莱姆法则}
      如果线性方程组(\ref{ls})的系数行列式不等于0,即
      $$
      D = \left|
      \begin{array}{ccc}
        a_{11}  & \cd  & a_{1n} \\
        \vd    &      & \vd  \\
        a_{n1}  & \cd  & a_{nn}
      \end{array}
      \right|\ne 0
      $$
      则方程组(\ref{ls})存在唯一解
      $$
      x_1 = \frac{D_1}D, \ x_2 = \frac{D_2} D, \ \cdots, \ x_n = \frac{D_n}D,
      $$
      其中
      \begin{center}
        \begin{tikzpicture}
          \matrix (M) [matrix of math nodes]  { 
            D_j = \\
          };
          \matrix(MM) [right=2pt of M, matrix of math nodes,nodes in empty cells,
            ampersand replacement=\&,left delimiter=|,right delimiter=|] {
            a_{11} \& \cd \& a_{1,j-1} \&  b_1 \& a_{1, j+1} \& \cd \& a_{1n} \\
            \vd   \&     \& \& \vd \& \vd \& \vd \& \& \vd\\       
            a_{n1} \& \cd \&  a_{n,j-1} \&  b_n \& a_{n, j+1} \& \cd \& a_{nn} \\
          };
          \node[below=7pt  of MM-3-4, blue]  {第$j$列};
        \end{tikzpicture}
      \end{center}
    \end{block}
  \end{footnotesize}
\end{frame}

\begin{frame}
  \begin{footnotesize}
    \proofname
    \red{先证存在性}: 
    将$x_i=\frac{D_i}D$代入第$i$个方程,则有
    $$
    \begin{array}{cl}
      & a_{i1}x_1 +\cd + a_{ii}x_i + \cd + a_{in}x_n \\[0.3cm]
      & \pause=  \disp \frac1D(a_{i1}\red{D_1} +\cd + a_{ii}\blue{D_i} + \cd + a_{in}\purple{D_n}) \\[0.3cm]
      & \pause=  \disp \frac1D \left[
        a_{i1}\red{(b_1 A_{11}+ \cd + b_nA_{n1})}
        +\cd+ a_{ii}\blue{(b_1 A_{1i} + \cd + b_nA_{ni})}  \right. \\[0.2cm]
     & \pause \left. \ \ \ \ \ \
        + \cd + a_{in}\purple{(b_1 A_{1n}+ \cd + b_nA_{nn})} \right]\\[0.3cm]
     &\pause =  \disp \frac1D \left[
        b_1(a_{i1} A_{11}+ a_{i2} A_{12}  \cd + a_{in}A_{1n}) 
        +\cd + b_i(a_{i1} A_{i1}+ a_{i2} A_{i2}  \cd + a_{in}A_{in})   \right. \\[0.2cm]
        & \pause \left. \ \ \ \ \ \ + \cd+ b_n(a_{i1} A_{n1}+ a_{i2} A_{n2}  \cd + a_{in}A_{nn}) \right] \\[0.3cm]
      &\pause =  \disp \frac 1D b_{i} D = b_i.
    \end{array}
    $$
  \end{footnotesize}
\end{frame}

\begin{frame} 
  \begin{footnotesize}
    \proofname\textbf{[续]}\\
    \red{再证唯一性}:设还有一组解$\purple{y_i, i = 1, 2, \cd, n}$,以下证明$\purple{y_i =D_i/D}$。
    \pause 现构造一个新行列式
    $$
    \begin{array}{rcl}
      y_1D &\pause = &\pause \left|
      \begin{array}{cccc}
        a_{11}y_1 & a_{12}    &  \cd   &   a_{1n}      \\[0.1cm] 
        a_{21}y_1 & a_{22}    &  \cd   &   a_{2n}      \\[0.1cm] 
        \vd    &    \vd     &        &     \vd      \\[0.1cm] 
        a_{n1}y_1 & a_{n2}    &  \cd   &   a_{nn}      
      \end{array}
      \right| \\[0.4in]
      & \pause\xlongequal{c_1 + y_2 c_2 + \cd + y_n c_n} &\pause
      \left|
      \begin{array}{cccc}
        \sum_{k=1}^n a_{1k}y_k & a_{12}    &  \cd   &   a_{1n}      \\[0.1cm] 
        \sum_{k=1}^n a_{2k}y_k & a_{22}    &  \cd   &   a_{2n}      \\[0.1cm] 
        \vd    &    \vd     &        &     \vd      \\[0.1cm] 
        \sum_{k=1}^n a_{2k}y_k & a_{n2}    &  \cd   &   a_{nn}      
      \end{array}
      \right|  \\[0.4in]
      & \pause=&\pause \left|
      \begin{array}{cccc}
       b_1 & a_{12}    &  \cd   &   a_{1n}      \\[0.1cm] 
       b_2 & a_{22}    &  \cd   &   a_{2n}      \\[0.1cm] 
       \vd    &    \vd     &        &     \vd      \\[0.1cm] 
       b_n & a_{n2}    &  \cd   &   a_{nn}      
      \end{array}
      \right| \pause = D_1
    \end{array}
    $$\pause
    所以$y_1 = D_1/D$。\pause同理可证$y_i=D_i/D, i = 2, \cd, n$。
  \end{footnotesize}
\end{frame}


\begin{frame}
  \begin{footnotesize}
    \uncover<1->{
      \begin{exampleblock}{例}
        \setlength{\arraycolsep}{1.0pt}
        $$\left\{
        \begin{array}{rcrcrcrcr}
          2x_1 &+ & x_2 &- & 5x_3&+ & x_4 &= & 8, \\[0.2cm]
          x_1 &- & 3x_2&  &     &- & 6x_4& = & 9, \\[0.2cm]
          &  & x_2 &- & x_3 &+ & 2x_4 &= & -5, \\[0.2cm]
          x_1 &+ & 4x_2&-& 7x_3 &+& 6x_4 &= & 0.
        \end{array}
        \right.
        $$
      \end{exampleblock}
    }
    \uncover<2->{
      \vspace{0.3cm}
      \textbf{解:}
      $$
      \begin{array}{ll}
        D &= \left|
        \begin{array}{rrrr}
          2 &  1 & -5 &  1\\
          1 & -3 &  0 & -6\\
          0 &  2 & -1 &  2\\
          1 &  4 & -7 &  6
        \end{array}
        \right|
        \xlongequal[r_4-r_2]{r_1-2r_2}
        \left|
        \begin{array}{rrrr}
          0 &  7 & -5 & 13\\
          1 & -3 &  0 & -6\\
          0 &  2 & -1 &  2\\
          0 &  7 & -7 & 12
        \end{array}
        \right|\\[0.8cm]
        & = - \left|
        \begin{array}{rrr}
          7  & -5 & 13\\
          2  & -1 &  2\\
          7  & -7 & 12
        \end{array}
        \right|
        \xlongequal[c_3+2c_2]{c_1+2c_2}
        - \left|
        \begin{array}{rrr}
          -3  & -5 &  3\\
          0  & -1 &  0\\
          -7 & -7 & -2
        \end{array}
        \right|\\[0.8cm]
        & =\left| 
        \begin{array}{rr}
          -3 &  3\\
          -7 & -2
        \end{array}
        \right| = 27.
      \end{array}
      $$
    }
  \end{footnotesize}
\end{frame}


\begin{frame}
  \begin{footnotesize}
    \textbf{解:}(续)
    $$
    \begin{array}{ll}
      D_1 = \left|
      \begin{array}{rrrr}
        \red{8}  &  1 & -5 &  1\\
        \red{9}  & -3 &  0 & -6\\
        \red{-5} &  2 & -1 &  2\\
        \red{0}  &  4 & -7 &  6
      \end{array}
      \right|= 81, &
      D_2 = \left|
      \begin{array}{rrrr}
        2 & \red{8}  & -5 &  1\\
        1 & \red{9}  &  0 & -6\\
        0 & \red{-5} & -1 &  2\\
        1 & \red{0}  & -7 &  6
      \end{array}
      \right|=-108,\\[0.8cm]
      D_3 = \left|
      \begin{array}{rrrr}
        2  &  1 & \red{8}  &  1\\
        1  & -3 & \red{9}  & -6\\
        0  &  2 & \red{-5} &  2\\
        1  &  4 & \red{0}  &  6
      \end{array}
      \right|= -27, &
      D_4 = \left|
      \begin{array}{rrrr}
        2 &  1 & -5 & \red{8} \\
        1 & -3 &  0 & \red{9} \\
        0 &  2 & -1 & \red{-5}\\
        1 &  4 & -7 & \red{0} 
      \end{array}
      \right|= 27
    \end{array}
    $$
    于是得
    $$
    x_1 = \frac{D_1}D = 3,  \ \ 
    x_2 = \frac{D_2}D = -4, \ \ 
    x_3 = \frac{D_3}D = -1, \ \ 
    x_4 = \frac{D_4}D = 1.
    $$
  \end{footnotesize}
\end{frame}


\begin{frame}
  \begin{footnotesize}
    \uncover<1->{
      \begin{exampleblock}{例}
        设曲线$y=a_0+a_1x + a_2 x^2 + a_3 x^3$通过四点$(1,3), (2,4), (3,3), (4,-3)$,
        求系数$a_0,a_1,a_2,a_3$。
      \end{exampleblock}
    }
    \uncover<2->{
      \vspace{0.3cm}
      \textbf{解:}依题意可得线性方程组

      $$
      \setlength{\arraycolsep}{1.0pt}
      \left\{
      \begin{array}{rcrcrcrcr}
        a_0 & + &  a_1 & + &  a_2 & + &   a_3 & = & 3, \\[0.2cm]
        a_0 & + & 2a_1 & + & 4a_2 & + &  8a_3 & = & 4, \\[0.2cm]
        a_0 & + & 3a_1 & + & 9a_3 & + & 27a_3 & = & 3, \\[0.2cm]
        a_0 & + & 4a_1 & + &16a_4 & + & 64a_3 & = & 3,
      \end{array}
      \right.
      $$
      其系数行列式为
      $$
      D = \left|
      \begin{array}{rrrr}
        1 & 1 &  1 &  1 \\
        1 & 2 &  4 &  8 \\
        1 & 3 &  9 & 27 \\
        1 & 4 & 16 & 64
      \end{array}
      \right|
      $$
      是一个范德蒙德行列式,其值为
      $$ 
      D = 1\cdot 2 \cdot 3 \cdot 1 \cdot 2 \cdot 1 = 12
      $$
    }

  \end{footnotesize}
\end{frame}


\begin{frame}
  \begin{footnotesize}
    \textbf{解:}(续)
    $$
    \begin{array}{ll}
      D_1 = \left|
      \begin{array}{rrrr}
        \red{3}  &  1 &  1 &  1\\
        \red{4}  &  2 &  4 &  8\\
        \red{3}  &  3 &  9 & 27\\
        \red{-3} &  4 & 16 & 64
      \end{array}
      \right|= 36, &
      D_2 = \left|
      \begin{array}{rrrr}
        1 & \red{3}  & 1  &  1\\
        1 & \red{4}  & 4  &  8\\
        1 & \red{3}  & 8  &  27\\
        1 & \red{-3} & 16 &  64
      \end{array}
      \right|=-18,\\[0.8cm]
      D_3 = \left|
      \begin{array}{rrrr}
        1 & 1 & \red{3}  &  1 \\
        1 & 2 & \red{4}  &  8 \\
        1 & 3 & \red{3}  & 27 \\
        1 & 4 & \red{-3} & 64
      \end{array}
      \right|= 24, &
      D_4 = \left|
      \begin{array}{rrrr}
        1 & 1 &  1 & \red{3} \\
        1 & 2 &  4 & \red{4} \\
        1 & 3 &  9 & \red{3} \\
        1 & 4 & 16 & \red{-3}
      \end{array}
      \right|= -6.
    \end{array}
    $$
    于是得
    $$
    a_0 = \frac{D_1}D = 3,  \ \ 
    a_1 = \frac{D_2}D = -3/2, \ \ 
    a_2 = \frac{D_3}D = 2, \ \ 
    a_3 = \frac{D_4}D = -1/2.
    $$
    即曲线方程为
    $$
    y = 3 - \frac 32 x + 2 x^2 - \frac 12 x^3.
    $$
  \end{footnotesize}
  
\end{frame}


\begin{frame}
  \begin{footnotesize}
    \begin{block}{定理}
      如果线性方程组(\ref{ls})的系数行列式$D\ne 0$,则(\ref{ls})一定有解,且解是惟一的。
    \end{block}

    \begin{block}{定理}
      如果线性方程组(\ref{ls})无解或有两个不同的解,则它的系数行列式必为0。
    \end{block}

    \vspace{0.3cm}
    \textbf{说明:} \\
    \begin{itemize}
    \item 
      线性方程组(\ref{ls})右端的常数项$b_1,b_2,\cd, b_n$不全为0时,
      线性方程组(\ref{ls})叫做\red{非齐次线性方程组}\\[0.2cm]
      当$b_1,b_2,\cd, b_n$全为0时,
      线性方程组(\ref{ls})叫做\red{齐次线性方程组}。
    \end{itemize}

  \end{footnotesize}
\end{frame}


\begin{frame}
  \begin{footnotesize}
    \begin{block}{齐次线性方程组}
      对于齐次线性方程组
      \begin{equation}\label{ls0}
        \left\{
        \begin{array}{l}
          a_{11}x_1 + a_{12}x_2 + \cdots + a_{1n}x_n = 0, \\[0.3cm]
          a_{21}x_1 + a_{22}x_2 + \cdots + a_{2n}x_n = 0, \\[0.3cm]
          \cd \\[0.2cm]
          a_{n1}x_1 + a_{n2}x_2 + \cdots + a_{nn}x_n = 0.
        \end{array}
        \right.
      \end{equation}
      一定有零解,但不一定有非零解。
    \end{block}
  \end{footnotesize}
\end{frame}

\begin{frame}
  \begin{footnotesize}
    \begin{block}{定理}
      如果齐次线性方程组(\ref{ls0})的系数行列式$D\ne 0$,则它没有非零解。
    \end{block}

    \vspace{1.0cm}
    
    \begin{block}{定理}
      如果齐次线性方程组(\ref{ls0})有非零解,则它的系数行列式必为0。
    \end{block}
    
  \end{footnotesize}
\end{frame}


\begin{frame}
  \begin{footnotesize}
    \uncover<1->{
      当$\lambda$为何值时,齐次线性方程组
      $$
      \left\{
      \begin{array}{rcrcrcr}
        (5-\lambda) x & + &           2 y & + & 2           z & = & 0, \\
        2           x & + & (6-\lambda) y &   &               & = & 0, \\
        2           x &   &               & + & (4-\lambda) z & = & 0.
      \end{array}
      \right.
      $$
      有非零解?
    }

    \uncover<2->{
      \vspace{0.3cm}
      \textbf{解:}由上述定理可知,若所给齐次线性方程组有非零解,则其系数行列式$D=0$。
      而
      $$
      \begin{array}{ll}
        D & = \left|
        \begin{array}{ccc}
          5-\lambda & 2         & 2  \\
          2         & 6-\lambda & 0  \\
          2         & 0         & 4-\lambda
        \end{array}
        \right| \\[0.8cm]
        & = (5-\lambda)(6-\lambda)(4-\lambda) - 4(4-\lambda) - 4(6-\lambda)\\[0.3cm]
        & = (5-\lambda)(2-\lambda)(8-\lambda),
      \end{array}
      $$
      故$\lambda=2$、$\lambda=5$或$\lambda=8$。
    }

  \end{footnotesize}
\end{frame}
