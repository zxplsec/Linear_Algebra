\subsection{$n$阶行列式的定义}

\begin{frame}
  \begin{block}{定义}
    由$n^2$个数$a_{ij}(i,j=1,2,\cdots,n)$组成的$n$阶行列式
    \begin{equation}\label{Dn}
      D = \left|
      \begin{array}{cccc}
        a_{11}  &  a_{12} & \cdots & a_{1n} \\
        a_{21}  &  a_{22} & \cdots & a_{2n} \\
        \vdots & \vdots & \ddots & \vdots\\  
        a_{n1}  &  a_{n2} & \cdots & a_{nn} 
      \end{array}
      \right|
    \end{equation}
    是一个数。
  \end{block}
\end{frame}

\begin{frame}
  \begin{block}{定义(续)}
    \begin{itemize}
    \item 当$n=1$时,定义$D=|a_{11}|=a_{11}$;\pause 
    \item 当$n\ge2$时,定义
      \begin{equation}
        D = a_{11} A_{11} + a_{12} A_{12} + \cdots + a_{1n} A_{1n},
      \end{equation}
      其中
      $$A_{1j} = (-1)^{1+j} M_{1j}$$
      而$M_{1j}$是$D$中划去第$1$行第$j$列后,按原顺序排成的$n-1$阶行列式,即
      $$
      M_{1j} =   \left|
      \begin{array}{cccccc}
        a_{21}  & \cdots&  a_{2,j-1}  &  a_{2,j+1}  & \cdots & a_{2n} \\
        a_{31}  & \cdots&  a_{3,j-1}  &  a_{3,j+1}  & \cdots & a_{3n} \\
        \vdots &       &  \vdots &  \vdots &  & \vdots\\  
        a_{n1}  & \cdots&  a_{n,j-1}  &  a_{n,j+1}  & \cdots & a_{nn} 
      \end{array}
      \right| \quad (j = 1,2,\cdots, n),
      $$
      并称$M_{1j}$为$a_{1j}$的余子式,$A_{1j}$为$a_{1j}$的代数余子式.
    \end{itemize}
  \end{block}

\end{frame}



\begin{frame}

  \begin{block}{注}
    \begin{enumerate}
    \item[1]
      在$D$中,$a_{11},a_{22},\cdots,a_{nn}$所在的对角线称为行列式的\red{主对角线},$a_{11},a_{22},\cdots,a_{nn}$称为\red{主对角元}。\\[0.1in]
    \item[2]
      行列式$D$是由$n^2$个元素构成的$n$次齐次多项式:\\[0.1in]
      \begin{itemize}
      \item 二阶行列式的展开式有$2!$项 \\[0.1in]
      \item 三阶行列式的展开式有$3!$项 \\[0.1in]
      \item $n$阶行列式的展开式有$n!$项,其中每一项都是不同行不同列的$n$个元素的乘积,带正号的项与带负号的项各占一半。
      \end{itemize}

    \end{enumerate}
    
  \end{block}
\end{frame}


\begin{frame}
  \begin{exampleblock}{例}
    证明:$n$阶下三角行列式
    $$
    D_n = \left|
    \begin{array}{cccc}
      a_{11}  &  0 & \cdots & 0 \\
      a_{21}  &  a_{22} & \cdots & 0 \\
      \vdots & \vdots & \ddots & \vdots\\  
      a_{n1}  &  a_{n2} & \cdots & a_{nn} 
    \end{array}
    \right| = a_{11} a_{22} \cdots a_{nn}.
    $$
  \end{exampleblock} \pause 

  \begin{block}{证明【数学归纳法】}
    \begin{itemize}
    \item 当$n=2$时,结论成立。\pause 
    \item 假设结论对$n-1$阶下三角阵成立,则由定义
      $$
      \begin{array}{rcl}
        \left|
        \begin{array}{cccc}
          a_{11}  &  0 & \cdots & 0 \\
          a_{21}  &  a_{22} & \cdots & 0 \\
          \vdots & \vdots & \ddots & \vdots\\  
          a_{n1}  &  a_{n2} & \cdots & a_{nn} 
        \end{array}
        \right| &=& \pause a_{11} \cdot (-1)^{1+1} \left|
        \begin{array}{cccc}
          a_{22}  &  0 & \cdots & 0 \\
          a_{31}  &  a_{33} & \cdots & 0 \\
          \vdots & \vdots & \ddots & \vdots\\  
          a_{n1}  &  a_{n2} & \cdots & a_{nn} 
        \end{array}
        \right| \\
        &=& \pause a_{11} (a_{22} a_{33} \cdots a_{nn}). \quad \qed        
      \end{array}
      $$    
    \end{itemize}
  \end{block}

\end{frame}

\begin{frame}
  $$
  \begin{array}{l}
    \left|
    \begin{array}{cccc}
      a_{11}  &  a_{12} & \cdots & a_{1n} \\
      a_{21}  &  a_{22} & \cdots & a_{2n} \\
      \vdots & \vdots & \ddots & \vdots\\  
      a_{n1}  &  a_{n2} & \cdots & a_{nn} 
    \end{array}
    \right| = 
    \left|
    \begin{array}{cccc}
      a_{11}  &  0 & \cdots & 0 \\
      0  &  a_{22} & \cdots & a_{2n} \\
      \vdots & \vdots & \ddots & \vdots\\  
      0  &  a_{n2} & \cdots & a_{nn} 
    \end{array}
    \right|  \\[0.4in]
    + \left|
    \begin{array}{ccccc}
      0  &  a_{12} & 0 & \cdots & 0 \\
      a_{21} & 0  &  a_{23} & \cdots & a_{2n} \\
      \vdots & \vdots & \vdots & \ddots & \vdots\\  
      a_{n1}  & 0&  a_{n3} & \cdots & a_{nn} 
    \end{array}
    \right| + \cdots  \\[0.4in]
    + 
        \left|
    \begin{array}{cccc}
      0 & \cdots & 0 & a_{1n} \\
      a_{21}  &   \cdots & a_{2,n-1} & 0 \\
      \vdots &  \ddots & \vdots & \vdots\\  
      a_{n1}  &   \cdots & a_{n,n-1} & 0
    \end{array}
    \right| 
  \end{array}
  $$
\end{frame}


\begin{frame}
  同理可证
  $$
  \left|
  \begin{array}{cccc}
    a_{11}  &  0 & \cdots & 0 \\
    0  &  a_{22} & \cdots & 0 \\
    \vdots & \vdots & \ddots & \vdots\\  
    0  &  0 & \cdots & a_{nn} 
  \end{array}
  \right| = a_{11}a_{22}\cdots a_{nn}
  $$

\end{frame}


\begin{frame}
  \begin{exampleblock}{例}
    计算$n$阶行列式
    $$
    \left|
    \begin{array}{ccccc}
      0  &  0 & \cdots & 0 & a_n \\
      0  &  0 & \cdots & a_{n-1} & * \\
      \vdots & \vdots & & \vdots & \vdots\\  
      0  &  a_2 & \cdots & * & * \\
      a_1 & * & \cdots & * & *
    \end{array}
    \right| 
    $$
  \end{exampleblock} \pause 

  \begin{block}{解}
    由行列式定义,
    $$
    \begin{array}{rcl}
      D_n &=& \left|
      \begin{array}{ccccc}
        0  &  0 & \cdots & 0 & a_n \\
        0  &  0 & \cdots & a_{n-1} & * \\
        \vdots & \vdots & & \vdots & \vdots\\  
        0  &  a_2 & \cdots & * & * \\
        a_1 & * & \cdots & * & *
      \end{array}
      \right| = \pause  (-1)^{1+n} a_n \left|
      \begin{array}{cccc}
        0  &  0 & \cdots &   a_{n-1} \\
        \vdots & \vdots &  & \vdots\\  
        0  &  a_2 & \cdots  & * \\
        a_1 & * & \cdots  & *
      \end{array} 
      \right| \\
      &=& \pause (-1)^{n-1} a_n D_{n-1}.      
    \end{array}
    $$
  \end{block}

\end{frame}


\begin{frame}
  \begin{block}{解(续)}
    同理递推,
    $$
    \begin{array}{rcl}
      D_n & =& (-1)^{n-1} a_n D_{n-1} \pause = (-1)^{n-1} a_n (-1)^{n-2} a_{n-1} D_{n-2} \\[0.1in]
      && \cdots\cdots \\[0.2cm]
      &=& \pause  (-1)^{(n-1)+(n-2)+\cdots+2+1} a_n a_{n-1} \cdots a_2 a_1 \\[0.1in]
      &=& \pause (-1)^{\frac{n(n-1)}{2}} a_n a_{n-1} \cdots a_2 a_1.
    \end{array}
    $$

  \end{block}

  \pause 
  例如,
  $$
  D_2 = -a_1a_2, \quad
  D_3 = -a_1a_2a_3, \quad
  D_4 = a_1a_2a_3a_4, \quad
  D_5 = a_1a_2a_3a_4a_5.
  $$

\end{frame}
