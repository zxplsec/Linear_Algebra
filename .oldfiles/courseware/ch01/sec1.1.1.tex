\subsection{二阶行列式}
%%%%%

\begin{frame}
    \begin{exampleblock}{引例}
      用消元法求解
      $$
      \left \lbrace
      \begin{array}{l}
        a_{11} x_1 + a_{12} x_2 = b_1, \\[0.2cm]
        a_{21} x_1 + a_{22} x_2 = b_2.
      \end{array}
      \right.
      $$
    \end{exampleblock}
    \pause 
    消去$x_2$得
    $$
    (a_{11}a_{22}-a_{12}a_{21})x_1 = b_1 a_{22} - b_2 a_{12},
    $$
    消去$x_1$得
    $$
    (a_{11}a_{22}-a_{12}a_{21})x_2 = b_2 a_{11} - b_1 a_{11}.
    $$
    \pause
    
    若$\boxed{\red{a_{11}a_{22}-a_{12}a_{21}\ne0}}$,则
    $$
    x_1 = \frac{b_1 a_{22} - b_2 a_{12}}{a_{11}a_{22}-a_{12}a_{21}}, \ \
    x_2 = \frac{b_2 a_{11} - b_1 a_{11}}{a_{11}a_{22}-a_{12}a_{21}}.
    $$
\end{frame}

\begin{frame}

    %%%%
    \begin{block}{二阶行列式}
      由$2^2=4$个数,按下列形式排成2行2列的方形
      $$
      \left|
      \begin{array}{cc}
        a_{11} & a_{12} \\[0.2cm]
        a_{21} & a_{22} 
      \end{array}
      \right|,
      $$
      其被定义成一个数
      $$
      \left|
      \begin{array}{cc}
        a_{11} & a_{12} \\[0.2cm]
        a_{21} & a_{22} 
      \end{array}
      \right| = a_{11}a_{22} - a_{12}a_{21} \equiv D,
      $$
      该数称为由这四个数构成的二阶行列式。
    \end{block}

\end{frame}

\begin{frame}

    %%%%
    \begin{itemize}
    \item 
      $\red{a_{ij}}$表示行列式的元素。\\[0.2cm]
      $i$为行标,表明该元素位于第$i$行;\\[0.2cm]
      $j$为列标,表明该元素位于第$j$列。\\[0.4cm]
    \item
      对角线法则\\
      \begin{center}
        \begin{tikzpicture}
          \matrix (A) [matrix of nodes,ampersand replacement=\&,row sep=15pt,column sep=15pt,left delimiter=|,
          right delimiter=|] {
            $a_{11}$ \& $a_{12}$  \\
            $a_{21}$ \& $a_{22}$  \\
          };
          \draw[blue, thick] (A-1-1.south east) -- (A-2-2.north west);
          \draw[red,  thick] (A-1-2.south west) -- (A-2-1.north east);
        \end{tikzpicture}
      \end{center}
    \end{itemize}
\end{frame}

\begin{frame}

    %%%%%%
    \begin{itemize}
    \item 
      类似地,
      $$
      \begin{array}{l}
        b_1 a_{22} - b_2 a_{12} = \left|
        \begin{array}{cc}
          b_1 & a_{12} \\
          b_2 & a_{22} 
        \end{array}
        \right|  \equiv D_1\\[0.4cm]
        b_2 a_{11} - b_1 a_{21} = \left|
        \begin{array}{cc}
          a_{11} & b_1 \\
          a_{21} & b_2
        \end{array}
        \right|  \equiv D_2\\
      \end{array}
      $$      
      则上述方程组的解可表示为
      $$
      x_1 = \frac{D_1}{D},\ \
      x_2 = \frac{D_2}{D}.
      $$
    \end{itemize}


\end{frame}


\begin{frame}
  
    \uncover<1->{
      \begin{block}{例1}
        求解二元线性方程组
        $$
        \left\{
        \begin{array}{l}
          3x_1 - 2x_2 = 12, \\[0.2cm]
          2x_1 + x_2  = 1.
        \end{array}
        \right.
        $$
      \end{block}
    }
    \uncover<2->{
      解:因为
      $$
      \begin{array}{l}
        D = \left|
        \begin{array}{cc}
          3 & -2 \\
          2 & 1 
        \end{array}
        \right| = 7 \ne 0,\\[0.4cm]
        D_1 = \left|
        \begin{array}{cc}
          12 & -2 \\
          1 & 1 
        \end{array}
        \right| = 14 , \\[0.4cm]
        D_2 = \left|
        \begin{array}{cc}
          3 & 12 \\
          2 & 1 
        \end{array}
        \right| = -21,
      \end{array}
      $$
      因此,
      $$
      x_1=\frac{D_1}{D}=2, \ \ x_2 = \frac{D_2}{D} = -3.
      $$
    }

\end{frame}
