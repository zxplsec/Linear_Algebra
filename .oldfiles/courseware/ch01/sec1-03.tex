%%%%%
\section{$n$阶行列式的定义}



\begin{frame}

  \begin{overprint}
    \onslide<1-2>
    \begin{center}
      \blue{$n$阶行列式的定义}\vspace{0.5cm}
    \end{center}
  \end{overprint}
  
  \begin{overprint}
    \onslide<1>
    \begin{block}{$n$阶行列式}
      由$n^2$个数组成的$n$行$n$列的$n$阶行列式定义如下
      \begin{footnotesize}
        $$
        \left|
        \begin{array}{ccccc}
          a_{11} & a_{12} & \cd & a_{1n} \\[0.2cm]
          a_{21} & a_{22} & \cd & a_{2n} \\[0.2cm]
          \vd   &  \vd  &     & \vd   \\[0.2cm]
          a_{n1} & a_{n2} & \cd & a_{nn} 
        \end{array}
        \right|
        = \sum_{n!} (-1)^{\tau(p_1p_2\cdots p_n)} a_{1p_1}a_{2p_2}\cd a_{np_n}.
        $$
      \end{footnotesize}
    \end{block}

    \begin{itemize}
    \item \red{$\disp \sum_{n!}--$} 对所有$n$阶排列$p_1p_2\cdots p_n$的种数进行相加,共有$P_n=n!$项\\[0.3cm]
    \item \red{$\tau(p_1p_2\cdots p_n)--$} $n$阶排列$p_1p_2\cdots p_n$的逆序数 \\[0.3cm]
    \item 记为\red{$D_n$} 或 \red{$\mathrm{det}(a_{ij})$} \\[0.3cm]
    \end{itemize}
    
    %%%
    \onslide<2>

    \blue{注意} \\
    \begin{itemize}
    \item 
      $n$阶行列式的定义具有与三阶行列式类似的特征。\\[0.3cm]
    \item
      定义一阶行列式($n=1$)为
      $$
      |a_{11}|=a_{11}
      $$
      (注意不要与绝对值符号相混淆)。
    \end{itemize}
  \end{overprint}
\end{frame}

\begin{frame}
    %%%
  \uncover<1->{
    \begin{exampleblock}{例5}
      在六阶行列式中,项$a_{23}a_{31}a_{42}a_{56}a_{14}a_{65}$应带哪种符号?
    \end{exampleblock}
  }
  
  \uncover<2->{
    \vspace{0.5cm}
    \textbf{解}: 重新排序为$a_{14}a_{23}a_{31}a_{42}a_{56}a_{65}$,而
    $$
    \tau(431265) = 0 + 1 + 2 + 2 + 0 + 1 = 6,
    $$
    故符号为正。
  }
\end{frame}

\begin{frame}
    %%%
  \uncover<1->{
    \begin{exampleblock}{例6}
      \begin{footnotesize}
        证明
        $$
        \left|
        \begin{array}{cccc}
          \lambda_1 &&&\\
          &\lambda_2&&\\
          &&\dd& \\
          &&&\lambda_n
        \end{array}
        \right| = \lambda_1\cdots\lambda_n, \ \ 
        \left|
        \begin{array}{cccc}
	  &&&\lambda_1 \\
          && \lambda_2&\\
          &  \dd      && \\
          \lambda_n   &&&
        \end{array}
        \right| = (-1)^{\frac{n(n-1)}2}\lambda_1\cdots\lambda_n        
        $$
      \end{footnotesize}
    \end{exampleblock}
  }
  
  \uncover<2->{
    \begin{footnotesize}
      \proofname
      第一式显然成立,下证第二式。记$\lambda_i = a_{i,n-i+1}$,则
      $$
      \begin{array}{ll}
        \left|
        \begin{array}{cccc}
	  &&&\lambda_1 \\
          && \lambda_2&\\
          &  \dd      && \\
          \lambda_n   &&&
        \end{array}
        \right| 
        &= 
        \left|
        \begin{array}{cccc}
	  &&&a_{1n} \\
          &&a_{2,n-1}&\\
          &  \dd   && \\
          a_{n1}    &&&
        \end{array}
        \right| \\[0.5cm]
        & =(-1)^{\tau(n\cdot(n-1)\cdot \cdots \cdot 1)} a_{1n}a_{2,n-1}\cdots a_{n1}\\[0.2cm]
        & =(-1)^{\frac{n(n-1)}2}\lambda_1\cdots\lambda_n        
      \end{array}
      $$
    \end{footnotesize}      
  }
\end{frame}


\begin{frame}
  \uncover<1->{
    \begin{exampleblock}{例7}
      \begin{footnotesize}
        证明
        $$
        \left|
        \begin{array}{cccc}
          a_{11}&&&\\
          a_{21}&a_{22}&&\\
          \vd&\vd&\dd& \\
          a_{n1}&a_{n2}&\cd&a_{nn}
        \end{array}
        \right| =a_{11}\cdots a_{nn}
        $$
      \end{footnotesize}
    \end{exampleblock}
  }
  \uncover<2->{
    \begin{footnotesize}
      \proofname
      注意当$j>i$时,$a_{ij}=0$,故$D$中可能不为0的元素$a_{ip_i}$的下标满足$p_i\le i$,即\red{$p_1\le 1, p_2 \le 2, \cdots, p_n \le n$}。\\[0.3cm]
      在所有排列$p_1p_2\cdots p_n$中,能满足上述关系的排列只有一个自然排列$12\cdots n$,所以$D$中可能不为0的项只有
      $$
      (-1)^{\tau(12\cdots n)} a_{11} a_{22} \cdots a_{nn} = a_{11} a_{22} \cdots a_{nn}.
      $$
    \end{footnotesize}
  }
\end{frame}



\begin{frame}
  类似地,
  \begin{footnotesize}
    $$
    \left|
    \begin{array}{cccc}
      a_{11} & a_{12} & \cd & a_{1n}\\
            & a_{22} &     & a_{2n}\\
            &       & \dd & \vd  \\
            &       &     &a_{nn}
    \end{array}
    \right| =a_{11}\cdots a_{nn}
    $$
  \end{footnotesize}
\end{frame}

\begin{frame}
  \uncover<1->{
    \begin{exampleblock}{例8}
      \begin{footnotesize}
        证明
        $$
        \left|
        \begin{array}{cccc}
          a_{11} & 0     & \cd & 0     \\
          a_{21} & a_{22} & \cd & a_{2n} \\
          \vd   &\vd    & \dd & \vd   \\
          a_{n1} & a_{n2} & \cd & a_{nn}
        \end{array}
        \right| =a_{11}
        \left|
        \begin{array}{ccc}
           a_{22} & \cd & a_{2n} \\
           \vd   & \dd & \vd   \\
           a_{n2} & \cd & a_{nn}
        \end{array}
        \right|
        $$
      \end{footnotesize}
    \end{exampleblock}
  }
  \uncover<2->{
    \proofname
    \begin{footnotesize}
    由定义
    $$
    D = \sum_{n!} (-1)^{\tau(p_1p_2\cdots p_n)} a_{1p_1}a_{2p_2}\cd a_{np_n}
    $$
    可知,只有当$p_1$取为1,$a_{11}a_{2p_2}\cd a_{np_n}$才有可能不为0,这些可能不为0的项有$(n-1)!$。
    当$p_1$取为1时,$p_2\cdots p_n$只能在$2, \cdots, n$中取值,又$\tau(1p_2\cdots p_n) = \tau(p_2\cdots p_n)$,故
    $$
    \begin{array}{ll}
    D & \disp = \sum_{(n-1)!} (-1)^{\tau(1p_2\cdots p_n)} a_{11}a_{2p_2}\cd a_{np_n} \\[0.4cm]
      & \disp = a_{11} \sum_{(n-1)!} (-1)^{\tau(p_2\cdots p_n)} a_{2p_2}\cd a_{np_n} \\[0.4cm]
      & \disp = 
        a_{11}
        \left|
        \begin{array}{ccc}
           a_{22} & \cd & a_{2n} \\
           \vd   & \dd & \vd   \\
           a_{n2} & \cd & a_{nn}
        \end{array}
        \right|
    \end{array}
    $$
      
    \end{footnotesize}
  }
\end{frame}
