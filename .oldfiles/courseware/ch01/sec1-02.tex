\section{全排列与逆序数}
%%%%%

\begin{frame}
  \begin{overprint}
    \onslide<1->
    \begin{center}
      \blue{全排列与逆序数}\vspace{0.3cm}
    \end{center}
  \end{overprint}
  
  \begin{overprint}
    %%%%
    \onslide<1>
    \begin{exampleblock}{引例}
      用1、2、3三个数,可以组成多少个没有重复数字的三位数?
    \end{exampleblock}
    利用乘法原理可知,这样的三位数有$3\times2\times1=3!=6$个。
    
    \begin{block}{全排列}
      把$n$个不同的元素排成一列,叫做这$n$个元素的全排列。$n$个不同元素的所有排列的种数,用$P_n$表示。
    \end{block}
    \begin{center}
      \red{$\boxed{P_n=n!}$}      
    \end{center}
    %%%%
    \onslide<2>
    \begin{block}{逆序数}
      在一个排列中,如果一对数的前后位置与大小顺序相反,即前面的数大于后面的数,则称它们为一个逆序。
      一个排列中逆序的总数称为该排列的逆序数。
    \end{block}
    \begin{itemize}
    \item  \blue{举例}\\[0.2cm]
      $2431$中,$21, 43, 41, 31$为逆序,逆序为4。\\[0.4cm]
    \item
      逆序数为奇数的排列叫奇排列,
      逆序数为偶数的排列叫偶排列。
    \end{itemize}
    %%%%
    \onslide<3>
    \shadowbox{逆序数的计算}
    
    \vspace{0.2cm}
    设$p_1p_2\cdots p_n$为一个排列
    \begin{itemize}
    \item
      定义该排列中某个元素$p_i$的逆序数为:
      \red{在$p_1p_2\cdots p_{i-1}$中比$p_i$大的个数,记为$t_i$。} \\[0.4cm]
    \item
      该排列的逆序数为
      $$
      \tau(p_1p_2\cdots p_n) = t_1 + t_2 + \cdots t_n.
      $$
    \end{itemize}
    
  \end{overprint}


\end{frame}


\begin{frame}
  \uncover<1->{
    \begin{block}{例4}
      计算$32514$与$n, n-1, n-2, \cdots, 2, 1$的逆序数。      
    \end{block}
  }

  \uncover<2->{
    解: 
    $$
    \begin{array}{l}
      \tau(32514) = 0 + 1+ 0 + 3 + 1  = 5\\[0.2cm]
      \tau(n, n-1, n-2, \cdots, 2, 1) = 0 + 1 + 2 + \cdots + (n-1) = \frac{n(n-1)}2.
    \end{array}
    $$
  }
\end{frame}
