\subsection{三阶行列式}
%%%%%

\begin{frame}

  
  \begin{overprint}
    %%%%
    \onslide<1>
    \begin{block}{三阶行列式}
      由$3^2=9$个数组成的3行3列的三阶行列式,则按如下形式定义一个数
      $$ 
      D_3 = 
      \left|
      \begin{array}{ccc}
        a_{11} & a_{12} & a_{13} \\[0.2cm]
        a_{21} & a_{22} & a_{23} \\[0.2cm]
        a_{31} & a_{32} & a_{33} 
      \end{array}
      \right|
      = 
      \begin{array}{l}
          a_{11}a_{22}a_{33} + a_{12}a_{23}a_{31} + a_{13}a_{21}a_{32} \\[0.2cm]
        - a_{13}a_{22}a_{31} - a_{11}a_{23}a_{32} - a_{12}a_{21}a_{33}
      \end{array}
      $$

    \end{block}
  \end{overprint}
\end{frame}



\begin{frame}
 
    \begin{itemize}
    \item
      沙路法\\

       
              \begin{tikzpicture}               
                
                \only<1->{
                   \matrix(A) [matrix of math nodes,nodes in empty cells,ampersand replacement=\&,row sep=15pt,column sep=15pt] {
                       \&  a_{11} \& a_{12} \& a_{13}  \& \red{a_{11}} \& \red{a_{12}} \&\\
                       \&  a_{21} \& a_{22} \& a_{23}  \& \red{a_{21}} \& \red{a_{22}} \&\\
                       \&  a_{31} \& a_{32} \& a_{33}  \& \red{a_{31}} \& \red{a_{32}} \&\\
                   -   \&    -        \&   -        \&               \&   +         \&   +        \& +\\
                };
                }
                \only<2->{
                 \draw[blue,thick] (A-1-2.center) -- (A-4-5.center);
                 \draw[blue,thick] (A-1-3.center)  -- (A-4-6.center);
                 \draw[blue,thick] (A-1-4.center) -- (A-4-7.center); 
                }
                \only<3->{
                 \draw[red,thick,->] (A-1-6.center) -- (A-4-3.center);
                 \draw[red,thick,->] (A-1-5.center) -- (A-4-2.center);
                 \draw[red,thick,->] (A-1-4.center) -- (A-4-1.center); 
                }

              \end{tikzpicture}          
    \end{itemize}
\end{frame}


\begin{frame}
  \uncover<1->{
    \begin{block}{例3}
      计算
      $$
      D_3 = 
      \left |
      \begin{array}{ccc}
        1  & 2 & -4 \\ 
        -2 & 2 & 1  \\
        -3 & 4 & -2
      \end{array}
      \right|
      $$
    \end{block}
  }
  \uncover<2->{
    解:由对角线法则可知,
    $$
    \begin{array}{ll}
      D_3 &=   1\times   2  \times (-2) +   2  \times 1 \times (-3) + (-2) \times 4 \times (-4)\\[0.2cm]
      & - 2\times (-2) \times (-2) - (-4) \times 2 \times (-3) +   1  \times 1 \times   4\\[0.2cm]
      & = -14.
    \end{array}
    $$
  }
\end{frame}


\begin{frame}
  \uncover<1->{
    \begin{block}{例3}
      求方程
      $$
      \left |
      \begin{array}{ccc}
        1  & 1 & 1 \\
        2  & 3 & x  \\
        4  & 9 & x^2
      \end{array}
      \right| = 0
      $$      
    \end{block}
  }
  \uncover<2->{
    解:行列式
    $$ 
    D = 3x^2 + 18 + 4x - 2x^2 - 12 - 9x 
    = x^2 - 5x + 6
    $$
    由此可知$x=2$或$3$。
  }
\end{frame}


\begin{frame}
  如果三元一次方程组
  $$
  \begin{array}{c}  
    a_{11}x_1 + a_{12}x_2 + a_{13}x_3 = b_1, \\
    a_{21}x_1 + a_{22}x_2 + a_{23}x_3 = b_2, \\
    a_{31}x_1 + a_{32}x_2 + a_{33}x_3 = b_3,
  \end{array}
  $$
  的系数行列式
  $$
  D = \left|
  \begin{array}{ccc}
    a_{11} & a_{12} & a_{13}\\
    a_{21} & a_{22} & a_{23}\\
    a_{31} & a_{32} & a_{33}
  \end{array}
  \right| \ne 0
  $$
  则用消元法求解可得
  $$
  x_1 = \frac{D_1}{D}, \quad
  x_2 = \frac{D_2}{D}, \quad
  x_3 = \frac{D_3}{D}, \quad
  $$
  其中
  $$
  D_1 = \left|
  \begin{array}{ccc}
    b_1 & a_{12} & a_{13}\\
    b_2 & a_{22} & a_{23}\\
    b_3 & a_{32} & a_{33}
  \end{array}
  \right|, \
  D_2 = \left|
  \begin{array}{ccc}
    a_{11} & b_1 & a_{13}\\
    a_{21} & b_2 & a_{23}\\
    a_{31} & b_3 & a_{33}
  \end{array}
  \right|, \
  D_3= \left|
  \begin{array}{ccc}
    a_{11} & a_{12} & b_1 \\
    a_{21} & a_{22} & b_2 \\
    a_{31} & a_{32} & b_3 
  \end{array}
  \right|.
  $$

\end{frame}


\begin{frame}
  从二、三阶行列式的展开式中可发现:
  $$
  \begin{array}{l}
    D  =  \left|
    \begin{array}{ccc}
      a_{11} & a_{12} & a_{13}\\
      a_{21} & a_{22} & a_{23}\\
      a_{31} & a_{32} & a_{33}
    \end{array}
    \right| \\[0.6cm]
     = 
    a_{11}a_{22}a_{33}+a_{12}a_{23}a_{31}+a_{13}a_{21}a_{32}
    -a_{13}a_{22}a_{31}-a_{12}a_{21}a_{33}-a_{11}a_{23}a_{32} \\[0.3cm]\pause 
     = 
    a_{11}(a_{22}a_{33}-a_{23}a_{32})-
    a_{12}(a_{21}a_{33}-a_{23}a_{31})+
    a_{13}(a_{21}a_{32}-a_{22}a_{31}) \\[0.3cm] \pause 
     = 
    a_{11} \underbrace{\left| \begin{array}{ccc} a_{22} & a_{33} \\ a_{23} & a_{32} \end{array} \right|}_{M_{11}} -
    a_{12} \underbrace{\left| \begin{array}{ccc} a_{21} & a_{23} \\ a_{31} & a_{33} \end{array} \right|}_{M_{12}} +
    a_{13} \underbrace{\left| \begin{array}{ccc} a_{21} & a_{22} \\ a_{31} & a_{32} \end{array} \right|}_{M_{13}}
  \end{array}
  $$
  \pause 
  这里,$M_{11},M_{12},M_{13}$分别称为$a_{11},a_{12},a_{13}$的\red{余子式},并称
  $$
  A_{11} = (-1)^{1+1} M_{11}, \quad
  A_{12} = (-1)^{1+2} M_{12}, \quad
  A_{13} = (-1)^{1+3} M_{13}
  $$
  分别称为$a_{11},a_{12},a_{13}$的\red{代数余子式}。\pause 这样,$D$可表示为
  $$
  D= a_{11}A_{11} + a_{11}A_{13} + a_{13}A_{13}.
  $$
\end{frame}


\begin{frame}
  同样地,
  $$
  D = \left| \begin{array}{ccc} a_{11} & a_{12} \\ a_{21} & a_{22} \end{array} \right|
  = a_{11} A_{11} + a_{12} A_{12},
  $$
  其中
  $$
  A_{11} = (-1)^{1+1}|a_{22}| =  a_{22},\quad
  A_{11} = (-1)^{1+2}|a_{21}| = -a_{21}.
  $$
  \pause 
  注意这里的$|a_{22}|,~|a_{21}|$是一阶行列式,而不是绝对值。

  \pause 
  \vspace{0.1in}
  \red{我们把一阶行列式$|a|$定义为$a$。}

\end{frame}
