\subsection{标准正交基}

\begin{frame}
  \begin{footnotesize}
    \begin{block}{定理}
      $\RRR^n$中两两正交且不含零向量的向量组(称为非零正交向量组)$\alphabd_1,\alphabd_2,\cd,\alphabd_s$是线性无关的。
    \end{block}
    \pause\proofname
    设
    $$
    k_1\alphabd_1+k_2\alphabd_2+\cd+k_s\alphabd_s=0,
    $$
    则
    $$
    (k_1\alphabd_1+k_2\alphabd_2+\cd+k_s\alphabd_s,\alpha_j)=0, \quad j=1,2,\cd,s,
    $$
    即
    $$
   k_j(\alphabd_j,\alphabd_j)=0, \quad j=1,2,\cd,s.
    $$
    由于$(\alphabd_j,\alphabd_j)>0$,故
    $$
    k_j=0, \quad j=1,2,\cd,s.
    $$
    因此,$\alphabd_1,\alphabd_2,\cd,\alphabd_s$线性无关。
  \end{footnotesize}
\end{frame}


\begin{frame}
  \begin{footnotesize}
    \begin{block}{定义(标准正交基)}
      设$\alphabd_1,\alphabd_2,\cd,\alphabd_n\in \RRR^n$,若
      $$
      (\alphabd_i,\alphabd_j)=\delta_{ij}=\left\{
      \begin{array}{ll}
        1,& i=j,\\
        0,& i\ne j.
      \end{array}
      \right. \quad i,j=1,2,\cd,n.
      $$
      则称$\{\alphabd_1,\alphabd_2,\cd,\alphabd_n\}$是$\RRR^n$中的一组标准正交基。
    \end{block}
  \end{footnotesize}
\end{frame}


\begin{frame}
  \begin{footnotesize}
    \begin{exampleblock}{例1}
      设$B=(\alphabd_1,\alphabd_2,\cd,\alphabd_n)$是$\RRR^n$中的一组标准正交基,求$\RRR^n$中向量$\betabd$在基$B$下的坐标。
    \end{exampleblock}
    \pause\jiename
    $$
    \begin{array}{rl}
    & \betabd=x_1\alphabd_1+x_2\alphabd_2+\cd+x_n\alphabd_n\\[0.1in]
    \Longrightarrow&   (\betabd,\alphabd_j)=(x_1\alphabd_1+x_2\alphabd_2+\cd+x_n\alphabd_n,\alphabd_j)=x_j(\alphabd_j,\alphabd_j)\\[0.1in]
     \Longrightarrow& \ds x_j =\frac{ (\betabd,\alphabd_j)}{(\alphabd_j,\alphabd_j)}
    \end{array}
     $$
    
  \end{footnotesize}
\end{frame}
