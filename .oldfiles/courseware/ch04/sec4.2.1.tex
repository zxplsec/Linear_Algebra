\subsection{$\RRR^n$中向量的内积,欧式空间}
\begin{frame}
  \begin{footnotesize}
    \begin{block}{定义}
      在$\RRR^n$中,对于$\alphabd=(a_1,a_2,\cd,a_n)^T$和$\betabd=(b_1,b_2,\cd,b_n)^T$,规定$\alphabd$和$\betabd$的内积为
      $$
      (\alphabd,\betabd)=a_1b_1+a_2b_2+\cd+a_nb_n.
      $$
    \end{block}
    当$\alphabd$和$\betabd$为列向量时,
    $$
    (\alphabd,\betabd)=\alphabd^T\betabd=\betabd^T\alphabd.
    $$
  \end{footnotesize}
\end{frame}

\begin{frame}
  \begin{footnotesize}
    \begin{block}{内积的运算性质}
      对于$\alphabd,\betabd,\gammabd\in\RRR^n$和$k\in\RRR$,
      \begin{itemize}
      \item[(i)]   $(\alphabd,\betabd)=(\betabd,\alphabd)$
      \item[(ii)]  $(\alphabd+\betabd,\gammabd)=(\alphabd,\gammabd)+(\betabd,\gammabd)$
      \item[(iii)] $(k\alphabd,\betabd)=k(\alphabd,\betabd)$
      \item[(iv)]  $(\alphabd,\alphabd)\ge0$, 等号成立当且仅当$\alphabd=\zero$.
      \end{itemize}
    \end{block}
    \pause
    \begin{block}{定义(向量长度)}
      向量$\alphabd$的长度定义为
      $$
      \|\alphabd\|=\sqrt{(\alphabd,\alphabd)}
      $$
    \end{block}
  \end{footnotesize}
\end{frame}


\begin{frame}
  \begin{footnotesize}
    \begin{block}{定理(柯西-施瓦茨(Cauchy-Schwarz)不等式)}
      $$
      |(\alphabd,\betabd)|\le\|\alphabd\|\|\betabd\|
      $$
    \end{block}
    \pause 
    \proofname
    $\forall t \in \RRR$,有
    $$
    (\alphabd+t\betabd,\alphabd+t\betabd) \ge 0
    $$
    即
    $$
    (\betabd,\betabd)t^2+2(\alphabd,\betabd)t+(\alphabd,\alphabd)\ge0
    $$
    此为关于$t$的二次函数,由一元二次方程理论可知
    $$
    \Delta = b^2-4ac = 4 (\alphabd,\betabd)^2-4(\alphabd,\alphabd)(\betabd,\betabd)\le 0
    $$
    即
    $$
    (\alphabd,\betabd)^2\le (\alphabd,\alphabd)(\betabd,\betabd)
    $$
    亦即
    $$
    |(\alphabd,\betabd)|\le\|\alphabd\|\|\betabd\|
    $$
  \end{footnotesize}
\end{frame}

\begin{frame}
  \begin{footnotesize}
    \begin{block}{定义(向量之间的夹角)}
    向量$\alphabd,\betabd$之间的夹角定义为
    $$
    <\alphabd,\betabd>=\arccos\frac{(\alphabd,\betabd)}{\|\alphabd\|\|\betabd\||}
    $$
    \end{block}
    \pause
    \begin{block}{定理}
    $$\alphabd\perp\betabd ~~\Longleftrightarrow~~
   (\alphabd,\betabd)=0
    $$
    \end{block}
    \pause
    注意:零向量与任何向量的内积为零,从而零向量与任何向量正交。
  \end{footnotesize}
\end{frame}



\begin{frame}
  \begin{footnotesize}
    \begin{block}{定理(三角不等式)}
      $$
      \|\alphabd+\betabd\|\le\|\alphabd\|+\|\betabd\|.
      $$
    \end{block}
    \pause\proofname
    $$
    \begin{array}{rl}
      (\alphabd+\betabd,\alphabd+\betabd)
      &= (\alphabd,\alphabd)+2(\alphabd,\betabd)+(\betabd,\betabd)\\[0.1in]
      &\le (\alphabd,\alphabd)+2|(\alphabd,\betabd)|+(\betabd,\betabd) \\[0.1in]
      &\le \|\alphabd\|^2+2\|\alphabd\|\|\betabd\|+\|\betabd\|^2 \\[0.1in]
    \end{array}
    $$
    
    \pause
    注意:当$\alphabd\perp\betabd$时,$\|\alphabd+\betabd\|=\|\alphabd\|+\|\betabd\|$。
  \end{footnotesize}
\end{frame}


\begin{frame}
  \begin{footnotesize}
    \begin{block}{定义(欧几里得空间)}
      定义了内积运算的$n$维实向量空间,称为$n$维欧几里得空间(简称欧氏空间),仍记为$\RRR^n$。
    \end{block}
  \end{footnotesize}
\end{frame}

\begin{frame}
  \begin{footnotesize}
    
  \end{footnotesize}
\end{frame}


\begin{frame}
  \begin{footnotesize}
    
  \end{footnotesize}
\end{frame}


\begin{frame}
  \begin{footnotesize}
    
  \end{footnotesize}
\end{frame}


\begin{frame}
  \begin{footnotesize}
    
  \end{footnotesize}
\end{frame}
