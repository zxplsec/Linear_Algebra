\subsection{施密特(Schmidt)正交化方法}
\begin{frame}
  \begin{footnotesize}
    \begin{block}{目标}
      从一组线性无关的向量$\alphabd_1,\alphabd_2,\cd,\alphabd_n$出发,构造一组标准正交向量组。
    \end{block}
  \end{footnotesize}
\end{frame}


\begin{frame}
  \begin{footnotesize}
    \begin{block}{施密特(Schmidt)正交化过程}
      给定$\RRR^n$中的线性无关组$\alphabd_1,\alphabd_2,\cd,\alphabd_n$,
      \begin{itemize}
      \item[(1)] 取$\betabd_1=\alphabd_1$
      \item[(2)] 令$\betabd_2=\alphabd_2+k_{21}\betabd_1$,
        $$
        \begin{array}{rl}
          &\betabd_2\perp\betabd_1\\[0.2cm]
          \Longrightarrow&(\betabd_1,\betabd_2)=0\\[0.2cm]
          \Longrightarrow& \ds k_{21}=-\frac{(\alphabd_2,\betabd_1)}{(\betabd_1,\betabd_1)}
        \end{array}
        $$
        故
        $$
        \betabd_2=\alphabd_2-\frac{(\alphabd_2,\betabd_1)}{(\betabd_1,\betabd_1)}\betabd_1
        $$
      \item[(3)] 令$\betabd_3=\alphabd_3+k_{31}\betabd_1+k_{32}\betabd_2$,
        $$
        \begin{array}{rll}
          &\betabd_3\perp\betabd_i, & i=1,2\\[0.2cm]
          \Longrightarrow&(\betabd_3,\betabd_i)=0, & i=1,2\\[0.2cm]
          \Longrightarrow& \ds k_{3i}=-\frac{(\alphabd_3,\betabd_i)}{(\betabd_i,\betabd_i)} & i=1,2
        \end{array}
        $$
      \end{itemize}
    \end{block}

  \end{footnotesize}
\end{frame}


\begin{frame}
  \begin{footnotesize}
    \begin{block}{施密特(Schmidt)正交化过程}
      
      \begin{itemize}
      \item[(4)] 继续以上步骤,假设已经求出两两正交的非零向量$\betabd_1,\betabd_2,\cd,\betabd_{j-1}$,
        取
        $$
        \betabd_j=\alphabd_j+k_{j1}\betabd_1+k_{j2}\betabd_2+\cd++k_{j,j-1}\betabd_{j-1},
        $$
        $$
        \begin{array}{rl}
          & \betabd_j \perp \betabd_i\quad (i=1,2,\cd,j-1)\\[0.2cm]
          \Longrightarrow &(\betabd_j,\betabd_i)=0 \quad (i=1,2,\cd,j-1)\\[0.2cm]
          \Longrightarrow &(\alphabd_j+k_{ji}\betabd_i,\betabd_i)=0 \quad (i=1,2,\cd,j-1)\\[0.4cm]
          \Longrightarrow& \ds k_{ji}=-\frac{(\alphabd_j,\betabd_i)}{(\betabd_i,\betabd_i)}
        \end{array}
        $$
        故
        $$
        \betabd_j=\alphabd_j-\frac{(\alphabd_j,\betabd_1)}{(\betabd_1,\betabd_1)}\betabd_1
        -\frac{(\alphabd_j,\betabd_2)}{(\betabd_2,\betabd_2)}\betabd_2
        -\cd
        -\frac{(\alphabd_j,\betabd_{j-1})}{(\betabd_{j-1},\betabd_{j-1})}\betabd_{j-1}.
        $$
      \end{itemize}
    \end{block}
  \end{footnotesize}
\end{frame}


\begin{frame}
  \begin{footnotesize}
    
  \end{footnotesize}
\end{frame}


\begin{frame}
  \begin{footnotesize}
    
  \end{footnotesize}
\end{frame}


\begin{frame}
  \begin{footnotesize}
    
  \end{footnotesize}
\end{frame}
