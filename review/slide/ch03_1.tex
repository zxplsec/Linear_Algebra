%%%%%%%%%%%%%%%%%%%%%%%%%%%%%%%%%%%%%%%%%%%%%%%%%%%%%%%%%%%%%%%%%%%% 

\subsection{典型例题1~~(线性相关性)}
\begin{frame}\ft{\subsecname}  
  \begin{li}[2007-2008第一学期,2010-2011第一学期]
    若有不全为零的数$\lambda_1,\lambda_2,\cd,\lambda_m$使得$\lambda_1\alphabd_1+\lambda_2\alphabd_2+\cd+\lambda_m\alphabd_m+\lambda_1\betabd_1+\lambda_2\betabd_2+\cd+\lambda_m\betabd_m=\M0$ 成立,则$\alphabd_1,\alphabd_2,\cd,\alphabd_m$线性相关,$\betabd_1,\betabd_2,\cd,\betabd_m$也线性相关。试讨论该结论是否正确?
  \end{li}    
  \pause
  该题可转换为:
  $$(\MA+\MB)\vx=\M0\mbox{有非零解}~~\xLongrightarrow[]{\ds \red{?}}~~ \MA\vx=\M0\mbox{和}\MB\vx=\M0\mbox{都有非零解}$$
  
\end{frame}

\begin{frame}\ft{\subsecname}
  
  \begin{li}[2007-2008第二学期]
    设$\MA$为$m\times n$矩阵,$\MB$为$n\times m$矩阵,$\MI$为单位矩阵,易知$\MB\MA=\MI$,试判断$\MA$的列向量组是否线性相关?为什么?
  \end{li}
  
  \pause 
  \begin{jie}
    一方面
    $$
    \rank(\MA) \ge \rank(\MB\MA) =n,
    $$
    另一方面
    $$
    \rank(\MA)\le n
    $$
    故$\rank(\MA)=n$,于是$\MA$的列向量组线性无关。
  \end{jie}
\end{frame}


\begin{frame}\ft{\subsecname}
  
  \begin{li}[2012-2013第二学期]
    设$\MA$为$n\times m$矩阵,$\MB$为$m\times n$矩阵,$n<m$且$\MA\MB=\MI$,证明$\MB$的列向量组线性无关。
  \end{li}

  
\end{frame}

\begin{frame}\ft{\subsecname}
  
  \begin{li}[2008-2009第一学期]
    证明:与基础解系等价的线性无关的向量组也是基础解系。
  \end{li}
  \pause\proofname
  设$A:\alphabd_1,\alphabd_2,\cd,\alphabd_r$为基础解系,$B:\betabd_1,\betabd_2,\cd,\betabd_s$是$A$的等价组,且线性无关。
  由于$B$等价于$A$,故$A,B$可以互相线性表示。因$A$为基础解系,齐次线性方程组的全部解能由$A$线性表示,而$A$可由$B$线性表示,故齐次线性方程组的全部解能由$B$线性表示。注意到$r(A)=r$和$r(B=s)$,而$A$与$B$等价,故$r(A)=r(B)$,即$r=s$。综上所述,$B$也为基础解系。
  
\end{frame}


\begin{frame}\ft{\subsecname}
  
  \begin{li}[2009-2010第一学期]
    已知向量组$\alphabd_1,\alphabd_2,\alphabd_3,\alphabd_4$线性无关,
    \begin{itemize}
    \item[1] 向量组$\alphabd_1,\alphabd_2,\alphabd_3$是否线性无关,并说明理由。
    \item[2] 常数$l,m$满足何种条件时,$l\alphabd_1+\alphabd_2,\alphabd_2+\alphabd_3,m\alphabd_3+\alphabd_1$线性无关,并说明理由。
    \end{itemize}
  \end{li}
  \pause \proofname
  \begin{itemize}
  \item[1] 整体无关,则部分无关。
  \item[2]   
    设$
    x_1(l\alphabd_1+\alphabd_2)+x_2(\alphabd_2+\alphabd_3)+x_3(m\alphabd_3+\alphabd_1)=\M0  
    $
    即
    $$
    (lx_1+x_3)\alphabd_1+(x_1+x_2)\alphabd_2 +(x_2+mx_3)\alphabd_3=\M0
    $$      
    由于$\alphabd_1,\alphabd_2,\alphabd_3$线性无关,故
    $$
    \left\{
      \begin{array}{rcl}
        lx_1+x_3&=&0\\
        x_1+x_2&=&0\\
        x_2+mx_3&=&0
      \end{array}
    \right.
    $$
    只有零解。
  \end{itemize}
  
\end{frame}




\subsection{典型例题2~~(极大无关组与向量组的秩)}

\begin{frame}\ft{\subsecname}
  
  \begin{li}[$\bigstar\bigstar\bigstar\bigstar\bigstar$]
    设向量组
    $$
    \alphabd_1=\left(
      \begin{array}{r}
        -1\\-1\\0\\0
      \end{array}
    \right),~~ \alphabd_2=\left(
      \begin{array}{r}
        1\\2\\1\\-1
      \end{array}
    \right),~~ \alphabd_3=\left(
      \begin{array}{r}
        0\\1\\1\\-1
      \end{array}
    \right),~~ \alphabd_4=\left(
      \begin{array}{r}
        1\\3\\2\\1
      \end{array}
    \right),~~ \alphabd_5=\left(
      \begin{array}{r}
        2\\6\\4\\-1
      \end{array}
    \right)
    $$
    求向量组的秩及其一个极大无关组,并将其余向量用该极大无关组线性表示。
  \end{li}
  \pause

  \begin{jie}
    作矩阵$\MA=(\alphabd_1,\alphabd_2,\alphabd_3,\alphabd_4,\alphabd_5)$,由
    $$
    \begin{array}{rl}
      \MA &= \left(
            \begin{array}{rrrrr}
              -1&1&0&1&2\\
              -1&2&1&3&6\\
              0&1&1&2&4\\
              0&-1&-1&1&-1
            \end{array}
                         \right) \xlongrightarrow[r_2+r_1]{ r_1\times(-1)}
                         \left(
                         \begin{array}{rrrrr}
                           1&-1&0&-1&-2\\
                           0&1&1&2&4\\
                           0&1&1&2&4\\
                           0&-1&-1&1&-1
                         \end{array}
                                      \right)\\[0.28in]
          &\xlongrightarrow[r_4+r_2]{r_3- r_2}
            \left(
            \begin{array}{rrrrr}
              1&-1&0&-1&-2\\
              0&1&1&2&4\\
              0&0&0&0&0\\
              0&0&0&3&3
            \end{array}
                       \right) \xlongrightarrow[r_3\leftrightarrow r_4]{r_4\div 3}
                       \left(
                       \begin{array}{rrrrr}
                         1&-1&0&-1&-2\\
                         0&1&1&2&4\\
                         0&0&0&1&1\\
                         0&0&0&0&0
                       \end{array}
                                  \right)
    \end{array}
    $$
  \end{jie}
  
\end{frame}

\begin{frame}\ft{\subsecname}
  
  $$
  \begin{array}{rl}
    & \xlongrightarrow[r_2 -2 r_3]{r_1+r_3}
      \left(
      \begin{array}{rrrrr}
        1&-1&0&0&-1\\
        0&1&1&0&2\\
        0&0&0&1&1\\
        0&0&0&0&0
      \end{array}
                 \right) \xlongrightarrow[]{r_1+r_2}
                 \left(
                 \begin{array}{rrrrr}
                   1&0&1&0&1\\
                   0&1&1&0&2\\
                   0&0&0&1&1\\
                   0&0&0&0&0
                 \end{array}
                            \right) = \MB
  \end{array}
  $$
  将最后一个阶梯矩阵$\MB$记为$(\betabd_1,\betabd_2,\betabd_3,\betabd_4,\betabd_5)$
  \pause 
  \vspace{0.1in}

  易知$\betabd_1,\betabd_2,\betabd_4$为$\MB$的列向量组的一个极大无关组,故$\alphabd_1,\alphabd_2,\alphabd_4$也为$\MA$的列向量组的一个极大无关组,故
  $$
  \rank(\alphabd_1,\alphabd_2,\alphabd_3,\alphabd_4,\alphabd_5)=3,
  $$
  且
  $$
  \begin{array}{l}
    \alphabd_3=\alphabd_1+\alphabd_2,\\
    \alphabd_5=\alphabd_1+2\alphabd_2+\alphabd_4,\\
  \end{array}
  $$
  
  
\end{frame}




\begin{frame}
  
  \begin{li}[$\bigstar\bigstar\bigstar\bigstar\bigstar$]
    设$\alphabd_1=(1,3,1,2), ~\alphabd_2=(2,5,3,3), ~\alphabd_3=(0,1,-1,a), ~\alphabd_4=(3,10,k,4)$,
    试求向量组$\alphabd_1,~\alphabd_2,~\alphabd_3,~\alphabd_4$的秩,并将$\alphabd_4$用$\alphabd_1,~\alphabd_2,~\alphabd_3$线性表示。
  \end{li}
  \pause 
  \begin{jie}
    将4个向量按列排成一个矩阵$\MA$,对$\MA$进行初等变换,将其化为阶梯形矩阵$\MU$,即
    $$
    \MA=\left(
      \begin{array}{rrrr}
        1&2&0&3\\
        3&5&1&10\\
        1&3&-1&k\\
        2&3&a&4
      \end{array}
    \right) \xlongrightarrow[]{\mbox{初等行变换}}
    \left(
      \begin{array}{rrcc}
        1&2&0&3\\
        0&-1&1&1\\
        0&0&a-1&-3\\
        0&0&0&k-2
      \end{array}
    \right)=\MU
    $$
    \pause 
    \begin{itemize}
    \item[(1)] 当$a=1$或$k=2$时,$\MU$只有3个非零行,故
      $$\rank(\alphabd_1,\alphabd_2,\alphabd_3,\alphabd_4)=\rank(\MA)=3. $$ 
    \item[(2)] \pause 当$a\ne1$且$k\ne2$时,
      $$\rank(\alphabd_1,\alphabd_2,\alphabd_3,\alphabd_4)=\rank(\MA)=4.$$
    \end{itemize}
  \end{jie}
\end{frame}


\begin{frame}
  
  $$
  \MA=\left(
    \begin{array}{rrrr}
      1&2&0&3\\
      3&5&1&10\\
      1&3&-1&k\\
      2&3&a&4
    \end{array}
  \right) \xlongrightarrow[]{\mbox{初等行变换}}
  \left(
    \begin{array}{rrcc}
      1&2&0&3\\
      0&-1&1&1\\
      0&0&a-1&-3\\
      0&0&0&k-2
    \end{array}
  \right)
  $$
  \begin{itemize}
  \item 当$k=2$且$a\ne1$时,$\alphabd_4$可由$\alphabd_1,~\alphabd_2,~\alphabd_3$线性表示,
    且
    $$
    \alphabd_4=-\frac{1+5a}{1-a}\alphabd_1+\frac{2+a}{1-a}\alphabd_2+\frac{3}{1-a}\alphabd_3.
    $$
  \item \pause 当$k\ne2$或$a=1$时,$\alphabd_4$不能由$\alphabd_1,~\alphabd_2,~\alphabd_3$线性表示。
  \end{itemize}
  
\end{frame}


\begin{frame}
  
  \begin{li}[$\bigstar\bigstar\bigstar\bigstar\bigstar$]
    设
    $$
    \MA=\left(
      \begin{array}{rrr}
        1&2&1\\
        2&2&-2\\
        -1&t&5\\
        1&0&-3
      \end{array}
    \right)
    $$
    已知$\rank(\MA)=2$,求$t$。
  \end{li}
  \pause
  \begin{jie}
    $$
    \MA \xlongrightarrow[]{\mbox{初等行变换}} \left(
      \begin{array}{ccr}
        1&2&1\\
        0&-2&-4\\
        0&2+t&6\\
        0&0&0
      \end{array}
    \right)=\MB
    $$ \pause
    由于$\rank(\MB)=\rank(\MA)$,故$\MB$中第2、3行必须成比例,即
    $$
    \frac{-2}{2+t}=\frac{-4}6,
    $$
    即得$t=1$。
  \end{jie}
\end{frame}


%% \begin{frame}\ft{\subsecname}
%%   
%%   \begin{li}[2005-2006第一学期]
%%     对于$\RANK^3$中的向量组$A:\alphabd_1,\alphabd_2,\alphabd_3$和$B:\betabd_1,\betabd_2,\betabd_3$,讨论下面的问题:
%%     \begin{itemize}
%%     \item[(1)] 向量组$B$能否成为$\RANK^3$中的基?能否用$A$线性表示$B$?如果可以,试求出由$\alphabd_1,\alphabd_2,\alphabd_3$到$\betabd_1,\betabd_2,\betabd_3$的过渡矩阵$P$,其中
%%       $$
%%       \begin{aligned}
%%         \alphabd_1=\left(
%%           \begin{array}{c}
               %%                1\\0\\0
               %%              \end{array}
               %%                \right),
               %%                \alphabd_2=\left(
               %%                \begin{array}{c}
               %%                1\\1\\0
               %%              \end{array}
               %%                \right),
               %%                \alphabd_3=\left(
               %%                \begin{array}{c}
               %%                1\\1\\1
               %%              \end{array}
               %%                \right),
               %%                \\
               %%                \betabd_1=\left(
               %%                \begin{array}{c}
               %%                1\\1\\a
               %%              \end{array}
               %%                \right),
               %%                \betabd_2=\left(
               %%                \begin{array}{c}
               %%                1\\1\\2-a
               %%              \end{array}
               %%                \right),
               %%                \betabd_3=\left(
               %%                \begin{array}{c}
               %%                -1\\1\\0
               %%              \end{array}
               %%                \right), (a\in \RANK)
               %%                \end{aligned}
               %%                $$
               %%                \item[(2)]
               %%                若$\betabd_1=k(2\alphabd_1+2\alphabd_2-\alphabd_3), \betabd_2=k(2\alphabd_1-\alphabd_2+2\alphabd_3),\betabd_3=k(\alphabd_1-2\alphabd_2-2\alphabd_3)$,$k$为非零实数,
               %%                \begin{itemize}
               %%                \item[(a)] 给出向量组$\betabd_1,\betabd_2,\betabd_3$线性无关的一个充要条件,并证明之;
               %%                \item[(b)] 给出矩阵$(\betabd_1,\betabd_2,\betabd_3)$为正交阵的一个充要条件,并证明之.
               %%                \end{itemize}
               %%                \end{itemize}
               %%                \end{li}
               %%                
               %%                \end{frame}



\begin{frame}\ft{\subsecname}
  
  \begin{li}[2005-2006第二学期]
    设$$\alphabd_1=(1,0,2,1),\alphabd_2=(2,0,1,-1),\alphabd_3=(1,1,0,1),\alphabd_4=(4,1,3,1),$$
    求向量组$\alphabd_1,\alphabd_2,\alphabd_3,\alphabd_4$的秩和一个极大无关组。
  \end{li}  
\end{frame}


\begin{frame}\ft{\subsecname}                              
  \begin{li}[2006-2007第二学期]
    计算向量组$$
    \begin{aligned}
      \alphabd_1=(1,-2,3,-1,2)^T,\alphabd_2=(2,1,2,-2,-3)^T, \\[1ex]
      \alphabd_3=(5,0,7,-5,-4)^T,\alphabd_4=(3,-1,5,-3,-1)^T
    \end{aligned}$$
    的秩和一个极大无关组,同时将其余向量表示成极大无关组的线性组合。
  \end{li}
  
\end{frame}






\begin{frame}\ft{\subsecname}
  
  \begin{li}[2007-2008第二学期]
    计算向量组
    $$\xibd_1=(1,2,3)^T,\xibd_2=(-8,4,8)^T,\xibd_3=(2,-1,-2)^T,
    \xibd_4=(10,5,6)^T
    $$的秩和一个极大无关组,同时将其余向量表示成极大无关组的线性组合。
  \end{li}
  
\end{frame}



\begin{frame}\ft{\subsecname}
  
  \begin{li}[2008-2009第一学期]
    计算向量组
    $$\xibd_1=(1,0,2,1)^T,\xibd_2=(2,0,1,-1)^T,\xibd_3=(1,1,0,1)^T,
    \xibd_4=(4,1,3,1)^T
    $$的秩和一个极大无关组,同时将其余向量表示成极大无关组的线性组合。
  \end{li}
  
\end{frame}




\begin{frame}\ft{\subsecname}
  
  \begin{li}[2008-2009第一学期]
    计算向量组$$\xibd_1=(1,1,0)^T,\xibd_2=(0,1,1)^T,\xibd_3=(1,1,1)^T,
    \xibd_4=(1,2,1)^T$$的秩和一个极大无关组,并给出向量组中不能由其余向量线性表示的向量。
  \end{li}
  
\end{frame}


\begin{frame}\ft{\subsecname}
  
  \begin{li}[2009-2010第一学期]
    已知线性方程组$\MA\vx=\vb$存在两个不同的解,其中$$\MA=\left(
      \begin{array}{ccc}
        \lambda&1&1\\
        0&\lambda-1&0\\
        1&1&\lambda
      \end{array}
    \right),\vb=\left(
      \begin{array}{c}
        a\\1 \\1
      \end{array}
    \right).$$
    \begin{itemize}
    \item[1] 求$\lambda,a$.
    \item[2] 求其通解。
    \end{itemize}
  \end{li}

  
\end{frame}


\begin{frame}\ft{\subsecname}
  
  \begin{li}[2009-2010第一学期]
    设有向量组$$\alphabd_1=(1,2,0)^T,\alphabd_2=(1,a+2,-3a)^T,\alphabd_3=(-1,-b-2,a+2b)^T,\betabd=(1,3,-3)^T,$$
    讨论当$a,b$为何值时,
    \begin{itemize}
    \item[1] $\betabd$不能由$\alphabd_1,\alphabd_2,\alphabd_3$线性表示;
    \item[2] $\betabd$可由$\alphabd_1,\alphabd_2,\alphabd_3$惟一地线性表示,并求出表示式;
    \item[3] $\betabd$可由$\alphabd_1,\alphabd_2,\alphabd_3$线性表示,但表示式不唯一,并求出表示式。
    \end{itemize}
  \end{li}
  
\end{frame}



\begin{frame}\ft{\subsecname}
  
  \begin{li}[2012-2013第二学期]
    已知
    $$
    \alphabd_1=\left(
      \begin{array}{c}
        1\\4\\0\\2
      \end{array}
    \right),\alphabd_2=\left(
      \begin{array}{c}
        2\\7\\1\\3
      \end{array}
    \right),\alphabd_3=\left(
      \begin{array}{r}
        0\\1\\-1\\a
      \end{array}
    \right),\betabd=\left(
      \begin{array}{c}
        3\\10\\b\\4
      \end{array}
    \right)
    $$
    问$a,b$为何值时,
    \begin{itemize}
    \item[1] $\betabd$不能由$\alphabd_1,\alphabd_2,\alphabd_3$线性表示;
    \item[2] $\betabd$可由$\alphabd_1,\alphabd_2,\alphabd_3$惟一地线性表示,并求出表示式;
    \item[3] $\betabd$可由$\alphabd_1,\alphabd_2,\alphabd_3$线性表示,但表示式不唯一,并求出表示式;
    \item[4] $\alphabd_1,\alphabd_2,\alphabd_3$线性相关,并在此时求它的秩和一个最大无关组,且用该最大无关组表示其余向量。
    \end{itemize}
  \end{li}  
\end{frame}
