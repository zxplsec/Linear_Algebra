\section{特征值问题}

%%%%%%%%%%%%%%%%%%%%%%%%%%%%%%% 
\subsection{知识点}


\begin{frame}\ft{特征值与特征向量}
  
  \begin{dingyi}[特征值与特征向量]
    设$\MA$为复数域$\mathbb C$上的$n$阶矩阵,如果存在数$\lambda\in\mathbb C$和非零的$n$维向量$\vx$使得
    $$
    \MA\vx=\lambda\vx
    $$
    则称$\lambda$为矩阵$\MA$的\blue{\underline{特征值}},$\vx$为$\MA$的对应于特征值$\lambda$的\blue{\underline{特征向量}}。
  \end{dingyi}  

  \begin{itemize}
  \item[(1)] 特征向量$\vx\ne\M0$;
  \item[(2)] 特征值问题是对方针而言的。 
  \end{itemize}
  
\end{frame}


\begin{frame}\ft{特征值与特征向量}
  
  由定义,$n$阶矩阵$\MA$的特征值,就是使齐次线性方程组
  $$
  (\MA-\lambda\MI)\vx=\M0
  $$
  有非零解的$\lambda$值,即满足方程
  $$
  \det(\MA-\lambda\MI)=0
  $$
  的$\lambda$都是矩阵$\MA$的特征值。


  
  
\end{frame}


\begin{frame}\ft{特征值与特征向量}
  
  \begin{dingyi}[特征多项式、特征矩阵、特征方程]
    设$n$阶矩阵$\MA=(a_{ij})$,则
    $$
    f(\lambda)=\det(\MA-\lambda\MI)
    =\left|
      \begin{array}{cccc}
        a_{11}- \lambda&a_{12}&\cd&a_{1n}\\[0.2cm]
        a_{21}&a_{22}-\lambda&\cd&a_{2n}\\[0.2cm]
        \vd&\vd&&\vd\\[0.2cm]
        a_{n1}&a_{n2}&\cd&a_{nn}-\lambda
      \end{array}
    \right|
    $$
    称为矩阵$\MA$的特征多项式,$\MA-\lambda\MI$称为$\MA$的特征矩阵,$\det(\MA-\lambda\MI)=0$称为$\MA$的特征方程。
  \end{dingyi}
  
  
\end{frame}

\begin{frame}\ft{特征值与特征向量}
  
  \begin{li}{例1}
    求矩阵
    $$
    \MA=\left(
      \begin{array}{rrr}
        5&-1&-1\\
        3&1&-1\\
        4&-2&1
      \end{array}
    \right)
    $$
    的特征值与特征向量。
  \end{li}
  \pause
  $$
  \begin{array}{rl}
    \det(\MA-\lambda\MI)&=\left|\begin{array}{rrr}
                                  5-\lambda&-1&-1\\
                                  3&1-\lambda&-1\\
                                  4&-2&1-\lambda
                                \end{array}\right| =-(\lambda-3)(\lambda-2)^2=0
  \end{array}
  $$
  故特征值为$\lambda_1=3,~\lambda_{2,3}=2\mbox{(二重特征值)}$。
  
\end{frame}

\begin{frame}\ft{特征值与特征向量}
  
  \begin{itemize}
  \item 对于特征值$\lambda_1=3$,齐次线性方程组$(\MA-3\MI)\vx=\M0$为
    $$
    \left(\begin{array}{rrr}
            2&-1&-1\\
            3&-2&-1\\
            4&-2&-2
          \end{array}\right)\left(
          \begin{array}{c}
            x_1\\
            x_2\\
            x_3
          \end{array}
        \right)=\left(
          \begin{array}{c}
            0\\
            0\\
            0
          \end{array}
        \right)
        $$
        基础解系为$\vx_1=(1,1,1)^T$,因此\red{$k_1\vx_1(k_1\ne 0)$}是$\MA$对应于$\lambda_1=3$的全部特征向量。\\[0.1in]
        
      \item 
        对于特征值$\lambda_{2,3}=2$,齐次线性方程组$(\MA-2\MI)\vx=\M0$为
        $$
        \left(\begin{array}{rrr}
                3&-1&-1\\
                3&-1&-1\\
                4&-2&-1
              \end{array}\right)\left(
              \begin{array}{c}
                x_1\\
                x_2\\
                x_3
              \end{array}
            \right)=\left(
              \begin{array}{c}
                0\\
                0\\
                0
              \end{array}
            \right)
            $$
            基础解系为$\vx_2=(1,1,2)^T$,因此\red{$k_2\vx_2(k_2\ne 0)$}是$\MA$对应于$\lambda_{2,3}=2$的全部特征向量。

          \end{itemize}•
          

          
        \end{frame}



        \begin{frame}\ft{特征值与特征向量的性质}
          
          \begin{dingli}
            设$n$阶矩阵$\MA=(a_{ij})$的$n$个特征值为$\lambda_1,\lambda_2,\cd,\lambda_n$,则
            \begin{itemize}
            \item[(1)] $\ds \sum_{i=1}^n\lambda_i=\sum_{i=1}^na_{ii}$
            \item[(2)] $\ds \prod_{i=1}^n\lambda_i=\det(\MA)$         
            \end{itemize}
          \end{dingli}
          

          \begin{itemize}
          \item 当$\det(\MA)\ne 0$,即$\MA$为可逆矩阵时,其特征值全为非零数;
          \item 奇异矩阵$\MA$至少有一个零特征值。      
          \end{itemize}
          
        \end{frame}


        


        \begin{frame}\ft{特征值与特征向量的性质}
          
          \begin{dingli}
            一个特征向量不能属于不同的特征值。
          \end{dingli}
          
          
        \end{frame}

        \begin{frame}\ft{特征值与特征向量的性质}
          
          \begin{xingzhi}{性质1}
            \begin{table}
              \caption{特征值与特征向量}
              
              \begin{tabular}{|c|c|c|}\hline
                &特征值&特征向量\\\hline
                \red{$\MA$}&\red{$\lambda$}&\red{$\vx$}\\ \hline 
                \hline 
                $k\MA$&$k\lambda$&$\vx$\\\hline
                $\MA^m$&$\lambda^m$&$\vx$\\\hline
                $\MA^{-1}$&$\lambda^{-1}$&$\vx$\\\hline
              \end{tabular}
            \end{table}
          \end{xingzhi}
          % 
          % \end{frame}
          % 
          % \begin{frame}\ft{特征值与特征向量的性质}
          %   
          \begin{xingzhi}{性质2}
            矩阵$\MA$与$\MA^T$的特征值相同。
          \end{xingzhi}
          
          
        \end{frame}

        

        % \begin{frame}\ft{特征值与特征向量的性质}
        %   
        %   \begin{li}{例}
        %     设$\MA=\left(
        %       \begin{array}{rrr}
        %         1&-1&1\\
        %         2&-2&2\\
        %         -1&1&-1
        %       \end{array}
        %     \right)$
        %     \begin{itemize}
        %     \item[(i)]求$\MA$的特征值与特征向量
        %     \item[(ii)] 求可逆矩阵$\MP$,使得$\MP^{-1}\MA\MP$为对角阵。 
        %     \end{itemize}
        %   \end{li}
        %   \pause
        %   $$
        %   \begin{array}{rl}
        %     |\lambda\MI-\MA|&=\left|
        %       \begin{array}{rrr}
        %         \lambda-1&1&-1\\
        %         -2&\lambda+2&-2\\
        %         1&-1&\lambda+1
        %       \end{array}
        %     \right| = \lambda^2(\lambda+2)
        %   \end{array}
        %   $$
        %   $\MA$的特征值为$\lambda_1=\lambda_2=0$(二重特征值)和$\lambda=-2$.
        %   
        % \end{frame}
        % 
        % \begin{frame}\ft{特征值与特征向量的性质}
        %   
        %   \begin{itemize}
        %   \item 当$\lambda_1=\lambda_2=0$时,
        %     $$
        %     \lambda_1\MI-\MA=\left(
        %       \begin{array}{rrr}
        %         -1&1&-1\\
        %         -2&2&-2\\
        %         1&-1&1
        %       \end{array}
        %     \right)\left(
        %       \begin{array}{c}
        %         x_1\\x_2\\x_3
        %       \end{array}
        %     \right)=\left(
        %       \begin{array}{c}
        %         0\\0\\0
        %       \end{array}
        %     \right)
        %     $$
        %     基础解系为$\vx_1=(1,1,0)^T$和$\vx_2=(-1,0,1)^T$,故$\MA$对应于$\lambda_1=0$的全体特征向量为
        %     $$
        %     k_1\vx_1+k_2\vx_2 ~~~(k_1,k_2\mbox{为不全为零的任意常数})
        %     $$
        %   \end{itemize}
        %   
        % \end{frame}
        % 
        % 
        % \begin{frame}\ft{特征值与特征向量的性质}
        %   
        %   \begin{itemize}
        %   \item 当$\lambda_3=-2$时,
        %     $$
        %     \lambda_1\MI-\MA=\left(
        %       \begin{array}{rrr}
        %         -3&1&-1\\
        %         -2&0&-2\\
        %         1&-1&-1
        %       \end{array}
        %     \right)\left(
        %       \begin{array}{c}
        %         x_1\\x_2\\x_3
        %       \end{array}
        %     \right)=\left(
        %       \begin{array}{c}
        %         0\\0\\0
        %       \end{array}
        %     \right)
        %     $$
        %     基础解系为$\vx_3=(-1,-2,1)^T$,故$\MA$对应于$\lambda_3=-2$的全体特征向量为
        %     $$
        %     k_3\vx_3 ~~~(k_3\mbox{为非零的任意常数})
        %     $$
        %   \end{itemize}
        %   
        % \end{frame}
        % 
        % 
        % \begin{frame}\ft{特征值与特征向量的性质}
        %   
        %   $$
        %   \MA(\vx_1,~\vx_2,~\vx_3)=(\vx_1,~\vx_2,~\vx_3)\left(
        %     \begin{array}{rrr}
        %       \lambda_1&&\\
        %       &\lambda_2&\\
        %       &&\lambda_3
        %     \end{array}
        %   \right)
        %   $$
        %   取
        %   $$
        %   \MP=(\vx_1,~\vx_2,~\vx_3)=\left(
        %     \begin{array}{rrr}
        %       1&-1&-1\\
        %       1&0&-2\\
        %       0&1&1
        %     \end{array}
        %   \right), ~~ \Lambdabd=\left(
        %     \begin{array}{rrr}
        %       0&&\\
        %       &0&\\
        %       &&-2
        %     \end{array}
        %   \right)
        %   $$
        %   则$\MP$可逆(因为$|\MP|\ne 0$),且
        %   $$
        %   \MP^{-1}\MA\MP=\Lambdabd
        %   $$
        %   
        % \end{frame}


        \begin{frame}\ft{特征值与特征向量的性质}
          
          \begin{li}{例}
            对于下列矩阵$\MA$的特征值,能做怎样的断言?
            \begin{itemize}
            \item[(1)] $\det(\MI-\MA^2)=0$
            \item[(2)] $\MA^k=0$
            \item[(3)] $\MA=k\MI-\MB$($\lambda_0$为$\MB$的特征值)
            \end{itemize}
          \end{li}
          
        \end{frame}


        \begin{frame}\ft{相似矩阵}
          
          \begin{dingyi}[相似矩阵]
            对于方阵$\MA$和$\MB$,若存在可逆矩阵$\MP$,使得
            $$
            \MP^{-1}\MA\MP=\MB,
            $$
            就称$\MA$相似于$\MB$,记作$\MA\sim\MB$.
          \end{dingyi}
          

          % 
          % 
          % \end{frame}
          % 
          % 
          % \begin{frame}
          %   
          \begin{dingli}
            相似矩阵的特征值相同。
          \end{dingli}
          
          
        \end{frame}



        \begin{frame}\ft{矩阵可对角化的条件}

          \red{矩阵可对角化,即矩阵与对角阵相似。 } 


          
          \begin{dingli}
            $\mbox{矩阵可对角化} ~~\Longleftrightarrow~~
            \mbox{$n$阶矩阵有$n$个线性无关的特征向量}$ 
          \end{dingli}
          
          % 
          % \end{frame}
          % 
          % 
          % \begin{frame}
          %   
          %   若$\MA$与$\Lambdabd$相似,则$\Lambdabd$的主对角元都是$\MA$的特征值。
          %   若不计$\lambda_k$的排列次序,则$\Lambdabd$是唯一的,称$\Lambdabd$为$\MA$的相似标准型。
          %   
          % \end{frame}
          % 
          % 
          % \begin{frame}
          %   
          \begin{dingli}
            $\MA$的属于不同特征值的特征向量是线性无关的。
          \end{dingli}
          
          % 
          % \end{frame}
          % 
          % \begin{frame}
          %   
          \begin{tuilun}{推论}
            若$\MA$有$n$个互不相同的特征值,则$\MA$与对角阵相似。
          \end{tuilun}
          
        \end{frame}

        \begin{frame}\ft{矩阵可对角化的条件}
          
          \begin{li}{例}
            设实对称矩阵
            $$
            \MA=\left(
              \begin{array}{rrrr}
                1&-1&-1&-1\\
                -1&1&-1&-1\\
                -1&-1&1&-1\\
                -1&-1&-1&1
              \end{array}
            \right)
            $$
            问$\MA$是否可对角化?若可对角化,求对角阵$\Lambdabd$及可逆矩阵$\MP$使得$\MP^{-1}\MA\MP=\Lambdabd$,再求$\MA^k$。
          \end{li}
          \pause\proofname
          $$
          |\MA-\lambda\MI|=\left|
            \begin{array}{rrrr}
              1-\lambda&-1&-1&-1\\
              -1&1-\lambda&-1&-1\\
              -1&-1&1-\lambda&-1\\
              -1&-1&-1&1-\lambda
            \end{array}
          \right|=(\lambda+2)(\lambda-2)^3
          $$
          故特征值为$\lambda_1=-2,~~\lambda_{2,3,4}=2$
          
        \end{frame}

        \begin{frame}\ft{矩阵可对角化的条件}
          
          \begin{itemize}
          \item
            对于特征值$\lambda_1=-2$,齐次线性方程组$(\MA+2\MI)\vx=\M0$为
            $$
            \left(
              \begin{array}{rrrr}
                3&-1&-1&-1\\
                -1&3&1&1\\
                -1&-1&3&1\\
                -1&-1&-1&3
              \end{array}
            \right)\left(
              \begin{array}{cccc}
                x_1\\x_2\\x_3\\x_4
              \end{array}
            \right)=\left(
              \begin{array}{cccc}
                0\\0\\0\\0
              \end{array}
            \right)
            $$
            基础解系为$$
            \vx_1=(1,1,1,1)^T,
            $$
            故对应于$\lambda_1=-2$的全部特征向量为
            $
            \red{ k_1\vx_1( k_1\ne 0)}.
            $
            \pause 
          \item
            对于特征值$\lambda_{2,3,4}=2$,齐次线性方程组$(\MA-2\MI)\vx=\M0$为
            $$
            (\MA-\lambda_2\MI)\vx= \left(
              \begin{array}{rrrr}
                -1&-1&-1&-1\\
                -1&-1&-1&-1\\
                -1&-1&-1&-1\\
                -1&-1&-1&-1
              \end{array}
            \right)\left(
              \begin{array}{cccc}
                x_1\\x_2\\x_3\\x_4
              \end{array}
            \right)=\left(
              \begin{array}{cccc}
                0\\0\\0\\0
              \end{array}
            \right)
            $$
            基础解系为
            $$
            \vx_2=(1,-1,0,0)^T,~~
            \vx_3=(1,0,-1,0)^T,~~
            \vx_4=(1,0,-1,0)^T,
            $$
            故对应于$\lambda_2=2$的全部特征向量为
            $\red{
              k_2\vx_2+k_3\vx_3+k_4\vx_4 (k_2,k_3,k_4\mbox{不全为零})}
            $
          \end{itemize}•
          
          
        \end{frame}


        \begin{frame}\ft{矩阵可对角化的条件}
          

          由特征值问题定义可知
          $$\red{\boxed{
              \MA(\vx_1,\vx_2,\vx_3,\vx_4)=(\vx_1,\vx_2,\vx_3,\vx_4)\left(
                \begin{array}{cccc}
                  \lambda_1&&&\\
                           &\lambda_2&&\\
                           &&\lambda_3&\\
                           &&&\lambda_4
                \end{array}
              \right)}}
          $$

          取
          $$	
          \MP=(\vx_1,\vx_2,\vx_3,\vx_4)=\left(
            \begin{array}{rrrr}
              1&1&1&1\\
              1&-1&0&0\\
              1&0&-1&0\\
              1&0&0&-1
            \end{array}
          \right)
          $$
          则$\MA\MP=\MP\Lambdabd$,注意到$\det(\MP)\ne 0$,于是
          $$
          \MA=\MP\Lambdabd\MP^{-1}.
          $$

          
        \end{frame}

        \begin{frame}\ft{矩阵可对角化的条件}
          
          \begin{li}{例2}
            设$\MA=(a_{ij})_{n\times n}$是主对角元全为$2$的上三角矩阵,且存在$a_{ij}\ne 0(i<j)$,问$\MA$是否可对角化?
          \end{li}
          \pause\proofname
          $$
          \MA=\left(
            \begin{array}{cccc}
              2&*&\cd&*\\
              0&2&\cd&*\\
              \vd&\vd&\dd&\vd\\
              0&0&\cd&2
            \end{array}
          \right)~~\Longrightarrow~~
          \det(\MA-\lambda\MI)=(2-\lambda)^n
          ~~\Longrightarrow~~
          \lambda=2\mbox{为$\MA$的$n$重特征值}
          $$ 
          $$
          \begin{array}{rl}
            \rank(2\MI-\MA)\ge 1 &
                                   \Longrightarrow~~
                                   (2\MI-\MA)\vx=\M0\mbox{的基础解系所含向量个数}\le n-1\\[0.1in] 
                                 &\Longrightarrow~~
                                   \MA\mbox{的线性无关的特征向量的个数}\le n-1\\[0.1in]
                                 &\Longrightarrow~~
                                   \MA\mbox{不与对角阵相似。}
          \end{array}
          $$
          
        \end{frame}



        \begin{frame}\ft{实对称矩阵的对角化}
          
          \begin{dingli}
            实对称矩阵$\MA$的任一特征值都是实数。
          \end{dingli}
          
          % 
          % \end{frame}
          % 
          % 
          % \begin{frame}
          %   
          \begin{dingli}
            实对称矩阵$\MA$对应于不同特征值的特征向量\red{正交}。
          \end{dingli}
          
          
        \end{frame}



        \begin{frame}\ft{实对称矩阵的对角化}
          
          \begin{dingli}
            对于$n$阶实对称矩阵$\MA$,存在$n$阶正交矩阵$\MT$,使得
            $$
            \MT^{-1}\MA\MT=\Lambdabd
            $$
          \end{dingli}
          
          
        \end{frame}



        \begin{frame}\ft{实对称矩阵的对角化}
          
          \begin{li}{例}
            设
            $$
            \MA=\left(
              \begin{array}{rrr}
                1&-2&2\\
                -2&-2&4\\
                2&4&-2
              \end{array}
            \right)
            $$
            求正交阵$\MT$,使$\MT^{-1}\MA\MT$为对角阵。      
          \end{li}
          \pause
          $$
          |\MA-\lambda\MI|=\left|
            \begin{array}{rrr}
              1-\lambda&-2&2\\
              -2&-2-\lambda&4\\
              2&4&-2-\lambda
            \end{array}
          \right| = -(\lambda-2)^2(\lambda+7)
          $$
          特征值为$\lambda_{1,2}=2$(二重)和$\lambda_3=-7$。
          
        \end{frame}


        \begin{frame}\ft{实对称矩阵的对角化}
          
          \begin{itemize}
          \item 对于特征值$\lambda_{1,2}=2$,齐次线性方程组$(\MA-2\MI)\vx=\M0$为
            $$
            \left(
              \begin{array}{rrr}
                -1&-2& 2\\
                -2&-3& 4\\
                2& 4&-4
              \end{array}
            \right)\left(
              \begin{array}{r}
                x_1\\
                x_2\\
                x_3
              \end{array}
            \right)=\left(
              \begin{array}{r}
                0\\
                0\\
                0
              \end{array}
            \right)
            $$
            得特征向量$\vx_1=(2,-1,0)^T,~~\vx_2=(2,0,1)^T$。 \\[0.1in]
          \item 
            对于特征值$\lambda_2=-7$,齐次线性方程组$(\MA-\lambda_2\MI)\vx=\M0$为
            $$
            \left(
              \begin{array}{rrr}
                8&-2&2\\
                -2&5&4\\
                2&4&5
              \end{array}
            \right)\left(
              \begin{array}{r}
                x_1\\
                x_2\\
                x_3
              \end{array}
            \right)=\left(
              \begin{array}{r}
                0\\
                0\\
                0
              \end{array}
            \right)
            $$
            得特征向量$\vx_3=(1,2,-2)^T$。
          \end{itemize}•
          


          
          
        \end{frame}


        \begin{frame}\ft{实对称矩阵的对角化}
          
          \begin{itemize}
          \item 对特征向量$\vx_1=(2,-1,0)^T,~~\vx_2=(2,0,1)^T$,先用\red{施密特正交化过程}正交化,然后单位化。
          \item[] 先正交化得
            $$
            \begin{array}{rl}
              \betabd_1&=\vx_1,\\[0.2cm]
              \betabd_2&\ds =\vx_2-\frac{(\vx_2,\betabd_1)}{(\betabd_1,\betabd_1)}\betabd_1\\[0.2in]
                       &\ds=\left(
                         \begin{array}{rrr}
                           2\\0\\1
                         \end{array}
              \right)-\frac45\left(
              \begin{array}{rrr}
                2\\-1\\0
              \end{array}
              \right)=\frac15\left(
              \begin{array}{rrr}
                2\\4\\5
              \end{array}
              \right)
            \end{array}
            $$
            \pause
            再单位化得
            $$
            \vy_1=\left(\frac{2\sqrt{5}}{5},~-\frac{\sqrt{5}}{5},~0\right)^T,~~
            \vy_2=\left(\frac{2\sqrt{5}}{15},~-\frac{4\sqrt{5}}{15},~\frac{\sqrt{5}}3\right)^T
            $$


          \item
            对特征向量$\vx_3=(1,2,-2)^T$单位化,得$\ds \vy_3=\left(\frac13,~~\frac23,~~-\frac23\right)^T$。
          \end{itemize}
          
        \end{frame}


        \begin{frame}\ft{实对称矩阵的对角化}
          
          取正交矩阵
          $$
          \MT=(\vy_1,~~\vy_2,~~\vy_3)=\left(
            \begin{array}{ccc}
              \ds\frac{2\sqrt{5}}{5}&\ds\frac{2\sqrt{5}}{15}&\ds\frac13\\[0.1in]
              \ds-\frac{\sqrt{5}}{5}&\ds-\frac{4\sqrt{5}}{15}&\ds\frac23\\[0.1in]
              0&\ds\frac{\sqrt{5}}3&\ds-\frac23
            \end{array}
          \right)
          $$
          则
          $$
          \MT^{-1}\MA\MT=\mathrm{diag}(2,2,-7).
          $$

          
        \end{frame}


        \begin{frame}\ft{实对称矩阵的对角化}
          
          \begin{li}{例}
            设实对称矩阵$\MA$和$\MB$是相似矩阵,证明:存在正交矩阵$\MP$,使$\MP^{-1}\MA\MP=\MB$。
          \end{li}
          \pause\proofname
          $$
          \begin{array}{rl}
            \MA\sim\MB & \Longrightarrow~~
                         \MA,~\MB\mbox{有相同的特征值$\lambda_1,\lambda_2,\cd,\lambda_n$}\\[0.1in]
                       & \Longrightarrow~~ \exists\mbox{正交阵}\MP_1,\MP_2,~~ s.t.~~
                         \MP_1^{-1}\MA\MP_1=\mathrm{diag}(\lambda_1,\lambda_2,\cd,\lambda_n)=\MP_2^{-1}\MA\MP_2\\[0.1in]
                       & \Longrightarrow~~
                         \MP_2\MP_1^{-1}\MA\MP_1 \MP_2^{-1}=\MB
          \end{array}
          $$
          取$\MP=\MP_1\MP_2^{-1}$,则$\MP$为正交阵,且
          $$
          \MP^{-1}\MA\MP=\MB
          $$
          
        \end{frame}

        


        \begin{frame}\ft{实对称矩阵的对角化}
          
          \begin{li}{例}
            设$\MA,\MB$都是$n$阶实对称矩阵,若存在正交矩阵$\MT$使$\MT^{-1}\MA\MT,~\MT^{-1}\MB\MT$都是对角阵,则$\MA\MB$是实对称矩阵。
          \end{li}
          \pause\proofname
          $$
          \begin{array}{rl}
            \left.
            \begin{array}{rl}
              \MT^{-1}\MA\MT=\Lambdabd_1\\[0.1in]
              \MT^{-1}\MB\MT=\Lambdabd_2
            \end{array}
            \right\} & \Longrightarrow~~
                       (\MT^{-1}\MA\MT)(\MT^{-1}\MB\MT)=\Lambdabd_1\Lambdabd_2=\Lambdabd_2\Lambdabd_1=(\MT^{-1}\MB\MT)(\MT^{-1}\MA\MT)      \\[0.2in]
                     & \Longrightarrow~~ \MT^{-1}\MA\MB\MT=\MT^{-1}\MB\MA\MT\\[0.2in]
                     & \Longrightarrow~~ \MA\MB=\MB\MA\\[0.2in]
                     & \Longrightarrow~~ (\MA\MB)^T=\MB^T\MA^T=\MB\MA=\MA\MB
          \end{array}
          $$
          
        \end{frame}

        \begin{frame}\ft{实对称矩阵的对角化}
            \begin{li}{$\bigstar\bigstar\bigstar$}
              三阶实对称矩阵$\MA$的特征值为$\lambda_1=-1,\lambda_2=\lambda_3=1$,对应于$\lambda_1=-1$的特征向量为$\alphabd_1=(0,1,1)^T$,求$\MA$。
            \end{li}\pause\proofname
            $$
            \MA \sim \diag(-1,~1,~1)
            $$ 
            注意\red{不同特征值对应的特征向量正交},在与$\alphabd_1$正交的平面上取两个线性无关的向量,如$\alphabd_2=(1,0,0)^T,\alphabd_3=(0,1,-1)^T$,则
            $$
            \MA(\alphabd_1,~\alphabd_2,~\alphabd_3)=(\alphabd_1,~\alphabd_2,~\alphabd_3)\left(
              \begin{array}{ccc}
                -1&&\\
                  &1&\\
                  &&1
              \end{array}
            \right)
            $$ 
            注意到$\alphabd_1,~\alphabd_2,~\alphabd_3$正交,单位化即得标准正交向量组
            $$
            \betabd_1=\frac1{\sqrt{2}}(0,1,1)^T,~~
            \betabd_2=(1,0,0)^T,~~
            \betabd_3=\frac1{\sqrt{2}}(0,1,-1)^T.
            $$ 
            令$\MP=(\betabd_1,~\betabd_2,~\betabd_3)$,则
            $$
            \MA=\MP\Lambdabd\MP^{-1}=\MP\Lambdabd\MP^{T}
            = \left(
              \begin{array}{rrr}
                1&0&0\\
                0&0&-1\\
                0&-1&0
              \end{array}
            \right)
            $$
        \end{frame}


        %%%%%%%%%%%%%%%%%%%%%%%%%%%%%%% 
        \subsection{往年试题}

        \begin{frame} 
          
          \begin{li}[05-06上]
            设二阶方阵$\MA$满足$\MA^2-3\MA+2\MI=\M0$,求$\MA$所有可能的特征值。
          \end{li}
        \end{frame}

        \begin{frame}
          \begin{li}[05-06下]
            设三阶方阵$\MA$有三个实特征值$\lambda_1,\lambda_2,\lambda_3$,且$\lambda_1=\lambda_2\ne\lambda_3$,如果$\lambda_1$对应两个线性无关的特征向量$\alphabd_1$和$\alphabd_2$,$\lambda_3$对应一个特征向量$\alphabd_3$,证明$\alphabd_1,\alphabd_2,\alphabd_3$线性无关。
          \end{li}
        \end{frame}

        \begin{frame}
          \begin{li}[05-06下]
            设$\MA=\left(
              \begin{array}{ccc}
                0&0&1\\
                1&1&-1\\
                x^2&0&0
              \end{array}
            \right)$,$x$为实数,试讨论$x$为何值时,$\MA$可与对角阵相似?
          \end{li}
          
        \end{frame}


        \begin{frame} 
          
          \begin{li}[06-07上,08-09上]
            设$\MA=\left(
              \begin{array}{ccc}
                1&k&1\\
                1&0&1\\
                0&1&0
              \end{array}
            \right)$,
            \begin{itemize}
            \item 当$k=1$时,是否存在正交矩阵$\MQ$,使得$\MQ^T\MA\MQ$为对角阵?如果存在,是否唯一?
            \item 当$k=0$时,$\MA$能否与对角阵相似?

            \end{itemize}•
          \end{li}
        \end{frame}

        \begin{frame}
          \begin{li}[07-08上]
            设$\MA=\left(
              \begin{array}{ccc}
                2&0&0\\
                1&2&-1\\
                1&0&1
              \end{array}
            \right)$,
            \begin{itemize}
            \item 求$\MA$的特征值和特征向量;
            \item 求$\MA^k$及其特征值和特征向量;
            \end{itemize}•
          \end{li}
          
        \end{frame}


        \begin{frame}                     
          \begin{li}[07-08下]
            已知$1,1,-1$是三阶实对称矩阵$\MA$的三个特征值,向量$\alphabd_1=(1,1,1)^T,\alphabd_2=(2,2,1)^T$是$\MA$的对应于$\lambda_1=\lambda_2=1$的特征向量。
            \begin{itemize}
            \item[(1)] 能否求出$\MA$的属于$\lambda_3=-1$的特征向量?如能,试求出该特征向量,若不能,请说明理由;
            \item[(2)]能否由此求得$\MA$?若能,试求之,若不能请说明理由。

            \end{itemize}•
          \end{li}
        \end{frame}

        \begin{frame}
          \begin{li}[08-09上]
            已知$\MA$是三阶方阵,且$\MA^2\ne \M0, \MA^3=\M0$。
            \begin{itemize}
            \item[(1)] 能否求出$\MA$的特征值?如能,试求出该特征值,若不能,请说明理由;
            \item[(2)] $\MA$能否对角化?若能,试求之,若不能请说明理由。
            \item[(3)] 已知$\MB=\MA^3-5\MA^2+3\MI$,能否求得$\det(\MB)$,若能,试求之,若不能请说明理由。

            \end{itemize}•
          \end{li}
          
        \end{frame}



        \begin{frame}                     
          \begin{li}[09-10下]
            设$\alphabd$是$n$维非零实列向量,$\MA=\MI-\frac2{\alphabd^T\alphabd}\alphabd\alphabd^T$,
            \begin{itemize}
            \item[(1)] 计算$\MA^T$,并回答$k\MI-\MA$能否对角化?请说明理由,其中$k$为常数;
            \item[(2)] 计算$\MA^2$,并回答$k\MI-\MA$是否可逆? 请说明理由,其中$k\ne\pm 1$为常数;
            \item[(3)] 给出$\MI-2\alphabd\alphabd^T$为正交矩阵的充分必要条件。
            \end{itemize}•
          \end{li}
        \end{frame}

        \begin{frame}
          \begin{li}[08-09上]
            已知$\MA$是三阶方阵,且$\MA^2\ne \M0, \MA^3=\M0$。
            \begin{itemize}
            \item[(1)] 能否求出$\MA$的特征值?如能,试求出该特征值,若不能,请说明理由;
            \item[(2)] $\MA$能否对角化?若能,试求之,若不能请说明理由。
            \item[(3)] 已知$\MB=\MA^3-5\MA^2+3\MI$,能否求得$\det(\MB)$,若能,试求之,若不能请说明理由。

            \end{itemize}•
          \end{li}

          
        \end{frame}



        \begin{frame} 
          

          \begin{li}[12-13下]
            已知$\MA$是三阶实对称阵,且$\MA^2+2\MA= \M0$,已知$\rank(\MA)=2$。
            \begin{itemize}
            \item[(1)] 求$\MA$的全部特征值? 
            \item[(2)] 计算$\det(\MA+4\MI)$
            \item[(3)] 当$k$为何值时,$\MA+k\MI$正定。

            \end{itemize}•
          \end{li}

        \end{frame}

        \begin{frame}
          \begin{li}[12-13下]
            已知三阶矩阵$\MA$的特征值为$1,2,3$,求$\det(\MA^3-5\MA^2+7\MA)$
          \end{li}
        \end{frame}

        \begin{frame}
          \begin{li}[12-13下]
            证明:设$\MA$为$n$阶非零实对称矩阵,则存在$n$维列向量$\vx$使得$\vx^T\MA\vx\ne 0$.
          \end{li}          
        \end{frame}



        \begin{frame} 
          
          \begin{li}[13-14上]
            设$\alphabd=(a_1,a_2,a_3)^T,\betabd=(b_1,b_2,b_3)^T$,且$\alphabd^T\betabd=2,\MA=\alphabd\betabd^T$,
            \begin{itemize}
            \item[(1)] 求$\MA$的特征值;
            \item[(2)] 求可逆阵$\MP$及对角阵$\Lambdabd$使得$\MP^{-1}\MA\MP=\Lambdabd$。

            \end{itemize}
          \end{li}

          
        \end{frame}
