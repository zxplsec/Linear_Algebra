\subsection{知识点}

\begin{frame}
  \begin{footnotesize}
    \begin{itemize}
    \item 行列式的定义
    \item[] 余子式、代数余子式、行列式的按行(列)展开
    \end{itemize}
  \end{footnotesize}
\end{frame}


\begin{frame}
  \begin{footnotesize}
    \begin{itemize}
    \item 行列式的性质
      \begin{itemize}
      \item 互换行与列,行列式不变
      \item 某行全为零,行列式为零
      \item 两行相等,行列式为零
      \item 两行成比例,行列式为零
      \item 行倍加,行列式不变
      \item 行倍乘,行列式倍乘
      \item 交换两行,行列式反号
      \item
        $$
        \sum_{k=1}^n a_{ik}A_{jk}=\delta_{ij}|\A|
        $$
      \end{itemize}
    \end{itemize}
  \end{footnotesize}
\end{frame}


\begin{frame}
  \begin{footnotesize}
    \begin{itemize}
    \item 行列式的计算
      \begin{itemize}
      \item 通过初等行变换化为上三角行列式
      \item 降阶法(使某行(列)只有一个非零元)
      \item 升阶法(加边法),适用于
        $$
    \left|
    \begin{array}{cccc}
      x_1 &  a  & \cd & a   \\
      a   & x_2 & \cd & a   \\
      \vd & \vd &     & \vd \\
      a   &  a  & \cd & x_n
    \end{array}
    \right|
    $$
    或
    $$
    \left|
    \begin{array}{cccc}
      x_1 & a_1  & \cd & a_n   \\
      a_1 & x_2 & \cd  & a_n   \\
      \vd & \vd &     & \vd \\
      a_1 & a_2  & \cd & x_n
    \end{array}
    \right|
    $$
      \end{itemize}
    \end{itemize}
  \end{footnotesize}
\end{frame}


\begin{frame}
  \begin{footnotesize}
    一些特殊的行列式
    \begin{itemize}
    \item 奇数阶反对称矩阵的行列式等于零
    \item 对角行列式
      $$
      \left|
      \begin{array}{cccc}
      a_{11}&&&\\
      &a_{22}&&\\
      &&\dd&\\
      &&&a_{nn}        
      \end{array}
      \right|      
      =a_{11}a_{22}\cd a_{nn}
      $$
    \item 三角行列式
      $$
      \left|
      \begin{array}{cccc}
      a_{11}&*&\cd&*\\
      &a_{22}&\cd&*\\
      &&\dd&\vd\\
      &&&a_{nn}        
      \end{array}
      \right|
      =\left|
      \begin{array}{cccc}
      a_{11}&&&\\
      *&a_{22}&&\\
      \vd&\vd&\dd&\\
      *&*&\cd&a_{nn}        
      \end{array}
      \right|
      =a_{11}a_{22}\cd a_{nn}
      $$
    \item 斜三角行列式
      $$
    \left|
    \begin{array}{ccccc}
        &   & &  & a_n \\
        &   & & a_{n-1} & * \\
       &  & & \vdots & \vdots\\  
        &  a_2 & \cdots & * & * \\
      a_1 & * & \cdots & * & *
    \end{array}
    \right| = (-1)^{\frac{n(n-1)}2}
    a_1a_2\cd a_{n}
    $$
    \end{itemize}
  \end{footnotesize}
\end{frame}


\begin{frame}
  \begin{footnotesize}
    \begin{itemize}
    \item 对角块行列式
      $$
      \left|
      \begin{array}{cccc}
      \A_{1}&&&\\
      &\A_{2}&&\\
      &&\dd&\\
      &&&\A_{n}        
      \end{array}
      \right|      
      =|\A_{1}|~|\A_{2}|~\cd~|\A_{n}|
      $$      

      $$
      \left|
      \begin{array}{cc}
        \A&\\
        \C&\B
      \end{array}
      \right|=|\A|~|\B|
      $$
    \end{itemize}
  \end{footnotesize}
\end{frame}


\begin{frame}
  \begin{footnotesize}
    \begin{itemize}
    \item 爪形行列式
        \begin{center}
    \begin{tikzpicture}
      \matrix(A) [matrix of math nodes,nodes in empty cells,ampersand replacement=\&,left delimiter=|,right delimiter=|] {
        a_{11} \& a_{12} \& a_{13} \& \cd \& a_{1n} \\
        a_{21} \& a_{22} \& 0     \& \cd \&  0    \\
        a_{31} \&  0    \& a_{33} \& \cd \&  0    \\
        \vd  \&  \vd  \&  \vd  \&     \&  \vd  \\
        a_{n1} \&  0    \& 0 \& \cd \& a_{nn} \\
      };
      \draw[red] (A-1-1.center) -- (A-1-5.center) (A-1-1.center) -- (A-5-1.center) (A-1-1.center) -- (A-5-5.center);
    \end{tikzpicture}
  \end{center}
  \pause 
  其解法固定,即从第二行开始,每行依次乘一个系数然后加到第一行,使得第一行除第一个元素外都为零,从而得到一个下三角行列式。
  
  \pause\vspace{0.1in}

  类似的方式还可用于求解如下形式的“爪型行列式”
  \begin{figure}
    \centering
    \subfigure[]{
      \begin{tikzpicture}
        \matrix(B) [matrix of math nodes,nodes in empty cells,ampersand replacement=\&,left delimiter=|,right delimiter=|] {
          \&  \& \\
          \&  \& \\
          \&  \& \\ 
        };
        \draw[red] (B-1-3.north east) -- (B-1-1.north west) 
        (B-1-3.north east) -- (B-3-1.south west) 
        (B-1-3.north east) -- (B-3-3.south east);
      \end{tikzpicture}
    }
    \subfigure[]{
      \begin{tikzpicture}
        \matrix(B) [matrix of math nodes,nodes in empty cells,ampersand replacement=\&,left delimiter=|,right delimiter=|] {
          \&  \& \\
          \&  \& \\
          \&  \& \\
        };
        \draw[red]
        (B-3-1.south west) -- (B-1-1.north west) 
        (B-3-1.south west) -- (B-1-3.north east) 
        (B-3-1.south west) -- (B-3-3.south east);
      \end{tikzpicture}
    }
    \subfigure[]{
      \begin{tikzpicture}
        \matrix(B) [matrix of math nodes,nodes in empty cells,ampersand replacement=\&,left delimiter=|,right delimiter=|] {
          \&  \& \\
          \&  \& \\
          \&  \& \\
        };
        \draw[red]
        (B-3-3.south east) -- (B-1-1.north west) 
        (B-3-3.south east) -- (B-1-3.north east) 
        (B-3-3.south east) -- (B-3-1.south west);
      \end{tikzpicture}
    }
      
    \end{figure}

    \end{itemize}
  \end{footnotesize}
\end{frame}



\begin{frame}
  \begin{footnotesize}
    \begin{itemize}
    \item 发散型行列式
      $$
        D_{2n} = \left|
        \begin{array}{cccccc}
          a &     & & & & b \\
          & \dd & & & \id & \\
          &   & a & b &  & \\
          &   & c & d &  &  \\
          & \id & & & \dd & \\
          c &     & & & & d
        \end{array}
        \right|=(ad-bc)^n
        $$        
    \end{itemize}
  \end{footnotesize}
\end{frame}



\begin{frame}
  \begin{footnotesize}
    \begin{itemize}
    \item 范德蒙德(Vandermonde)行列式
        $$
        D_n = \left|
        \begin{array}{cccc}
          1        &  1        & \cd &    1     \\                    
          x_1      &  x_2      & \cd &    x_n    \\ 
          x_1^2    &  x_2^2     & \cd &   x_n^2   \\ 
          \vd      &  \vd      &     &    \vd      \\
          x_1^{n-1} & x_2^{n-1} &  \cd &  x_n^{n-1}
        \end{array}
        \right|
        = \prod_{n \ge i > j \ge 1}(x_i-x_j).
        $$
        
        \begin{exampleblock}{常见题型}
        $$
        \left|
        \begin{array}{ccc}
          1   &   1   &   1\\
          a   &   b   &   c\\
          a^3 &   b^3 &   c^3
        \end{array}
        \right|=(a+b+c)(b-a)(c-a)(c-b)
        $$\vspace{0.1in}
        
        $$
        \left|
        \begin{array}{ccc}
          1&a^2&a^3\\
          1&b^2&b^3\\
          1&c^2&c^3
        \end{array}
        \right|=(ab+bc+ca)
        \left|
        \begin{array}{ccc}
          1&a&a^2\\
          1&b&b^2\\
          1&c&c^2
        \end{array}
        \right|
        $$
      \end{exampleblock}

    \end{itemize}
  \end{footnotesize}
\end{frame}
