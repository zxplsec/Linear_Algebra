\subsection{往年试题}

\begin{frame}  
    \begin{li}[2005-2006第二学期]
      设$\MA=\left(
      \begin{array}{rrrr}
        1&2&-3&-1\\
        1&-2&1&1\\
        0&1&-1&2\\
        3&0&2&4
      \end{array}
      \right)$,求$|\MA\MA^T|$。
    \end{li}
    \pause

    \begin{jie}
    因
    $\red{
    |\MA\MA^T|=|\MA|~|\MA^T|=|\MA|^2}
    $,而
    $$
    \begin{array}{rl}
      |\MA|&=\left|
      \begin{array}{rrrr}
        1&2&-3&-1\\
        0&-4&4&2\\
        0&1&-1&2\\
        0&-6&11&7
      \end{array}
      \right|=\left|
      \begin{array}{rrr}
        -4&4&2\\
        1&-1&2\\
        -6&11&7
      \end{array}
      \right|\\[0.3in]
      &=\left|
      \begin{array}{rrr}
        -4&0&2\\
        1&0&2\\
        -6&5&7
      \end{array}
      \right|=-5\left|
      \begin{array}{rrr}
        -4&2\\
        1&2\\
      \end{array}
      \right|=50
    \end{array}
    $$
    故$$|\MA\MA^T|=2500$$
  \end{jie}
\end{frame}


\begin{frame}
  
    \begin{li}[2009-2010第二学期]
      设$\MA=\left(
      \begin{array}{rrr}
        1&2&3\\
        4&5&6\\
        7&8&9\\
        10&11&12
      \end{array}
      \right)$,求$|\MA\MA^T|$。
    \end{li}
    \pause

    \begin{jie}
    因$\red{\rank(\MA\MA^T)\le \rank(\MA)\le 3}$,故$\MA\MA^T$为降秩矩阵,从而$|\MA\MA^T|=0$。
    \end{jie}
\end{frame}


\begin{frame}
  
    \begin{li}[2006-2007第一学期]
      设$\MA=(a_{ij})$为$2007$阶方阵,其中$a_{ij}=i-j$,求$|\MA|$
    \end{li}
    \pause
    \begin{jie}
    注意到
    $$
    a_{ij}=i-j=-(j-i)=-a_{ji},
    $$
    故$\MA$为反对称矩阵,由\red{奇数阶反对称矩阵的行列式为零}可知,$$|\MA|=0.$$
    \end{jie}
\end{frame}



\begin{frame}
  
    \begin{li}[2006-2007第二学期]
      计算
      $$
      D=\left|
      \begin{array}{rrrrrr}
        1&2&3&\cd&n-1&n\\
        -1&0&3&\cd&n-1&n\\
        -1&-2&0&\cd&n-1&n\\
        \vd&\vd&\vd&\dd&\vd&\vd\\
        -1&-2&-3&\cd&-(n-1)&0
      \end{array}
      \right|~~(n\ge 1)
      $$
    \end{li}
    \pause 
    \begin{jie}
    $$
    D\xlongequal[\ds i=2,\cd,n]{\ds r_i+r_1}\left|
      \begin{array}{rrrrrr}
        1&2&3&\cd&n-1&n\\
        0&2&6&\cd&2(n-1)&2n\\
        0&0&3&\cd&2(n-1)&2n\\
        \vd&\vd&\vd&\dd&\vd&\vd\\
        0&0&0&\cd&0&n
      \end{array}
      \right|=n!
    $$
    \end{jie}
\end{frame}

\begin{frame}
  
    \begin{li}[2006-2007第二学期]
      计算
      $$
      D=\left|
      \begin{array}{rrrrrr}
        1&2&3&\cd&n-1&n\\
        2&3&4&\cd&n&n+1\\
        3&4&5&\cd&n+1&n+2\\
        \vd&\vd&\vd&\dd&\vd&\vd\\
        n&n+1&n+2&\cd&2n-2&2n-1
      \end{array}
      \right|
      $$
    \end{li}
    \pause

    \begin{jie}
    $$
    \begin{array}{rl}
      D&\xlongequal[i=n,n-1,\cd,2]{\ds r_{i}-r_{i-1}}\left|
      \begin{array}{rrrrrr}
        1&2&3&\cd&n-1&n\\
        1&1&1&\cd&1&1\\
        1&1&1&\cd&1&1\\
        \vd&\vd&\vd&\dd&\vd&\vd\\
        1&1&1&\cd&1&1
      \end{array}
      \right|=0
    \end{array}
    $$
    \end{jie}
    
\end{frame}


\begin{frame}
  
    \begin{li}[2007-2008第一学期,2010-2011第二学期,2011-2012第一学期]
      计算
      $
      D=\left|
      \begin{array}{rrrrr}
        x+a_1&a_2&a_3&\cd&a_n\\
        a_1&x+a_2&a_3&\cd&a_n\\
        a_1&a_2&x+a_3&\cd&a_n\\
        \vd&\vd&\vd&\dd&\vd\\
        a_1&a_2&a_3&\cd&x+a_n
      \end{array}
      \right|
      $
    \end{li}
    

    \pause
    \begin{small}
    \begin{jie}[加边法]
    \begin{itemize}
    \item 当$x=0$时,$D=0$
    \item 当$x\ne 0$时,
      $$
    \begin{array}{rl}
            D&=\left|
      \begin{array}{rrrrrr}
        1&a_1&a_2&a_3&\cd&a_n\\
        0&x+a_1&a_2&a_3&\cd&a_n\\
        0&a_1&x+a_2&a_3&\cd&a_n\\
        0&a_1&a_2&x+a_3&\cd&a_n\\
        \vd&\vd&\vd&\vd&\dd&\vd\\
        0&a_1&a_2&a_3&\cd&x+a_n
      \end{array}      \right|=\left|
      \begin{array}{rrrrrr}
        1&a_1&a_2&a_3&\cd&a_n\\
        -1&x&0&0&\cd&0\\
        -1&0&x&0&\cd&0\\
        -1&0&0&x&\cd&0\\
        \vd&\vd&\vd&\vd&\dd&\vd\\
        -1&0&0&0&\cd&x
      \end{array}
      \right|\\[0.4in]
      &\xlongequal[]{r_1-\frac{a_1}xr_2-\cd--\frac{a_n}xr_{n+1}}
      \left|
      \begin{array}{rrrrrr}
        1+\frac{a_1+a_2+\cd+a_n}x&0&0&0&\cd&0\\
        -1&x&0&0&\cd&0\\
        -1&0&x&0&\cd&0\\
        -1&0&0&x&\cd&0\\
        \vd&\vd&\vd&\vd&\dd&\vd\\
        -1&0&0&0&\cd&x
      \end{array}
      \right|=\ds x^{n-1}(x+\sum_{i=1}^na_i)
    \end{array}
      $$
    \end{itemize}
  \end{jie}
  \end{small}
\end{frame}


\begin{frame}
  
    \begin{li}[2008-2009第一学期]
      计算$
      D=\left|
      \begin{array}{ccccc}
        a_1-b&a_2&a_3&\cd&a_n\\
        a_1&a_2-b&a_3&\cd&a_n\\
        a_1&a_2&a_3-b&\cd&a_n\\
        \vd&\vd&\vd&\dd&\vd\\
        a_1&a_2&a_3&\cd&a_n-b
      \end{array}
      \right|
      $
    \end{li}
  
\end{frame}

\begin{frame}
  
    \begin{li}[2011-2012第二学期]
      计算
      $
      D=\left|
      \begin{array}{cccccc}
        x&1&2&\cd&8&9\\
        1&x&2&\cd&8&9\\
        1&2&x&\cd&8&9\\
        \vd&\vd&\vd&\dd&\vd&\vd\\
        1&2&3&\cd&x&9\\
        1&2&3&\cd&9&x
      \end{array}
      \right|
      $
    \end{li}
    
    \pause
    \begin{small}
    \begin{jie}
    $$
    \begin{array}{rl}
      D&=(x+45)\left|
    \begin{array}{cccccc}
      1&1&2&\cd&8&9\\
      1&x&2&\cd&8&9\\
      1&2&x&\cd&8&9\\
      \vd&\vd&\vd&\dd&\vd&\vd\\
      1&2&3&\cd&x&9\\
      1&2&3&\cd&9&x
      \end{array}
    \right|=(x+45)\left|
    \begin{array}{cccccc}
      1&1&2&\cd&8&9\\
      0&x-1&0&\cd&0&0\\
      0&1&x-2&\cd&0&0\\
      \vd&\vd&\vd&\dd&\vd&\vd\\
      0&1&1&\cd&x-8&0\\
      0&1&1&\cd&1&x-9
      \end{array}
    \right|\\[0.4in]
    &=(x+45)\left|
    \begin{array}{cccccc}      
      x-1&0&\cd&0&0\\
      1&x-2&\cd&0&0\\
      \vd&\vd&\dd&\vd&\vd\\
      1&1&\cd&x-8&0\\
      1&1&\cd&1&x-9
      \end{array}
    \right|=(x+45)(x-1)(x-2)\cd(x-9)
    \end{array}
    $$
  \end{jie}
  \end{small}
\end{frame}

\begin{frame}
  
    \begin{li}[2012-2013第二学期]
      计算
      $$
      \left|
      \begin{array}{ccccc}
        4&3&\cd&3&3\\
        3&4&\cd&3&3\\
        \vd&\vd&\dd&\vd&\vd\\
        3&3&\cd&4&3\\
        3&3&\cd&3&4
      \end{array}
      \right|
      $$
    \end{li}
    \pause

    \begin{jie}
      可用加边法
    \end{jie}
  
\end{frame}


\begin{frame}
  
    \begin{li}[2007-2008第一学期,2010-2011第一学期]
      设$\alphabd_1,\alphabd_2,\alphabd_3,\betabd_1,\betabd_2$都是四维列向量,且四阶行列式
      $$
      |\alphabd_1,\alphabd_2,\alphabd_3,\betabd_1|=m,~~
      |\alphabd_1,\alphabd_2,\betabd_2,\alphabd_3|=n,      
      $$
      求四阶行列式$|\alphabd_1,\alphabd_2,\alphabd_3,\betabd_1+\betabd_2|$
    \end{li}
    \pause

    \begin{jie}
    $$
    \begin{array}{rl}
      |\alphabd_1,\alphabd_2,\alphabd_3,\betabd_1+\betabd_2|&=
      |\alphabd_1,\alphabd_2,\alphabd_3,\betabd_1|+
      |\alphabd_1,\alphabd_2,\alphabd_3,\betabd_2|\\[0.1in]
      &=
      |\alphabd_1,\alphabd_2,\alphabd_3,\betabd_1|-
      |\alphabd_1,\alphabd_2,\betabd_2,\alphabd_3|=m-n
    \end{array}
    $$
  \end{jie}
\end{frame}


\begin{frame}
  
    \begin{li}[2007-2008第二学期]
      设$\alphabd_1,\alphabd_2,\alphabd_3$都是三维列向量,记三阶矩阵
      $$
      \MA=(\alphabd_1,\alphabd_2,\alphabd_3),~~
      \MB=(\alphabd_1+\alphabd_2+\alphabd_3,\alphabd_1+2\alphabd_2+4\alphabd_3,\alphabd_1+3\alphabd_2+9\alphabd_3),
      $$已知$|\MA|=1$,求$|\MB|$。
    \end{li}
    \pause

    \begin{jie}
    因
    $$
    \MB=(\alphabd_1,\alphabd_2,\alphabd_3)\left(
    \begin{array}{ccc}
      1&1&1\\
      1&2&3\\
      1&4&9
    \end{array}
    \right)=\MA\left(
    \begin{array}{ccc}
      1&1&1\\
      1&2&3\\
      1&4&9
    \end{array}
    \right)
    $$
    而
    $$
    \left|
    \begin{array}{ccc}
      1&1&1\\
      1&2&3\\
      1&4&9
    \end{array}
    \right|=\left|
    \begin{array}{ccc}
      1&1&1\\
      0&1&2\\
      0&3&8
    \end{array}
    \right|=\left|
    \begin{array}{ccc}
      1&1&1\\
      0&1&2\\
      0&0&2
    \end{array}
    \right|=2    
    $$
    故$$|\MB|=2|\MA|=2.$$
  \end{jie}
\end{frame}


\begin{frame}
  
    \begin{li}[2008-2009第一学期]
      计算
      $
      D=\left|
      \begin{array}{rrr}
        -ab&ac&ae\\
        bd&-cd&de\\
        bf&cf&-cf
      \end{array}
      \right|
      $
    \end{li}
    \pause

    \begin{jie}
    $$
    \begin{array}{rl}
    D&=adf\left|
      \begin{array}{rrr}
        -b&c&e\\
        b&-c&e\\
        b&c&-c
      \end{array}
      \right|=adf\left|
      \begin{array}{rrr}
        -b&c&e\\
        0&0&2e\\
        0&2c&e-c
      \end{array}
      \right|\\[0.2in]
      &=-adf\left|
      \begin{array}{rrr}
        -b&c&e\\
        0&2c&e-c\\
        0&0&2e
      \end{array}
      \right|  
      =-adf\cdot(-b)\cdot 2c \cdot 2e=4abcdef
    \end{array}
    $$
    \end{jie}
\end{frame}




\begin{frame}
  
    \begin{li}[2012-2013第二学期]
      计算
      $
      D=\left|
      \begin{array}{rrrr}
        1&2&0&0\\
        -1&1&0&0\\
        0&-1&-2&4\\
        0&0&-1&-3
      \end{array}
      \right|
      $
    \end{li}
    \pause

    \begin{jie}
    $$
    D=\left|
    \begin{array}{rr}
      1&2\\
      -1&1
    \end{array}
    \right|\left|
    \begin{array}{rr}
      -2&4\\
      -1&-3
    \end{array}
    \right|=3\cdot10=30
    $$
    \end{jie}
\end{frame}


\begin{frame}
  
    \begin{li}[2013-2014第一学期]
      在$n$阶行列式$D$中,如果把第一列移到最后一列,而其余各列保持原来次序各向左移动了一列,得到行列式$\Delta$,问$\Delta$与$D$有何关系?
    \end{li}
    \pause

    \begin{jie}
    $$
    \Delta=(-1)^{n-1}D
    $$
  \end{jie}
  
\end{frame}


