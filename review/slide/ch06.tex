\section{第六章~~二次型}

\subsection{知识点}

\begin{frame}\ft{二次型的定义和矩阵表示}
  
  \begin{dingyi}
    $n$元变量$x_1,x_2,\cd,x_n$的二次齐次多项式
    $$
    \begin{array}{rcccccccc}
      f(x_1,x_2,\cd,x_n) &=& \\[0.1in]
      a_{11}x_1^2&+&2a_{12}x_1x_2&+&2a_{13}x_1x_3&+&\cd&+&2a_{1n}x_1x_n\\[0.1in]
                         &+&a_{22}x_2^2&+&2a_{23}x_2x_3&+&\cd&+&2a_{2n}x_2x_n\\[0.1in]
                         &&&&\cd&&\cd\\[0.1in]
                         &&&&&&&+&a_{nn}x_n^2\\[0.1in]
    \end{array}
    $$
    其矩阵形式为
    $$
    f(x_1,x_2,\cd,x_n)=   (x_1,x_2,\cd,x_n)\left(
      \begin{array}{cccc}
        a_{11}&a_{12}&\cd&a_{1n}\\
        a_{21}&a_{22}&\cd&a_{2n}\\
        \vd&\vd&&\vd\\
        a_{n1}&a_{n2}&\cd&a_{nn}\\
      \end{array}
    \right)\left(
      \begin{array}{c}
        x_1\\
        x_2\\
        \vd\\
        x_n
      \end{array}\right) = \vx^T\MA\vx
    $$
  \end{dingyi}
  
\end{frame}




\begin{frame}\ft{二次型的定义和矩阵表示}
  
  \begin{li}
    设$f(x_1,x_2,x_3,x_4)=2x_1^2+x_1x_2+2x_1x_3+4x_2x_4+x_3^2+5x_4^2$,则它的矩阵为
    $$
    \MA=\left(
      \begin{array}{cccc}
        2&\ds1/2&1&0\\[0.2cm]
        \ds1/2&0&0&2\\[0.2cm]
        1&0&1&0\\[0.2cm]
        0&2&0&5
      \end{array}
    \right)
    $$
  \end{li}
  
\end{frame}


\begin{frame}\ft{二次型的定义和矩阵表示}
  
  设$\alphabd$在两组基$\{\epsilonbd_1,\epsilonbd_2,\cd,\epsilonbd_n\}$和$\{\etabd_1,\etabd_2,\cd,\etabd_n\}$下的坐标向量分别为
  $$
  \vx=(x_1,x_2,\cd,x_n)^T\mbox{~~和~~}\vy=(y_1,y_2,\cd,y_n)^T
  $$
  又
  $$
  (\etabd_1,\etabd_2,\cd,\etabd_n)=(\epsilonbd_1,\epsilonbd_2,\cd,\epsilonbd_n)\MC
  $$
  故
  $$
  \vx=\MC\vy
  $$
  从而
  $$
  f(\alphabd)=\vx^T\MA\vx=\vy^T(\MC^T\MA\MC)\vy
  $$ \pause


  \blue{
    二次型$f(\alphabd)$在两组基$\{\epsilonbd_1,\epsilonbd_2,\cd,\epsilonbd_n\}$和$\{\etabd_1,\etabd_2,\cd,\etabd_n\}$下所对应的矩阵分别为
    $$
    \MA \mbox{~~和~~} \MC^T\MA\MC     
    $$
  }
  
\end{frame}






\begin{frame}\ft{矩阵的合同}
  
  \begin{dingyi}[矩阵的合同]
    对于两个矩阵$\MA$和$\MB$,若存在可逆矩阵$\MC$,使得
    $$
    \MC^T\MA\MC=\MB,
    $$
    就称$\MA$合同于$\MB$,记作$\MA\simeq\MB$。
  \end{dingyi}
  
\end{frame}


\begin{frame}\ft{用正交变换法将二次型化为标准型}
  
  \begin{itemize}
  \item 含平方项而不含混合项的二次型称为\red{标准二次型}。\\[0.2cm]
  \item 化二次型为标准型,就是对实对称矩阵$\MA$,寻找可逆阵$\MC$,使$\MC^T\MA\MC$成为对角矩阵。
  \end{itemize}
  
\end{frame}






\begin{frame}\ft{用正交变换法将二次型化为标准型}
  
  \begin{dingli}[主轴定理]
    对于任一个$n$元二次型
    $$
    f(x_1,x_2,\cd,x_n)=\vx^T\MA\vx,
    $$
    存在正交变换$\vx=\MQ\vy$~($\MQ$为正交阵),使得
    $$
    \vx^T\MA\vx=\vy^T(\MQ^T\MA\MQ)\vy=\lambda_1y_1^2+\lambda_2y_2^2+\cd+\lambda_ny_n^2,
    $$
    其中$\lambda_1,\lambda_2,\cd,\lambda_n$为$\MA$的$n$个特征值,
    $\MQ$的$n$个列向量$\alphabd_1,\alphabd_2,\cd,\alphabd_n$是$\MA$对应于$\lambda_1,\lambda_2,\cd,\lambda_n$的标准正交特征向量。
  \end{dingli}
  
\end{frame}



\begin{frame}\ft{用正交变换法将二次型化为标准型}
  
  \begin{li}{\blue{$\bigstar\bigstar\bigstar\bigstar\bigstar$}}
    用正交变换法,将二次型
    $$
    f(x_1,x_2,x_3)=2x_1^2+5x_2^2+5x_3^2+4x_1x_2-4x_1x_3-8x_2x_3
    $$
    化为标准型。
  \end{li}
  \pause
  对应方程为
  $$
  \MA=\left(
    \begin{array}{rrr}
      2&2&-2\\
      2&5&-4\\
      -2&-4&5
    \end{array}
  \right)
  $$
  \pause
  其特征多项式为
  $$
  \det(\MA-\lambda\MI)=-(\lambda-1)^2(\lambda-10)
  $$
  得特征值$\lambda_{1,2}=1$和$\lambda_3=10$.
  
\end{frame}


\begin{frame}\ft{用正交变换法将二次型化为标准型}
  
  $$
  \begin{array}{rl}
    (\MA-\MI)\vx=\M0 & \Rightarrow~~
                       \left(
                       \begin{array}{rrr}
                         1&2&-2\\
                         2&4&-4\\
                         -2&-4&4
                       \end{array}
                                \right)\left(
                                \begin{array}{c}
                                  x_1\\
                                  x_2\\
                                  x_3
                                \end{array}
    \right)=\M0\\[0.3in]  
                     & \Rightarrow~~
                       \vx_1=(-2,1,0)^T, \quad
                       \vx_2=(2,0,1)^T. \\[0.2in] 
    (\MA-10\MI)\vx=\M0 & \Rightarrow~~
                         \left(
                         \begin{array}{rrr}
                           -8&2&-2\\
                           2&-5&-4\\
                           -2&-4&-5
                         \end{array}
                                  \right)\left(
                                  \begin{array}{c}
                                    x_1\\
                                    x_2\\
                                    x_3
                                  \end{array}
    \right)=\M0\\[0.3in]  
                     & \Rightarrow~~
                       \vx_3=(1,2,-2)^T.
  \end{array}
  $$ \pause 

  对$\vx_1,\vx_2$用施密特正交化过程先正交化,再单位化,得
  $$
  \xibd_1=\left(-\frac{2\sqrt{5}}5,\frac{2\sqrt{5}}5,0\right)^T,~~~~
  \xibd_2=\left(\frac{2\sqrt{5}}{15},\frac{4\sqrt{5}}{15},\frac{\sqrt{5}}3\right)^T
  $$ 
  对$\vx_3$单位化,得
  $$
  \xibd_3=\left(\frac13,\frac23,-\frac23\right)^T
  $$
  
\end{frame}


\begin{frame}\ft{用正交变换法将二次型化为标准型}
  
  取正交矩阵
  $$
  \MQ=(\xibd_1,\xibd_2,\xibd_3)=\left(
    \begin{array}{rrr}
      \ds-\frac{2\sqrt{5}}5&\ds\frac{2\sqrt{5}}{15}&\ds\frac13\\[0.2cm]
      \ds \frac{2\sqrt{5}}5&\ds\frac{4\sqrt{5}}{15}&\ds\frac23\\[0.2cm]
      \ds 0&\ds\frac{\sqrt{5}}3&\ds-\frac23
    \end{array}
  \right)
  $$
  则
  $$
  \MQ^{-1}\MA\MQ=\MQ^{T}\MA\MQ=\diag(1,1,10).
  $$ \pause 
  令$\vx=(x_1,x_2,x_3)^T,\vy=(y_1,y_2,y_3)^T$,\red{做正交变换$\vx=\MQ\vy$,原二次型就化成标准型}
  $$
  \vx^T\MA\vx=\vy^T(\MQ^T\MA\MQ)\vy=y_1^2+y_2^2+10y_3^2.
  $$
  
\end{frame}








\begin{frame}\ft{惯性定理和二次型的规范形}
  
  \begin{dingli}[惯性定理]
    对于一个$n$元二次型$\vx^T\MA\vx$,不论做怎样的坐标变换使之化为标准形,其中正平方项的项数$p$和负平方项的项数$q$都是唯一确定的。或者说,对一个$n$阶实对称矩阵$\MA$,不论取怎样的可逆矩阵$\MC$,只要使
    $$
    \MC^T\MA\MC=\left(
      \begin{array}{cccccccccc}
        d_1&&&&&&&&\\
           &\dd&&&&&&&\\
           &&d_p&&&&&&\\
           &&&-d_{p+1}&&&&&\\
           &&&&\dd&&&&\\
           &&&&&-d_{p+q}&&&\\
           &&&&&&0&&\\
           &&&&&&&\dd&\\
           &&&&&&&&0
      \end{array}
    \right)
    $$
    其中$d_i>0(i=1,2,\cd,p+q),p+q\le n$成立,则$p$和$q$是由$\MA$唯一确定的。
  \end{dingli}
  
\end{frame}



\begin{frame}\ft{惯性定理和二次型的规范形}
  
  \begin{dingyi}
    二次型$\vx^T\MA\vx$的标准形中,
    \begin{itemize}
    \item  正平方项的项数(与$\MA$合同的对角阵中正对角元的个数),称为二次型(或$\MA$)的\red{正惯性指数};
    \item  负平方项的项数(与$\MA$合同的对角阵中负对角元的个数),称为二次型(或$\MA$)的\red{负惯性指数};
    \item
      正、负惯性指数的差称为符号差;
    \item 矩阵$\MA$的秩也成为\red{二次型$\vx^T\MA\vx$的秩}。
    \end{itemize}
  \end{dingyi}
  \pause
  设$\rank(\MA)=r$,正惯性指数为$p$,则
  \begin{itemize}
  \item 负惯性指数为$q=r-p$
  \item 符号差为$p-q=2p-r$
  \item 与$\MA$合同的对角阵的零对角元个数为$n-r$。
  \end{itemize}
  
  
\end{frame}

\begin{frame}\ft{惯性定理和二次型的规范形}
  
  \begin{tuilun}
    设$\MA$为$n$阶实对称矩阵,若$\MA$的正、负惯性指数分别为$p$和$q$,则
    $$\blue{
      \MA\simeq\underbrace{\diag(\underbrace{1,\cd,1}_{p\mbox{个}},\underbrace{-1,\cd,-1}_{q\mbox{个}},\underbrace{0,\cd,0}_{n-p-q\mbox{个}})}_{\red{\ds\MA\mbox{的合同规范形}}}
    }
    $$ \pause 
    或者说,对于二次型$\vx^T\MA\vx$,存在坐标变换$\vx=\MC\vy$,使得
    $$
    \blue{
      \vx^T\MA\vx=\underbrace{y_1^2+\cd+y_p^2-y_{p+1}^2-\cd-y_{p+q}^2}_{\red{\ds\vx^T\MA\vx\mbox{的规范形}}}.
    }
    $$
  \end{tuilun}
  
\end{frame}




\begin{frame}\ft{正定二次型和正定矩阵}
  
  \begin{dingyi}
    如果对于任意的非零向量$\vx=(x_1,x_2,\cd,x_n)^T$,恒有
    $$
    \vx^T\MA\vx=\sum_{i=1}^n\sum_{j=1}^na_{ij}x_ix_j>0,
    $$
    就称$\vx^T\MA\vx$为正定二次型,称$\MA$为正定矩阵。
  \end{dingyi}
  \pause\vspace{0.1in}

  
  注:正定矩阵是针对对称矩阵而言的。
  
  
\end{frame}

\begin{frame}
  
  \begin{jielun}
    二次型$f(y_1,y_2,\cd,y_n)=d_1y_1^2+d_2y_2^2+\cd+d_ny_n^2$正定
    $~~~\Longleftrightarrow~~~d_i>0~~(i=1,2,\cd,n)$
  \end{jielun} 
  \begin{jielun}
    一个二次型$\vx^T\MA\vx$,经过非退化线性变换$\vx=\MC\vy$,化为$\vy^T(\MC^T\MA\MC)\vy$,其正定性保持不变。即当
    $$\vx^T\MA\vx~~~\xLongleftrightarrow[]{\ds \vx=\MC\vy}~~~\vy^T(\MC^T\MA\MC)\vy\quad (\MC\mbox{可逆})$$
    时,等式两端的二次型有相同的正定性。
  \end{jielun} 
  
\end{frame}


\begin{frame}
  
  \begin{dingli}
    若$\MA$是$n$阶实对称矩阵,则以下命题等价:
    \begin{itemize}
    \item[(1)]$\MA$正定;
    \item[(2)]$\MA$的正惯性指数为$n$,即$\MA\simeq\MI$;
    \item[(3)]存在可逆矩阵$\MP$使得$\MA=\MP^T\MP$;
    \item[(4)]$\MA$的$n$个特征值$\lambda_1,\lambda_2,\cd,\lambda_n$全大于零。
    \item[(5)]$\MA$的$n$个顺序主子式全大于零。
    \end{itemize}
  \end{dingli}
  
  \begin{dingli}
    $$
    \MA\mbox{正定}~~\Longrightarrow~~
    a_{ii}>0(i=1,2,\cd,n) \mbox{~~且~~}
    \det(\MA)>0
    $$
  \end{dingli}
  
\end{frame}

\begin{frame}
  
  \begin{li}
    $\MA\mbox{正定} ~~\Longrightarrow~~ \MA^{-1}\mbox{正定}$
  \end{li}
\end{frame}

\begin{frame}
  \begin{li}
    判断二次型
    $$
    f(x_1,x_2,x_3)=x_1^2+2x_2^2+3x_3^2+2x_1x_2-2x_2x_3
    $$
    是否为正定二次型。
  \end{li}
  
\end{frame}

\begin{frame}
    
  \begin{li}
    判断二次型
    $$
    f(x_1,x_2,x_3)=3x_1^2+x_2^2+3x_3^2-4x_1x_2-4x_1x_3+4x_2x_3
    $$
    是否为正定二次型。
  \end{li}
  
\end{frame}






%%%%%%%%%%%%%%%%%%%%%%%%%%%%% 
\subsection{典型例题}


\begin{frame}
  
  \begin{li}[2005-2006第一学期]
    求二次型
    $
    f(x_1,x_2,x_3)=(x_1+x_2)^2+(x_2-x_3)^2+(x_3+x_1)^2
    $
    的秩。
  \end{li}
\end{frame}

\begin{frame}
  \begin{li}[2005-2006第一学期]
    设二次型
    $
    f(x_1,x_2,x_3)=x_1^2+x_2^2+x_3^2-2x_1x_2-2x_2x_3-2x_3x_1,
    $
    \begin{itemize}
    \item[(1)] 求二次型$f$的矩阵$\MA$的全部特征值;
    \item[(2)] 求可逆矩阵$\MP$,使得$\MP^{-1}\MA\MP$为对角阵;
    \item[(3)] 计算$\det(\MA^m)$.
    \end{itemize}•
  \end{li}  
\end{frame}


\begin{frame}  
  \begin{li}[2005-2006第二学期]
    判断二次型
    $
    f(x_1,x_2,x_3)=x_1^2+2x_2^2+6x_3^2+2x_1x_2+2x_1x_2+6x_2x_3
    $
    的正定性。
  \end{li}
\end{frame}

\begin{frame}
  \begin{li}[2006-2007第一学期]
    设二次型
    $
    f(x_1,x_2,x_3)=x_1^2+4x_2^2+2\lambda  x_1x_2-2x_1x_3+4x_2x_3,
    $
    试求该二次型的矩阵,并指出$\lambda$取何值时,$f$正定?
  \end{li}
  
\end{frame}


\begin{frame}
  \begin{li}[2006-2007第二学期]
    判断二次型
    $
    f(x,y,z)=3x^2+2y^2+2z^2+2xy+2xz
    $
    \begin{itemize}
    \item[(1)] 用正交变换化二次型$f$为标准型,并写出相应的正交阵;
    \item[(2)]  求$f(x,y,z)$在单位球面$x^2+y^2+z^2=1$上的最大值和最小值。
    \end{itemize}•
  \end{li}
\end{frame}

\begin{frame}
  \begin{li}[2006-2007第二学期]
    设二次型
    $
    f(x_1,x_2,x_3)=2x_1x_3+2x_1x_3-2x_2x_3,
    $
    \begin{itemize}
    \item[(1)] 写出二次型$f$的矩阵$\MA$;
    \item[(2)]  求出$\MA$的全部特征值和特征向量;
    \item[(3)] 化$f$为标准型;
    \item[(4)] 判断$f$是否正定.
    \end{itemize}•
  \end{li}
  
\end{frame}


\begin{frame}
  
  \begin{li}[2007-2008第一学期,2009-2010第一学期]
    对于二次型
    $
    f(x_1,x_2,x_3)=ax_1^2+2x_2^2-2x_3^2+2bx_1x_3(b>0),
    $其中二次型的矩阵$\MA$的特征值之和为$1$,特征值之积为$-12$.
    \begin{itemize}
    \item[(1)] 求$a,b$;
    \item[(2)]  化$f$为标准型,并写出所用的正交变换和正交矩阵。
    \end{itemize}•
  \end{li}
\end{frame}

\begin{frame}
  \begin{li}[2007-2008第二学期]
    设二次型的矩阵为
    $
    \left(
      \begin{array}{ccc}
        5&-a&2b-1\\
        a-b&c&2-c\\
        c-2&-3&3
      \end{array}
    \right), a,b,c
    $为常数,则
    \begin{itemize}
    \item[(1)] 写出二次型$f$的具体形式;
    \item[(2)]  求出$\MA$的全部特征值和特征向量;
    \item[(3)] 求正交变换$\vx=\MP\vy$,化$f$为标准型;
    \item[(4)] 在$\|\vx\|=1$的条件下,求$f$的最大值和最小值.
    \end{itemize}•
  \end{li}
  
\end{frame}


\begin{frame}
  
  \begin{li}[2008-2009第一学期]
    设二次型
    $
    f(x_1,x_2,x_3)=x_1^2+x_2^2+x_3^2+2ax_1x_2+2bx_2x_3+2x_1x_3,
    $经正交变换$\vx=\MP\vy$化为标准型$f=y_2^2+2y_3^2$,试求$a,b$。      
  \end{li}
\end{frame}

\begin{frame}
  \begin{li}[2008-2009第一学期]
    设二次型$f(x_1,x_2,x_3)=x_1^2+x_2^2+x_3^2-2x_1x_2-2x_2x_3-2x_3x_1$,      
    \begin{itemize}
    \item[(1)] 求出$\MA$的全部特征值和特征向量;
    \item[(2)] 求正交变换$\vx=\MP\vy$,化$f$为标准型;
    \item[(3)] 计算$\det(\MA^m)$
    \end{itemize}•
  \end{li}
  
\end{frame}


\begin{frame}
  
  \begin{li}[2009-2010第二学期]
    设二次型
    $
    f(x_1,x_2,x_3)=2x_1x_3+x_2^2
    $,\begin{itemize}
    \item[(1)] 求出$\MA$的全部特征值和特征向量;
    \item[(2)] 求正交变换$\vx=\MP\vy$,化$f$为标准型。
    \end{itemize}      
  \end{li}
\end{frame}

\begin{frame}
  \begin{li}[2010-2011第一学期]
    设二次型$f(x_1,x_2,x_3)=4x_2^2-3x_3^2+4x_1x_2-4x_1x_3+8x_2x_3$,      
    \begin{itemize}
    \item[(1)] 写出$\MA$;
    \item[(2)] 求正交变换$\vx=\MP\vy$,化$f$为标准型。
    \end{itemize}•
  \end{li}
  
\end{frame}


\begin{frame}
  

  \begin{li}[2010-2011第二学期]
    设二次型
    $
    f(x_1,x_2,x_3)=x_1^2+2x_2^2+x_3^2+2tx_1x_2+2x_1x_3
    $的矩阵是奇异阵,
    \begin{itemize}
    \item[(1)] 写出$\MA$并求$t$的值;
    \item[(2)] 根据所求$t$的值,求一个可逆矩阵$\MP$和一个对角阵$\Lambdabd$,使得$\MP^{-1}\MA\MP=\Lambdabd$;
    \item[(3)] 求$\MA^n(n\ge 2)$.
    \end{itemize}      
  \end{li}
\end{frame}

\begin{frame}
  \begin{li}[2011-2012第二学期]
    在正交变换$\vx=\MQ\vy$将二次型$f(x_1,x_2,x_3)=2x_1x_2+2x_1x_3+2x_2x_3$化为标准型。
  \end{li}
  
\end{frame}

\begin{frame}
  

  \begin{li}[2012-2013第二学期]
    已知二次型
    $
    f(x_1,x_2,x_3)=(1-a)x_1^2+(1-a)x_2^2+2x_3^2+2(1+a)x_1x_2 
    $的秩为2,
    \begin{itemize}
    \item[(1)] 求$a$;
    \item[(2)] 求正交变换$\vx=\MP\vy$,将$f$化为标准型.
    \end{itemize}      
  \end{li}
\end{frame}

\begin{frame}
  \begin{li}[2012-2013第二学期]
    已知二次型
    $
    f(x_1,x_2,x_3)=2x_1^2+3x_2^2+3x_3^2+4x_2x_3 
    $的秩为2,
    \begin{itemize}
    \item[(1)] 把$f$写成$f=\vx^T\MA\vx$的形式;
    \item[(2)] 求$\MA$的特征值和特征向量;
    \item[(3)] 求正交变换$\vx=\MP\vy$,将$f$化为标准型.
    \end{itemize}      
  \end{li}
  
\end{frame}

\begin{frame} 
  \begin{li}[2013-2014第一学期]
    用正交变换化二次型
    $
    f(x_1,x_2,x_3)=x_1^2+x_2^2+2x_3^2-2x_1x_2+4x_1x_3+4x_2x_3$为标准型.
  \end{li}  
\end{frame}






